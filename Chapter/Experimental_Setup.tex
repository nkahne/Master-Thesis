\chapter{Experimental Setup}

So far we discussed general aspects of dc-SQUIDs and how their working principle allows for highly sensitive magnetic flux measurements. As already briefly seen in subsection \ref{subsec_para_res}, when it comes to practical SQUIDs many theoretical considerations regarding parameter optimization need to be reevaluated to account for usability in practical experiments. We begin this chapter with general concepts of a practical SQUID design and introduce a typical low-noise setup with a room temperature readout electronic. In this working group, SQUIDs are mainly developed for the readout of \textit{Metallic Magnetic Microcalorimeters (MMCs)} (see section \ref{sec_MMC}). We will see in the following how those SQUIDs need to be designed to optimize their coupling to these detectors. Furthermore, this chapter will cover various methods to reduce quality factors of parasitic resonances, such as adding shunt resistors or coupling to normal conducting gold layers.

\section{Practical dc-SQUIDs}\label{sec_practical_SQUID}

The SQUIDs developed in this working group are used as current sensors for the MMC readout by sending the detected signals from the \textit{pickup coil} of the MMC to the input coil of the SQUID. The requirement of the latter entails a parasitic capacitance resulting in numerous resonances, as discussed in subsection \ref{subsec_para_res}. To achieve high inductive coupling between input coil and SQUID loop, it is necessary to fabricate them closely on top of each other, only separated by a thin insulating layer. The coupling strength is given by the dimensionless parameter 

\gl{
k_{\rm is} = \frac{M_{\rm is}}{\sqrt{L_{\rm i}L_{\rm s}}} \ \ , 
}{} 

where $M_{\rm is}=\Delta\Phi_{\rm s}/\Delta I_{\rm i}$ is the mutual inductance, describing how much flux $\Delta\Phi_{\rm s}$ is generated in the SQUID loop for a current change $\Delta I_{\rm i}$ in the input coil. This allows us to define the so-called coupled energy sensitivity $\epsilon_{\rm c}(f)$ with respect to the input coil, which by using equation \ref{energy_sens} is given as

\gl{
\epsilon_{\rm c}(f) = \frac{\epsilon(f)}{k_{\rm is}^2} = \frac{L_{\rm i}S_{I\rm ,i}}{2} \ \ .
}{}

This expression refers to the apparent current noise $S_{I\rm ,i}=S_{\rm \Phi_s}/M_{\rm is}^2$, which is generated by the flux noise from the SQUID loop through the coupling $M_{\rm is}$. A strong coupling can be achieved by the commonly used square \textit{washer}-geometry with a planar input coil \cite{Jay1981}, as shown in figure \ref{abb:washer}. 

\figurecenter {t!}
{width=\textwidth}
{../Figures/Washergeometrie}
{0cm}   %vspace
{Schematic drawing of a typical planar thin-film dc-SQUID. The SQUID loop is realized as a square washer-geometry interrupted by a narrow slit, only connected at the junction area. A thin insulating layer separates the washer from the planar multi-turn input coil above. Left: View from the top. Right: Cross section marked by the dashed line.}
{washer}

Here, the SQUID loop is represented by the washer, whereas each turn of the input coil is symmetrically located on top of it to maximize the coupling between each system. A cross section of this setup is depicted in figure \ref{abb:washer} (right), showing the insulating dielectric layer separating each coil. The washer is intersected by a slit, which starts at the square hole in the middle and ends at the remotely situated junction area that connects each side of the loop. The total inductance of the SQUID loop can be calculated by adding the dominating washer hole inductance $L_{\rm h}$, the slit inductance $L_{\rm t}\approx \qty{0.3}{\pH\per\um}$ and the much smaller parasitic inductance $L_{\rm j}$ associated with the junction area, giving \cite{Ketchen1991}

\gl{
L_{\rm s} = L_{\rm h} + L_{\rm t} + L_{\rm j} \ \ .
}{}

The latter is referred to as parasitic due to it's position outside of the input coil, thus not contributing to the coupling. By neglecting $L_{\rm t}$ and $L_{\rm j}$, we can approximate the washer inductance in the limit of $d\ll w$ to $L_{\rm s}\approx L_{\rm h}\approx 1.25{\rm\mu_0}d$, where $d$ and $w$ are the inner and outer side lengths, respectively \cite{Ketchen1981}. This is a reasonable result considering that the supercurrent will only flow along the inner edge of the washer \cite{Ketchen1982}, thereby being independent of the outer side length $w$. The effective area $A_{\rm eff}$ of the SQUID loop has been calculated to $A_{\rm eff}\approx dw$ \cite{Ketchen1985}, showing that this geometry allows for high sensitivity while keeping the SQUID inductance small. The input coil inductance on the other hand can be approximated by $L_{\rm i}=L_{\rm str}+n^2L_{\rm s}$, where $L_{\rm str}$ is the stripline inductance (see section \ref{sec_damping}) and $n$ is the number of input coil turns \cite{Ketchen1981}. The dc-SQUID designs used in this working group, however, are too complex to provide such analytical expressions and therefore need to be calculated numerically using simulation software such as \textit{InductEX}. 

\subsection{Gradiometer}

\figurecenter {t!}
{width=\textwidth}
{../Figures/Gradiometer}
{0cm}   %vspace
{caption}
{label}


\section{Operation of a dc-SQUID}

\subsection{Flux-Locked Loop}

\subsection{Two-Stage Configuration}

\section{Metallic Magnetic Microcalorimeters} \label{sec_MMC}

\subsection{Readout with Coupled dc-SQUIDs}

\section{dc-SQUID Design}

\subsection{dc-SQUID with a Two-Turn Input Coil}

%\subsection{Integrated Two-Stage Chip}

\section{Damping Methods} \label{sec_damping}

\subsection{Lossy Input Coil}

\subsection{Inductive Damping}