\chapter{dc-SQUID Design} \label{ch_SQUIDdesign}

The main objective for this thesis was the optimization of an existing Front-End SQUID design for an improved coupling to one of the detectors being developed in this working group. As we have seen in the previous section \ref{sec_MMC}, adjusting $L_{\rm i}$ to the detector coil ensures the maximization of the flux-to-flux coupling and therefore minimizes the extrinsic energy sensitivity. The previous SQUID design exhibits a design value of $L_{\rm i}=1.64$ for the input coil inductance \cite{Bauer2022}, which is well suited for the pickup coil inductance of the ECHo-100k detector of $L_{\rm p}=\qty{1.14}{\nano\henry}$ \cite{Mantegazzini2021}. The parasitic inductance $L_{\rm par}$, that arises from the aluminum bonds between the SQUID and the detector substrate to form the flux transformer, has been estimated to \qty{0.5}{\nano\henry} \cite{Hengstler2017}. Other MMC detectors from this working group such as the 4k-pixel molecule camera MOCCA and the X-ray detector maXs100 require higher input inductances, as their pickup coil inductances are $L_{\rm p}=\qty{8.8}{\nano\henry}$ and $L_{\rm p}=\qty{6.65}{\nano\henry}$, respectively. In \cite{Bauer2022} SQUIDs with matching input inductances for the MOCCA and maXs100 detector were developed for the first time using an intermediary coupling transformer. These improved the energy resolution $\Delta E_{\rm FWHM}$ of the detectors, although the effect was minimal for the latter due to a reduction of the effective coupling constant $k_{\rm is}'$, compensating the gain through an increased input inductance of $L_{\rm i}'=\qty{5.47}{\nano\henry}$. Specifically for the maXs100 detector, a different approach was followed in the framework of this thesis to achieve a better coupling while avoiding a significant increase of the detector noise.  

\section{dc-SQUID with a Two-Turn Input Coil}

Increasing the input coil inductance can be realized either by changing the geometry of the coil itself, or by implementing a double flux transformer structure, with the benefit of easily adapting the inductance independently of the SQUID design. The latter, however, was accompanied with a reduction of the effective coupling constant $k_{\rm is}'$ in the work of \cite{Bauer2022} regarding the maXs100 detector readout, which led to a lower flux-to-flux coupling despite the higher inductance of $L_{\rm i}'=\qty{5.47}{\nano\henry}$. Only the white noise reduction of the SQUID, which was caused by the shielding effects of the added flux transformer, resulted in a small overall improvement of $\epsilon_{\rm p}$ and $\Delta E_{\rm FWHM}$. \\
In this work we designed a new detector SQUID with \textit{window-type}\footnote{A well-established microfabrication technique to produce Josephson junctions for SQUIDs} Josephson junctions, which is based on the design developed in \cite{Bauer2022}. A schematic of this SQUID is shown in figure \ref{abb:NL_FE}. Four large oval loops form the second order gradiometer described in subsection \ref{subsec_gradio}, where the lower fabricated niobium layer (Nb1) represents the SQUID loop as a parallel gradiometer. The feed line coming from the top supplies the serially connected input coil, which was fabricated as a second niobium layer (Nb2) on top of the SQUID loop, only separated by an insulating SiO$_{\rm 2}$ layer. This geometry allows to combine a small SQUID loop inductance $L_{\rm s}=\frac{L_{\rm l}}{4}$ with a large input inductance $L_{\rm i}=4L_{\rm l}$. Also in the Nb2 layer, a feedback coil is located below the input coil whose feed lines are shown on the bottom left. The feed lines for the SQUID loop at the bottom center lead to the junction area, which is shown in the zoomed in section. Both square-shaped Nb/Al-Al$\rm O_x$/Nb junctions are realized with the dimensions $\qtyproduct{4.5 x 4.5}{\micro\meter}$ and a targeted critical current of $I_{\rm c}=\qty{6}{\micro\ampere}$, leading to a critical current density of $j_{\rm c}=\qty{30}{\ampere\per\centi\meter\squared}$. Two AuPd junction shunt resistors $R_{\rm s}$ with $\qty{6}{\ohm}$ each are located on the left and right side of the junction area, respectively. Both are attached to a large heat sink made of two gold layers, a thick galvanized layer on top of a sputtered, thin one. These so-called \textit{cooling fins} provide thanks to their large volume a better electron-phonon coupling, which reduces the electron temperature of the normal-conducting shunts and thus mitigates the corresponding thermal noise \cite{Mazibrada2024}. A third resistor $R_{\rm d}$ with the same dimensions is placed above the junction area and connected in parallel with the washer loop. This so-called \textit{washer shunt} provides damping properties to reduce quality factors of parasitic resonances, as will be discussed in section \ref{sec_damping}. \\

\figurecenter {t!}
{width=\textwidth}
{../Figures/screenshot_test}
{0cm}
{caption} 
{NL_FE}

As opposed to the previous design, this input coil is realized with two turns instead of one. Neglecting the stripline inductance $L_{\rm str}$ (see section \ref{sec_resonance_results}), the input inductance becomes approximately proportional to the number of turns squared $n^2$ \cite{Ketchen1981,Jaycox1981}, giving a theoretical value of $L_{\rm i}\approx2^{2}\cdot \qty{1.64}{\nano\henry}\approx\qty{6.56}{\nano\henry}$. To implement the second turn a wider washer loop line width of $w_{\rm s}=\qty{10}{\micro\meter}$ was needed. The line width of the input coil $w_{\rm i}$ with two turns remained at $\qty{3}{\micro\meter}$, which would have not been possible to fabricate on top of the previous washer width of \qty{5}{\micro\meter} without sacrificing coupling strength. As the input inductance was the only parameter necessary to adjust, we attempted to keep the other design parameters unaltered. The widened washer width therefore needed to be compensated with an approximately 10 \% larger washer hole circumference in order to maintain the same SQUID loop inductance. This was estimated by modelling the rather complicated oval washer loop geometry as a ring-shaped structure whose inductance can be calculated with the relation $L={\rm \mu_0}R\left(\ln\left(\frac{8R}{a}\right)-2\right)$, where \textit{R} denotes the loop radius and \textit{a} the radius of the wire \cite{Dengler2016}. \\

The design values for $R_{\rm s}$ and $L_{\rm s}$ can be obtained by minimizing the extrinsic energy sensitivity given in equation \ref{extr_energy_sens}, which requires the maximization of the flux-to-flux coupling ($L_{\rm i}=L_{\rm p}+L_{\rm par}$) and the minimization of the intrinsic white noise of the SQUID. In \cite{Bauer2022} this numerical calculation led to .          


Talk about the value of $\frac{\Delta\Phi_{\rm s}}{\Delta\Phi_{\rm p}}$ that we would expect given the design values. The same for $\epsilon_p$.)))

\section{Integrated Two-Stage Chip}

\textit{Figure}: Int. 2stage chip (klayout and microscope?)

Briefly cover its features

\section{Damping Methods} \label{sec_damping}

Talk about how resistors are used to damp resonances, like Rd and RxCx.\\ 
\textit{Figure}: Comparison of two IVC plots, one with RxCx, one without (see Fabienne Diss.) (?) \\
\textit{Figure}: Schematic circuit diagram of our SQUIDs with all L's, C's and R's

\subsection{Lossy Input Coil}

\textit{Figure}: FE with lossy layer highlighted (klayout and microscope?)

Explain briefly why it could help, citing the only paper we found for this specific application so far

\subsection{Inductive Damping}

\textit{Figure}: Feedlines with goldpads on top.

I have not found any info about this method in papers or other publications.. I have only a handwavy explanation in mind on how it should work. 