\chapter{dc-SQUID Design} \label{ch_SQUIDdesign}

The main objective for this thesis was the optimization of an existing Front-End SQUID design for an improved coupling to one of the detectors being developed in this working group. As we have seen in the previous section \ref{sec_MMC}, adjusting $L_{\rm i}$ to the detector coil ensures the maximization of the flux-to-flux coupling and therefore minimizes the extrinsic energy sensitivity. The previous SQUID design exhibits a design value of $L_{\rm i}=1.64$ for the input coil inductance \cite{Bauer2022}, which is well suited for the pickup coil inductance of the ECHo-100k detector of $L_{\rm p}=\qty{1.14}{\nano\henry}$ \cite{Mantegazzini2021}. The parasitic inductance $L_{\rm par}$, that arises from the aluminum bonds between the SQUID and the detector substrate to form the flux transformer, has been estimated to \qty{0.5}{\nano\henry} \cite{Hengstler2017}. Other MMC detectors from this working group such as the 4k-pixel molecule camera MOCCA and the X-ray detector maXs100 require higher input inductances, as their pickup coil inductances are $L_{\rm p}=\qty{8.8}{\nano\henry}$ and $L_{\rm p}=\qty{6.65}{\nano\henry}$, respectively. In \cite{Bauer2022} SQUIDs with matching input inductances for the MOCCA and maXs100 detector were developed for the first time using an intermediary coupling transformer. These improved the energy resolution $\Delta E_{\rm FWHM}$ of the detectors, although the effect was minimal for the latter due to a reduction of the effective coupling constant $k_{\rm is}'$, compensating the gain through an increased input inductance of $L_{\rm i}'=\qty{5.47}{\nano\henry}$. Specifically for the maXs100 detector, a different approach was followed in the framework of this thesis to achieve a better coupling while avoiding a significant increase of the detector noise.  

\section{dc-SQUID with a Two-Turn Input Coil} \label{sec_FEdesign}

Increasing the input coil inductance can be realized either by changing the geometry of the coil itself, or by implementing a double flux transformer structure, with the benefit of easily adapting the inductance independently of the SQUID design. The latter, however, was accompanied with a reduction of the effective coupling constant $k_{\rm is}'$ in the work of \cite{Bauer2022} regarding the maXs100 detector readout, which led to a lower flux-to-flux coupling despite the higher inductance of $L_{\rm i}'=\qty{5.47}{\nano\henry}$. Only the white noise reduction of the SQUID, which was caused by the shielding effects of the added flux transformer, resulted in a small overall improvement of $\epsilon_{\rm p}$ and $\Delta E_{\rm FWHM}$. \\
In this work we designed a new detector SQUID with \textit{window-type}\footnote{A well-established microfabrication technique to produce Josephson junctions for SQUIDs} Josephson junctions, which is based on the design developed in \cite{Bauer2022}. A schematic of this SQUID is shown in figure \ref{abb:NL_FE}. Four large oval loops form the second order gradiometer described in subsection \ref{subsec_gradio}, where the lower fabricated niobium layer (Nb1) contains the SQUID loop as a parallel gradiometer. The feed line coming from the top supplies the serially connected input coil, which was fabricated as a second niobium layer (Nb2) on top of the SQUID loop, only separated by an insulating SiO$_{\rm 2}$ layer. This geometry allows to combine a small SQUID loop inductance $L_{\rm s}=\frac{L_{\rm l}}{4}$ with a large input inductance $L_{\rm i}=4L_{\rm l}$. Also in the Nb2 layer, another serial gradiometer of second order representing the feedback coil is located below the input coil, whose feed lines are shown on the bottom left. This coil has a design inductance of $L_{\rm f}=\qty{336}{\pico\henry}$ with a line width of \qty{3}{\micro\meter}. Both coils exhibit the same geometry as the corresponding underlying washer strip to maximize the overlap and therefore the coupling. At the same time, the proximity to the input coil was kept at a minimum to minimize cross talk between the two coils. The feed lines for the SQUID loop at the bottom center lead to the junction area, which is shown in the zoomed in section. Both square-shaped Nb/Al-Al$\rm O_x$/Nb junctions are realized with the dimensions $\qtyproduct{4.5 x 4.5}{\micro\meter}$ and a targeted critical current of $I_{\rm c}=\qty{6}{\micro\ampere}$, leading to a critical current density of $j_{\rm c}=\qty{30}{\ampere\per\centi\meter\squared}$. Two AuPd junction shunt resistors $R_{\rm s}$ are located on the left and right side of the junction area, respectively. Both are attached to a large heat sink made of two gold layers, a thick galvanized layer on top of a sputtered, thin one. These so-called \textit{cooling fins} provide thanks to their large volume a better electron-phonon coupling, which reduces the electron temperature of the normal-conducting shunts and thus mitigates the corresponding thermal noise \cite{Mazibrada2024}. A third AuPd resistor $R_{\rm d}$ with the same dimensions is placed above the junction area and connected in parallel with the washer loop. This so-called \textit{washer shunt} provides damping properties to reduce quality factors of parasitic resonances, as will be discussed in section \ref{sec_damping}. \\

\figurecenter {t!}
{width=\textwidth}
{../Figures/FE_NL_design}
{0cm}
{Halb klayout-halb foto wie bei fabian?.. Schematic of the new planar dc-SQUID design with a two-turn input coil, including a zoom into the junction area (bottom). Four large oval-shaped loops form the microstrip transmission line structure given by the washer SQUID loop in the lower niobium layer Nb1 and the input coil in the upper niobium layer Nb2. Fabricated in the same manner is a small feedback coil between the junction area and the input coil. Both coils are free of $\rm{SiO_2}$, visualized through the inverse drawing of the $\rm{SiO_2}$ layer.} 
{NL_FE}

As opposed to the previous design, this input coil is realized with two turns instead of one. Neglecting the stripline inductance $L_{\rm str}$ (see section \ref{sec_resonance_results}), the input inductance becomes approximately proportional to the number of turns squared $n^2$ \cite{Ketchen1981,Jaycox1981}, giving a theoretical value of $L_{\rm i}^{\rm theo}\approx2^{2}\cdot \qty{1.64}{\nano\henry}\approx\qty{6.56}{\nano\henry}$. To implement the second turn a wider washer loop line width of $w_{\rm s}=\qty{10}{\micro\meter}$ was needed. The line width of the input coil $w_{\rm i}$ with two turns remained at $\qty{3}{\micro\meter}$, which would have not been possible to fabricate on top of the previous washer width of $w_{\rm s}=\qty{5}{\micro\meter}$ without sacrificing coupling strength. A general increase in $w_{\rm i}$, however, entails the risk of capturing noise inducing flux vortices. These form if the B-field pointing perpendicular to the SQUID plane exceeds a critical field given by \cite{Kuit2008} 

\gl{
B_{\rm v,crit} = 1.65\frac{\unit{\fq}}{w_{\rm s}^2}	\ \ .
}{}

At the earths surface, its magnetic field reaches a maximum amplitude of \qty{65}{\micro\tesla}, which would give a threshold width of $w_{\rm s}=\qty{7.2}{\micro\meter}$. The dilution refrigerators used to cool down both the SQUIDs and the detectors typically provide magnetic shielding, such that we consider a width of $w_{\rm s}=\qty{10}{\micro\meter}$ to have a negligible impact on the flux noise attributed to flux vortices. As the input inductance was the only parameter necessary to adjust, we attempted to keep the other design parameters unaltered. The widened washer width would therefore need to be compensated with an approximately \qty{15}{\%} larger washer hole circumference in order to maintain the same SQUID loop inductance. For the sake of safety, the circumference was increased only by \qty{10}{\%} to prevent possible hysteretic behavior through a too large SQUID loop inductance and therefore screening parameter $\beta_{\rm L}$. The increase was estimated by modeling the rather complicated oval washer loop geometry as a ring-shaped structure whose inductance can be calculated with the relation $L={\rm \mu_0}R\left(\ln\left(\frac{8R}{a}\right)-2\right)$, where \textit{R} denotes the loop radius and \textit{a} the radius of the wire \cite{Dengler2016}. In addition to these geometric adjustments, a comprehensive structuring of the SiO2 layer in the area of the washer loops was also omitted, allowing the empty interior to remain free from the insulator layer. This is intended to prevent potential flux noise induced by unavoidable magnetic impurities in the $\rm SiO_2$. The same method was applied to the smaller loops formed by the feedback coil. The absence of insulation within the loops is visualized by the green-colored areas in figure \ref{abb:NL_FE}. \\

The design values for $R_{\rm s}$ and $L_{\rm s}$ can be obtained by minimizing the extrinsic energy sensitivity given in equation \ref{extr_energy_sens}, which requires the maximization of the flux-to-flux coupling ($L_{\rm i}=L_{\rm p}+L_{\rm par}$) and the minimization of the intrinsic white noise of the SQUID. In \cite{Bauer2022} this numerical calculation led to the optimal parameters $\beta_{\rm C}=0.7$ and $\beta_{\rm L}=0.86$, which were given the restraints $\beta_{\rm C}\leq 0.7$ and $\beta_{\rm L}\leq 1$ to avoid hysteretic behavior. Consequently, additional noise through voltage jumps caused by hysteretic IVCs as well as Nyquist noise from higher harmonics of the Josephson frequencies  \cite{Clarke1996} have been neglected, which was not the case for the derivation of equation \ref{voltagenoise_psd}. The intrinsic white noise of the SQUID used for the numerical calculation is therefore given by \cite{Knuutila1988}

\gl{
S_{\rm \Phi_s} = 2k_{\rm B}T\frac{L_{\rm s}^2}{R_{\rm s}}\left[(1-k_{\rm is}^2s_{\rm in})^2+\frac{\sqrt{2}(1+\beta_{\rm L})^2}{\beta_{\rm L}}\right] \ \ .	
}{intr_FEnoise_minimize}

The targeted critical current of $I_{\rm c}=\qty{6}{\micro\ampere}$ provides with the given junction dimensions a critical current density of $j_{\rm c}=\qty{29.63}{\ampere\per\cm\squared}$, which can be used to calculate the junction capacitance \textit{C} by using the empirical relation $\frac{1}{C'}=p_1 + p_2\log_{10}j_{\rm c}$ \cite{Maezawa1995}. Here, the intrinsic capacitance $C'$ excludes any parasitic capacitances arising from the window-type fabrication technique. For simplicity reasons, we assume $C\approx C'$ and thus obtain $C=\qty{0.95}{\pico\farad}$. The optimal Stewart McCumber and screening parameter then provide the values $R_{\rm s}=\qty{6.3}{\ohm}$ and $L_{\rm s}=\qty{147}{\pico\henry}$, respectively. The designed shunt resistor in both the previous and the new design was rounded to \qty{6}{\ohm}, which results with Ohm's circuit law in a normal resistance of $R_{\rm n}=\qty{3}{\ohm}$ for the whole SQUID. This consequently corresponds to a slightly lower damping parameter of $\beta_{\rm L}=0.62$. The coupling constant was set to an upper limit of $k_{\rm is}=0.75$, which is typically the highest achievable value for the SQUIDs produced in this working group. Under the assumption of $k_{\rm is}$ being maximal and $L_{\rm i}=L_{\rm p}+L_{\rm par}=\qty{7.15}{\nano\henry}$ for the maXs100 detector read-out, the theoretically obtainable flux-to-flux coupling regarding a single pickup coil is $\frac{\Delta\Phi_{\rm s}}{\Delta\Phi_{\rm p}}=5.38\%$. Lastly, for the extrinsic energy sensitivity we would obtain with equation \ref{intr_FEnoise_minimize} $\epsilon_{\rm p}=\qty{0.53}{h}$, given the typical detector operation temperature of $T=\qty{20}{\milli\kelvin}$.   


\section{Integrated Two-Stage Chip}\footnote{wird evtl weg gelassen und dann nur in ein, zwei sätzen erwähnt} 

\textit{Figure}: Int. 2stage chip (klayout and microscope?)

Briefly cover its features

\section{Damping Methods} \label{sec_damping}

As discussed in subsection \ref{subsec_para_res}, several SQUID parameters can be optimized to mitigate the influence of various resonances in the circuit. However, most of these were determined by the minimization of the extrinsic energy sensitivity (see section \ref{sec_FEdesign}), which imposes substantial limitations on the extent to which additional parameter modifications could be implemented. For instance, increasing the length of the input coil $l_{\rm i}$ would shift the corresponding strip line resonance given by equation \ref{stripline_res_general} away from the operation frequency, which in turn would lead to a larger input inductance, thus impeding the maximization of the flux-to-flux coupling. Furthermore, even resonances far away from the operation frequency can result problematic as thermally activated transitions between different states increase the noise level \cite{Sepp1987}. This motivates to follow a more practical approach to suppress \textit{LC} resonances, which can be realized through damping with attenuators, such as the damping resistor $R_{\rm d}$ shown in the top of junction area in figure \ref{abb:NL_FE}. These are typically connected in parallel to the resonant circuit, as this reduces the quality factor \textit{Q} of the corresponding \textit{LC} resonance given an appropriate dimensioning of the resistor. Consequently, the $L_{\rm s}C_{\rm p}$ and $L_{\rm s}C$ resonances can be damped by implementing a \textit{washer shunt} $R_{\rm d}$ \cite{Ono1997, Ryh1992}. Although the IVC intersection caused by the latter cannot be eliminated, previous works in this group showed a significant smoothing of the curves as the accompanying step structures decreased \cite{Bauer2018}. The current noise introduced through this resistor, on the other hand, deteriorates the energy sensitivity and thus limits the damping benefit. However, this effect is minimal for the condition $R_{\rm d}\approx R_{\rm s}$ with $\beta_{\rm L}=1$ \cite{Enpuku1986, Ryh1992}, which is why we choose $R_{\rm d}=\qty{6}{\ohm}$. The input circuit is shunted with an $R_{\rm x}C_{\rm x}$ attenuator to damp the $L_{\rm i}C_{\rm p}$ resonance, where the added capacitance $C_{\rm x}$ blocks low frequency current noise \cite{Sepp1987}. Both damping techniques to suppress $C_{\rm p}$-related resonances were proven to be effective both in literature and our working group \cite{Sepp1987,Knuutila1987, Enpuku1986,Can1991,Bauer2018,Bauer2022}\footnote{overkill? besser weniger/keine quellen oder satz ganz streichen?}. As for the $\lambda/2$ resonances, both the $R_{\rm d}$ and the $R_{\rm x}C_{\rm x}$ attenuator provide good damping as well \cite{Can1991}, since they terminate the microstrip lines and thus avoid impedance mismatches given a suitable dimensioning of the resistors (see section \ref{sec_resonance_results}). A schematic of the resulting circuit diagram of the coupled dc-SQUID with all damping components is depicted in figure \ref{abb:RxCx_circuit} (left). Shown on the right is the design of the $R_{\rm x}C_{\rm x}$ shunt, which has been adapted from \cite{Bauer2022}. As the needed capacitance $C_{\rm x}=\qty{10}{\pico\farad}$ is rather large, two parallelly connected square-shaped capacitors with $\frac{C_{\rm x}}{2}$ are used to reduce the needed space. The top and bottom plate of each capacitor are fabricated in the Nb2 and Nb1 layer, respectively. The shunt $R_{\rm x}$ is split as well into two parallel AuPd resistors with $2R_{\rm x}$ each. Optimal values for these components will be discussed in chapter \ref{ch_results}.

\twofigurescenter{t!}
{width=0.48\textwidth}
{../Figures/gekoppeltes_SQUID}
{0.02\textwidth} %hspace
{width=0.48\textwidth}
{../Figures/FE_RxCx}
{0.5cm} %vspace
{Halb klayout-halb foto wie bei fabian? ... Left: Schematic circuit of a coupled dc-SQUID with damping components $R_{\rm d}$ and $R_{\rm x}C_{\rm x}$. The input coil forms a flux transformer with a pickup coil of inductance $L_{\rm p}$. The parasitic capacitance between the input circuit and the SQUID loop is connected in parallel to both $L_{\rm i}$ and $L_{\rm s}$. Right: Structure design of the $R_{\rm x}C_{\rm x}$ attenuator located above the dc-SQUID. Both devices $R_{\rm x}$ and $C_{\rm x}$ are each realized as two parallel components.}
{RxCx_circuit}

\subsection{Lossy Input Coil}\label{subsec_L_FE}

Several approaches to reduce \textit{Q} values of $\lambda/2$ resonances associated with microstrip lines have been investigated in \cite{Boyd2022}. Two parallel meanders fabricated on top of each other in a direct-coupled MMC setup\footnote{As opposed to the flux transformer setup in section \ref{sec_MMC}, the SQUID is directly coupled to the sensor.} produced high \textit{Q} resonances at integer and half-integer wavelengths. Whereas placing an individual resistor in parallel to one meander only damped the half-integer modes, a more distributed damping scheme in the form of an insulated gold layer between both meanders provided strong damping of all microstrip resonances while maintaining a low detector noise level. The noise and the \textit{Q} values have been further reduced by structuring the Au layer with the same geometry as the microstrip lines and electrically connecting it to one of the meanders, thereby preventing large noise inducing, normal-conducting loops. The attenuation constant $\alpha$ of this lossy microstrip line increased with frequency and thickness of the Au layer, while the noise stayed insensitive upon thickness changes. \\
Within the scope of this thesis, such damping techniques were implemented and tested on the dc-SQUID described in section \ref{sec_FEdesign}. For this, we sputtered a gold layer between $\rm SiO_2$ and Nb2 with the same geometry as the input coil running above the SQUID loop. This gold layer was omitted around the vias between the washer loops to avoid additional normal resistances. For fabrication reasons, gold and niobium were structured together as a bilayer in the same microfabrication step by sputtering the upper niobium layer directly after the gold. To connect the four input coil segments over vias, the original second niobium layer Nb2 was used. A schematic of our Front-End design with this lossy input coil, including an image obtained with an optical microscope, is shown in figure \ref{abb:FE_L}. Our Front-End SQUIDs with this gold layer adjacent to the input coil will be referred to as 'lossy' for the upcoming discussions. 

\figurecenter {t!}
{width=\textwidth}
{../Figures/FE_L_design}
{0cm}
{Halb klayout-halb foto wie bei fabian? ... dc-SQUID design with a two-turn input coil realized as a lossy microstrip line. A gold layer is structured between the insulating $\rm SiO_2$ and an upper niobium layer and has been fabricated in a single step as a bilayer containing both gold and niobium. The overlying and gold-free niobium in the Nb2 layer connects all four input coil segments over their respective washer loop to prevent the vias to acquire a normal resistance.} 
{FE_L}

\subsection{Inductive Damping}

Together with the attempt of suppressing resonances through a lossy microstrip line, we introduce a second new damping technique denoted as inductive damping. This method is based on the principle of magnetic damping, where a change in magnetic flux creates eddy currents in a nearby conductor, which following Lenz's law induces a flux trying to compensate the initial one. The flux change is therefore damped by effectively transferring part of the magnetic energy to the kinetic energy of the induced current, which in the case of a normal conductor emits heat. This phenomenon also causes the reduction of the geometric inductance of the SQUID loop through the shielding effect of the flux transformer (see subsection \ref{subsec_extr_sens_theo}). A strong indicator that this mechanism could be applied to SQUIDs showed the first experiment in \cite{Boyd2022}, where a square gold layer representing the MMC sensor was placed at a hight of \qty{300}{\nano\meter} above an isolated meander. This lead to a significant reduction of the high \textit{Q} values of the meander modes. The concept is now applied to the feed lines on our SQUID chip, where large gold pads have been placed across the SQUID washer and feedback coil feed lines, as shown in figure \ref{abb:damped_chip}. The sharp voltage spikes associated with high \textit{Q} resonances, that might be present within those lines would therefore be damped by partly converting their energy into heat in the normal conducting gold layer. The gold pads are fabricated in the same layers as the heat sinks for the shunt resistors and are consequently sputtered first before being electroplated. The latter step significantly increases their hight and therefore volume, which allows for larger and more effective eddy currents. Additionally, the generated heat is expected to better dissipate into the chip substrate as the electron-phonon coupling increases with volume. All feed lines are typically realized as microstrip lines, which for the gold pads needed to be adjusted into a coplanar structure as there is no insulation layer after Nb2. Figure \ref{abb:damped_chip} shows the chip design consisting of four distinct Front-End variants, each provided with the inductive damping scheme. The first channel at the top is realized without an input coil in order to better allocate possible resonance structures visible in the SQUID's IVCs. The design introduced in \ref{sec_FEdesign}, also referred to as 'non-lossy', is represented in channel 2, followed by the lossy variant in channel 3. The last channel contains a lossy Front-End as well, however, the washer loop interiors were not kept $\rm SiO_2$-free. This allows to investigate the influence of possible magnetic impurities in the insulation material and assess whether it can be regarded as negligible or not. The same chip design has been used without gold pads on the feed lines, resulting in 8 different Front-End SQUIDs that were developed and tested within the scope of this thesis. Some first results are presented in the following chapter.     

\figureleft {t!}
{width=\textwidth} %sets how much of the fig space is used
{../Figures/damped_chip}
{9cm} %sets width of the fig space
{0cm}
{Inductive damping scheme on the Front-End SQUID chip of the type 4x100i6 v1.4. All feedback coil and SQUID loop feed lines are covered with large rectangular, insulated gold pads. Each SQUID channel is occupied with a different SQUID variant. Channel 1: Front-End without an input coil. Channel 2: Non-lossy SQUID design presented in section \ref{sec_FEdesign}. Channel 3: Lossy SQUID design from section \ref{subsec_L_FE}. Channel 4: Lossy design with $\rm SiO_2$ inside the washer loop interiors.}
{damped_chip}

