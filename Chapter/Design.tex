\chapter{dc-SQUID Design} \label{ch_SQUIDdesign}

The main objective for this thesis was the optimization of an existing Front-End SQUID design for an improved coupling to one of the detectors being developed in this working group. As we have seen in the previous section \ref{sec_MMC}, adjusting $L_{\rm i}$ to the detector coil ensures the maximization of the flux-to-flux coupling and therefore minimizes the extrinsic energy sensitivity. The previous SQUID design exhibits a design value of $L_{\rm i}=1.64$ for the input coil inductance \cite{Bauer2022}, which is well suited for the pickup coil inductance of the ECHo-100k detector of $L_{\rm p}=\qty{1.14}{\nano\henry}$ \cite{Mantegazzini2021}. The parasitic inductance $L_{\rm par}$, that arises from the aluminum bonds between the SQUID and the detector substrate to form the flux transformer, has been estimated to \qty{0.5}{\nano\henry} \cite{Hengstler2017}. Other MMC detectors from this working group such as the 4k-pixel molecule camera MOCCA and the X-ray detector maXs100 require higher input inductances, as their pickup coil inductances are $L_{\rm p}=\qty{8.8}{\nano\henry}$ and $L_{\rm p}=\qty{6.65}{\nano\henry}$, respectively. In \cite{Bauer2022} SQUIDs with matching input inductances for the MOCCA and maXs100 detector were developed for the first time using an intermediary coupling transformer. These improved the energy resolution $\Delta E_{\rm FWHM}$ of the detectors, although the effect was minimal for the latter due to a reduction of the effective coupling constant $k_{\rm is}'$, compensating the gain through an increased input inductance of $L_{\rm i}'=\qty{5.47}{\nano\henry}$. Specifically for the maXs100 detector, a different approach was followed in the framework of this thesis to achieve a better coupling while avoiding a significant increase of the detector noise.  

\section{dc-SQUID with a Two-Turn Input Coil} \label{sec_FEdesign}

Increasing the input coil inductance can be realized either by changing the geometry of the coil itself, or by implementing a double flux transformer structure, with the benefit of easily adapting the inductance independently of the SQUID design. The latter, however, was accompanied with a reduction of the effective coupling constant $k_{\rm is}'$ in the work of \cite{Bauer2022} regarding the maXs100 detector readout, which led to a lower flux-to-flux coupling despite the higher inductance of $L_{\rm i}'=\qty{5.47}{\nano\henry}$. Only the white noise reduction of the SQUID, which was caused by the shielding effects of the added flux transformer, resulted in a small overall improvement of $\epsilon_{\rm p}$ and $\Delta E_{\rm FWHM}$. \\
In this work we designed a new detector SQUID with \textit{window-type}\footnote{A well-established microfabrication technique to produce Josephson junctions for SQUIDs} Josephson junctions, which is based on the design developed in \cite{Bauer2022}. A schematic of this SQUID is shown in figure \ref{abb:NL_FE}. Four large oval loops form the second order gradiometer described in subsection \ref{subsec_gradio}, where the lower fabricated niobium layer (Nb1) contains the SQUID loop as a parallel gradiometer. The feed line coming from the top supplies the serially connected input coil, which was fabricated as a second niobium layer (Nb2) on top of the SQUID loop, only separated by an insulating SiO$_{\rm 2}$ layer. This geometry allows to combine a small SQUID loop inductance $L_{\rm s}=\frac{L_{\rm l}}{4}$ with a large input inductance $L_{\rm i}=4L_{\rm l}$. Also in the Nb2 layer, another serial gradiometer of second order representing the feedback coil is located below the input coil, whose feed lines are shown on the bottom left. This coil has a design inductance of $L_{\rm f}=\qty{336}{\pico\henry}$ with a line width of \qty{3}{\micro\meter}. Both coils exhibit the same geometry as the corresponding underlying washer strip to maximize the overlap and therefore the coupling. At the same time, the proximity to the input coil was kept at a minimum to minimize cross talk between the two coils. The feed lines for the SQUID loop at the bottom center lead to the junction area, which is shown in the zoomed in section. Both square-shaped Nb/Al-Al$\rm O_x$/Nb junctions are realized with the dimensions $\qtyproduct{4.5 x 4.5}{\micro\meter}$ and a targeted critical current of $I_{\rm c}=\qty{6}{\micro\ampere}$, leading to a critical current density of $j_{\rm c}=\qty{30}{\ampere\per\centi\meter\squared}$. Two AuPd junction shunt resistors $R_{\rm s}$ are located on the left and right side of the junction area, respectively. Both are attached to a large heat sink made of two gold layers, a thick galvanized layer on top of a sputtered, thin one. These so-called \textit{cooling fins} provide thanks to their large volume a better electron-phonon coupling, which reduces the electron temperature of the normal-conducting shunts and thus mitigates the corresponding thermal noise \cite{Mazibrada2024}. A third AuPd resistor $R_{\rm d}$ with the same dimensions is placed above the junction area and connected in parallel with the washer loop. This so-called \textit{washer shunt} provides damping properties to reduce quality factors of parasitic resonances, as will be discussed in section \ref{sec_damping}. \\

\figurecenter {t!}
{width=\textwidth}
{../Figures/FE_NL_design}
{0cm}
{Schematic of the new planar dc-SQUID design with a two-turn input coil, including a zoom into the junction area (bottom). Four large oval-shaped loops form the microstrip transmission line structure given by the washer SQUID loop in the lower niobium layer Nb1 and the input coil in the upper niobium layer Nb2. Fabricated in the manner is a small feedback coil between the junction area and the input coil. Both coils are free of $\rm{SiO_2}$, visualized through the inverse drawing of the $\rm{SiO_2}$ layer.} 
{NL_FE}

As opposed to the previous design, this input coil is realized with two turns instead of one. Neglecting the stripline inductance $L_{\rm str}$ (see section \ref{sec_resonance_results}), the input inductance becomes approximately proportional to the number of turns squared $n^2$ \cite{Ketchen1981,Jaycox1981}, giving a theoretical value of $L_{\rm i}\approx2^{2}\cdot \qty{1.64}{\nano\henry}\approx\qty{6.56}{\nano\henry}$. To implement the second turn a wider washer loop line width of $w_{\rm s}=\qty{10}{\micro\meter}$ was needed. The line width of the input coil $w_{\rm i}$ with two turns remained at $\qty{3}{\micro\meter}$, which would have not been possible to fabricate on top of the previous washer width of $w_{\rm s}=\qty{5}{\micro\meter}$ without sacrificing coupling strength. A general increase in $w_{\rm i}$, however, entails the risk of capturing noise inducing flux vortices. These form if the B-field pointing perpendicular to the SQUID plane exceeds a critical field given by \cite{Kuit2008} 

\gl{
B_{\rm v,crit} = 1.65\frac{\unit{\fq}}{w_{\rm s}^2}	\ \ .
}{}

At the earths surface, its magnetic field reaches a maximum amplitude of \qty{65}{\micro\tesla}, which would give a threshold width of $w_{\rm s}=\qty{7.2}{\micro\meter}$. The dilution refrigerators used to cool down both the SQUIDs and the detectors typically provide magnetic shielding, such that we consider a width of $w_{\rm s}=\qty{10}{\micro\meter}$ to have a negligible impact on the flux noise attributed to flux vortices. As the input inductance was the only parameter necessary to adjust, we attempted to keep the other design parameters unaltered. The widened washer width would therefore need to be compensated with an approximately \qty{15}{\%} larger washer hole circumference in order to maintain the same SQUID loop inductance. For the sake of safety, the circumference was increased only by \qty{10}{\%} to prevent possible hysteretic behavior through a too large SQUID loop inductance and therefore screening parameter $\beta_{\rm L}$. The increase was estimated by modeling the rather complicated oval washer loop geometry as a ring-shaped structure whose inductance can be calculated with the relation $L={\rm \mu_0}R\left(\ln\left(\frac{8R}{a}\right)-2\right)$, where \textit{R} denotes the loop radius and \textit{a} the radius of the wire \cite{Dengler2016}. 
\\

The design values for $R_{\rm s}$ and $L_{\rm s}$ can be obtained by minimizing the extrinsic energy sensitivity given in equation \ref{extr_energy_sens}, which requires the maximization of the flux-to-flux coupling ($L_{\rm i}=L_{\rm p}+L_{\rm par}$) and the minimization of the intrinsic white noise of the SQUID. In \cite{Bauer2022} this numerical calculation led to the optimal parameters $\beta_{\rm C}=0.7$ and $\beta_{\rm L}=0.86$, which were given the restraints $\beta_{\rm C}\leq 0.7$ and $\beta_{\rm L}\leq 1$ to avoid hysteretic behavior. Consequently, additional noise through voltage jumps caused by hysteretic IVCs as well as Nyquist noise from higher harmonics of the Josephson frequencies  \cite{Clarke1996} have been neglected, which was not the case for the derivation of equation \ref{voltagenoise_psd}. The intrinsic white noise of the SQUID used for the numerical calculation is therefore given by \cite{Knuutila1988}

\gl{
S_{\rm \Phi_s} = 2k_{\rm B}T\frac{L_{\rm s}^2}{R_{\rm s}}\left[(1-k_{\rm is}^2s_{\rm in})^2+\frac{\sqrt{2}(1+\beta_{\rm L})^2}{\beta_{\rm L}}\right] \ \ .	
}{intr_FEnoise_minimize}

The targeted critical current of $I_{\rm c}=\qty{6}{\micro\ampere}$ provides with the given junction dimensions a critical current density of $j_{\rm c}=\qty{30}{\ampere\per\cm\squared}$, which can be used to calculate the junction capacitance \textit{C} by using the empirical relation $\frac{1}{C'}=p_1 + p_2\log_{10}j_{\rm c}$ \cite{Maezawa1995}. Here, the intrinsic capacitance $C'$ excludes any parasitic capacitances arising from the window-type fabrication technique. For simplicity reasons, we assume $C\approx C'$ and thus obtain $C=\qty{0.95}{\pico\farad}$. The optimal Stewart McCumber and screening parameter then provide the values $R_{\rm s}=\qty{6.3}{\ohm}$ and $L_{\rm s}=\qty{147}{\pico\henry}$, respectively. The designed shunt resistor in both the previous and the new design was rounded to \qty{6}{\ohm}, which results with Ohm's circuit law in a normal resistance of $R_{\rm n}=\qty{3}{\ohm}$ for the whole SQUID. This consequently corresponds to a slightly lower damping parameter of $\beta_{\rm L}=0.62$. The coupling constant was set to an upper limit of $k_{\rm is}=0.75$, which is typically the highest achievable value for the SQUIDs produced in this working group. Under the assumption of $k_{\rm is}$ being maximal and $L_{\rm i}=L_{\rm p}+L_{\rm par}=\qty{7.15}{\nano\henry}$ for the maXs100 detector read-out, the theoretically obtainable flux-to-flux coupling is $\frac{\Delta\Phi_{\rm s}}{\Delta\Phi_{\rm p}}=2.69\%$. Lastly, for the extrinsic energy sensitivity we would obtain with equation \ref{intr_FEnoise_minimize} $\epsilon_{\rm p}=\qty{0.53}{h}$, given the typical detector operation temperature of $T=\qty{20}{\milli\kelvin}$.   


\section{Integrated Two-Stage Chip}

\textit{Figure}: Int. 2stage chip (klayout and microscope?)

Briefly cover its features

\section{Damping Methods} \label{sec_damping}

As discussed in subsection \ref{subsec_para_res}, several SQUID parameters can be optimized to mitigate the influence of various resonances in the circuit. However, most of these were determined by the minimization of the extrinsic energy sensitivity (see section \ref{sec_FEdesign}), which imposes substantial limitations on the extent to which additional parameter modifications could be implemented. For instance, increasing the length of the input coil $l_{\rm i}$ would shift the corresponding strip line resonance given by equation \ref{stripline_res_general} away from the operation frequency, which in turn would lead to a larger input inductance, thus impeding the maximization of the flux-to-flux coupling. A more practical approach to suppress \textit{LC} resonances can be realized through damping with attenuators, such as the damping resistor $R_{\rm d}$ shown in the top of junction area in figure \ref{abb:NL_FE}. These are typically connected in parallel to the resonant circuit, as this reduces the quality factor \textit{Q} of the corresponding \textit{LC} resonance given an appropriate dimensioning of the resistor. Consequently, the $L_{\rm s}C_{\rm p}$ and $L_{\rm s}C$ resonances can be damped by implementing a \textit{washer shunt} $R_{\rm d}$ \cite{Ono1997, Rhyänen1992}. Although the IVC intersection caused by the latter cannot be eliminated, previous works in this group showed a significant smoothing of the curves as the accompanying step structures decreased \cite{Bauer2018}. The current noise introduced through this resistor, on the other hand, deteriorates the energy sensitivity. This tradeoff is minimal for the condition $R_{\rm d}\approx R_{\rm s}$ with $\beta_{\rm L}=1$ \cite{Enpuku1986, Ryhänen1992}. The input circuit is shunted with an $R_{\rm x}C_{\rm x}$ attenuator to damp the $L_{\rm i}C_{\rm p}$ resonance, where the added capacitance $C_{\rm x}$ blocks low frequency current noise \cite{Sepp1987}. Both damping techniques to suppress $C_{\rm p}$-related resonances were proven to be effective both in literature and our working group \cite{Sepp1987, Enpuku1986,Can1991,Bauer2018,Bauer2022}. As for the $\lambda/2$ resonances, both the $R_{\rm d}$ and the $R_{\rm x}C_{\rm x}$ attenuator provide good damping as well \cite{Can1991}, since they terminate the microstrip lines and thus avoid impedance mismatches given a suitable dimensioning of the resistors (see section \ref{sec_resonance_results}). A schematic of the resulting circuit diagram of the coupled dc-SQUID with all damping components is depicted in figure \ref{abb:} (left). Shown on the right is the design of the $R_{\rm x}C_{\rm x}$ shunt, which has been adapted from \cite{Bauer2022}. As the needed capacitance $C_{\rm x}=\qty{10}{\pico\farad}$ is rather large, two parallelly connected square-shaped capacitors with $\frac{C_{\rm x}}{2}$ are used to reduce the needed space. The top and bottom plate of each capacitor are fabricated in the Nb2 and Nb1 layer, respectively. The shunt $R_{\rm x}$ is split as well into two parallel AuPd resistors with $2R_{\rm x}$ each. Optimal values for these components will be discussed in chapter \ref{ch_results}.


\textit{Figure}: RxCx (klayout/microscope) \\
%\textit{Figure}: Comparison of two IVC plots, one with RxCx, one without (see Fabienne Diss.) (?) \\
\textit{Figure}: Schematic circuit diagram of our SQUIDs with all L's, C's and R's

\subsection{Lossy Input Coil}

\textit{Figure}: FE with lossy layer highlighted (klayout and microscope?)

Explain briefly why it could help, citing the only paper we found for this specific application so far

\subsection{Inductive Damping}

\textit{Figure}: Feedlines with goldpads on top.

I have not found any info about this method in papers or other publications.. I have only a handwavy explanation in mind on how it should work. 