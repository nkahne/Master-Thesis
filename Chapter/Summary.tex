\chapter{Summary and Outlook}

In this work we introduced a new dc-SQUID design optimized for the readout of the MMC-based X-ray detector maXs100, which is developed in this working group. For this, the design based of the work of \cite{Bauer2022} was adapted by increasing the input coil inductance to match the inductance of the detector pickup coil. Contrary to the intermediary flux transformer approach tested in \cite{Bauer2022}, the inductance was adjusted by adding a second turn to the input coil. This measure maximizes the flux-to-flux coupling $\Delta\Phi_{\rm s}/\Delta\Phi$, which is a crucial parameter to maximize the amplitude of the readout signal as well as to minimize the noise coupled into the detector, thereby increasing the energy resolution $\Delta E_{\rm FWHM}$.

To assess whether the design changes lead to additional noise or to a deteriorated resonance behavior, we measured the SQUID both in a single-stage setup to investigate general properties including the current-voltage characteristics, and in a two-stage setup to measure the flux noise. Additionally, we implemented new damping techniques to suppress various resonances originating from either $LC$ resonant circuits or from microstrip structures. These involved the deposition of additional galvanically isolated gold onto the washer and feedback coil feed lines, as well as the fabrication of the input coil as a bilayer containing both gold and niobium. The new SQUID design was produced with different damping scheme configurations to examine the effectiveness of the employed damping approaches. 

The single-stage experiments were conducted both in a liquid helium transport vessel at $\qty{4}{\kelvin}$ and in a dilution cryostat at $\qty{10}{\milli\kelvin}$. In the case of the latter, we measured the characteristic parameters of 10 SQUIDs, which except for the SQUID loop inductance resulted to be close to their target values. 
%It should be noted, however, that the critical current and consequently $\beta_{\rm L}$ and the current swing exhibited a large fabrication-related variance. Furthermore, the measurement uncertainty regarding $I_{\rm c}$ and $\Delta I$ leads to strongly varying values of $L_{\rm s}$. 
We determined a SQUID loop inductance of $L_{\rm s}=\qty{108}{\pH}$, deviating \qty{27}{\percent} from the target value of \qty{147}{\pH}. This can be explained by the conservative adjustment of the washer loops, which had to be enlarged to compensate for the inductance loss through widening the lines to fit the second input coil turn. Additionally, with the previous SQUID design the inductance $L_{\rm s}$ had already been overestimated, yielding only $\qty{135}{\pH}$. This could be confirmed in this work with similar values as the previous design was included on the wafer produced within the scope of this thesis. The input coil inductance, on the other hand, was measured at $\qty{4}{\kelvin}$ and determined to $L_{\rm i}=\qty{6.4}{\nH}$ as the average of two measurements. This represents a deviation of \qty{14}{\percent} from the simulated value of \qty{5.6}{\nH}, which is in agreement with observations from \cite{Ferring2015, Bauer2022}. 

The IVCs of all new SQUID variants presented in chapter \ref{ch_results} were obtained at \qty{10}{\milli\kelvin} in the cryostat. We compared 8 SQUIDs each with a different damping scheme to explore the influence of the added damping on the resonant structures. The most effective method resulted to be the inductive damping of the feed lines, as it showcased significantly smoother IV curves as compared to the non-damped counterparts. The implementation of the input coil as a lossy microstrip line, however, seemed to provide no considerable reduction of the observed resonance artefacts. It is noteworthy though, that the input coils were in an open loop and not shorted or connected to another coil. This might affect input coil related resonance frequencies, which should be investigated in future works. 

The flux noise measurements were performed alongside the IVC measurements in the same cryostat and at the same temperature. Two front-end SQUIDs with lossy input coils, one without inductive damping from the chip 3A13 and one with from the chip 2A16, were measured in a two-stage setup. The obtained intrinsic flux noise of the front-ends were  $\sqrt{S_{\rm \Phi,w}}=\qty{0.23}{\micro\fq\per\sqrthz}$ and $\sqrt{S_{{\rm \Phi},1/f}}(\qty{1}{\Hz})=\qty{3.52}{\micro\fq\per\sqrthz}$ with an exponent of $\alpha=0.85$ for the front-end without inductive damping and $\sqrt{S_{\rm \Phi,w}}=\qty{0.30}{\micro\fq\per\sqrthz}$, $\sqrt{S_{{\rm \Phi},1/f}}(\qty{1}{\Hz})=\qty{3.54}{\micro\fq\per\sqrthz}$ and $\alpha=0.82$ for the inductively damped front-end. Another front-end of the same type as the former has been measured in \cite{Mazibrada2024} using a cross correlated setup, which provided comparable values to the front-end from 3A13. The non-lossy variant has been measured in an integrated two-stage setup and yielded, possibly due to selfheating effects, a larger white noise of $\sqrt{S_{\rm \Phi,w}}=\qty{0.42}{\micro\fq\per\sqrthz}$, whereas the $1/f$ component resulted in a lower value of $\sqrt{S_{{\rm \Phi},1/f}}(\qty{1}{\Hz})=\qty{2.0}{\micro\fq\per\sqrthz}$ with an also smaller exponent of $\alpha=0.60$. These values are similar to the noise of the previous design with an intermediary flux transformer setup measured in \cite{Bauer2022}. Further measurements are needed to verify if these values are representative for the respective SQUID type. The results are nevertheless comparable to other low-noise SQUIDs developed in this working group, even though smaller values up to $<\qty{1}{\micro\fq\per\sqrthz}$ have been achieved in the low frequency regime \cite{Ferring2015}. The noise values of the front-end from 2A16 indicate, however, that the inductive damping technique did not significantly raise the overall noise levels, thus representing a technique worth pursuing in the future. The flux-to-flux coupling $\Delta\Phi_{\rm s}/\Delta\Phi$ regarding the maXs100 detector yielded \qty{2.25}{\percent}, which represents an increase of \qty{29}{\percent} over the value of \qty{1.75}{\percent} obtained with the intermediary flux transformer in \cite{Bauer2022}. Consequently, our extrinsic energy sensitivities are comparatively low, where the best value regarding the white noise yielded $\epsilon_{\rm p,w}=23.11\, h$ and regarding the $1/f$ noise $\epsilon_{{\rm p}, 1/f}=1910\, h$.

A possible measure to further increase the flux-to-flux coupling while also reducing the white noise would be a better adjustment of the SQUID loop and input coil inductance, which can be realized by expanding the washer loop size, such that the inductances are shifted towards their target values $L_{\rm s}=\qty{147}{\pH}$ and $L_{\rm i}=\qty{7.15}{\nH}$, respectively. Furthermore, substituting window-type with cross-type junctions allows for the use of shunt resistances twice as large, leading to a significant reduction of the intrinsic white noise by a factor of 2. The four times smaller junction capacitance would also shift the fundamental SQUID resonance further away from the working point, ensuring optimal performance. Nevertheless, the new SQUID design represents an improved detector SQUID for the maXs100 detector and should be preferably used in the future. 
