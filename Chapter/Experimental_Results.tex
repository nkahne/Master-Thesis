\chapter{Experimental Results} \label{ch_results}

This chapter will provide an overview over the general performance of the first stage dc-SQUIDs developed within the framework of this thesis. Particularly, the overall resonance behavior and the noise spectra were investigated. We will begin with a summary of characteristic parameters obtained by our measurements and compare them with the target values. The SQUIDs were tested both in a single- and a two-stage setup as described in section \ref{sec_operation}. The single-stage measurement was carried out at $T=\qty{4.2}{\kelvin}$ in a liquid helium transport vessel as well as in a dilution cryostat with $T=\qty{10}{\milli\kelvin}$. The former submerges the SQUIDs in liquid helium via a dipstick, which provides a sample holder for PCBs. The SQUIDs are glued onto those PCBs and can be electrically connected to them through aluminum bond wires, by utilizing the bond pads shown around the border of the chip in figure \ref{abb:damped_chip}. The sample holder is equipped with both a superconducting (niobium) and a soft-magnetic shield to suppress external magnetic fields. The read-out is enabled by connecting the broadband SQUID electronic of the type XXF-1 (see subsection \ref{subsec_fll}) to the dipstick and using it to both supply the necessary bias and ramp current signals to the SQUID as well as provide the FLL feedback circuit. The SQUID electronic is then controlled via software and the voltage output observed on a keysight InfiniiVision\footnote{Keysight Technologies Deutschland GmbH, Herrenberger Straße 130, 71034 Böblingen} oscilloscope. The noise measurement has been conducted with the two-stage setup at $T=\qty{10}{\milli\kelvin}$ in the cryostat, whereas the single-stage measurements to obtain characteristic Front-End properties have been done both in the helium vessel and the cryostat. 

\section{Characteristic dc-SQUID Parameters}

Despite all above-mentioned distinctions between the Fron-End variants, most characteristic parameters such as the SQUID loop inductance $L_{\rm s}$ or shunt resistance $R_{\rm s}$ are unaffected by these variations. We therefore consider the following measurements to be representative for all variants and assume the errors to stem from fabrication-related deviations across the wafer. \\
To qualitatively explain the following considerations, we present the current-voltage as well as the voltage-flux characteristics of one of the measured SQUIDs, shown in figure \ref{abb:}. This SQUID with the label HDSQ17-W1-3C16 is of the type 'non-lossy' with inductive damping and has been measured at $T=\qty{10}{\milli\kelvin}$. We start the characterization by measuring the slope of the IV curves across the ohmic regime, which corresponds to the normal resistance $R_{\rm n}$ of the SQUID. Including the SQUID chosen for figure \ref{abb:damped_chip}, we measured 10 SQUIDs from the wafer HDSQ17-W1 covering all above-mentioned variants. The median normal resistance taken from these SQUIDs yielded the value $R_{\rm n}=\qty{3}{\ohm}$. This corresponds to a shunt resistance $R_{\rm s}=2R_{\rm n}$ of $R_{\rm s}=\qty{6}{\ohm}$, which exactly corresponds to the targeted value of $R_{\rm s}=\qty{6}{\ohm}$. The maximum voltage swing shown in figure \ref{abb:} is obtained by applying a bias current $I_{\rm b}^{\rm max}$ supplied by the SQUID electronic, giving median values of $\Delta V_{\rm max}=\qty{29.95}{\micro\volt}$ for $I_{\rm b}^{\rm max}=\qty{11.62}{\micro\ampere}$. For $T=\qty{0}{\kelvin}$, $I_{\rm b}^{\rm max}$ should correspond to twice the critical current $I_{\rm c}$ (compare figure \ref{abb:IV_VPhi_theo}). Due to the finite temperature we obtain noticeable thermal smoothening visible in the IVC of figure \ref{abb:}, such that the critical current is better approximated by \cite{Drung1996}

\gl{
I_{\rm c} \approx \frac{I_{\rm b}^{\rm max}}{2} + \frac{k_{\rm B}T}{\unit{\fq}}\left(1+\sqrt{1+\frac{I_{\rm b}^{\rm max}\unit{\fq}}{k_{\rm B}T}}\right) \ \ .
}{}

We therefore calculate the critical current to $I_{\rm c}=\qty{5.84}{\micro\ampere}$\footnote{Calculated with the median of all 10 Ic's. If I calculate Ic once with the median of Ib,max, then the result is Ib,max/2 $\approx$ 5.81. Which is better? For the above argument, 5.84 fits better.}. This corresponds to a critical current density of $j_{\rm c}=\qty{28.8}{\ampere\per\centi\meter\squared}$, such that both values only deviate \qty{2.6}{\percent} form the design values. 


%Discuss various measured/simulated parameters such as Rs (mention sputtered thickness and sheet resistance),Rn,$\Delta V$,Ic,beta's,Ls,Li,M's and calculate $\frac{\Delta\Phi_{\rm s}}{\Delta\Phi_{\rm p}}$, $\epsilon_p$. Summarize the values in a table. Compare with previous FEs and literature. 

\begin{table}%[htb]
	\centering
	\begin{tabular}{l*{6}{c}}
		test & 1 & 2 & 3 & 4 & 5 & 6 \\
		\hline
		a & 1 & 2 & 3 & 4 & 5 & 6 \\
		b & 1 & 2 & 3 & 4 & 5 & 6 \\
	\end{tabular}
	\caption{SQUID parameters}
	\label{tab:SQUIDparameters}
\end{table}

\subsection{Input Coil Inductance}

Explain the method to measure Li, compare the result with expected value.

\section{Resonance Behavior}\label{sec_resonance_results}


To do: plot and fit C' of the jjs with the empirical relation...
Discuss in more detail all types of resonances with formulae and calculate the frequencies (there are still two of them that are unclear/need to be discussed in private or in a meeting)

\textit{Figures}: All measured FE IVCs, like non-lossy, lossy, damping variants, iso, no-iso, no input coil, Fabiennes FE... (should I also show the corresponding VPhi curves?)

Try to identify the resonances in the plots. Discuss what actually helped damping and what not. 

\section{Noise Performance}

\subsection{Lumped Element Two-Stage Setup}

\textit{Figures}: Show noise measurements at mK (3 setups from SQUID Cryo, 1 setup from Mocca Cryo) 

Discuss all contributions, especially from the FE -> Difference if input coil is shorted or not?

Mention how it would change with a cross JJ SQUID (lower white noise) -> here or in the summary?

Are 2stage VPhi curves interesting or do we restrict ourselves to noise plots?

\subsection{Integrated Two-Stage Setup}

\textit{Figure}: Int. 2stage noise measurement

Summarize Fabians measurements and compare with ours. 