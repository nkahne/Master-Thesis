\chapter{Experimental Results}

\section{Characteristic dc-SQUID Parameters}

Discuss various measured/simulated parameters such as Rs,Rn,$\Delta V$,Ic,beta's,Ls,Li,M's and calculate $\frac{\Delta\Phi_{\rm s}}{\Delta\Phi_{\rm p}}$, $\epsilon_p$. Summarize the values in a table. Compare with previous FEs and literature. 

\subsection{Input Coil Inductance}

Explain the method to measure Li, compare the result with expected value.

\section{Resonance Behavior}\label{sec_resonance_results}

Discuss in more detail all types of resonances with formulae and calculate the frequencies (there are still two of them that are unclear/need to be discussed in private or in a meeting)

\textit{Figures}: All measured FE IVCs, like non-lossy, lossy, damping variants, iso, no-iso, no input coil, Fabiennes FE... (should I also show the corresponding VPhi curves?)

Try to identify the resonances in the plots. Discuss what actually helped damping and what not. 

\section{Noise Performance}

\subsection{Lumped Element Two-Stage Setup}

\textit{Figures}: Show noise measurements at mK (3 setups from SQUID Cryo, 1 setup from Mocca Cryo) 

Discuss all contributions, especially from the FE -> Difference if input coil is shorted or not?

Mention how it would change with a cross JJ SQUID (lower white noise) -> here or in the summary?

Are 2stage VPhi curves interesting or do we restrict ourselves to noise plots?

\subsection{Integrated Two-Stage Setup}

\textit{Figure}: Int. 2stage noise measurement

Summarize Fabians measurements and compare with ours. 