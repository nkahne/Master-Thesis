\chapter{Experimental Results} \label{ch_results}

This chapter will provide an overview over the general performance of the first stage dc-SQUIDs developed within the framework of this thesis. Particularly, the overall resonance behavior and the noise spectra were investigated. We will begin with a summary of characteristic parameters obtained by our measurements and compare them with the target values. The SQUIDs were produced in the institute's cleanroom and then tested both in a single- and a two-stage setup as described in section \ref{sec_operation}. The single-stage measurement was carried out at $T=\qty{4.2}{\kelvin}$ in a liquid helium transport vessel as well as in a dilution cryostat with $T=\qty{10}{\milli\kelvin}$. The former submerges the SQUIDs in liquid helium via a dipstick, which provides a sample holder for PCBs. The SQUIDs are glued onto those PCBs and can be electrically connected to them through aluminum bond wires, by utilizing the bond pads shown around the border of the chip in figure \ref{abb:damped_chip}. The sample holder is equipped with both a superconducting (niobium) and a soft-magnetic shield to suppress external magnetic fields. The read-out is enabled by connecting the broadband SQUID electronics of the type XXF-1 (see subsection \ref{subsec_fll}) to the dipstick and using it to both supply the necessary bias and ramp current signals to the SQUID as well as provide the FLL feedback circuit. The SQUID electronics is then controlled via software and the voltage output observed on a keysight InfiniiVision\footnote{Keysight Technologies Deutschland GmbH, Herrenberger Straße 130, 71034 Böblingen} oscilloscope. The noise measurement has been conducted with the two-stage setup at $T=\qty{10}{\milli\kelvin}$ in the cryostat, whereas the single-stage measurements to obtain characteristic Front-End properties have been done both in the helium vessel and the cryostat. 

\section{Characteristic dc-SQUID Parameters} \label{sec_charac}

Despite all above-mentioned distinctions between the Fron-End variants, most characteristic parameters such as the SQUID loop inductance $L_{\rm s}$ or shunt resistance $R_{\rm s}$ are unaffected by these variations. We therefore consider the following measurements to be representative for all variants and assume the errors to stem from fabrication-related deviations across the wafer. \\
To qualitatively explain the following considerations, we present the current-voltage as well as the voltage-flux characteristics of one of the measured SQUIDs, shown in figure \ref{abb:iv_vphi_result}. This SQUID with the label HDSQ17-W1-3C16 is of the type 'non-lossy' with inductive damping and has been measured at $T=\qty{10}{\milli\kelvin}$. We start the characterization by measuring the slope of the IV curves across the ohmic regime, which corresponds to the normal resistance $R_{\rm n}$ of the SQUID. Including the SQUID chosen for figure \ref{abb:damped_chip}, we measured 10 SQUIDs from the wafer HDSQ17-W1 covering all above-mentioned variants. The median normal resistance taken from these SQUIDs yielded the value $R_{\rm n}=\qty{3}{\ohm}$. This corresponds to a shunt resistance $R_{\rm s}=2R_{\rm n}$ of $R_{\rm s}=\qty{6}{\ohm}$, which exactly corresponds to the targeted value of $R_{\rm s}=\qty{6}{\ohm}$. The maximum voltage swing as seen in figure \ref{abb:iv_vphi_result} (right) is obtained by applying a bias current $I_{\rm b}^{\rm max}$ supplied by the SQUID electronics, while driving a ramp signal through one of the coupled coils to provide a varying external flux. This yielded median values of $\Delta V_{\rm max}=\qty{29.95}{\micro\volt}$ and $I_{\rm b}^{\rm max}=\qty{11.62}{\micro\ampere}$. For $T=\qty{0}{\kelvin}$, $I_{\rm b}^{\rm max}$ should correspond to twice the critical current $I_{\rm c}$ (compare figure \ref{abb:IV_VPhi_theo}). Due to the finite temperature we obtain noticeable thermal smoothening visible in the IVC of figure \ref{abb:iv_vphi_result} (left), such that the critical current is better approximated by \cite{Drung1996}

\twofigurescenter {t!}
{width=0.48\textwidth}
{../Figures/IV_3c16_ch2}
{0.02\textwidth} %hspace b/w figures
{width=0.48\textwidth}
{../Figures/VPhi_3c16_ch2}
{0.5cm} %vspace
{Current-voltage (left) and voltage-flux characteristics (right) of a non-lossy SQUID with inductive damping, labelled HDSQ17-W1-3C16. The family of IV curves are measured by varying the bias current for either the input or the feedback coil, while a current ramp signal is driven through the SQUID. To obtain the corresponding $V\Phi$ characteristic, the roles for the bias and ramp signal are switched. The curve was measured at $I_{\rm b}^{\rm max}=\qty{11.33}{\uA}$, resulting in a maximal voltage swing $\Delta V_{\rm max}=\qty{30.7}{\uV}$. Note that both plots have different scaling on the y-axis.}
{iv_vphi_result}


\gl{
I_{\rm c} \approx \frac{I_{\rm b}^{\rm max}}{2} + \frac{k_{\rm B}T}{\unit{\fq}}\left(1+\sqrt{1+\frac{I_{\rm b}^{\rm max}\unit{\fq}}{k_{\rm B}T}}\right) \ \ .
}{}

We therefore calculate the critical current to\footnote{Calculated with the median of all 10 Ic's. If I calculate Ic once with the median of Ib,max, then the result is Ib,max/2 $\approx$ 5.81. Which is better? For the above argument, 5.84 fits better.} $I_{\rm c}=\qty{5.84}{\micro\ampere}$. This corresponds to a critical current density of $j_{\rm c}=\qty{28.8}{\ampere\per\centi\meter\squared}$, such that both values only deviate \qty{2.7}{\percent} from the design values. The junction capacitance thus becomes slightly smaller than the design value according to the empirical relation introduced in section \ref{sec_FEdesign}. Using the fitted function $\frac{1}{C'}=0.132-0.053\log_{10}j_{\rm c}$ obtained in \cite{Bauer2022} and by again assuming $C'=C$, the capacitance yields $C=\qty{0.948}{\pico\farad}$, which is in excellent agreement with the target value of \qty{0.95}{\pico\farad}. The McCumber parameter can now be calculated to $\beta_{\rm C}=0.61$, also fitting well with the design value of 0.62. The screening parameter is typically derived by using the $\beta_{\rm L}$-dependent normalized current swing $\frac{\Delta I_{\rm max}}{2I_{\rm c}}$ at $V=0$, which has been numerically simulated in \cite{Tesche1977}. We estimated the current swing by extrapolating both extremal IV curves for the theoretical case of $T=0$, thus neglecting the thermal rounding and obtaining a minimal and maximal critical current $I_{\rm c,1}$ and $I_{\rm c,2}$. For our SQUIDs, the extremal IV curve with the lower critical current $I_{\rm c,1}$ does not correspond exactly to $\Phi = (n+\frac{1}{2}\Phi_0)$ (compare figure \ref{abb:iv_vphi_result}), 
which is a consequence of an asymmetric current injection (see below). The median of the maximal current swing $\Delta I_{\rm max}$ therefore yields \qty{7.18}{\micro\ampere}, ranging from \qty{6.09}{\micro\ampere} to \qty{7.73}{\micro\ampere} for the lowest and highest measured value. The value for $\Delta I_{\rm max}$, which approximately corresponds to the current swing of the voltage-biased SQUID in a two-stage configuration, indicates that the SQUID is well adapted to the arrays produced in this working group \cite{Kempf2015}\footnote{Die arrays in diesem paper sind andere als die die wir nutzen und haben $M^{-1}=12.9$ µA/phi0. Wo finde ich veröffentlichungen zu HDSQ15w3 arrays?}. These arrays have a typical reciprocal mutual inductance of $\frac{1}{M_{\rm ix}}=\qty{11.7}{\micro\ampere\per\fq}$, such that the flux $\Delta\Phi_{\rm x}$ coupled to the array is $\Delta\Phi_{\rm x}=M_{\rm ix}\Delta I_{\rm max}=\qty{0.61}{\fq}$, which is close to the optimal value of $\frac{\unit{\fq}}{2}$ to achieve a flux gain of $G_{\rm \Phi}\approx 3$ (see section \ref{subsec_2stage_theo}). The current swing allows us to calculate the screening parameter to $\beta_{\rm L}=0.60$, which is rather low compared to the design value of $\beta_{\rm L}=0.86$. Hysteretic behavior related to the obtained $\beta_{\rm L}$ and $\beta_{\rm C}$ should nevertheless not occur and thus be absent in the IVCs. The intersections and current steps seen in figure \ref{abb:iv_vphi_result} likely stem from resonances, which will be discussed in section \ref{sec_resonance_results}. \\ 

With $I_{\rm c}=5.84$ and $\beta_{\rm L}=0.60$ we obtain a SQUID loop inductance of $L_{\rm s}=\qty{103}{\pico\henry}$, which fits very well with a simulated value of $L_{\rm s}=\qty{106}{\pico\henry}$. This significantly deviates from the design value of $L_{\rm s}=\qty{147}{\pico\henry}$, which can be partly explained by the conservative estimation for the geometric adjustment of the washer loop size, tendentially resulting in a smaller inductance. Another reason is a that the SQUID loop inductance of the previous design resulted in a lower value than intended in \cite{Bauer2022}. On the wafer HDSQ17-W1 we included for comparison the original design with a single turn input coil, where we measured two channels of the chip HDSQ17-W1-3A09 alongside the other 10 new SQUIDs discussed in this section. These provided the respective values $L_{\rm s}=\qty{127}{\pico\henry}$ ($\beta_{\rm L}=0.80$) and $L_{\rm s}=\qty{138}{\pico\henry}$ ($\beta_{\rm L}=0.87$), which further supports the assumption of the original SQUID loop design being too small. A better adjustment of $L_{\rm s}$ can be realized in future works. \\ 
 
The mutual inductances of the input and feedback coil were obtained by both sending a bias current to the SQUID loop and a current ramp signal to the coils. The input coil was only connected to the SQUID electronics during the measurements in the helium vessel at \qty{4.2}{\kelvin}, where we calculated $M_{\rm is}$ of 22 Front-End channels from 9 chips, covering all non-lossy and lossy variants. Counting the obtained voltage oscillations per given current ramp amplitude leads to the measured median values $M_{\rm is}=\qty{611}{\pico\farad}$ and $M_{\rm fs}=\qty{51}{\pico\farad}$. The latter coincides well with the measured value of \qty{50}{\pico\farad} from \cite{Bauer2022}. The input coil current sensitivity is almost twice as large as the value determined with the previous design with a single-turn input coil, which yielded $M_{\rm is}=\qty{328}{\pico\farad}$ \cite{Bauer2022}. This result can be understood by the general linear dependence of the mutual inductance from the number of turns \textit{n} of the input coil and the SQUID loop inductance, i.e. $M_{\rm is}\approx nL_{\rm s}$ \cite{Ketchen1981}. This expression would need to be multiplied by a factor of 4 due to the parallel connection of the four washer loops. The result doesn't match the lower measured value of the previous design for $n=1$, however, one needs to account for the parasitic inductance at the junction area that doesn't contribute to M, which in turn reduces the mutual inductance. The fact that $M_{\rm is}\approx 2\cdot \qty{328}{\pico\farad}>\qty{611}{\pico\farad}$ can be explained by two reasons. On one hand, the inductance $L_{\rm s}$ has been reduced by approximately \qty{20}{\percent} as compared to the previous design, which according to equation \ref{kis_theo} leads to a decreased current sensitivity $M_{\rm is}$. On the other hand, the geometric two-turn input coil structure might deteriorate the coupling constant $k_{\rm is}$, as compared to a single turn design. This parameter will be determined together with the input coil inductance in the following subsection \ref{subsec_Li}.

As for the transfer coefficients, we obtained different values for the positive and negative slopes of the $I\Phi$ curve. This is attributed to the intended asymmetric current injection, which is realized by differing SQUID loop inductances between each arm of the loop, ultimately leading to a unilaterally larger transfer coefficient $I_{\rm \Phi}$ \cite{Ferring2015}. The values are $I_{\rm \Phi,+}=\qty{10.2}{\micro\ampere\per\fq}$ and $I_{\rm \Phi,-}=\qty{23.1}{\micro\ampere\per\fq}$ for the positive and negative slope, respectively. The voltage transfer coefficient is approximately symmetric and yielded $V_{\rm \Phi,+}=\qty{94.0}{\micro\volt\per\fq}$ and $V_{\rm \Phi,-}=\qty{93.0}{\micro\volt\per\fq}$. Except for $I_{\rm \Phi,+}$, the variance of these transfer parameters was rather large, with deviations from the median of up to \qty{39}{\percent}. These results are nevertheless comparable to previous SQUIDs produced in this working group.      


%Discuss various measured/simulated parameters such as Rs (mention sputtered thickness and sheet resistance),Rn,$\Delta V$,Ic,beta's,Ls,Li,M's and calculate $\frac{\Delta\Phi_{\rm s}}{\Delta\Phi_{\rm p}}$, $\epsilon_p$. Summarize the values in a table. Compare with previous FEs and literature. 


\subsection{Input Coil Inductance} \label{subsec_Li}

To determine whether the new SQUID design with increased input inductance exhibits better coupling to the maXs100 detector, it is essential to measure $L_{\rm i}$. For this measurement, the input coil is electrically shorted via its bond pads with aluminum bond wires while the Front-End is in a two-stage setup at \qty{4.2}{\kelvin}. In this case, the detector SQUID is in an integrated chip that contains both the new non-lossy Front-End design as well as an array for the second stage, which has been developed and tested by \cite{Kraemer2023}. These integrated chips were produced on the wafer labeled as HDSQ16-W1 and were designed with the same material thicknesses, such that we expect the following considerations to be representative for the non-lossy SQUIDs from HDSQ17-W1. The aluminum bond wires are normal conducting at these temperatures and thus provide a resistance $R_{\rm bond}$. This closed loop couples via $M_{\rm is}$ thermal noise from the resistive wires into the SQUID loop, which is added to the apparent noise of the first stage SQUID. The resistance $R_{\rm bond}$ forms together with the total inductance $L_{\rm tot}$ consisting of the input coil, its feed line and the wire inductance an \textit{RL} lowpass filter. The attributed cut-off frequency damps higher noise frequencies up to the point where this noise contribution becomes negligible, as can be seen in the measured noise spectrum of figure \ref{abb:Li_meas}. The second drop at even higher frequencies represents the lowpass characteristic of the SQUID electronics, as it provides a limited bandwidth of up to \qty{7}{\mega\hertz}\footnote{6 oder 7?}. The total apparent noise of the Front-End SQUID is then given by     

\figureleft {t!}
{width=\textwidth} %sets how much of the fig space is used
{../Figures/Li_meas}
{9cm} %sets width of the fig space
{0cm}
{Noise spectrum of the input coil inductance $L_{\rm i}$ measurement. The input coil is resistively shorted through aluminum bond wires with resistance $R_{\rm bond}$, which results in thermal noise that couples into the SQUID. The added noise has a cut-off frequency $f_{\rm c}=\frac{R_{\rm bond}}{2\pi L_{\rm tot}}$ that allows to extract the added white noise component by a numerical fit. The input inductance can be derived by substracting the parasitic inductances from $L_{\rm tot}$, which is a fit parameter alongside $R_{\rm bond}$.}
{Li_meas}

\gl{
S_{\rm \Phi_s,SQ} = M_{\rm is}^2\frac{4k_{\rm B}T}{R_{\rm bond}}\left[
\frac{1}{1+(\frac{2\pi fL_{\rm tot}}{R_{\rm bond}})^{2}}\right]	+ S_{\rm \Phi_s,w} \ \ ,
}{}

where the first term describes the added current noise of the shorted circuit, which is schematically shown in the inset of figure \ref{abb:Li_meas}. The second term represents the apparent white noise of the SQUID. This expression is numerically fitted to the measured data to obtain both the white noise component $ M_{\rm is}^2\frac{4k_{\rm B}T}{R_{\rm bond}}$ and the cut-off frequency $f_{\rm c}=\frac{R_{\rm bond}}{2\pi L_{\rm tot}}$. Two measurements from separate chips (2C14 and 2C23) were carried out, where the one from 2C14 is depicted in figure \ref{abb:Li_meas}. The fits resulted in $R_{\rm bond}^{\rm 2C14}=\qty{2.07}{\milli\ohm}$ and $L_{\rm tot}^{\rm 2C14}=\qty{6.86}{\nH}$ for the chip 2C14, whereas $R_{\rm bond}^{\rm 2C23}=\qty{1.86}{\milli\ohm}$ and $L_{\rm tot}^{\rm 2C23}=\qty{6.74}{\nH}$ for the chip 2C23. It has been shown in \cite{Hengstler2017}, that the aluminum wires exhibit an inductance of $L_{\rm bond}=\qty{0.14}{\nano\henry\per\milli\ohm}$, which consequently results in $L_{\rm bond}^{\rm 2C14}=\qty{0.29}{\nH}$ and  $L_{\rm bond}^{\rm 2C23}=\qty{0.26}{\nH}$. The other parasitic inductance that stems from the input coil feed lines can be estimated with the microstrip inductance per unit length \cite{Chang1979}

\gl{
L_{\rm str}=\mu_0\frac{h+2\lambda}{w_{\rm i}+2h+4\lambda} \ \ ,	
}{Lstr} 

with $\lambda$ being the London penetration depth of the superconductor, $w_{\rm i}$ the width of the upper line and $h$ the thickness of the insulating layer. The length of the feed lines is approximately \qty{724}{\um}, the width \qty{3}{\um} and $\rm{SiO_2}$ has been fabricated as a \qty{375}{\nm} thick layer. We also assume $\lambda=\qty{90}{\nm}$, which is a typical value for the fabricated niobium in this working group. This yields a smaller contribution of $\qty{0.12}{\nH}$ for the feed line inductance. Thus, we finally obtain the input inductances $L_{\rm i}^{\rm 2C14}=\qty{6.45}{\nH}$ and $L_{\rm i}^{\rm 2C23}=\qty{6.36}{\nH}$. Even though these are close to the in the beginning of section \ref{sec_FEdesign} approximated $L_{\rm i}^{\rm theo}\approx n^{2}\qty{1.64}{\nH}\approx\qty{6.56}{\nH}$, we should take into account the smaller measured inductance $L_{\rm i}=\qty{1.27}{\nH}$ for the single turn input coil \cite{Bauer2022}. Furthermore, the microstrip inductance should not be neglected anymore as the length $l_{\rm i}$ of the input coil with two turns is now approximately twice as large. $L_{\rm i}$ is only linear proportional to the total microstrip inductance $l_{\rm i}L_{\rm str}$ \cite{Ketchen1991}, which would further reduce the increase of $L_{\rm i}$ upon adding a second turn. However, the enlargement of the washer loop has a stronger influence on $L_{\rm i}$ than on $L_{\rm s}$ due to the series connection of the input coil, such that the inductance increase per washer loop is multiplied by 4. This effect seems to narrowly compensate the two reductions, such that the measured values are only slightly below $L_{\rm i}^{\rm theo}$. With InductEx we obtained a simulated value of $L_{\rm i}=\qty{5.6}{\nH}$, which deviates \qty{13}{\percent} and \qty{12}{\percent} from $L_{\rm i}^{\rm 2C14}$ and $L_{\rm i}^{\rm 2C23}$, respectively. These large deviations are in accordance with previous inductance simulations in several works of this working group \cite{Ferring2015, Bauer2022}. For the following discussions, we will use the average $L_{\rm i}=\qty{6.40}{\nH}$ of the two measurements. \\

We have now all parameters to determine the coupling constant $k_{\rm is}$. Using equation \ref{kis_theo}, we obtain $k_{\rm is}=0.75$, which corresponds to the target value that has been set as a realistic upper limit for the minimization of the extrinsic energy sensitivity $\epsilon_{\rm p}$. This suggests that the reduction of the input coil current sensitivity $M_{\rm is}$ solely stems from the small measured SQUID loop inductance $L_{\rm s}$. It is noteworthy, however, that the variance of both the measured critical currents and the current swings was rather large with deviating up to \qty{11}{\percent} and \qty{15}{\percent}, respectively. This of course provides an uncertainty for $\beta_{\rm L}$ and $L_{\rm s}$ that needs to be taken into account. Such variances have also been observed in \cite{Bauer2022}, where it was shown that SQUIDs with \textit{cross-type} junctions exhibit a smaller variance across the wafer, making them more reliable than window-type junctions regarding $I_{\rm c}$. \\    
An overview of the most relevant measured parameters compared to their respective design values is shown in table \ref{tab:SQUIDparameters}.

\begin{table}[htb]
	\centering
	\begin{tabular}{c|*{9}{c}}
	\toprule
		Parameter & $R_{\rm s}$ & $I_{\rm c}$ & $M_{\rm is}$ & $M_{\rm fs}$ & $L_{\rm s}$ & $L_{\rm i}$ & $\beta_{\rm L}$ & $\beta_{\rm C}$ & $k_{\rm is}$ \\
		 & $[\Omega]$ & $[\unit{\micro\ampere}]$ & $[\unit{\pH}]$ & $[\unit{\pH}]$ & $[\unit{\pH}]$ & $[\unit{\nH}]$ &  &  &  \\
		\midrule
		Measured & 6 & 5.84 & 611 & 51 & 103 & 6.40 & 0.60 & 0.61 & 0.75 \\
		Design & 6 & 6 & - & 50 & 147 & 6.56 & 0.86 & 0.62 & 0.75 \\
	\end{tabular}
	\caption{Summary of measured characteristic parameters of the new dc-SQUID design with a two-turn input coil, which are compared with the corresponding target values.}
	\label{tab:SQUIDparameters}
\end{table}

\section{Resonance Behavior}\label{sec_resonance_results}

In this section we will investigate how various \textit{LC} resonances might affect the IVCs of the SQUIDs tested in the previous section and whether the applied damping techniques proved to be effective or not. Figure \ref{abb:all_IV_merge} showcases the IVCs of 8 distinct SQUIDs from the 10 chosen for the characterization. The order from top to bottom corresponds to the order shown in the inductively damped chip with various SQUID types in figure \ref{abb:damped_chip}. These damped SQUIDs are on the right side (e) through h)), while the corresponding SQUID types without inductive damping are on the left side (a) through d)) in the same order. There are a few differences between certain SQUID variants, however, they collectively share a prominent feature at $V_{\rm s}\approx\qty{40}{\uV}$. This artifact shows a pronounced current step with corresponding intersections between the IV curves. The fact that it is also present in the SQUIDs without input coil (d) and h)) rules out any input coil related resonances to be the underlying cause. Regarding the feedback coil, due to the small serially connected inductance its resonance $f_{L_{\rm f}C_{\rm p}}$ lies far above the operation frequency, as we will see in the following. Consequently, the pronounced current step at approximately \qty{40}{\uV} will be attributed to the fundamental SQUID resonance $f_{L_{\rm s}C}$, which by using equation \ref{fund_res} is calculated to $f_{L_{\rm s}C}=\frac{\qty{47}{\uV}}{\unit{\fq}}=\qty{22.7}{\GHz}$, using the measured values of $L_{\rm s}=\qty{103}{\pH}$ and $C=\qty{0.95}{\pF}$ This value does not exactly correspond to the observed \qty{40}{\uV}, which would indicate an underestimation of either the SQUID loop inductance $L_{\rm s}$, junction capacitance $C$ or both. As discussed in section \ref{sec_charac}, the parasitic capacitance arising from the insulation window structure of the junction is typically neglected, i.e. $C=C'$. The real junction capacitance should therefore slightly larger, leading to a decreased resonance frequency $f_{L_{\rm s}C}$. The measured inductance $L_{\rm s}$ can also deviate from the real one, which could be explained by the asymmetric $I\Phi$ characteristic as this aspect has not been included in the numerical simulation for $\Delta I(\beta_{\rm L})$ in \cite{Tesche1977}.  \\
The comparison of each side of figure \ref{abb:all_IV_merge} demonstrates a significant smoothening of all IVCs of the inductively damped SQUIDs, as compared to their non-damped counterpart. We can therefore conclude that damping with insulated gold pads proved to effectively suppress all visible resonances to some extent, regardless of the SQUID design. Consequently, one would expect the resulting intrinsic flux noise of the damped SQUIDs to be lower. This will be investigated in section \ref{sec_noise_results}. Whether the various step structures apart from the large step at \qty{40}{\uV} can be assigned to the different microstrip and \textit{LC} resonances described in subsection \ref{subsec_para_res} will be the subject of the following discussion. \\

For the $C_{\rm p}$-related resonances, we need to take into account that the SQUID loops are connected in parallel and the input coil in series. The parasitic capacitance $C_{\rm p}^{\rm loop}$ per washer loop will therefore need to be multiplied by a factor of 4 or $\frac{1}{4}$ to obtain the total parasitic capacitance $C_{\rm p}$. A theoretical value has been calculated to \cite{EnpukuI1991, EnpukuIII1992}    
  
\gl{
C_{\rm p}^{\rm loop}=\frac{l_{\rm i,1/4}C_{\rm str}}{8}=\frac{l_{\rm i,1/4}}{8}\frac{\epsilon_0\epsilon_{\rm r}w_{\rm i}K_{\rm f}(w_{\rm i},t_{\rm i},h)}{h} \ \ ,
}{}

\figurecenter {htb!}
%{scale=0.25}
%{width=\textwidth}
{height=\textheight}
{../Figures/all_IV/all_IV_merge}
{0cm}
{a) - d): Measured IVCs of the SQUID types presented in figure ?? in the same order, but without inductive damping. e) - h): The same measurement for the respective SQUID types but with inductive damping. Measured at $T=\qty{10}{\milli\kelvin}$.} 
{all_IV_merge}

where $l_{\rm i,1/4}$ denotes the input coil length on a single washer loop and $C_{\rm str}$
the microstrip line capacitance per unit length. The latter depends on the vacuum and relative permittivity $\epsilon_{\rm 0}$ and $\epsilon_{\rm r}$, as well as the width of the input coil $w_{\rm i}$, the fringing factor $K_{\rm f}(w_{\rm i},t_{\rm i},h)$ and the insulation thickness \textit{h}. The fringing factor also depends on $w_{\rm i}$, \textit{h} and the thickness of the input coil denoted as $t_{\rm i}$ \cite{Chang1979}. With $h=\qty{375}{\nm}$, $w_{\rm i}=\qty{3}{\um}$ and a sputtered Nb2 layer with $t_{\rm i}=\qty{400}{\nm}$ we obtain the value $K_{\rm f}=1.45$. This leads to a microstrip inductance of $C_{\rm str}=\qty{0.4}{\nF}$. The input coil length per washer loop is approximately given by $l_{\rm i,1/4}\approx\frac{l_{\rm i}}{4}\approx\qty{860}{\um}$, with the total length of the input coil being $l_{\rm i}\approx\qty{3.44}{\mm}$. Consequently, we obtain for the parasitic capacitance per washer loop $C_{\rm p}^{\rm loop}=\qty{0.043}{\pF}$. The total parasitic capacitance regarding the washer is therefore $C_{\rm p}=4C_{\rm p}^{\rm loop}=\qty{0.17}{\pF}$, which is still small compared to the junction capacitance $C=\qty{0.95}{\pF}$, i.e. $\frac{C_{\rm p}}{C}\ll 2$. The influence of this parameter on the energy sensitivity should consequently be small, as explained in subsection \ref{subsec_para_res}. The $L_{\rm s}C_{\rm p}$ resonance can now be calculated to $f_{L_{\rm s}C_{\rm p}}=\frac{1}{2\pi\sqrt{L_{\rm s}4C_{\rm p}^{\rm loop}}}=\qty{37.8}{\GHz}$, where we used the measured SQUID loop inductance $L_{\rm s}=\qty{103}{\pH}$. This corresponds to a voltage drop of $V_{\rm s}=\qty{78.1}{\uV}$ for the condition $f_{\rm J}=f_{L_{\rm s}C_{\rm p}}$. This is well above the optimal operation frequency given by $f_{\rm op}\approx 0.3f_{\rm c}\approx\qty{5.1}{\GHz}$ ($V_{\rm s}=\qty{10.5}{\uV}$), with $f_{\rm c}=\frac{I_{\rm c}R_{\rm s}}{\unit{\fq}}$ \cite{Cantor1996}. Mostly the plots a) through c) from figure \ref{abb:all_IV_merge} showcase small distinctive step-like structures around \qty{75}{\uV}. These could, however, also stem from a higher harmonic of the fundamental SQUID resonance, which should be the case for the SQUID without input coil (a)), as there is no corresponding parasitic capacitance. \\

For the $f_{L_{\rm i}C_{\rm p}}$ resonance we need to take the $R_{\rm x}C_{\rm x}$ attenuator into account. The capacitances $C_{\rm x}$ and $C_{\rm p}$ can approximately be added up to the total capacitance $C_{\rm tot}=C_{\rm x}+\frac{C_{\rm p}^{\rm loop}}{4}$ due to the serial connection of the input coil. The geometric input coil inductance is shielded analogously to $L_{\rm s}$ (compare equation \ref{eff_Ls}), giving the relation $L_{\rm i}'=(1-k_{\rm is}^2s_{\rm s})L_{\rm i}$ \cite{Cantor1996}. Here, the screening factor was derived to $s_{\rm s}=\frac{\beta_{\rm L}s_{\rm i}k_{\rm is}^2}{6+2\beta_{\rm L}+\beta_{\rm L}s_{\rm i}k_{\rm is}^2}$. The input inductance and pickup coil inductance also form a parallel connection, resulting in $L_{\rm tot}=\frac{L_{\rm i}'L_{\rm p}}{L_{\rm i}'+L_{\rm p}}$. The resonance frequency then reads 

\gl{
f_{L_{\rm i}C_{\rm p}}\approx\frac{1}{2\pi\sqrt{C_{\rm tot}L_{\rm tot}}} \ \ ,
}{}

which by using our measured and calculated parameters yields \qty{0.9}{\GHz}. The corresponding voltage drop is $V_{\rm s}=\qty{1.8}{\uV}$, which is well below the above-mentioned operation voltage and is therefore not visible in the IVCs from figure \ref{all_IV_merge}. The capacitance $C_{\rm x}$ is chosen small to minimze the attributed $Q_{L_{\rm i}C_{\rm p}}$ value, which for an RCL parallel circuit is given by 

\gl{
Q_{L_{\rm i}C_{\rm p}} \approx R_{\rm x}\sqrt{\frac{C_{\rm tot}}{L_{\rm tot}}} \ \ .
}{Q}

%\gl{
%Q_{L_{\rm i}C_{\rm p}} = R_{\rm x}\sqrt{\frac{(C_{\rm x}+C_{\rm p}/4)(L_{\rm i}'+L_{\rm p})}{L_{\rm i}'L_{\rm p}}}
%}{}

Evidently, $C_{\rm x}$ cannot be chosen arbitrarily small as it would shift the resonance to the vicinity of the working point. An optimal value has been found to be $Q_{L_{\rm i}C_{\rm p}}\approx 2$ \cite{Cantor1996}. The resistive componenent of the attenuator is typically dimensioned such that it corresponds to the nominal impedance $Z_0$ of the microstrip line in order to mitigate wave reflections occuring where the input coil leaves the SQUID loop. This impedance can be expressed as \cite{EnpukuI1991} 

\gl{
Z_{\rm 0}=\sqrt{\frac{L_{\rm str}}{C_{\rm str}}} \ \ .
}{} 

With the produced geometric proportions of the microstrip, the optimal value yields $R_{\rm x}=\qty{20.6}{\ohm}$, which equals the chosen design value. Therefore, with equation \ref{Q} we obtain an optimal capacitance of $C_{\rm x}=\qty{31}{\pF}$. Unfortunately, such a large capacitance has not yet been implemented into the new design. We adapted the lower value of the previous design (\qty{10}{\pF}), which had been chosen in \cite{Bauer2022} with the premise to work with the ECHo-100k detector. In our case we obtain $Q\approx 1.13$, however, the corresponding resonance frequency is still small enough to consider the effect of this resonance as negligible. \\
We can also further neglect the $C_{\rm p}$-related resonance with respect to the feedback coil, as its small design inductance of $L_{\rm f}=\qty{336}{\pH}$ leads to a resonance frequency of $f_{L_{\rm f}C_{\rm p}}=\frac{\qty{160}{\uV}}{\unit{\fq}}=\qty{77}{\GHz}$, well above $f_{\rm op}$. 
\\

Regarding the $\lambda/2$ resonances a distinction is made whether the inpt coil or the SQUID loop act as the signal carrying line. They will consequently be referred to as the input coil or washer resonance with the resonance frequencies $f_{l_{\rm i}}$ or $f_{l_{\rm w}}$, respectively. These should ideally fulfill the condition $4f_{l_{\rm i}}<f_{\rm op}<\frac{f_{l_{\rm w}}}{4}$ to mitigate their noise inducing influence at the working point \cite{Can1991}. The wave propagation velocity $c_{\rm str}$ depends on $L_{\rm str}$ and $C_{\rm str}$, such that equation \ref{stripline_res_general} becomes \cite{EnpukuIII1992}        

\gl{
f_{l_{\rm i}}=\frac{c_{\rm str}}{2l_{\rm i}}=\frac{1}{2l_{\rm i}\sqrt{L_{\rm str}C_{\rm str}}} \ \ .
}{}

With the length of the input coil $l_{\rm i}$, we obtain for the input coil resonance $f_{l_{\rm i}}\qty{17.6}{\GHz}$ ($V_{\rm s}=\qty{36.4}{\uV}$). However, we need to include the length $l_{\rm i}^{\rm feed}\approx \qty{620}{\um}$ of the input coil feed lines which are also structured as a microstrip transmission line. This results in $\qty{14.9}{\GHz}$ and accordingly $V_{\rm s}=\qty{30.9}{\uV}$. Even though the condition $4f_{l_{\rm i}}<f_{\rm op}$ is not fulfilled, there are no visible step structures in the vicinity of this frequency, which suggests successful damping of this microstrip resonance. The effectiveness of the $R_{\rm x}C_{\rm x}$ attenuator has also been demonstrated in \cite{Bauer2022}, where drastic step strutures at $f_{l_{\rm i}}=\frac{\qty{54}{\uV}}{\unit{\fq}}$ were eliminated entirely after adding the $R_{\rm x}C_{\rm x}$ component to the SQUID design with $L_{\rm i}=\qty{1.27}{\nH}$. \\
The washer resonance is more difficult to determine for the given SQUID loop design, as the complex washer structure surrounding the junction area affects the effective length $l_{\rm w}$ of the signal carrying line. A crude approximation can be done by adding the SQUID loop feed line length to the circumference of a single washer loop due to the parallel connection to estimate $l_{\rm w}$. We thus obtain $f_{l_{\rm w}}\approx\frac{\qty{50.3}{\uV}}{\unit{\fq}}=\qty{26.9}{\GHz}$, which is more than four times the operation frequency. Just like for $f_{l_{\rm i}}$, there aren't any artifacts in the SQUID IVCs that could be assigned to this resonance. This indicates that the washer shunt exhibits good damping properties, as has already been demonstrated by \cite{Richter2017} and \cite{Bauer2018}. \\

Another prominent step structure seen in the IVCs of figure \ref{abb:all_IV_merge} occurs at approximately \qty{15}{\uV} for the extremal curve with the smallest critical current $I_{\rm c,1}$. This feature is absent in the IVCs of the SQUIDs without input coil (a) and e)) as well as the previous design with a single-turn input coil (not shown here). Consequently, the second turn seems to cause an additional resonance effect thas has not yet been identified. It is also unclear why this artifact does not appear in plot d) of figure \ref{abb:all_IV_merge}. 
\\
Interestingly, there are no considerable differences between the characteristics of lossy and non-lossy variants. The noise behavior of the former will be investigated in the following section.

-LsC 
%Discuss in more detail all types of resonances with formulae and calculate the frequencies (there are still two of them that are unclear/need to be discussed in private or in a meeting)

%\textit{Figures}: All measured FE IVCs, like non-lossy, lossy, damping variants, iso, no-iso, no input coil, Fabiennes FE... (should I also show the corresponding VPhi curves?)

%Try to identify the resonances in the plots. Discuss what actually helped damping and what not. 

%Possible reason for LsC being too low: betaL theo adapted for symmetric squids, might need correction for asymm. -> could mean that betaL is too small, and therefore Ls. Would make sense as we did not expect Ls to be \textit{that} small, also, InductEX seems to underestimate inductances and it simulated Ls=106pH, favoring a larger "real" Ls.

\section{Noise Performance} \label{sec_noise_results}

\subsection{Lumped Element Two-Stage Setup}

\textit{Figures}: Show noise measurements at mK (3 setups from SQUID Cryo, 1 setup from Mocca Cryo) 

Discuss all contributions, especially from the FE -> Difference if input coil is shorted or not?

Mention how it would change with a cross JJ SQUID (lower white noise) -> here or in the summary?

Are 2stage VPhi curves interesting or do we restrict ourselves to noise plots?

\subsection{Integrated Two-Stage Setup}

\textit{Figure}: Int. 2stage noise measurement

Summarize Fabians measurements and compare with ours. 