\chapter{Experimental Results} \label{ch_results}

This chapter provides an overview over the general performance of the dc-SQUIDs developed within the framework of this thesis. Particularly, the resonance behavior and the noise spectra were investigated. We begin with a summary of characteristic parameters obtained by our measurements and compare them with the target values. 
%The SQUIDs were produced in the institute's cleanroom and then tested both in a single- and a two-stage setup as described in section \ref{sec_operation}. The single-stage measurements were carried out at $T=\qty{4.2}{\kelvin}$ in a liquid helium transport vessel as well as in a dilution cryostat BF-LD250 from BlueFors\footnote{BlueFors Cryogenics Oy, Arinatie, 00370 Helsinki, Finnland} with a base temperature of $T=\qty{10}{\milli\kelvin}$. The former submerges the SQUIDs in liquid helium via a dipstick, which provides a sample holder for PCBs. The SQUIDs are glued onto those PCBs and are electrically connected to them via aluminum bond wires, by utilizing the bond pads shown in figure \ref{abb:damped_chip}. The sample holder is equipped with both a superconducting (niobium) and a soft-magnetic Cryoperm shield to suppress external magnetic fields. The read-out is done by a broadband SQUID electronics of the type XXF-1 (see subsection \ref{subsec_fll}) to both supply the necessary bias and ramp current signals to the SQUID as well to provide the FLL feedback. The SQUID electronics is controlled via software and the voltage output observed on a Keysight InfiniiVision\footnote{Keysight Technologies Deutschland GmbH, Herrenberger Straße 130, 71034 Böblingen} oscilloscope. 
The noise measurement has been conducted with the two-stage setup at $T=\qty{10}{\milli\kelvin}$ in the cryostat, whereas the single-stage measurements to obtain characteristic front-end properties have been done both in the helium vessel and the cryostat. 

\section{Characteristic dc-SQUID Parameters} \label{sec_charac}

Despite all distinctions between the front-end variants mentioned in chapter \ref{ch_SQUIDdesign}, most characteristic parameters such as the SQUID loop inductance $L_{\rm s}$ or shunt resistance $R_{\rm s}$ are unaffected by these variations. We, therefore, consider the following measurements to be representative for all variants and assume possible variations to be caused from fabrication-related variations across the wafer. \\
Figure \ref{abb:iv_vphi_result} shows the current-voltage as well as the voltage-flux characteristics of one of the measured SQUIDs. This SQUID (chip no. HDSQ17-W1-3C16) is of the type 'non-lossy' with inductive damping and has been measured at a cryostat temperature of $T=\qty{10}{\milli\kelvin}$. First, we measured the slope of the \textit{IV} curves across the ohmic regime, which allows to determine the normal resistance $R_{\rm n}$ of the SQUID. In total 10 SQUIDs from the wafer HDSQ17-W1 covering all variants we measured. The median normal resistance taken from these SQUIDs yields the value $R_{\rm n}=\qty{3.0}{\ohm}$, with the largest deviation being \qty{13.3}{\percent}. This corresponds to a shunt resistance of $R_{\rm s}=2R_{\rm n}=\qty{6.0}{\ohm}$, which exactly corresponds to the targeted value. 

The maximum voltage swing as seen in figure \ref{abb:iv_vphi_result} (right) is obtained by applying a bias current $I_{\rm b}^{\rm max}$ supplied by the SQUID electronics, while driving a ramp signal through one of the coupled coils to provide a varying external flux. This yielded median values of $\Delta V_{\rm max}=\qty{29.95}{\micro\volt}$ and $I_{\rm b}^{\rm max}=\qty{11.62}{\micro\ampere}$. For $T=\qty{0}{\kelvin}$, $I_{\rm b}^{\rm max}$ should correspond to twice the critical current $I_{\rm c}$ (compare figure \ref{abb:IV_VPhi_theo}). Due to the finite temperature we obtain noticeable thermal smoothening visible in the IVC of figure \ref{abb:iv_vphi_result} (left), such that the critical current is better approximated by \cite{Drung1996}

\twofigurescenter {t!}
{width=0.48\textwidth}
{../Figures/IV_3c16_ch2}
{0.02\textwidth} %hspace b/w figures
{width=0.48\textwidth}
{../Figures/VPhi_3c16_ch2}
{0.5cm} %vspace
{Current-voltage (left) and voltage-flux characteristics (right) of a non-lossy SQUID with inductive damping (chip no. HDSQ17-W1-3C16). 
The extremal IV curve with the lower critical current $I_{\rm c,1}$ does not correspond exactly to \\ $\Phi = (n+\Phi_0/2)$, which is a consequence of an asymmetric current injection.
	%The family of \textit{IV} curves are measured by varying the bias current for either the input or the feedback coil, while a current ramp signal is driven through the SQUID. To obtain the corresponding $V\Phi$ characteristic, the roles for the bias and ramp signal are switched. 
The $V\Phi$ curve was measured at $I_{\rm b}^{\rm max}=\qty{11.33}{\uA}$, resulting in a maximal voltage swing $\Delta V_{\rm max}=\qty{30.7}{\uV}$. Note that both plots have different scaling on the \textit{V}-axis.}
{iv_vphi_result}


\gl{
I_{\rm c} \approx \frac{I_{\rm b}^{\rm max}}{2} + \frac{k_{\rm B}T}{\unit{\fq}}\left(1+\sqrt{1+\frac{I_{\rm b}^{\rm max}\unit{\fq}}{k_{\rm B}T}}\right) \ \ .
}{}

We, therefore, calculate the median critical current to $I_{\rm c}=\qty{5.84}{\micro\ampere}$. With a junction size of \qtyproduct{4.5 x 4.5}{\micro\metre}, this corresponds to a critical current density of $j_{\rm c}=\qty{28.8}{\ampere\per\centi\meter\squared}$, such that both values only deviate \qty{2.7}{\percent} from the design values $I_{\rm c}=\qty{6}{\micro\ampere}$ and $j_{\rm c}=\qty{30}{\ampere\per\centi\meter\squared}$. The junction capacitance thus becomes slightly smaller than the design value according to the empirical relation introduced in section \ref{sec_FEdesign}. Using the empirical dependence $1/C'=0.132-0.053\log_{10}j_{\rm c}$ for the intrinsic junction capacitance $C'$ obtained in \cite{Bauer2022} and by again assuming $C'=C$, we get the capacitance $C=\qty{0.948}{\pico\farad}$, which is in excellent agreement with the target value of \qty{0.95}{\pico\farad}. The McCumber parameter can now be calculated to $\beta_{\rm C}=0.61$, also fitting well with the design value of 0.62. 

As for the transfer coefficients, different values for the positive and negative slopes of the $I\Phi$ curve are expected due to the asymmetric current injection, which is realized by differing SQUID loop inductances between each arm of the loop, ultimately leading to a unilaterally larger transfer coefficient $I_{\rm \Phi}$ \cite{Ferring2015}. The measured values are $I_{\rm \Phi,+}=\qty{10.2}{\micro\ampere\per\fq}$ and $I_{\rm \Phi,-}=\qty{23.1}{\micro\ampere\per\fq}$ for the positive and negative slope, respectively. The voltage transfer coefficient is approximately symmetric (compare figure \ref{abb:iv_vphi_result}) and yields $V_{\rm \Phi,+}=\qty{94.0}{\micro\volt\per\fq}$ and $V_{\rm \Phi,-}=\qty{93.0}{\micro\volt\per\fq}$. The variance of these transfer parameters is rather large, with deviations from the median of up to \qty{39}{\percent}. These results are nevertheless comparable to previous SQUIDs produced in this working group \cite{Richter2017}. \\  

{\large{\textbf{Screening parameter $\beta_{\rm L}$}}}
\\

The screening parameter is typically derived by using the $\beta_{\rm L}$-dependent normalized current swing $\Delta I_{\rm max}/2I_{\rm c}$ at $V=0$, which has been numerically simulated in \cite{Tesche1977}. We estimated the current swing by extrapolating both extremal IV curves for the theoretical case of $T=0$, thus neglecting the thermal rounding and obtaining a minimal and maximal critical current $I_{\rm c,1}$ and $I_{\rm c,2}$. %For our SQUIDs, the extremal IV curve with the lower critical current $I_{\rm c,1}$ does not correspond exactly to $\Phi = (n+\frac{1}{2}\Phi_0)$ (compare figure \ref{abb:iv_vphi_result}), which is a consequence of an asymmetric current injection (see below). 
The median of the maximal current swing $\Delta I_{\rm max}$ yields \qty{7.18}{\micro\ampere}, ranging from \qty{6.09}{\micro\ampere} to \qty{7.73}{\micro\ampere} for the lowest and highest measured value. The value for $\Delta I_{\rm max}$, which approximately corresponds to the current swing of the voltage-biased SQUID in a two-stage configuration, indicates that the SQUID is well adapted to the arrays produced in this working group. These arrays have a typical reciprocal mutual inductance of $1/M_{\rm ix}=\qty{11.7}{\micro\ampere\per\fq}$, such that the flux $\Delta\Phi_{\rm x}$ coupled to the array is $\Delta\Phi_{\rm x}=M_{\rm ix}\Delta I_{\rm max}=\qty{0.61}{\fq}$, which is close to the optimal value of $\unit{\fq}/2$ to achieve a flux gain of $G_{\rm \Phi}\approx 3$ (see section \ref{subsec_2stage_theo}). The current swing of each SQUID allows us to calculate the median screening parameter to $\beta_{\rm L}=0.65$, which is rather low compared to the design value of $\beta_{\rm L}=0.86$, indicating an overestimation of the SQUID loop inductance $L_{\rm s}$. According to the obtained $\beta_{\rm L}$ and $\beta_{\rm C}$, hysteretic behavior should not occur and thus be absent in the IVCs. The intersections and current steps seen in figure \ref{abb:iv_vphi_result}, therefore, likely stem from resonances, which will be discussed in section \ref{sec_resonance_results}. \\ 

{\large{\textbf{SQUID loop inductance $L_{\rm s}$}}}
\\

With the above-determined $I_{\rm c}$ and $\beta_{\rm L}$ values we obtain a median SQUID loop inductance of  $L_{\rm s}=\qty{108}{\pico\henry}$. This represents a significant deviation of \qty{27}{\percent} from the target value of $L_{\rm s}=\qty{147}{\pico\henry}$ discussed in section \ref{sec_FEdesign}, which can be partly explained by the conservative estimation for the geometric adjustment of the washer loop size, resulting in a smaller inductance. On the wafer HDSQ17-W1 we included for comparison the original design \cite{Bauer2022} with a single turn input coil, where we measured two channels of the chip HDSQ17-W1-3A09. These provided the respective values $L_{\rm s}=\qty{128}{\pico\henry}$ ($\beta_{\rm L}=0.80$) and $L_{\rm s}=\qty{138}{\pico\henry}$ ($\beta_{\rm L}=0.87$), which is in accordance with the value of \qty{135}{\pH} that was measured in \cite{Bauer2022}. This confirms that the washer loops were designed too small, resulting in a smaller SQUID loop inductance than the targeted \qty{147}{\pH}. The measured value of \qty{108}{\pH}, however, fits well with the simulated value of \qty{106}{\pH}, which was simulated with InductEx after the SQUIDs were already produced. The reduction of $L_{\rm s}$ reduces the coupling $M_{\rm is}$ between input coil and SQUID loop, which in turn reduces the flux-to-flux coupling $\Delta\Phi_{\rm s}/\Delta\Phi$ and thus the extrinsic energy sensitivity $\epsilon_{\rm p}$ given by equations \ref{ftf_Lp} and \ref{extr_energy_sens}, respectively. \\
 
{\large{\textbf{Mutual inductances}}}\\ 
 
The mutual inductances of the input and feedback coil were obtained by sending a current ramp to the corresponding coils. The input coil was only connected to the SQUID electronics during the measurements in the helium vessel at \qty{4.2}{\kelvin}, where $M_{\rm is}$ was measured from 22 front-end channels from 9 chips, covering all non-lossy and lossy variants. The measured median values are $M_{\rm is}=\qty{611}{\pico\farad}$ and $M_{\rm fs}=\qty{51}{\pico\farad}$, where the latter coincides well with the measured value of \qty{50}{\pico\farad} from \cite{Bauer2022}, as it is expected because the geometry of the feedback coil was not changed. The input coil current sensitivity is almost twice as large as the value determined with the previous design with a single-turn input coil, which yielded $M_{\rm is}=\qty{328}{\pico\farad}$ \cite{Bauer2022}. This result is expected from the general linear dependence of the mutual inductance on the number of turns \textit{n} of the input coil and the SQUID loop inductance, i.e. $M_{\rm is}\approx nL_{\rm s}$ \cite{Ketchen1981}. 
%This expression needs to be multiplied by a factor of 4 due to the parallel connection of the four washer loops. This results in a theoretical value of $M_{\rm is}=4\cdot \qty{135}{\pH}=\qty{540}{\pH}$ for $n=1$, which doesn't match the measured value of $\qty{328}{\pH}$. However, one needs to account for the parasitic inductance at the junction area that doesn't contribute to the $L_{\rm s}$ used in this approximation, which in turn reduces the mutual inductance. Furthermore, this expression assumes perfect coupling, which is not given for the SQUIDs produced in this working group as they typically exhibit $k\approx 0.75$. 
\\ The fact that $M_{\rm is}(n=2)/M_{\rm is}(n=1)<2$ can be explained by two reasons. On the one hand, the inductance $L_{\rm s}$ has been reduced by approximately \qty{19}{\percent} as compared to the previous design, which according to equation \ref{kis_theo} leads to a decreased current sensitivity $M_{\rm is}$. On the other hand, the geometric two-turn input coil structure might deteriorate the coupling constant $k_{\rm is}$, as compared to a single turn design.

Taking into account the input inductance calculated in section \ref{subsec_Li} and the other parameters determined above, we can calculate the coupling constant via equation \ref{kis_theo} to $k_{\rm is}=0.73$, which is close to the target value 0.75 that has been set as a realistic upper limit for the minimization of the extrinsic energy sensitivity $\epsilon_{\rm p}$. This suggests that the reduction of the input coil current sensitivity $M_{\rm is}$ is mostly due to the smaller SQUID loop inductance $L_{\rm s}$. It is noteworthy, however, that the variations of both the measured critical currents and the current swings were rather large with up to \qty{11}{\percent} and \qty{15}{\percent}, respectively. This of course provides an uncertainty for $\beta_{\rm L}$ and $L_{\rm s}$ that needs to be taken into account. Such variances have also been observed in \cite{Bauer2022}, where it was shown that SQUIDs with \textit{cross-type} junctions exhibit a smaller variance across the wafer, making them more reliable than window-type junctions regarding $I_{\rm c}$. 
%This parameter will be determined together with the input coil inductance in the following subsection \ref{subsec_Li}.
%However, as we will see in section \ref{subsec_Li} the coupling constant is close to the expectation. Therefore, the comparatively low value primarily results from the reduction of $L_{\rm s}$.  

%\large{\textbf{transfer coefficients}}\\ 

%As for the transfer coefficients, different values for the positive and negative slopes of the $I\Phi$ curve are expected due to the asymmetric current injection, which is realized by differing SQUID loop inductances between each arm of the loop, ultimately leading to a unilaterally larger transfer coefficient $I_{\rm \Phi}$ \cite{Ferring2015}. The measured values are $I_{\rm \Phi,+}=\qty{10.2}{\micro\ampere\per\fq}$ and $I_{\rm \Phi,-}=\qty{23.1}{\micro\ampere\per\fq}$ for the positive and negative slope, respectively. The voltage transfer coefficient is approximately symmetric and yields $V_{\rm \Phi,+}=\qty{94.0}{\micro\volt\per\fq}$ and $V_{\rm \Phi,-}=\qty{93.0}{\micro\volt\per\fq}$. The variance of these transfer parameters is rather large, with deviations from the median of up to \qty{39}{\percent}. These results are nevertheless comparable to previous SQUIDs produced in this working group \cite{Richter2017}.      


%Discuss various measured/simulated parameters such as Rs (mention sputtered thickness and sheet resistance),Rn,$\Delta V$,Ic,beta's,Ls,Li,M's and calculate $\frac{\Delta\Phi_{\rm s}}{\Delta\Phi_{\rm p}}$, $\epsilon_p$. Summarize the values in a table. Compare with previous FEs and literature. 


\section{Input Coil Inductance} \label{subsec_Li}

To determine whether the new SQUID design with increased input inductance exhibits better coupling to the maXs100 detector, it is essential to measure the input coil inductance $L_{\rm i}$. For this measurement, the input coil is electrically shorted via its bond pads with aluminum bond wires while the front-end is in a two-stage setup at \qty{4.2}{\kelvin}. In this special case, the detector SQUID was part of an integrated chip that contains both the new non-lossy front-end design as well as an array for the second stage, which has been developed and tested in \cite{Kraemer2023}. These integrated chips were produced on the wafer HDSQ16-W1 and were designed with the same material thicknesses, such that we expect the measurement to be representative for the non-lossy SQUIDs from HDSQ17-W1. The aluminum bond wires are normal conducting at the temperature of \qty{4.2}{\kelvin} and thus provide a resistance $R_{\rm bond}$. The closed loop couples frequency-independent thermal noise from the resistive wires into the SQUID loop via the mutual inductance $M_{\rm is}$, which adds to the apparent noise of the first stage SQUID. The resistance $R_{\rm bond}$ forms an \textit{RL} lowpass filter together with the total inductance $L_{\rm tot}$ consisting of the input coil, its feed line and the wire inductance. The attributed cut-off frequency damps higher noise frequencies up to the point where this noise contribution becomes negligible, as can be seen in the measured noise spectrum of figure \ref{abb:Li_meas}. The second cut-off at a frequency around \qty{5}{\MHz} represents the lowpass characteristic of the SQUID electronics, as it provides a limited bandwidth of up to \qty{6}{\MHz}. The total apparent noise of the Front-End SQUID is then given by     

\figureleft {t!}
{width=\textwidth} %sets how much of the fig space is used
{../Figures/Li_meas}
{9cm} %sets width of the fig space
{0cm}
{Noise spectrum of the input coil inductance $L_{\rm i}$ measurement. The input coil is resistively shorted through aluminum bond wires with resistance $R_{\rm bond}$, which results in thermal noise that couples into the SQUID. The added noise has a cut-off frequency\\ $f_{\rm c}=R_{\rm bond}/(2\pi L_{\rm tot})$ that allows to extract the added white noise component by a numerical fit. The input inductance can be derived by substracting the parasitic inductances from $L_{\rm tot}$, which is a fit parameter alongside $R_{\rm bond}$.}
{Li_meas}

\gl{
S_{\rm \Phi_s,SQ} = M_{\rm is}^2\frac{4k_{\rm B}T}{R_{\rm bond}}\left[
\frac{1}{1+(\frac{2\pi fL_{\rm tot}}{R_{\rm bond}})^{2}}\right]	+ S_{\rm \Phi_s,w} \ \ ,
}{}

where the first term describes the additional current noise of the shorted circuit, which is schematically shown in the inset of figure \ref{abb:Li_meas}. The second term represents the apparent white noise of the SQUID. This expression is numerically fitted to the measured data to obtain both the resistance of the bonding wires and the total inductance $L_{\rm tot}$. Two measurements from separate chips (2C14 and 2C23) were carried out, where the one from 2C14 is depicted in figure \ref{abb:Li_meas}. The fits result in $R_{\rm bond}^{\rm 2C14}=\qty{2.07}{\milli\ohm}$ and $L_{\rm tot}^{\rm 2C14}=\qty{6.86}{\nH}$ for the chip 2C14, whereas $R_{\rm bond}^{\rm 2C23}=\qty{1.86}{\milli\ohm}$ and $L_{\rm tot}^{\rm 2C23}=\qty{6.74}{\nH}$ for the chip 2C23. For the aluminum wires it has been shown in \cite{Hengstler2017} that they, depending on their arrangement, exhibit an inductance of $L_{\rm bond}\approx\qty{0.14}{\nano\henry\per\milli\ohm}$, which consequently results in $L_{\rm bond}^{\rm 2C14}=\qty{0.29}{\nH}$ and  $L_{\rm bond}^{\rm 2C23}=\qty{0.26}{\nH}$. The parasitic inductance from the input coil feed lines can be estimated with the microstrip inductance per unit length \cite{Chang1979}

\gl{
L_{\rm str}=\mu_0\frac{h+2\lambda}{w_{\rm i}+2h+4\lambda} \ \ ,	
}{Lstr} 

with $\lambda$ being the London penetration depth of the superconductor, $w_{\rm i}$ the width of the upper line and $h$ the thickness of the insulating layer. The length of the feed lines is approximately \qty{724}{\um}, the width \qty{3}{\um} and $\rm{SiO_2}$ has been fabricated as a \qty{375}{\nm} thick layer. We also assume $\lambda=\qty{90}{\nm}$, which is a typical value for the fabricated niobium in this working group. This yields a small contribution of $\qty{0.12}{\nH}$ for the feed line inductance. Thus, we finally obtain the input inductances $L_{\rm i}^{\rm 2C14}=\qty{6.45}{\nH}$ and $L_{\rm i}^{\rm 2C23}=\qty{6.36}{\nH}$. 
%With the measured input coil inductance of the previous design, the estimated value for the new design with an additional turn reads $L_{\rm i}^{\rm theo}\approx n^{2}\qty{1.27}{\nH}\approx\qty{5.08}{\nH}$
These are only slightly smaller than the originally targeted value of $L_{\rm i}^{\rm theo}\approx n^{2}\qty{1.64}{\nH}\approx\qty{6.56}{\nH}$ mentioned in section \ref{sec_FEdesign}. However, two effects overestimate the expected value. On the one hand, the measured inductance of the previous design only yielded \qty{1.27}{\nH}. On the other hand, the approximation does not take the additional microstrip inductance $L_{\rm str}$ into account (see section \ref{sec_practical_SQUID}), which grows only linearly with $n$ rather than $n^2$ \cite{Ketchen1991}. This effect would reduce the increase of $L_{\rm i}$ upon adding a second turn.
%, giving only $L_{\rm i}^{\rm theo}\approx\qty{4.5}{\nH}$. 
In contrast to that, a third effect leads to an underestimation of the target value. The enlargement of the washer loop has a strong influence on $L_{\rm i}$ due to the series connection of the input coil, where the resulting inductance increase per washer loop is multiplied by 4. 
%This effect seems to have a larger effect on $L_{\rm i}$ than the reduction through $L_{\rm str}$. 
Taking all effects into account, the enlargement of the washer loops should lead to additional \qty{0.48}{\nH} per turn, which after multiplying by $n^2$ results in the average input coil inductance $L_{\rm i}=\qty{6.40}{\nH}$ obtained from the two measurements. With InductEx we obtained a simulated value of $L_{\rm i}=\qty{5.6}{\nH}$, which deviates \qty{13}{\percent} and \qty{12}{\percent} from $L_{\rm i}^{\rm 2C14}$ and $L_{\rm i}^{\rm 2C23}$, respectively. These moderate deviations are in accordance with previous inductance simulations in several works of this working group \cite{Ferring2015, Bauer2022}. For the following discussions, we will use the average $L_{\rm i}=\qty{6.40}{\nH}$ of the two measurements.

%We now have all parameters to determine the coupling constant $k_{\rm is}$. Using equation \ref{kis_theo}, we obtain $k_{\rm is}=0.73$, which is close to the target value 0.75 that has been set as a realistic upper limit for the minimization of the extrinsic energy sensitivity $\epsilon_{\rm p}$. This suggests that the reduction of the input coil current sensitivity $M_{\rm is}$ mostly stems from the small measured SQUID loop inductance $L_{\rm s}$. It is noteworthy, however, that the variance of both the measured critical currents and the current swings was rather large with deviating up to \qty{11}{\percent} and \qty{15}{\percent}, respectively. This of course provides an uncertainty for $\beta_{\rm L}$ and $L_{\rm s}$ that needs to be taken into account. Such variances have also been observed in \cite{Bauer2022}, where it was shown that SQUIDs with \textit{cross-type} junctions exhibit a smaller variance across the wafer, making them more reliable than window-type junctions regarding $I_{\rm c}$. \\    
An overview of the most relevant measured parameters compared to their respective design values is given by table \ref{tab:SQUIDparameters}, showing an overall good agreement.

\begin{table}[htb]
	\centering
	\begin{tabular}{c|*{9}{c}}
	\toprule
		Parameter & $R_{\rm s}$ & $I_{\rm c}$ & $M_{\rm is}$ & $M_{\rm fs}$ & $L_{\rm s}$ & $L_{\rm i}$ & $\beta_{\rm L}$ & $\beta_{\rm C}$ & $k_{\rm is}$ \\
		 & $[\Omega]$ & $[\unit{\micro\ampere}]$ & $[\unit{\pH}]$ & $[\unit{\pH}]$ & $[\unit{\pH}]$ & $[\unit{\nH}]$ &  &  &  \\
		\midrule
		Measured & 6 & 5.84 & 611 & 51 & 108 & 6.40 & 0.65 & 0.61 & 0.73 \\
		Design & 6 & 6 & - & 50 & 147 & 6.56 & 0.86 & 0.62 & 0.75 \\
	\end{tabular}
	\caption{Summary of measured characteristic parameters of the new dc-SQUID design with a two-turn input coil, which are compared with the corresponding target values.}
	\label{tab:SQUIDparameters}
\end{table}

\section{Resonance Behavior}\label{sec_resonance_results}

In this section we will investigate how various \textit{LC} resonances might affect the IVCs of the SQUIDs tested in the previous section and whether the applied damping techniques proved to be effective or not. Figure \ref{abb:all_IV_merge} showcases the IVCs of 8 distinct SQUIDs from the 10 chosen for the characterization. The order from top to bottom corresponds to the order shown in the inductively damped chip with various SQUID types in figure \ref{abb:damped_chip}. These damped SQUIDs are on the right side (e) through h)), while the corresponding SQUID types without inductive damping are on the left side (a) through d)) in the same order. There are a few differences between certain SQUID variants, however, they collectively share a prominent feature at $V_{\rm s}\approx\qty{40}{\uV}$. This artifact shows a pronounced current step with corresponding intersections between the IV curves. The fact that it is also present in the SQUIDs without input coil (d) and h)) rules out any input coil related resonances to be the underlying cause. Regarding the feedback coil, due to the small serially connected inductance its resonance $f_{L_{\rm f}C_{\rm p}}$ lies far above the operation frequency, as we will see in the following. Consequently, we attribute the pronounced current step at approximately \qty{40}{\uV} to the fundamental SQUID resonance $f_{L_{\rm s}C}$, which by using equation \ref{fund_res} is calculated to $f_{L_{\rm s}C}=\qty{22.2}{\GHz}=\qty{46}{\uV\per\fq}$, using the measured values of $L_{\rm s}=\qty{108}{\pH}$ and $C=\qty{0.95}{\pF}$. This value does not exactly correspond to the observed \qty{40}{\uV}, which would indicate an underestimation of either the SQUID loop inductance $L_{\rm s}$, junction capacitance $C$ or both. As discussed in section \ref{sec_charac}, the parasitic capacitance arising from the insulation window structure of the junction is typically neglected, i.e. $C=C'$. The real junction capacitance should therefore be slightly larger, leading to a decreased resonance frequency $f_{L_{\rm s}C}$. The measured inductance $L_{\rm s}$ can also deviate from the real value, which could be explained by the asymmetric $I\Phi$ characteristic as this aspect has not been included in the numerical simulation for $\Delta I(\beta_{\rm L})$ in \cite{Tesche1977}. Additionally, the error of $L_{\rm s}$ is dominated by the measurement uncertainty regarding the maximal current swing. Further measurements are needed to increase the accuracy of the approximated value.\\

\figurecenter {p}
%{scale=0.25}
%{width=\textwidth}
{height=0.9\textheight}
{../Figures/all_IV/all_IV_mergeV2}
{0cm}
{a) - d): Measured IVCs of the SQUID types presented in figure \ref{abb:damped_chip} in the same order, but without inductive damping. e) - h): The same measurement for the respective SQUID types but with inductive damping. Measured at $T=\qty{10}{\milli\kelvin}$.} 
{all_IV_merge}

The comparison of each side of figure \ref{abb:all_IV_merge} demonstrates a significant smoothening of all IVCs of the inductively damped SQUIDs, as compared to their non-damped counterparts. We can therefore conclude that damping with insulated gold pads proved to effectively suppress all visible resonances to some extent, regardless of the SQUID design. Consequently, one would expect the resulting intrinsic flux noise of the damped SQUIDs to be lower. This will be investigated in section \ref{sec_noise_results}. Whether the various step structures apart from the large step at \qty{40}{\uV} can be assigned to the different microstrip and \textit{LC} resonances described in subsection \ref{subsec_para_res} will be the subject of the following discussion. \\

For the $C_{\rm p}$-related resonances, we need to take into account that the SQUID loops are connected in parallel and the input coil in series. The parasitic capacitance $C_{\rm p}^{\rm loop}$ per washer loop will therefore need to be multiplied by a factor of 4 or $1/4$ to obtain the total parasitic capacitance $C_{\rm p}$. A theoretical value has been calculated to \cite{EnpukuI1991, EnpukuIII1992}    
  
\gl{
C_{\rm p}^{\rm loop}=\frac{l_{\rm i,1/4}C_{\rm str}}{8}=\frac{l_{\rm i,1/4}}{8}\frac{\epsilon_0\epsilon_{\rm r}w_{\rm i}K_{\rm f}(w_{\rm i},t_{\rm i},h)}{h} \ \ ,
}{}

where $l_{\rm i,1/4}$ denotes the input coil length on a single washer loop and $C_{\rm str}$
the microstrip line capacitance per unit length. The latter depends on the vacuum and relative permittivity $\epsilon_{\rm 0}$ and $\epsilon_{\rm r}$, as well as the width of the input coil $w_{\rm i}$, the fringing factor $K_{\rm f}(w_{\rm i},t_{\rm i},h)$ and the insulation thickness \textit{h}. The fringing factor also depends on $w_{\rm i}$, \textit{h} and the thickness of the input coil denoted as $t_{\rm i}$ \cite{Chang1979}. With $h=\qty{375}{\nm}$, $w_{\rm i}=\qty{3}{\um}$ and a sputtered Nb2 layer with $t_{\rm i}=\qty{400}{\nm}$ we obtain the value $K_{\rm f}=1.45$. In case of the lossy SQUID variants, the niobium thickness of the input coil only yielded $t_{\rm i}=\qty{200}{\nm}$, leading to $K_{\rm f}=1.42$. This has a small effect on the resulting resonance frequencies calculated below, as they only deviate approximately \qty{2}{\percent} from the respective frequencies obtained for the non-lossy type. For the sake of clarity, we will therefore only consider the non-lossy SQUID in the following discussion, which will qualitatively represent the lossy type as well. \\ With a fringing factor of $K_{\rm f}=1.45$, we obtain a microstrip inductance of $C_{\rm str}=\qty{0.4}{\nF}$. The input coil length per washer loop is approximately given by $l_{\rm i,1/4}\approx l_{\rm i}/4\approx\qty{860}{\um}$, with the total length of the input coil being $l_{\rm i}\approx\qty{3.44}{\mm}$. Consequently, we obtain for the parasitic capacitance per washer loop $C_{\rm p}^{\rm loop}=\qty{0.043}{\pF}$. The total parasitic capacitance regarding the washer is therefore $C_{\rm p}=4C_{\rm p}^{\rm loop}=\qty{0.17}{\pF}$, which is still small compared to the junction capacitance $C=\qty{0.95}{\pF}$, i.e. $C_{\rm p}/C\ll 2$. The influence of this parameter on the energy sensitivity should consequently be small, as explained in subsection \ref{subsec_para_res}. The $L_{\rm s}C_{\rm p}$ resonance can now be calculated to $f_{L_{\rm s}C_{\rm p}}=1/(2\pi\sqrt{L_{\rm s}4C_{\rm p}^{\rm loop}})=\qty{34}{\GHz}$, where we used the measured SQUID loop inductance $L_{\rm s}=\qty{108}{\pH}$. This corresponds to a voltage drop of $V_{\rm s}=\qty{70.2}{\uV}$ for the condition $f_{\rm J}=f_{L_{\rm s}C_{\rm p}}$. This is well above the optimal operation frequency given by $f_{\rm op}\approx 0.3f_{\rm c}\approx\qty{5.1}{\GHz}$ ($V_{\rm s}=\qty{10.5}{\uV}$), with $f_{\rm c}=I_{\rm c}R_{\rm s}/\unit{\fq}$ \cite{Cantor1996}. Mostly the plots a) through c) from figure \ref{abb:all_IV_merge} showcase small distinctive step-like structures around \qty{75}{\uV}. These could, however, also stem from a higher harmonic of the fundamental SQUID resonance, which should be the case for the SQUID without input coil (a)), as there is no corresponding parasitic capacitance. \\

For the $f_{L_{\rm i}C_{\rm p}}$ resonance we need to take the $R_{\rm x}C_{\rm x}$ attenuator into account. The capacitances $C_{\rm x}$ and $C_{\rm p}$ can approximately be added up to the total capacitance $C_{\rm tot}=C_{\rm x}+\frac{C_{\rm p}^{\rm loop}}{4}$ due to the serial connection of the input coil. The geometric input coil inductance is shielded analogously to $L_{\rm s}$ (compare equation \ref{eff_Ls}), giving the relation $L_{\rm i}'=(1-k_{\rm is}^2s_{\rm s})L_{\rm i}$ \cite{Cantor1996}. Here, the screening factor was derived to $s_{\rm s}=\frac{\beta_{\rm L}s_{\rm i}k_{\rm is}^2}{6+2\beta_{\rm L}+\beta_{\rm L}s_{\rm i}k_{\rm is}^2}$. The input inductance and pickup coil inductance also form a parallel connection, resulting in $L_{\rm tot}=\frac{L_{\rm i}'L_{\rm p}}{L_{\rm i}'+L_{\rm p}}$. The resonance frequency then reads 

\gl{
f_{L_{\rm i}C_{\rm p}}\approx\frac{1}{2\pi\sqrt{C_{\rm tot}L_{\rm tot}}} \ \ ,
}{}

which by using our measured and calculated parameters yields \qty{0.9}{\GHz}. The corresponding voltage drop is $V_{\rm s}=\qty{1.8}{\uV}$, which is well below the above-mentioned operation voltage and is therefore not visible in the IVCs from figure \ref{abb:all_IV_merge}. The capacitance $C_{\rm x}$ is chosen small to minimze the attributed $Q_{L_{\rm i}C_{\rm p}}$ value, which for an RCL parallel circuit is given by 

\gl{
Q_{L_{\rm i}C_{\rm p}} \approx R_{\rm x}\sqrt{\frac{C_{\rm tot}}{L_{\rm tot}}} \approx \frac{R_{\rm x}}{f_{L_{\rm i}C_{\rm p}}L_{\rm tot}} \ \ .
}{Q}

%\gl{
%Q_{L_{\rm i}C_{\rm p}} = R_{\rm x}\sqrt{\frac{(C_{\rm x}+C_{\rm p}/4)(L_{\rm i}'+L_{\rm p})}{L_{\rm i}'L_{\rm p}}}
%}{}

Evidently, $C_{\rm x}$ cannot be chosen arbitrarily small as it would shift the resonance to the vicinity of the working point. An optimal value has been found to be $Q_{L_{\rm i}C_{\rm p}}\approx 2$ \cite{Cantor1996}. The resistive componenent of the attenuator is typically dimensioned such that it corresponds to the nominal impedance $Z_0$ of the microstrip line in order to mitigate wave reflections occuring where the input coil leaves the SQUID loop. This impedance can be expressed as \cite{EnpukuI1991} 

\gl{
Z_{\rm 0}=\sqrt{\frac{L_{\rm str}}{C_{\rm str}}} \ \ .
}{} 

With the produced geometric proportions of the microstrip, the optimal value yields $R_{\rm x}=\qty{20.6}{\ohm}$, which was chosen as the design value. Therefore, with equation \ref{Q} we obtain an optimal capacitance of $C_{\rm x}=\qty{31}{\pF}$. Unfortunately, such a large capacitance has not yet been implemented into the new design. We adapted the lower value of the previous design (\qty{10}{\pF}), which had been chosen in \cite{Bauer2022} with the premise to work with the pickup coil of the ECHo-100k detector ($L_{\rm p}=\qty{1.14}{\nH}$). In our case we obtain $Q\approx 1.13$, however, the corresponding resonance frequency is still small enough to consider the effect of this resonance as negligible. \\
We can also further neglect the $C_{\rm p}$-related resonance with respect to the feedback coil, as its small design inductance of $L_{\rm f}=\qty{336}{\pH}$ leads to a resonance frequency of $f_{L_{\rm f}C_{\rm p}}=\frac{\qty{160}{\uV}}{\unit{\fq}}=\qty{77}{\GHz}$, well above $f_{\rm op}$. 
\\

Regarding the $\lambda/2$ resonances a distinction is made whether the inpt coil or the SQUID loop act as the signal carrying line. They will consequently be referred to as the input coil or washer resonance with the resonance frequencies $f_{l_{\rm i}}$ or $f_{l_{\rm w}}$, respectively. These should ideally fulfill the condition $4f_{l_{\rm i}}<f_{\rm op}<\frac{f_{l_{\rm w}}}{4}$ to mitigate their noise inducing influence at the working point \cite{Can1991}. The wave propagation velocity $c_{\rm str}$ depends on $L_{\rm str}$ and $C_{\rm str}$, such that equation \ref{stripline_res_general} becomes \cite{EnpukuIII1992}        

\gl{
f_{l_{\rm i}}=\frac{c_{\rm str}}{2l_{\rm i}}=\frac{1}{2l_{\rm i}\sqrt{L_{\rm str}C_{\rm str}}} \ \ .
}{}

With the length of the input coil $l_{\rm i}$, we obtain for the input coil resonance $f_{l_{\rm i}}=\frac{\qty{36.4}{\uV}}{\unit{\fq}}=\qty{17.6}{\GHz}$. However, we need to include the length $l_{\rm i}^{\rm feed}\approx \qty{620}{\um}$ of the input coil feed lines which are also structured as a microstrip transmission line. This results in $\qty{14.9}{\GHz}$ and accordingly $V_{\rm s}=\qty{30.9}{\uV}$. Even though the condition $4f_{l_{\rm i}}<f_{\rm op}$ is not fulfilled, there are no visible step structures in the vicinity of this frequency, which suggests successful damping of this microstrip resonance. The effectiveness of the $R_{\rm x}C_{\rm x}$ attenuator has also been demonstrated in \cite{Bauer2022}, where drastic step structures at $f_{l_{\rm i}}=\frac{\qty{54}{\uV}}{\unit{\fq}}$ were eliminated entirely after adding the $R_{\rm x}C_{\rm x}$ component to the SQUID design with $L_{\rm i}=\qty{1.27}{\nH}$. \\
The washer resonance is more difficult to determine for the given SQUID loop design, as the complex washer structure surrounding the junction area affects the effective length $l_{\rm w}$ of the signal carrying line. A crude approximation can be done by adding the SQUID loop feed line length to the circumference of a single washer loop due to the parallel connection to estimate $l_{\rm w}$. We thus obtain $f_{l_{\rm w}}\approx\frac{\qty{50.3}{\uV}}{\unit{\fq}}=\qty{26.9}{\GHz}$, which is more than four times the operation frequency. Just like for $f_{l_{\rm i}}$, there aren't any artifacts in the SQUID IVCs that could be assigned to this resonance. This indicates that the washer shunt exhibits good damping properties, as has already been demonstrated by \cite{Richter2017} and \cite{Bauer2018}. \\

Another prominent step structure seen in the IVCs of figure \ref{abb:all_IV_merge} occurs at approximately \qty{15}{\uV} for the extremal curve with the smallest critical current $I_{\rm c,1}$. This feature is absent in the IVCs of the SQUIDs without input coil (a) and e)) as well as the previous design with a single-turn input coil (not shown here). Consequently, the second turn seems to cause an additional resonance effect thas has not yet been identified. It is also unclear why this artifact does not appear in plot d) of figure \ref{abb:all_IV_merge}. 
\\
Interestingly, there are no considerable differences between the characteristics of lossy and non-lossy variants. It is, however, worth noting that the input coils were not shorted or connected to a pickup coil during the experiments described in this chapter, which might be accompanied with an altered resonance behavior. The resonance as well as the noise properties of the new SQUID design with a shorted input coil should therefore be investigated in future works.

%\footnote{Hier erwähnen, dass die input coils alle nicht angeschlossen waren? Dass sie kurz zu schließen oder an eine andere Induktivität dran hängen (zb Lp) evtl ein anderes Resonanzverhalten verursachen könnte?}
 
%Discuss in more detail all types of resonances with formulae and calculate the frequencies (there are still two of them that are unclear/need to be discussed in private or in a meeting)

%\textit{Figures}: All measured FE IVCs, like non-lossy, lossy, damping variants, iso, no-iso, no input coil, Fabiennes FE... (should I also show the corresponding VPhi curves?)

%Try to identify the resonances in the plots. Discuss what actually helped damping and what not. 

%Possible reason for LsC being too low: betaL theo adapted for symmetric squids, might need correction for asymm. -> could mean that betaL is too small, and therefore Ls. Would make sense as we did not expect Ls to be \textit{that} small, also, InductEX seems to underestimate inductances and it simulated Ls=106pH, favoring a larger "real" Ls.

\section{Noise Performance} \label{sec_noise_results}

%\subsection{Lumped Element Two-Stage Setup}

The noise measurement was conducted at approximately $\qty{10}{\milli\kelvin}$ in the dilution cryostat, using the two-stage setup described in subsection \ref{subsec_2stage_theo}, where the second stage SQUID is provided by a separate SQUID array chip. Figure \ref{abb:noise_meas} depicts the two-stage noise measurement of two lossy SQUIDs with ${\rm SiO_2}$ inside of the SQUID loop, one without and one with inductive damping from the chips 3A13 and 2A16, respectively. In the following we will use the name of the front-end chip to represent the corresponding two-stage setup. The flux noise $\sqrt{S_\Phi}$ is given as the square root of the power spectral density in and is plotted against the frequency, ranging from aproximately \qty{30}{\milli\hertz} to \qty{25}{\kHz}. The measured data is shown as black dots and consists of a frequency-independent white noise and a low frequency $1/f$ noise component, as described by equation \ref{noise_components}. The fit of the total apparent flux noise are represented by a blue curve. The corresponding fit values are shown in table \ref{}, yielding for 3A13 $\sqrt{S_{\rm \Phi,w}}=\qty{0.41}{\micro\fq\per\sqrthz}$ and $\sqrt{S_{{\rm \Phi},1/f}}(\qty{1}{\Hz})=\qty{3.9}{\micro\fq\per\sqrthz}$ with an exponent of $\alpha=0.87$, whereas the fit for 2A16 provided $\sqrt{S_{\rm \Phi,w}}=\qty{0.57}{\micro\fq\per\sqrthz}$, $\sqrt{S_{{\rm \Phi},1/f}}(\qty{1}{\Hz})=\qty{4.5}{\micro\fq\per\sqrthz}$ and $\alpha=0.84$.

\begin{table}[h!]
	\centering
	\begin{tabular}{c|*{7}{c}}
		\toprule
		Parameter & $\sqrt{S_{\rm \Phi_{\rm s},w}}$ & $\sqrt{S_{\Phi_{\rm s},1/f}}(\qty{1}{\Hz})$ & $\alpha$ & $\epsilon_{\rm s,w}$ & $\epsilon_{{\rm s},1/f}$ & $\epsilon_{\rm p,w}$ & $\epsilon_{{\rm p},1/f}$ \\
		& $[\unit[per-mode=symbol]{\micro\fq\per\sqrthz}]$ & $[\unit[per-mode=symbol]{\micro\fq\per\sqrthz}]$ &  & [$h$] & [$h$] & [$h$] & [$h$] \\
		\midrule
		3A13 (L) & 0.23 & 3.52 & 0.85 & 1.58 & 369.3 & 25.26 & 5916 \\
		2A16 (LD) & 0.30 & 3.54 & 0.82 & 2.68 & 373.5 & 43.0 & 5983 \\
		int. chip (NL) & 0.42 & 2.0 & 0.60 & 5.26 & 119.2 & 84.2 & 1910 \\
		2A02 (L) & 0.22 & 3.0 & 0.85 & 1.44 & 268.2 & 23.11 & 4297 \\
		transf. (CJJ) & 0.38 & 3.33 & 0.70 & 3.45 & 265.0 & 114.4 & 8784 \\
		%transf. (WJJ) & 0.98 & 2.64 & 0.66 & 22.95 & 166.6 & 760.8 & 5521 \\
	\end{tabular}
	\caption{Overview of the determined noise and energy sensitivities of the SQUIDs described in this section. The acronyms stand for lossy (L), lossy and inductively damped (LD), non-lossy (NL), and cross-type Josephson junctions (CJJ). The latter represents the \qty{6}{\nH} double flux transformer SQUIDs measured in \cite{Bauer2022}.}
	%\vspace{12in}
	\label{tab:noise}
\end{table}
 
These values are, apart from the white noise of 3A13, relatively large which stems from larger contributions from the first and second stage SQUID, as discussed in the following. Both plots additionally contain the individual noise contributions from equation \ref{total_apparent_fluxnoise_2stage}. To disentangle contributions it is necessary to also perform a noise measurement of the isolated SQUID array within the two-stage setup, which is achieved by driving no currents through either the front-end or its coupled coils. The noise measured by this method corresponds to the apparent flux noise in the array, which is calculated by equation  \ref{total_apparent_fluxnoise_2stage} for $G_{\rm \Phi}=1$, $S_{\rm \Phi_s}=0$ and $R_{\rm dyn}=0$. For the array chip in this operating mode we assumed a temperature of $T=\qty{150}{\milli\kelvin}$, which is caused by self-heating.
\twofigurescenter{t!}
{width=0.48\textwidth}
{../Figures/3a13_noise}
{0.02\textwidth} %hspace
{width=0.48\textwidth}
{../Figures/2a16_noise}
{0.5cm} %vspace
{Measured noise spectra at $T=\qty{10}{\milli\kelvin}$ using a two-stage setup, where the low temperature amplifier is provided by a separate SQUID array chip. The blue curve represents the numerical fit of the measured total noise, whereas the remaining curves are the individual contributions of the first- and second-stage SQUID as well as the SQUID electronics consisting of the room temperature amplifier and the current source. Left: Lossy variant with ${\rm SiO_2}$ from the chip 3A13. Right: Lossy variant with ${\rm SiO_2}$, including inductively damped feed lines (2A16).}
{noise_meas}
For the shunt resistors determined in section \ref{sec_charac}, we obtain a sheet resistance of $\qty{1.23}{\ohm\per\Box}$. The gain resistance $R_{\rm g}$ with a length of 0.3 squares thus yields \qty{0.37}{\ohm}. The transfer coefficients $V_{\rm \Phi_x}$ and $I_{\rm \Phi_x}$ are obtained prior to the noise measurement and yield $V_{\rm \Phi_x}=\qty{602.2}{\micro\volt\per\fq}$ and $I_{\rm \Phi_x}=\qty{11.71}{\micro\ampere\per\fq}$ for 3A13, whereas for 2A16 $V_{\rm \Phi_x}=\qty{762.3}{\micro\volt\per\fq}$ and $I_{\rm \Phi_x}=\qty{9.02}{\micro\ampere\per\fq}$. The intrinsic flux noise of the array $S_{\rm \Phi_x}$ can, therefore, be determined by subtracting the noise terms due to the gain resistor, current source and room temperature amplifier from the measured total noise. For the two-stage setup, the array noise contribution is reduced by the flux gain squared, as explained in subsection \ref{subsec_2stage_theo}. For the measurements displayed in figure \ref{abb:noise_meas}, we obtained $G_{\rm \Phi}=2.34$ for the setup with the 3A13 chip and $G_{\rm \Phi}=1.31$ for 2A16. This resulted in an array white noise component of $\sqrt{S_{\rm \Phi,w}}=\qty{0.24}{\micro\fq\per\sqrthz}$ and a $1/f$ component of $\sqrt{S_{{\rm \Phi},1/f}}(\qty{1}{\Hz})=\qty{1.50}{\micro\fq\per\sqrthz}$ for 3A13, whereas with 2A16 we obtained $\sqrt{S_{\rm \Phi,w}}=\qty{0.31}{\micro\fq\per\sqrthz}$ and $\sqrt{S_{{\rm \Phi},1/f}}(\qty{1}{\Hz})=\qty{2.44}{\micro\fq\per\sqrthz}$. The higher noise contribution of the array in the 2A16 setup as well as the lower $G_\Phi$ value can be explained by dephasing of the array. This might be prevented in future experiments with better shielding of the experimental setup. Insufficient shielding allows external magnetic flux to be trapped inside the individual SQUID cells during the cooldown of the cryostat. If this added flux bias varies across the cells, the corresponding $V\Phi$-characteristics of the cells will be out-of-phase, resulting in a deteriorated $V\Phi$-characteristic of the array, which reduces $\Delta V_{\rm max}$ and $V_{\rm \Phi}$. \\

For the two-stage measurement we assumed a larger array chip temperature of $\qty{500}{\milli\kelvin}$ due to the current flowing through $R_{\rm g}$, resulting in considerable more dissipated heat. However, comparing with the array only measurement, the contribution of the gain resistor (yellow curve) is strongly reduced due to $G_{\rm \Phi}>1$ and $R_{\rm dyn}\gg R_{\rm g}$, thus becoming negligible. The noise contribution of the current source for the array flux bias has been calculated with the values given in \cite{Kaap2020} and is shown by the pink curve in figure \ref{abb:noise_meas}, which is also negligible. The green curve represents the room temperature amplifier noise, consisting of the current and voltage noise as described in equation \ref{total_apparent_fluxnoise_2stage}. It is calculated with the values given in \cite{Drung2006} (see subsection \ref{subsec_fll}). In the white noise regime it is comparable to the array contribution (brown curve), while for low frequencies the contribution of array noise is larger. The total noise, however, is dominated by the intrinsic noise of the first-stage SQUID (orange curve), which is obtained analogously to the intrinsic array noise by subtracting all other contributions from the total noise. The corresponding fit according to equation \ref{total_apparent_fluxnoise_2stage} provides the parameters $\sqrt{S_{\rm \Phi,w}}=\qty{0.23}{\micro\fq\per\sqrthz}$ and $\sqrt{S_{{\rm \Phi},1/f}}(\qty{1}{\Hz})=\qty{3.52}{\micro\fq\per\sqrthz}$ with an exponent of $\alpha=0.85$ for the front-end of 3A13, whereas the front-end of 2A16 yielded $\sqrt{S_{\rm \Phi,w}}=\qty{0.30}{\micro\fq\per\sqrthz}$, $\sqrt{S_{{\rm \Phi},1/f}}(\qty{1}{\Hz})=\qty{3.54}{\micro\fq\per\sqrthz}$ and $\alpha=0.82$. The white noise levels of both front-ends are comparable to previous low-noise SQUIDs developed in this working group \cite{Ferring2015}, however, the $1/f$ noise is only slightly larger than the best devices produced in the working group, which are around 2-3 \unit{\micro\fq\per\sqrthz}. Surprisingly, the intrinsic noise of the first-stage SQUID from 3A13 is slightly lower than the inductively damped SQUID from 2A16. The smoother IVCs obtained by the inductive damping scheme should have a positive effect on the flux noise level. The extent to which this effect occurs is, however, unclear. The difference of the white noise level between both SQUIDs can also stem from differing SQUID loop inductances $L_{\rm s}$ or varying chip temperatures \textit{T}, as the noise depends on these parameters according to equation \ref{intr_FEnoise_minimize}. It is thus crucial to further investigate the noise spectra of all SQUID variants produced within the scope of this thesis to provide a more conclusive comparison.

Two other noise measurements involving the new SQUID design, but with different setups, were conducted in this working group and are now briefly compared to the aforementioned results. The first setup involves the integrated two-stage chip HDSQ16-W1-3A15 developed and tested in the work of \cite{Kraemer2023}. This chip contains a modified version of the array that was used for our two-stage measurements, which exhibits 18 SQUID cells instead of 16. Located next to it is the first-stage SQUID, where the non-lossy variant described in section \ref{sec_FEdesign} has been chosen. The noise measurement was performed with the same two-stage configuration as described above. All noise contributions were determined the same way as well, which resulted in $\sqrt{S_{\rm \Phi,w}}=\qty{0.42}{\micro\fq\per\sqrthz}$, $\sqrt{S_{{\rm \Phi},1/f}}(\qty{1}{\Hz})=\qty{2.0}{\micro\fq\per\sqrthz}$ and $\alpha=0.60$ for the  intrinsic noise of the front-end. The larger white noise contribution could be explained by the increased self-heating due to the proximity of the array to the front-end. Interestingly, in contrast, the low-frequency noise and the corresponding $1/f$ exponent are reduced to values, which are comparable with the best SQUIDs produced in this working group. Further noise measurements of the non-lossy variant will allow to assess whether these values are attributed to the integrated two-stage setup, the Front-End design itself or just represent a statistical variation. \\
Lastly, a different type of noise measurement has been conducted by \cite{Mazibrada2024}, which involved a cross-correlation setup at \qty{10}{\milli\kelvin} in the cryostat. The setup utilizes two array channels of a single array chip that are both connected to the front-end via their input coils \cite{Herbst2023}. As this method allows to cancel out the array noise, it is possible to directly obtain the intrinsic noise of the detector SQUID. The measured front-end from the chip 2A02 is of the same type as the one from 3A13 used for our two-stage setup, i.e. lossy with ${\rm SiO_2}$ inside of the SQUID loop. The noise was determined to be $\sqrt{S_{\rm \Phi,w}}=\qty{0.22}{\micro\fq\per\sqrthz}$, $\sqrt{S_{{\rm \Phi},1/f}}(\qty{1}{\Hz})=\qty{3.0}{\micro\fq\per\sqrthz}$ and $\alpha=0.85$, which is comparable to the values for the SQUID from 3A13. This measurement was part of a series of temperature-dependent noise measurements, which allowed to determine the minimum temperature $T_{\rm min}$ of the front-end chip. The thermal decoupling of the SQUID from the cryostat occurred at $T_{\rm min}\approx\qty{90}{\milli\kelvin}$, which we will use for the following discussion regarding the energy sensitivity. \\

\section{Energy Sensitivity}

With the obtained flux noise we are able to determine the intrinsic and extrinsic energy sensitivities introduced in subsections \ref{subsec_noise_theo} and \ref{subsec_extr_sens_theo}. The theoretical values are obtained by using equation \ref{intr_FEnoise_minimize} for the theoretical intrinsic white noise of the SQUID. For the sake of clarity, they will be compared only to the noise measurement of 3A13, whereas the values for the other chips are summarized in table \ref{tab:noise}. With the measured parameters discussed in section \ref{sec_charac} and for the assumed chip temperature $T\approx \qty{90}{\milli\kelvin}$ we get $\sqrt{S_{\rm \Phi_{\rm s},w}^{\rm theo}}=\qty{0.11}{\micro\fq\per\sqrthz}$, which compared to the measured values is reduced by a factor of 2 to 4. This could be explained by added noise that is mixed down due to the resonance-induced hysteretic behavior shown in the IVCs. Additionally, the current noise from the washer shunt is not taken into account in equation \ref{intr_FEnoise_minimize}. The measured intrinsic energy sensitivity $\epsilon_{\rm s,w}$ will therefore deviate accordingly, yielding $\epsilon_{\rm s,w}=1.58\, h$, whereas the theoretical value is $\epsilon_{\rm s,w}^{\rm theo}=0.36\, h$. The measured value of $\epsilon_{\rm s,w}$ is surprisingly low compared to typical values achieved in this working group, however, for the \textit{1/f} noise the value of $\epsilon_{{\rm s},1/f}=369.3\, h$ is rather large. The flux-to-flux coupling $\Delta\Phi_{\rm s}/\Delta\Phi_{\rm p}$ regarding a single pickup coil with $L_{\rm p}=L_{\rm m}/2=\qty{6.65}{\nH}$ yields \qty{4.51}{\percent}, using the obtained values $M_{\rm is}=\qty{611}{\pH}$ and  $L_{\rm i}=\qty{6.4}{\nH}$. If we consider the gradiometric geometry of the pickup coil of the maXs100 detector, the coupling is reduced by a factor of 2 to $\Delta\Phi_{\rm s}/\Delta\Phi_{\rm }=\qty{2.25}{\percent}$ (compare subsection \ref{subsec_extr_sens_theo}). This represents an increase over the value obtained in \cite{Bauer2022}, where the use of a double flux transformer even deteriorated the coupling from \qty{1.95}{\percent} to \qty{1.75}{\percent}. In \cite{Bauer2022}, the noise with the \qty{6}{\nH} double flux transformer setup with the previous design using both window-type and cross-type junctions, had been measured using the cross-correlation method. The window-type variant exhibited an unexplained increase of the targeted critical current by a factor of 2, which led to strong hysteretic behavior and higher noise. We will therefore restrict the comparison to the cross-type variant, which is labeled 'SQ-CJJ-4w1 Chip10 $\#$1'. The measured noise yields $\sqrt{S_{\rm \Phi_{\rm s},w}}=\qty{0.38}{\micro\fq\per\sqrthz}$ for the white noise, $\sqrt{S_{\Phi_{\rm s},1/f}}(\qty{1}{\Hz})=\qty{3.33}{\micro\fq\per\sqrthz}$ for the $1/f$ component and $\alpha$ = 0.70. The measured white noise of the new design therefore represents an improvement, which together with the coupling gain will result in an overall smaller extrinsic energy sensitivity. Just like with $\epsilon_{\rm s,w}$, the measured extrinsic energy sensitivity exceeds the predicted value, where we obtain $\epsilon_{\rm p,w}^{\rm theo}=5.71\, h$ and $\epsilon_{\rm p,w}=25.26\, h$ by using equation \ref{extr_energy_sens}. The \textit{1/f} contribution results accordingly also in a large value of $\epsilon_{{\rm p},1/f}=5916\, h$. These values, however, represent a significant reduction as compared to the previous design with an intermediary coupling transformer. Given the aforementioned parameters, the cross-type version yields $\epsilon_{\rm p,w}=114.4\, h$ and $\epsilon_{{\rm p},1/f}=8784\, h$. \\

To further improve the noise level one could adapt cross-type junctions into the new SQUID design, which allows the production of up to 4 times smaller junction areas in our own cleanroom. Consequently, the junction capacitance is reduced by a factor of 4, allowing the use of a shunt resistance twice as large without altering $\beta_{\rm C}$. According to equation \ref{intr_FEnoise_minimize}, this reduces the noise by a factor of 2. Additionally, the SQUID loop inductance $L_{\rm s}$ should be adjusted to the target value of $\qty{147}{\pH}$. This can be implemented by again increasing the size of each washer loop, which would in turn also shift the input coil inductance closer to the optimal value of $L_{\rm i}=L_{\rm p}+L_{\rm par}=7.15$. This would lead to an increase of the mutual inductance $M_{\rm is}$ and thus the flux-to-flux coupling $\Delta\Phi_{\rm s}/\Delta\Phi_{\rm }$, thereby further reducing $\epsilon_{\rm p}$.               

\begin{table}[t!]
	\centering
	\begin{tabular}{c|*{7}{c}}
	\toprule
		Parameter & $\sqrt{S_{\rm \Phi_{\rm s},w}}$ & $\sqrt{S_{\Phi_{\rm s},1/f}}(\qty{1}{\Hz})$ & $\alpha$ & $\epsilon_{\rm s,w}$ & $\epsilon_{{\rm s},1/f}$ & $\epsilon_{\rm p,w}$ & $\epsilon_{{\rm p},1/f}$ \\
		 & $[\unit[per-mode=symbol]{\micro\fq\per\sqrthz}]$ & $[\unit[per-mode=symbol]{\micro\fq\per\sqrthz}]$ &  & [$h$] & [$h$] & [$h$] & [$h$] \\
		\midrule
		3A13 (L) & 0.23 & 3.52 & 0.85 & 1.58 & 369.3 & 25.26 & 5916 \\
		2A16 (LD) & 0.30 & 3.54 & 0.82 & 2.68 & 373.5 & 43.0 & 5983 \\
		int. chip (NL) & 0.42 & 2.0 & 0.60 & 5.26 & 119.2 & 84.2 & 1910 \\
		2A02 (L) & 0.22 & 3.0 & 0.85 & 1.44 & 268.2 & 23.11 & 4297 \\
		transf. (CJJ) & 0.38 & 3.33 & 0.70 & 3.45 & 265.0 & 114.4 & 8784 \\
		%transf. (WJJ) & 0.98 & 2.64 & 0.66 & 22.95 & 166.6 & 760.8 & 5521 \\
	\end{tabular}
	\caption{Overview of the determined noise and energy sensitivities of the SQUIDs described in this section. The acronyms stand for lossy (L), lossy and inductively damped (LD), non-lossy (NL), and cross-type Josephson junctions (CJJ). The latter represents the \qty{6}{\nH} double flux transformer SQUIDs measured in \cite{Bauer2022}.}
	%\vspace{12in}
	\label{tab:noise}
\end{table}

% energy sens. !

%Table with noise results?

%Compare noise levels with cross type noise? -> da aber nur eine windung und kein gold.

%Discuss all contributions, especially from the FE -> Difference if input coil is shorted or not?

%Mention how it would change with a cross JJ SQUID (lower white noise) -> here or in the summary?

%Are 2stage VPhi curves interesting or do we restrict ourselves to noise plots?

%\subsection{Integrated Two-Stage Setup}

%\textit{Figure}: Int. 2stage noise measurement

%Summarize Fabians measurements and compare with ours. 

