\chapter{Theoretical Background} \label{ch_theo}

This chapter provides a short introduction into Josephson junctions and their role in dc-SQUIDs (\textbf{d}irect \textbf{c}urrent \textbf{S}uperconducting \textbf{QU}antum \textbf{I}nterference \textbf{D}evice), which will be the main focus of this thesis. We start with a brief overview on macroscopic quantum phenomena such as the Josephson effect and explain the general working principle of superconductor-isolator-superconductor (SIS) tunnel contacts, followed by a summary of their basic properties. This theoretical description of Josephson tunnel junctions enables us to form the corresponding framework for SQUIDs, which are developed in this group and optimized within the scope of this thesis. Lastly, we will take a closer look into their parasitic resonance behavior and investigate different methods to mitigate the quality factors. We will closely follow the derivations from the textbooks \cite{Clarke2004} and \cite{Gross2016}.

\section{Josephson junctions}


The \textit{Josephson junction}, named after Brian D. Josephson, consists of two identical superconductors weakly coupled to each other. In the case of the junctions produced in this working group, such coupling is realized through a few nm of a thin insulating layer between the superconducting electrodes. Consequently, they are also referred to as SIS junctions. This is realized with a trilayer structure whereby in the framework of this thesis we use a Nb/Al-Al$\rm O_x$/Nb trilyer, with niobium being used for the superconductors and the insulating layer being provided by the aluminum oxide. The schematic structure of a SIS-type junction is shown in figure \ref{abb:fig:JJschem}. %When the junction is maintained at cold temperatures ($T\leq \qty{4}{\kelvin}$) and connected to a current source a supercurrent is measurable.
By connecting the tunnel junction to a current source they exhibit a non-trivial current-voltage behavior, which will be covered in the following. 

%$\theta$ %$\textbf{\textit{\theta}_\textbf{1}}$ and  $\textbf{\textit{\theta}_\textbf{2}}$ represent the macroscopic phases of each superconductor.}
        
\subsection{Josephson effect}\label{subchap_Jeffect}

According to the BCS theory developed by Bardeen, Cooper and Schrieffer in 1957 \cite{Bardeen1957}, electrons in a superconductor form pairs below a material dependent critical temperature $T_{\rm c}$. These composite particles are also referred to as \textit{Cooper pairs} and they represent the superconducting charge carriers with twice the mass and charge of a single electron. Within the framework of the BCS-theory the dissipationless current flow in superconductors can be explained by the formation of the energy gap in the density of states, suppressing the scattering of Cooper pairs and other particles in the solid body. Alongside the supercurrent with zero dc-resistance, the Meissner-Ochsenfeld effect \cite{Meissner1933} is the most characteristic feature of a superconductor. The latter describes the magnetic field expulsion provided the external magnetic field is smaller than a critical field $B_{\rm c}$. A detailed description of the microscopic theory of superconductivity can be found in \cite{deGennes1964}.

\figurecenter {t!}
{width=\textwidth}
{../Figures/jj_schematic}
%{7.8cm}
{0cm}
{Schematic of a Josephson (SIS) junction. Both superconducting electrodes $\textbf{\textit{S}}_\textbf{1}$ and $\textbf{\textit{S}}_\textbf{2}$ are weakly coupled with each other through a thin tunnel barrier \textbf{\textit{I}}. \bm{$\theta_{\rm 1}$} and \bm{$\theta_{\rm 2}$} represent the macroscopic phases of each superconductor.} 
{fig:JJschem}

%If at $T < \qty{4}{\kelvin}$ an external current source is connected to a Nb/Al-Al$\rm O_x$/Nb Josephson junction, a supercurrent will flow despite the tunnel barrier, implying the tunneling of Cooper pairs as niobium is predominantly superconducting at these temperatures ($T_\mathrm{c} = \qty{9.3}{\kelvin}$ \cite{Inaba_1980}). Since the tunneling probability of an individual electron is approximately $p \lesssim \num{e-4}$ \cite{Gross2016}, a much lower probability is to be expected for a Cooper pair consisting of two electrons. However, Josephson predicted that the tunneling behavior of Cooper pairs and individual conduction electrons must be the same. This is justified by the so-called \textit{Macroscopic Quantum Model}, formulated by Fritz London in 1953.
For temperatures below the critical temperature of niobium $T_{\rm c}  = \qty{9.3}{\kelvin}$ \cite{Inaba_1980} Cooper pairs are able to tunnel across a Nb/Al-Al$\rm O_x$/Nb Josephson junction despite the insulating barrier. In fact, their tunneling behavior resembles that of a single electron, which can be motivated by the  \textit{Macroscopic Quantum Model} that was formulated by Fritz London in 1953. This model heavily focuses on the quantum mechanical phase $\theta$ of ensemble of Cooper pair, whose macroscopic nature can be understood by the following arguments. On one hand, the distance between both electrons in a Cooper pair is approximately 10 to \qty{1000}{\nm} which is significantly larger than the spacing between Cooper pairs, resulting in strongly overlapping wave functions. On the other hand, Cooper pairs have to obey Bose-Einstein statistics. Thus, all Cooper pairs share the same ground state, and as a consequence, the energies and temporal evolutions of the phases are equal. These two effects lead to what is known as \textit{phase-lock} \cite{Gross2016}. The phases of neighboring pairs synchronize such that this quantum mechanical property now holds on a macroscopic scale. This gives rise to the phase-dependent macroscopic wave function

\begin{equation}
\Psi(\textbf{r},t) = \Psi_0(\textbf{r},t)e^{i\theta(\textbf{r},t)} \ \ ,
\end{equation}

which describes all charge carriers of a bulk superconductor. Here, the charge carrier density is given by $\left|\Psi_0(\textbf{r},t)\right|^2 = n_{\rm s}$. The phase of the Cooper pair ensemble depends on the time \textit{t} and the position \textbf{r}. As a result of sharing the same phase, both electrons of a Cooper pair consequently possess the same tunneling probability as an individual electron, enabling the supercurrent. This coherence phenomenon is referred to as the \textit{Josephson effect} \cite{Josephson1962}. Another significant consequence of the macroscopic quantum model is flux quantization. 

\figureleft {t!}
{width=\textwidth}
{../Figures/quantized_flux}
{7cm}
{0cm}
{Superconducting ring-shaped cylinder threaded by an external magnetic field. By applying the field at low temperatures, shielding currents arise to expel the field from the superconductor. Upon turning off the external field the shielding currents will remain due to the lack of resistance, causing magnetic flux to be trapped. The dotted blue path \textit{C} is situated at the center of the cylinder wall, which we assume to be current-free due to the London penetration depth $\lambda_{\rm L}$ being much smaller than the thickness of the cylinder wall.} 
{fig:quantflux}

This phenomenon can be explained through the capture of an external magnetic flux within a superconducting cylinder (see figure \ref{abb:fig:quantflux}). The wave function must remain unchanged after circumnavigating the cylinder due to $e^{i\theta} = e^{i\theta + 2\pi n}$. As a result, upon integrating along the current-free center of the cylinder wall (path $C$), the following equation holds for the captured flux \cite{Deaver1961}

\begin{equation}\label{fq}
\Phi = \frac{h}{q_\mathrm{s}}n = \frac{h}{2e}n \equiv \Phi_0n \ \ .
\end{equation}

Here, $n\in\mathbb{N}$ and \unit{\fq} = \qty{2.07e-15}{\tesla\metre\squared} \cite{CODATA2018} represents the so-called magnetic flux quantum. The captured flux is thus quantized, a consequence solely arising from the macroscopic nature of the phase. This quantity plays a crucial role in the theoretical description of Josephson junctions.


The current and voltage behavior in a SIS junction is described by the \textit{Josephson equations}. Crucial to this description is a critical current \textit{$I_\mathrm{c}$}, which marks the boundary between two operational modes; the zero-voltage state and the voltage state. Additionally, due to the macroscopic nature of the superconducting phase factor, the supercurrent $I_\mathrm{s}$ oscillates with the gauge-invariant phase difference between the two macroscopic wavefunctions of the superconducting electrodes $\varphi$, leading to the \textbf{first Josephson equation} \cite{Josephson1965}

\begin{equation}
\label{1.JE}
I_\mathrm{s} = I_\mathrm{c}\sin(\varphi) \ \ .
\end{equation}

The critical current $I_\mathrm{c}$ is proportional to the coupling strength $\kappa$, which describes the overlap of the wave functions $\Psi_1$ and $\Psi_2$ within the insulating layer. The relationship is given by

\begin{equation}
I_\mathrm{c} = \frac{4e\kappa V n_\mathrm{s}}{\hbar} \ \ ,
\end{equation}

where \textit{V} represents the volume of the superconducting electrode and \textit{e} denotes the elementary charge of an electron. We assume that the Cooper pair density $n_\mathrm{s}$ of the two superconductors $S_1$ and $S_2$ is identical, meaning $n_{\mathrm{s}1} = n_{\mathrm{s}2} = n_\mathrm{s}$.

The gauge-invariant phase difference refers to the phases $\theta_1$ and $\theta_2$ of the respective electrodes at the boundary of the insulating layer (positions 1 and 2, see figure \ref{abb:fig:JJschem}). Taking into account possible external electromagnetic fields within the barrier, the general form using the vector potential \textbf{A} is given by \cite{Gross2016}

\begin{equation}
\label{EichInv_Phase}
\varphi(\textbf{r},t) = \theta_2(\textbf{r},t) - \theta_1(\textbf{r},t) - \frac{2\pi}{\Phi_0}\int_{1}^{2}\textbf{A}(\textbf{r},t)\cdot \mathrm{d}\textbf{l} \ \ .
\end{equation}

Assuming a constant supercurrent density $J_\mathrm{s}$ across the junction, taking the time derivative of equation \eqref{EichInv_Phase} yields the \textbf{second Josephson equation} \cite{Josephson1965}

\begin{equation}
\label{2.JE}
\frac{\partial\varphi}{\partial t} = \frac{2\pi}{\Phi_0}V \ \ .
\end{equation}

The first of the two above-mentioned operating modes describes the zero-voltage state, i.e.  $I<I_\mathrm{c}$. Here, the entire injected current is carried by Cooper pairs, so $I=I_\mathrm{s}=\mathrm{const}$. As a result, $\varphi$ is constant over time, which, according to equation \eqref{2.JE}, leads to $V=0$. This is known as the \textit{dc Josephson effect}.

For $I>I_\mathrm{c}$ however, Cooper pairs begin to break up into quasiparticles which carry that portion of the current leading to a voltage drop \textit{V}. According to the second Josephson equation, the phase $\varphi$ becomes time dependent, and after integration one obtains

\begin{equation}
\label{phi(t)}
\varphi = \frac{2\pi}{\unit{\fq}}Vt + \varphi_0 = w_\mathrm{J}t + \varphi_0 \ \ \ \mathrm{with} \ \ \ w_\mathrm{J} = \frac{2\pi}{\unit{\fq}}V \ \ .
\end{equation}

Thus, if we insert equation \eqref{phi(t)} into equation \eqref{1.JE}, we observe that  the current $I_\mathrm{s}$ oscillates with the \textit{Josephson frequency} $\frac{f_\mathrm{J}}{V} = \frac{w_\mathrm{J}}{2\pi V} = \frac{1}{\unit{\fq}} \approx \SI{483.6}{\MHz\per\uV}$. Accordingly, this phenomenon is referred to as the \textit{ac Josephson effect}.



\subsection{Josephson Junctions in a Magnetic Field}\label{subsec_jjmag}

\figurecenter {b!}
{width=0.8\textwidth}
{../Figures/shortjj_mag}
%{7.8cm}
{0cm}
{Short Josephson junction connected to a current source in the presence of an external B-field in y-direction, parallel to the junction area. Inside the electrodes the magnetic field decays exponentially according to the London penetration depths $\lambda_{\rm L,1}$ and $\lambda_{\rm L,2}$, visually shown by the purple color gradient. The closed contour \textit{C} is used to derive expressions for the spatially dependent phase difference $\varphi$ and current density $J_{\rm s}$.} 
{fig:JJMag}

To motivate the structure of a dc-SQUID, it is essential to first investigate the current behavior of an extended Josephson junction in the presence of an external magnetic field. So far, all previous formulae apply for point-like junctions, assuming a spatially constant phase difference $\varphi$ and Josephson current density $J_{\rm s}$ across the junction area. This is not the case for three-dimensional (extended) junctions with a length \textit{L} and width \textit{W}. The \textit{Josephson penetration depth} $\lambda_{\rm J}$ is a quantity used to classify an extended junction as short ($\rm W,L \leq\lambda_{\rm J}$) or long ($\rm W,L \geq\lambda_{\rm J}$) and is defined as \cite{Weihnacht1969} 

\gl{
\lambda_{\rm J} = \sqrt{\frac{\unit{\fq}}{2\pi\unit{\micro_0}J_{\rm c}t_{\rm B}}
} \ \ .}{lamdaJ}

Here, the magnetic thickness is defined as $t_{\rm B} = d + \lambda_{\rm L,1} + \lambda_{\rm L,2}$, where $d$ is the geometric thickness of the isolator and $\lambda_{\rm L}$ the London penetration depth of respective superconducting electrode. It describes how far an external magnetic field penetrates both superconducting electrodes if applied parallel to the junction area, as depicted in figure \ref{abb:fig:JJMag}. The respective London penetration depths are $\lambda_{\rm L,1}$ and $\lambda_{\rm L,2}$ and $J_{\rm c} = \frac{I_{\rm c}}{WL}$ is the critical current density.
This distinction is needed to determine whether the magnetic self-field generated by the supercurrent is negligible in comparison to the external field (short junctions) or not (long junctions). Within the scope of this thesis, we only use short junctions.  

To analyze the current and phase distribution of such a junction we consider the setup shown in figure \ref{abb:fig:JJMag}. A short junction is connected to a current source and is penetrated by an external B-field in y-direction, parallel to the junction area. Now, obtaining an expression for the phase requires a similar approach as the calculation for the quantized flux, where we assumed that the phase changes by $2\pi n$ around a closed loop. Here, we again integrate over a closed contour \textit{C}, with the points $P_{\rm 1}-P_{\rm 4}$ marking the transitions between superconductor and isolator. Using equation \ref{EichInv_Phase}, we find 

\gl{
\frac{\del\varphi}{\del z} = \frac{2\pi}{\unit{\fq}}B_{\rm y}t_{\rm B} \ \ \ \mathrm{and} \ \ \ \frac{\del\varphi}{\del y} = -\frac{2\pi}{\unit{\fq}}B_{\rm z}t_{\rm B} \ \ .
}{phi(z,y)}

The magnetic field points in y-direction only, meaning $\varphi$ will only vary along the z-axis. Integrating the first expression in equation \ref{phi(z,y)} then leads to

\gl{
\varphi(z) = \frac{2\pi}{\unit{\fq}}B_{\rm y}t_{\rm B}z + \varphi_0 \ \ .
}{phi(z)} 

Here, the integration constant $\varphi_0$ represents the phase difference for the case $z=0$. Inserting equation \ref{phi(z)} into the first Josephson equation and using $J_{\rm s} = \frac{I_{\rm s}}{WL}$ gives 

\gl{
J_{\rm s}(y,z,t) = J_{\rm c}(y,z)\sin(kz + \varphi_0) \ \ \ \mathrm{with} \ \ \ k = \frac{2\pi}{\unit{\fq}}B_{\rm y}t_{\rm B} \ \ .
}{Js(y,z,t)}

If we now assume the critical current density $J_{\rm c}$ to be constant across the junction area, we can integrate equation \ref{Js(y,z,t)} to get a flux-dependent maximum Josephson current

\gl{
I_{\rm s}^{\rm m}(\Phi)	= I_{\rm c}\left|\frac{\sin\left(\frac{kL}{2}\right)}{\frac{kL}{2}}\right| = I_{\rm c}\left|\frac{\sin\left(\frac{\pi\Phi}{\unit{\fq}}\right)}{\frac{\pi\Phi}{\unit{\fq}}}\right| \ \ ,
}{Ismax}

where $\Phi = B_{\rm y}t_{\rm B}L$ is the total flux threading through the junction. This expression describes the so-called Fraunhofer diffraction pattern, shown in figure \ref{abb:fig:fraunhofer}. The result resembles the single slit experiment, where the same pattern is found for the light intensity behind the slit. Here, the analogy works by considering the integral of the critical current density $J_{\rm c}$ as a transmission function which is constant inside the junction and zero outside. 

\figureleft {t!}
{width=\textwidth}
{../Figures/fraunhofer}
{9cm}
{1cm}
{Normalized flux-dependent maximum Josephson current $I_{\rm s}^{\rm m}(\Phi)$ showing a Fraunhofer pattern. It modulates with the flux quantum $\unit{\fq}$, peaking at $\Phi=0$ with subsequent maxima at $\Phi=\pm(\frac{3}{2}+n)\unit{\fq}$ with $n\in\mathbb{N}_0$. For $\Phi=\pm(n+1)\unit{\fq}$ the total net current is zero.} 
{fig:fraunhofer}


\subsection{RCSJ Model}\label{subsec_RCSJ}

The Fraunhofer pattern describes the flux-dependent current for the case of $I<I_{\rm c}$, staying in the so-called zero-voltage state. In this regime, only the dc Josephson effect applies as discussed in subsection \ref{subchap_Jeffect}. Switching to the voltage stage, i.e. $I>I_{\rm c}$, Cooper pairs start breaking up into quasiparticles if the electric energy $eV$ exceeds the sum of both electrodes' gap energies $\Delta_1(T) + \Delta_2(T)$ \cite{Bardeen1957}. Consequently, at the \textit{gap-voltage} 

\gl{
V_{\rm g} = \frac{\Delta_1(T)+\Delta_2(T)}{e}
}{Vgap}

quasiparticles start to cross the tunnel barrier resulting in a steep rise of a resistive normal current $I_{\rm n}$. This process also occurs at finite temperatures for $k_{\rm B}T>\Delta_1(T) + \Delta_2(T)$, leading to a reduction of $I_{\rm c}$ as well as $V_{\rm g}$. Under a dc current source, the condition $I=I_{\rm s}+I_{\rm n}$ must be constantly fulfilled. This results in an oscillating normal current and therefore voltage, since $I_{\rm s}$ oscillates with $f_{\rm J}$ according to the ac Josephson effect. According to the second Josephson equation (\ref{2.JE}) the oscillating voltage thus causes the term $\frac{\mathrm{d}\varphi}{\mathrm{d}t}$ to vary sinusoidally, causing both $I_{\rm s}$ and $I_{\rm n}$ and in turn the resulting voltage to oscillate in a complex manner. As a voltage with such a high frequency cannot be measured, only the time-averaged voltage will be considered in the following discussion. \\
Now, further increasing the energy of the quasiparticles ($T>T_{\rm c}$ and/or $V>V_{\rm g}$) leads to a transition into normal-conducting electrons, which exhibit an ohmic dependence. This behavior can be seen in the typical current-voltage-characteristic (IVC) depicted in figure \ref{abb:IVCs}. \\ 

\twofigurescenter {t!}
{width=0.48\textwidth}
{../Figures/HDSQ15w1_2A19_9_windowJJ_IVC}
{0.02\textwidth} %hspace b/w figures
{width=0.48\textwidth}
{../Figures/JJoverdamped}
{0.5cm} %vspace
{Left: Measured IVC of an underdamped junction manufactured in this working group, showing the typical hysteresis. Right: Theoretical IVC of an overdamped junction with a current-voltage shape that is independent of the current sweep direction.}
{IVCs}

For real junctions, however, one needs to take into account that they are comprised of two electrodes separated by a thin insulating layer, which represent a parallel plate capacitor with the Al-Al$\rm O_x$ layer being the dielectric material. Therefore, a junction capacitance \textit{C} needs to be taken into account. A displacement current $I_{\rm d}$ will flow as a consequence, given we are in the voltage state. Lastly, thermal and 1/f noise cause a small fluctuating current $I_{\rm f}$. All these current channels were defined in the so-called Resistively and Capacitively Shunted Junction (RCSJ) model \cite{Cumber1968}, \cite{Stewart1968}, which models the total current of a lumped (0-dimensional) junction to a sufficiently high accuracy. A schematic of an effective circuit diagram is shown in figure \ref{abb:fig:rcsj} (left). Combining every current channel utilizing Kirchhoff's law leads to the \textit{Basic Junction Equation}, which is defined as \cite{Gross2016}

\gl{
I=I_{\rm s}+I_{\rm n}+I_{\rm d}+I_{\rm f}=I_{\rm c}\sin(\varphi)+\frac{1}{R(V)}\frac{\unit{\fq}}{2\pi}\frac{\mathrm{d}\varphi}{\mathrm{d}t}+C\frac{\unit{\fq}}{2\pi}\frac{\mathrm{d}^2\varphi}{\mathrm{d}t^2}+I_{\rm f} \ \ .
}{BJE}

By defining the Josephson coupling energy $U_{\rm J0}=\frac{\hbar I_{\rm c}}{2e}$ and the normalized currents $i=\frac{I}{I_{\rm c}}$ and $i_{\rm f}(t)=\frac{I_{\rm f}(t)}{I_{\rm c}}$, equation \ref{BJE} can be rewritten to 

\gl{
\left(\frac{\hbar}{2 e}\right)^2 C \frac{\mathrm{d}^2 \varphi}{\mathrm{d} t^2}+\left(\frac{\hbar}{2 e}\right)^2 \frac{1}{R(V)} \frac{\mathrm{d} \varphi}{\mathrm{d} t}+\frac{\mathrm{d}}{\mathrm{d} \varphi}\left\{U_{\rm J 0}\left[1-\cos \varphi-i \varphi+i_{\rm f}(t) \varphi\right]\right\}=0 \ \ . 
}{BJE2} 

\twofigurescenter {t!}
{width=0.48\textwidth}
{../Figures/RCSJ-Modell}
{0.02\textwidth} %hspace b/w figures
{width=0.48\textwidth}
{../Figures/washboard}
{0.5cm} %vspace
{Left: Schematic circuit of a lumped Josephson junction with all four current channels connected in parallel, according to the RCSJ model. The junction is represented by the cross symbol on the left, marking the supercurrent $I_{\rm s}$. The normal current $I_{\rm n}$ is realized with a resistance $R$, while the displacement current $I_{\rm d}$ and the noise $I_{\rm f}$ are attributed to a capacitor $C$ and a current source, respectively. Right: Tilted washboard potential for different currents, ranging from 0 to $1.5I_{\rm c}$. The tilt increases with the injected current $I$.}
{fig:rcsj}

The expression in the curly brackets represents the potential energy in the system $U_{\rm J}$, allowing equation \ref{BJE} to be compared to 

\gl{
M\frac{\mathrm{d}^2 x}{\mathrm{d}t^2} + \eta\frac{\mathrm{d} x}{\mathrm{d}t} + \nabla U = 0 \ \ ,
}{}

which describes a particle with mass \textit{M} and damping $\eta$ moving inside the potential \textit{U}. This mechanical analogue therefore allows us to interpret as the equation of motion for a \textit{phase particle}, where it's motion corresponds to a change of the gauge-invariant phase difference $\varphi$ within a potential $U_{\rm J}$ \cite{Clarke2004}. Consequently, it is attributed with a mass $M=\left(\frac{\hbar}{2 e}\right)^2C$ and damping $\eta=\left(\frac{\hbar}{2 e}\right)^2\frac{1}{R(V)}$. Figure \ref{abb:fig:rcsj} (right) visualizes how this phase particle behaves for different currents \textit{I}. Given the shape of $U_{\rm J}(\varphi)$, the potential is referred to as the \textit{tilted washboard potential}. \\
For $I=0$, the phase particle will remain within one of the potential minima. As the current increases, however, the potential starts to tilt such that the depth of the minima reduces until it vanishes for $I=I_{\rm c}$, thus becoming a saddle point. Up until this point, the phase particle can't overcome the potential barrier to move downward, which agrees with the second Josephson equation as the phase difference $\varphi$ should remain constant on average for $I<I_{\rm c}$. Further increasing the current and therefore the tilt of the potential causes the phase particle to fall along the potential, resulting in a voltage drop across the junction ($\frac{\del\varphi}{\del t}>0$). \\

Reversing the current sweep showcases the importance of the particle's mass $M$ and damping $\eta$, as they determine if the return path equals the current shape described above or not. For the case of a small mass (small $C$) and large damping (small $R$), the phase particle will, due to a lack of momentum and strong damping, come to a halt as soon as minima reappear in the washboard potential by reducing the current below $I_{\rm c}$. The current path will therefore remain unchanged as $I$ is reduced back to 0, as shown in figure \ref{abb:IVCs} (right). Such a junction is consequently called an \textit{overdamped} junction. \\
The other case describes an \textit{underdamped} junction (figure \ref{abb:IVCs} (left)) and involves a large mass (large $C$) and small damping (large $R$). This allows the phase particle to continue to move downward as it now carries enough momentum to overcome the arising maxima and minima. The finite voltage drop despite the current being below $I_{\rm c}$ is displayed as the steep quasiparticle current curve, which ends with a return current $I_{\rm R}$ that arises with the recapture of the particle in a minimum. This leads to a hysteretic IVC, as depicted in figure \ref{abb:IVCs} (left). $I_{\rm R}$ can be calculated via \cite{Likharev1986}

\gl{
I_{\rm R} = \frac{4}{\pi\sqrt{\beta_{\rm C}}}I_{\rm c} \ \ ,
}{IR}

with $\beta_{\rm C}$ being the dimensionless Stewart-McCumber parameter, that is used to quantitatively distinguish between both junction types. It is given by 

\gl{
\beta_{\rm C} = \frac{2\pi}{\unit{\fq}}I_{\rm c}R^2C
}{betaC} 

with $\beta_{\rm C}\gg 1$ corresponding to a strongly underdamped junction, whereas $\beta_{\rm C}\ll 1$ represents a strongly overdamped junction. The junctions developed and produced within the scope of this thesis aim to be overdamped, which is why we take a closer look on the time-averaged voltage for $I>I_{\rm c}$ in the case of $ \beta_{\rm C}\ll 1$. Neglecting the noise in equation \ref{BJE2}, as well as assuming the resistance to be linear below and above the gap voltage $V_{\rm g}$, i.e. $R(V)=R$, the time-averaged voltage can be derived to \cite{Clarke2004}

\gl{
\langle V(t)\rangle=I_{\mathrm{c}} R \sqrt{\left(\frac{I}{I_{\mathrm{c}}}\right)^2-1} \ \ \ \mathrm{for} \ \ \ \frac{I}{I_{\mathrm{c}}}>1 \ \ .
}{VJJ}

This equation will be crucial to determine the voltage drop of a dc-SQUID, as its derivation is analogous to that of a single junction, which will be covered in the next section.

\section{dc-SQUIDs}

We have now covered the theoretical framework necessary to understand the working principle of a dc-SQUID, which consists of a superconducting ring intersected by two identical Josephson junctions with critical Josephson currents $I_{\rm c}$, as depicted in figure \ref{abb:dcSQUID}. Both junctions are shunted with shunt resistors $R_{\rm s}$ to avoid hysteretic behavior in the respective IVCs. If the SQUID is biased with a bias current $I_{\rm b}$, it is possible to convert small flux variations inside the loop into a measurable voltage change. dc-SQUIDs are, therefore, used as highly sensitive flux-to-voltage transducers.

\subsection{Zero Voltage State}

\figureleft {b!}
{width=\textwidth} %sets how much of the fig space is used
{../Figures/dc-SQUID}
{9cm} %sets width of the fig space
{0cm}
{Schematic circuit diagram of a shunted dc-SQUID. A superconducting loop with inductance $L_{\rm s}$ is interrupted by two Josephson junctions such that they form a parallel connection. With bias current is $I_{\rm b}$ and external magnetic flux $\Phi$. To avoid hysteresis effects, shunt resistors $R_{\rm s}$ are connected in parallel to each junction.} 
{dcSQUID}

In order to fully understand the working principle of a dc-SQUID it is again necessary to first cover the zero voltage stage as we did for a single junction. 
The parallel connection of the two junctions allows the bias current with identical critical currents, i.e. $I_{\rm c,1}=I_{\rm c,2}=I_{\rm c}$ to split into two supercurrents $I_{\rm s1}$, $I_{\rm s2}$. Here we assume $I_{\rm b}<2I_{\rm c}$ to ensure that no voltage drop across both junctions occurs ($V_{\rm s}=0$). Applying Kirchhoff's law we then obtain the expression 

\gl{
I_{\rm b} = I_{\mathrm{s}}=I_{\mathrm{c}} \sin \varphi_1+I_{\mathrm{c}} \sin \varphi_2=2 I_{\mathrm{c}} \cos \left(\frac{\varphi_1-\varphi_2}{2}\right) \sin \left(\frac{\varphi_1+\varphi_2}{2}\right) \ \ .
}{squid_Is_tot}

In chapter \ref{subsec_jjmag} we concluded that a magnetic flux $\Phi$ causes the supercurrent to modulate with $\unit{\fq}$. A dc-SQUID can be considered as a single junction with a much larger effective area $A_{\rm eff}$ (loop area), that could be penetrated vy an external magnetic flux. It is, therefore, reasonable to expect a similar behavior for a dc-SQUID. The same approach as with a single junction is used to determine the flux dependence of the total supercurrent, where a closed loop integral is performed around the SQUID loop. The calculation leads to the relation \cite{Gross2016} 

\gl{
\varphi_2-\varphi_1=\frac{2 \pi \Phi}{\Phi_0}
 \ \ ,}{}

which can be directly inserted into equation \ref{squid_Is_tot} to obtain

\gl{
I_{\mathrm{s}}=2 I_{\mathrm{c}} \cos \left(\pi \frac{\Phi}{\Phi_0}\right) \sin \left(\varphi_1+\pi \frac{\Phi}{\Phi_0}\right) \ \ .
}{IsGeneral_SQUID0v}

In the most general case, however, one needs to take into account the inductance $L_{\rm s}$ of the SQUID loop and, therefore, a circulating current $I_{\rm cir} = ({I_{\rm s1} - I_{\rm s2}})/{2}$ that induces the additional flux $\Phi_{\rm cir} = L_{\rm s}I_{\rm cir}$. With the external flux $\Phi_{\rm e}$ we can thus write for the total flux
\begin{align}
	\Phi &= \Phi_{\rm e} + \Phi_{\rm cir} \\ &= \Phi_{\rm e} - L_{\rm s} I_{\rm  c} \sin \left(\pi \frac{\Phi}{\Phi_0}\right) \cos \left(\varphi_1+\pi \frac{\Phi}{\Phi_0}\right) \\ &= \Phi_{\mathrm{e}}-\frac{1}{2} \beta_L \Phi_0 \sin \left(\pi \frac{\Phi}{\Phi_0}\right) \cos \left(\varphi_1+\pi \frac{\Phi}{\Phi_0}\right) \ \ .
\end{align}
Here, we introduced the dimensionless screening parameter $\beta_{\rm L}=\frac{2L_{\rm s}I_{\rm c}}{\unit{\fq}}$, which relates the maximum possible flux $\Phi_{\rm cir}^{\rm max} = L_{\rm s}I_{\rm cir}^{\rm max} = L_{\rm s}I_{\rm c}$ produced by screening currents to ${\unit{\fq}}/{2}$. This quantity describes the influence the screening currents have on the total flux $\Phi$, which in turn affects $I_{\rm s}$ in equation \ref{IsGeneral_SQUID0v}. We will now simplify the expression above by considering the limiting case for small currents, i.e. $I_{\rm s}\ll 2I_{\rm c}$. This condition implies that $\sin\varphi_1\approx -  \sin\varphi_2$ and thus $\varphi_1\approx - \varphi_2$, leading to a vanishing argument $\varphi_1 + \pi\frac{\Phi}{\unit{\fq}} \approx 0$ in the cosine resulting to

%\gl{
%\Phi=\Phi_{\mathrm{e}}-\frac{1}{2} \beta_L \Phi_0 \sin \left(\pi \frac{\Phi}{\Phi_0}\right) \cos \left(\varphi_1+\pi \frac{\Phi}{\Phi_0}\right)
%}{}

\gl{
\Phi=\Phi_{\mathrm{e}}-\frac{1}{2} \beta_{\rm L} \Phi_0 \sin \left(\pi \frac{\Phi}{\Phi_0}\right) \ \ .	
}{} 

\figureleft {t!}
{width=\textwidth}
{../Figures/Phi_betaL}
{9cm}
{1cm}
{Normalized flux $\Phi$ modulated by the external flux $\Phi_{\rm e}$. The amplitude of the modulation depends on the screening parameter $\beta_{\rm L}$, where $\Phi(\Phi_{\rm e})$ remains a single-valued function for $\beta_{\rm L}\leq2/\pi$.} 
{Phi_betaL}

Figure \ref{abb:Phi_betaL} showcases this relation for several values of $\beta_{\rm L}$. High values ($\beta_{\rm L}>2/\pi$) correspond to hysteretic characteristics, meaning there can be multiple values of total flux $\Phi$ for the same applied flux $\Phi_{\rm e}$. For practical dc-SQUIDs, it is, therefore, desirable to avoid this ambiguous behavior. The intersections of each curve represents the case for $\Phi = n\unit{\fq}$, such that the screening currents vanish and the total flux equals the external flux ($\Phi=\Phi_{\rm e}$). This is to be expected as the flux in a superconducting ring needs to be quantized (see equation \ref{fq}). Consequently, the SQUID tries to maintain the total flux at integer values of $\unit{\fq}$ for the limiting case of $\beta_{\rm L}\gg 1$, where $\Phi_{\rm cir}$ dominates over any applied flux. This compensation is visualized by the strong modulation for high $\beta_{\rm L}$ in figure \ref{abb:Phi_betaL}, where a wide range of $\Phi_{\rm e}$ values remain in the proximity of $n\Phi$. The other limiting case, i.e. $\beta_{\rm L}\ll 1$, allows us to neglect the circulating currents such that we can write $\Phi \approx \Phi_{\rm e}$. From equation \ref{IsGeneral_SQUID0v} we then obtain the maximum possible supercurrent 

\gl{
I_{\mathrm{s}}^{\mathrm{m}}(\Phi_{\rm e})=2 I_{\mathrm{c}}\left|\cos \left(\pi \frac{\Phi_{\rm e}}{\Phi_0}\right)\right| \ \ .
}{maxIs}

The modulation of this current quickly diminishes for increasing $\beta_{\rm L}$, as was derived in \cite{Clarke2004} to 

\gl{
\frac{\Delta I_{\mathrm{s}}^{\mathrm{m}}\left(\Phi_{\mathrm{e}}\right)}{2 I_{\mathrm{c}}} \approx 1-\frac{2 \Phi_{\mathrm{e}}}{\Phi_0 \beta_{\mathrm{L}}} \ \ .
}{}

For the SQUIDs produced within the scope of this thesis, values of $\beta_{\rm L}\approx 1$ were cosidered optimal to minimize resonant behavior without reducing the SQUID inductance $L_{\rm s}$ too much. 

\subsection{Voltage State} \label{sec_voltagestate}

To utilize dc-SQUIDs as sensitive magnetometers, it is necessary to operate them in the voltage state by applying a large enough bias current $I_{\rm b}$, such that $I_{\rm b}>2I_{\rm c}$. In the case of negligible screening ($\beta_{\rm L}\ll 1, \Phi\approx\Phi_{\rm e}$) and strong damping ($\beta_{\rm C}\ll 1$), i.e. by chosing a small junction capacitance $C$ and SQUID inductance $L_{\rm s}$, it is possible to derive the flux dependence of the resulting voltage drop across the SQUID. Following the RCSJ model, we are only left with the supercurrent $I_{\rm s}$ and the resistive current $I_{\rm n}$, such that by using equation \ref{IsGeneral_SQUID0v} we can write for the bias current 

\gl{
I_{\rm b} = 2 I_{\mathrm{c}} \cos \left(\pi \frac{\Phi_{\rm e}}{\Phi_0}\right) \sin \left(\varphi_1+\pi \frac{\Phi_{\rm e}}{\Phi_0}\right) + 2\frac{V_{\rm s}}{R} \ \ ,
}{}

where we again assumed identical junctions, each shunted by a small shunt resistor $R_{\rm s}\ll R_{\rm n}$. Here $R_{\rm n}$ denotes the normal resistance of a single, unshunted junction. Therefore, the total normal resistance $R$ for each branch of the SQUID is approximately $R\approx R_{\rm s}$. Additionally, we can define the phase $\varphi=\varphi_1 + \pi\frac{\Phi}{\unit{\fq}}$ and with the maximum supercurrent from equation \ref{maxIs} we obtain a current-phase relation that resembles that of a single junction:

\gl{
I_{\rm b} = I_{\rm s}^{\rm m}(\Phi_{\rm e}) \sin \left(\varphi\right) + \frac{2}{R_{\rm s}}\frac{\unit{\fq}}{2\pi}\frac{\del\varphi}{\del t} \ \ .
}{}

This equivalence of a dc-SQUID and a single junction underlines the above-mentioned fact that the SQUID loop represents a single Josephson contact that provides a larger effective area external fields can penetrate. It is therefore possible to derive the voltage drop across the SQUID in the same manner as in subsection \ref{subsec_RCSJ}. With the critical current $I_{\rm s}^{\rm m}(\Phi_{\rm e})$ being flux-dependent with a modulation period of $\unit{\fq}$, we can compare to equation \ref{VJJ} and obtain for the time averaged voltage \cite{Clarke2004}
\begin{align}\label{squidV(t)}
	\langle V(t)\rangle &= \frac{R_{\rm s}}{2}\sqrt{I_{\rm b}^2 - I_{\rm s}^{\rm m}(\Phi_{\rm e})^2} \\ &= I_{\rm c}R_{\rm s}\sqrt{\left(\frac{I_{\rm b}}{2I_{\rm c}}\right)^2-\left[\cos\left(\pi\frac{\Phi_{\rm e}}{\unit{\fq}}\right)\right]^2} \ \ .	
\end{align}     
Evidently, both the critical current and the voltage are flux dependent and are modulated by $\unit{\fq}$. Figure \ref{abb:IV_VPhi_theo} (left) showcases this behavior by considering the case for the minimum and maximum critical current, i.e. for $\Phi_{\rm e} = n\unit{\fq}$ and $\Phi_{\rm e} = (n+\frac{1}{2})\unit{\fq}$, with $n\in\mathbb{Z}$. The current-voltage-characteristics at these flux values are particularly interesting, as they can be used to extract crucial SQUID parameters like the voltage swing $\Delta V_{\rm max}$. This property describes how the voltage varies with the applied flux $\Phi_{\rm e}$, at a given current $I_{\rm b}$. It is maximal at $I_{\rm b}\approx 2I_{\rm c}$, as depicted in figure \ref{abb:IV_VPhi_theo} (right). \\

\twofigurescenter {t!}
{ width=0.49\textwidth}
{../Figures/IV}
{0.001\textwidth} %hspace b/w figures
{width=0.49\textwidth}
{../Figures/VPhi}
{0.5cm} %vspace
{Left: IV-characteristics for the total flux $\Phi\approx\Phi_{\rm e}$ being an integer and half integer number of flux quanta, given that $\beta_{\rm C}\ll 1$ and $\beta_{\rm L}\ll 1$. The maximum voltage swing $\Delta V_{\rm max}$ is approximately given at $I_{\rm b}\approx 2I_{\rm c}$ and corresponds to $I_{\rm c}R_{\rm s}$ for a resistively shunted dc-SQUID. Right: The projection of equation \ref{squidV(t)} onto the $V\Phi$-plane shows the flux dependence of the voltage at the bias current values $I_{\rm b}=2I_{\rm c}$, $I_{\rm b}=3I_{\rm c}$ and $I_{\rm b}=4I_{\rm c}$. The amplitude of the modulation decreases for increasing $I_{\rm b}$.}
{IV_VPhi_theo}

It is, however, important to note that equation \ref{squidV(t)} doesn't hold for practical SQUIDs, as they are typically not fabricated to fulfill the limiting case of $\beta_{\rm C}\ll 1$ and $\beta_{\rm L}\ll 1$. With few adjustments, the conclusions reached here will nevertheless be applicable to practical SQUIDS. 

\subsection{Optimal Parameters}\label{subsec_optparam_theo}

Negligible screening is not feasible, as it would require an extremely small SQUID inductance $L_{\rm s}$, which in turn deteriorates the sensitivity to magnetic fields. The main reason to construct a dc-SQUID was to obtain a highly sensitive magnetometer by creating a large area for magnetic fields to penetrate. Also, the fabrication process doesn't allow to produce an arbitrarily small junction capacitance $C$. The parameter $\beta_{\rm C}$ will, therefore, reach a lower limit as well, since also decreasing $R_{\rm s}$ too much reduces the voltage swing $\Delta V_{\rm max}$ and increases the energy sensitivity $\epsilon(f)$, as we will see in subsection \ref{subsec_noise_theo}. Taking into account displacement and fluctuation currents, the current and voltage expressions for the dc-SQUID become analytically unsolvable and therefore have to be solved numerically. In \cite{Tesche1977} such numerical simulations lead to optimal values of $\beta_{\rm C}\approx 1$ and $\beta_{\rm L}\approx 1$ to minimize the energy sensitivity. \\

To further fine-tune the SQUID parameters, e.g. loop inductance and critical current, it is essential to look at how dc-SQUIDs are typically operated to achieve the highest possible flux sensitivity. Here, we distinguish between a current and a voltage bias, where the former was assumed in figure \ref{abb:IV_VPhi_theo}. Maximizing sensitivity in this mode is done by keeping the flux constant through a feedback loop at the steepest point in the $V\Phi$-curve, which is referred to as the working point. This allows for the largest possible voltage change $\Delta V$ at a given flux change $\Delta \Phi$. Similarly, at a voltage bias the working point (WP) will mark the steepest point in the $I\Phi$-curve. To quantify this, we introduce the transfer coefficients at the working point

\begin{align}
V_{\rm \Phi} &\equiv {\left|\frac{\del V}{\del\Phi}\right|}_{\text{WP}} \\ I_{\rm \Phi} &\equiv {\left|\frac{\del I}{\del\Phi}\right|}_{\text{WP}} \ \ .
\end{align}

As mentioned above, at $I_{\rm b}\approx 2I_{\rm c}$ (current bias) the amplitude of the voltage modulation is maximal. This is modified for practical SQUIDs, where thermal fluctuations can't be neglected. The resulting thermal current $I_{\rm th}$ causes a rounding of the edge at $I_{\rm b}= 2I_{\rm c}$ (figure \ref{abb:IV_VPhi_theo} (left)), thereby reducing $\Delta V_{\rm max}$ and $V_{\rm \Phi}$ \cite{Ivanchenko1968}. To minimize this effect, numerical simulations were made that lead to the condition \cite{Clarke1988}

\gl{
\frac{I_{\rm c}}{5}\geq I_{\rm th}\equiv\frac{2\pi k_{\rm B}T}{\unit{\fq}} \ \ .
}{}

A lower bound for $I_{\rm c}$ at $T=\qty{4.2}{\kelvin}$ will, therefore, be approximately \qty{1}{\uA}. This effect shifts the current $I_{\rm b,max}$, at which the voltage swing is maximal, according to \cite{Drung1996} by a temperature correction factor leading to 

\gl{
I_{\rm b,max}\approx 2I_{\rm c}(1-\sqrt{\Gamma/\pi}) \ \ ,
}{}

where $\Gamma$ is defined as $\Gamma=I_{\rm th}/I_{\rm c}$. Lastly, the thermal current can also be used to set an upper limit to the SQUID inductance. We can define a thermal inductance $L_{\rm th}= \unit{\fq}/({2I_{\rm th}})$ for the thermal current inducing half a flux quantum. This should be significantly larger than the SQUID inductance $L_{\rm s}$ to minimize the impact of these thermal fluctuations. Again, simulations provide a constraint for optimization, giving the relation \cite{Clarke1988}

\gl{
5L_{\rm s}\leq L_{\rm th}\equiv \frac{\unit{\fq}}{2I_{\rm th}}=\frac{\unit{\fq}^2}{4\pi k_{\rm B}T} \ \ .
}{} 

For $T=\qty{4.2}{\kelvin}$ we would obtain $L_{\rm s}\leq\qty{1}{n\henry}$, which is typically fulfilled for practical dc-SQUIDs.
%\\ \\ \\
%{\large\textbf{Noise}}
%\\ \\

\subsection{Noise}
\label{subsec_noise_theo}

The above-mentioned energy sensitivity, also called spectral noise energy density or energy resolution, is defined as the flux noise per SQUID inductance $L_{\rm s}$ and is typically expressed through a power spectral density as 

\gl{
\epsilon(f) = \frac{S_{\rm \Phi}(f)}{2L_{\rm s}} \ \ .
}{energy_sens}

This conveniently allows to compare noise properties from SQUIDs with different loop inductances \cite{Ferring-Siebert2024}. The flux noise power spectral density $S_{\rm \Phi}(f)$ is typically calculated from the voltage noise using the transfer coefficient introduced above: 

\gl{
S_{\rm \Phi}(f) = \frac{S_{\rm V}(f)}{V_{\rm \Phi}^2} \ \ .
}{fluxnoise_psd}

The flux noise in SQUIDs is typically separated into a frequency-independent white noise at higher frequencies and a low frequency 1/\textit{f}-noise component \cite{Koch2007}. To derive an expression for $S_{\rm V}(f)$ we consider the white noise only, limiting ourselves to higher frequencies to avoid any significant influence of 1/\textit{f}-noise. For this we need to distinguish between the noise stemming from the bias current and the circulating current inside the SQUID-loop. The former is affected by the total resistance of the SQUID which corresponds to ${R_{\rm s}}/{2}$ for $I\gg I_{\rm b,max}$ and to the dynamic resistance $R_{\rm dyn}={\del V}/{\del I}$ for operation at the working point. The circulating current, however, is affected by two resistances in series, i.e. 2$R_{\rm s}$. By taking into account noise inducing resonances due to the simulated optimal values of $\beta_{\rm C}\approx 1$ and $\beta_{\rm L}\approx 1$ that exhibit hysteretic behavior, both noise currents can be used to obtain the voltage noise given by \cite{Tesche1977}, \cite{Bruines1982}
 
\gl{
S_{\rm V}(f) = \frac{4k_{\rm B}T}{R_{\rm s}}\left[2R_{\rm dyn}^2+\frac{L_{\rm s}^2V_{\rm \Phi}^2}{2}\right] \ \ .
}{voltagenoise_psd}

Here, we used the approximation $V_{\rm \Phi}\approx\frac{I_{\rm c}R_{\rm s}}{\unit{\fq}/2}\approx\frac{R_{\rm s}}{L_{\rm s}}$ valid for $\beta_{\rm L}\approx 1$. This approximation leads to the expression $R_{\rm dyn}\approx \sqrt{2}R_{\rm s}$ for the dynamical resistance, which can be used to rewrite equation \ref{voltagenoise_psd} to 

\gl{
S_{\rm V}(f) = 18k_{\rm B}TR_{\rm s} \ \ .
}{}

Inserting this into equations \ref{fluxnoise_psd} and \ref{energy_sens} leads to the expression

\gl{
\epsilon(f) \approx 16k_{\rm B}T\sqrt{\frac{L_{\rm s}C}{\beta_{\rm C}}} \approx 16k_{\rm B}T\sqrt{L_{\rm s}C} \ \ \ \rm{for} \ \ \ \beta_{\rm C}\approx 1 \ \ .
}{}

For a moderately damped dc-SQUID ($\beta_{\rm C}\approx 1$) at a temperature $T=\qty{100}{\milli\kelvin}$ and with a suitably small inductance $L_{\rm s}=\qty{100}{\pico\henry}$ we would, therefore, obtain a theoretical value of $\epsilon\approx \qty{0.32}{h}$. \\

Finally we will look at the total flux noise spectrum by considering the $1/f$ component, which is added to the above derived white noise dominating in the low frequency regime. The origin of this component is still unclear \cite{Kempf2016}. It is parameterized by a power density at \qty{1}{\hertz}, giving a total noise power density of      


\gl{
S_{\rm \Phi}(f) = S_{\rm \Phi,w} + \frac{S_{\rm \Phi, 1/f}(\qty{1}{\hertz})}{f^\alpha} \ \ ,
}{noise_components}

with $S_{\rm \Phi,w}$ being the white noise component and $\alpha$ ranging from 0.5 to 1 \cite{Drung2011}, which can be determined by experiment.

\subsection{Parasitic Resonances}\label{subsec_para_res}

We have seen that optimizing properties of dc-SQUIDs requires careful fine-tuning of various parameters to ensure high sensitivity and low noise at the same time. Another constraint in this regard involves the presence of resonances in the system that we need to take into account. As discussed in subsection \ref{subsec_RCSJ}, a Josephson contact also represents a parallel plate capacitor with capacitance $C$. Here we take into account the total capacitance of $C/2$, due to the series connection of the two junctions within the SQUID loop. The SQUID with its loop inductance $L_{\rm s}$ will, therefore, form an $LC$ resonator that can be excited if the Josephson currents oscillate with the resonance frequency $f_{L_{\rm s}C}={1}/({2\pi\sqrt{L_{\rm s}C/2}})$. This is fulfilled for the condition 

\gl{
\frac{V_{\rm s}}{\unit{\fq}} = \frac{1}{2\pi\sqrt{L_{\rm s}C/2}} \ \ ,
}{fund_res}

where $V_{\rm s}$ denotes the voltage drop across the SQUID that is associated with the Josephson frequency $f_{\rm J}=V_{\rm s}/{\unit{\fq}}$ resulting from the ac Josephson effect. With optimal $\beta_{\rm C}$ and $\beta_{\rm L}$ values the voltage corresponding to this resonance frequency will move towards the vicinity of the working point, thus becoming relevant for the SQUID's performance.
This so-called \textit{fundamental SQUID resonance} negatively affects the operation range by manifesting itself through a current step in the IVC, as shown in figure \ref{abb:fund_resonance_theo}. As a result, the IV curves for integer and half integer flux quanta intersect as a result multiple times, due to higher harmonics of the resonance causing additional current steps \cite{Clarke2004}. The voltage swing $\Delta V$ and consequently $V_{\rm \Phi}$ are limited by this resonance, which favors the choice of small values for $L_{\rm s}$ and $C$. \\

\figureleft {t!}
{width=\textwidth}
{../Figures/IV_3d13_ch1}
{9cm}
{1cm}
{Measured IV-characteristic of a dc-SQUID developed within the scope of this thesis. The first current step at $V_{\rm s}\approx \qty{40}{\uV}$ corresponds to the fundamental SQUID resonance for $L_{\rm s}=\qty{119}{\pH}$ and $C=\qty{0.95}{\pF}$.}
{fund_resonance_theo}

Resonance inducing structures are, however, necessary for practical reasons. Those include the need for effectively coupling  external flux changes into the SQUID in order to take advantage of its high sensitivity. For this an input coil with inductance $L_{\rm i}$ is usually fabricated on top of the SQUID loop, separated by an insulating layer to ensure the coupling to be solely inductive (see section \ref{sec_practical_SQUID}). The SQUID can, therefore, be used as a current sensor by converting small current signals in the input coil into small magnetic flux changes in the SQUID. This additional coil, however, provides a parasitic capacitance $C_{\rm p}$ resulting from its fabrication on top of the SQUID loop. Consequently, rf currents can couple from one system into the other, causing further parasitic effects. Particularly, problematic $LC$ resonances arise from the added inductance $L_{\rm i}$ and capacitance $C_{\rm p}$, namely the $f_{L_{\rm s}C_{\rm p}}$ and $f_{L_{\rm i}C_{\rm p}}$ resonance. The latter can be excited for $f_{\rm J}=f_{L_{\rm i}C_{\rm p}}$, since the Josephson currents are able to couple into the input coil. The parasitic capacitance $C_{\rm p}$ and consequently both resonance frequencies depend on geometrical factors such as the length of the input coil, as well as the widths and heights of the individual components that form the resonator. It has been shown, that these resonances lead to the energy sensitivity increasing proportional to $\sqrt{1+2C_{\rm p}/C}$, as long as $C_{\rm p}/C\leq 2$. For $C_{\rm p}/C\geq 2$ the energy sensitivity saturates due to the resonance frequencies shifting below the working point \cite{Ryh1992}. It is, therefore, desirable to minimize $C_{\rm p}$ as much as possible in order to reduce the resulting voltage noise. \\

The fabrication of the input coil on top of the SQUID loop also represents a microstrip transmission line, consisting of a conductor carrying the signal (input coil) and a ground plane (SQUID loop), separated by a dielectric layer. This forms a waveguide allowing electromagnetic waves to propagate alongside it, which undergo reflections where impedance mismatches occur. These would arise whenever the input coil leaves the underlying ground plane. Standing waves occur if the length $l$ of the signal carrying line above the ground plane corresponds to an integer multiple of half the wavelength of the Josepshon frequency. This is consequently also called the $\lambda/2$ resonance. The corresponding resonance frequency 

\gl{
f_l = \frac{mc_{\rm str}}{2l} \ \ ,   
}{stripline_res_general}  

where $m\in\mathbb{N}$, depends on $l$ and the material-dependent wave propagation velocity $c_{\rm str}$. This resonance behavior also emerges if we consider the SQUID loop to be the signal carrying line with the input coil acting as the corresponding ground plane. Here it is again possible to move the resonance further away from the operation frequency at the working point by choosing adequate geometric proportions, e.g. by varying the length of the input coil or the SQUID loop. These methods to mitigate the influence of resonances can be complemented by direct damping through attenuators, which will be discussed in section \ref{sec_damping}. In chapter \ref{ch_SQUIDdesign} we will cover how various parameters are chosen in the SQUID design to suppress and avoid possible resonances, given the constraints we derived in subsection \ref{subsec_optparam_theo} and \ref{subsec_noise_theo}.