\chapter{Introduction}

Particle detectors requiring a high energy-independent resolution combined with a large bandwidth have been successfully realized in the form of cryogenic metallic magnetic calorimeters (MMC) \cite{Enss2005a}. These high-precision low-temperature detectors are used in a broad range of applications, such as for mass spectrometry of heavy particles \cite{Hengstler2017}, spectroscopy of highly charged heavy ions \cite{Gamer2919} or the investigation of the mass of the electron-neutrino \cite{Gastaldo2017}. Their working principle is based on the conversion of the energy of an incoming particle into a magnetic flux change. This is realized by the use of a paramagnetic sensor in a weak magnetic field, which experiences a temperature change upon the absorption of an incoming particle through an absorber that is thermally coupled to the sensor. The temperature increase is accompanied by a change in magnetization of the paramagnet, which creates a flux change that is read out by a superconducting quantum interference device (SQUID). SQUIDs represent highly sensitive state of the art magnetometers with a broad bandwidth. The SQUID can couple to the MMC via a flux transformer setup, where its input coil is connected to the pickup coil of the MMC experiencing the initial flux change. The induced current change creates a flux change in the input coil that couples into the superconducting SQUID loop, thereby acting as a current sensor SQUID. The SQUID loop is intersected by either one or two Josephson junctions, representing an rf-SQUID or dc-SQUID, respectively. A current-biased dc-SQUID produces a finite voltage across both arms of the loop upon a magnetic flux change. The resulting non-linear current-voltage-characteristics require a broadband FLL feedback readout setup to linearize and amplify the output voltage. To mitigate the room temperature noise contribution of the feedback electronics, a secondary SQUID is implemented to act as a low temperature preamplifier. The apparent noise in the current-sensor SQUID, however, couples into the MMC detector  
The energy resolution of the detector is not only limited by the intrinsic energy resolution, but also by the SQUID noise coupled into the MMC, which dominates the flux noise spectrum above a few kHz \cite{Kempf2018}.     