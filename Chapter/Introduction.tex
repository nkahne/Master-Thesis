\chapter{Introduction}

Particle detectors requiring a high energy-independent resolution combined with a large bandwidth have been successfully realized in the form of cryogenic metallic magnetic calorimeters (MMC) \cite{Enss2005a}. These high-precision low-temperature detectors are used in a broad range of applications, such as for mass spectrometry of heavy particles \cite{Hengstler2017}, spectroscopy of highly charged heavy ions \cite{Gamer2919} or the investigation of the mass of the electron-neutrino \cite{Gastaldo2017}. Their working principle is based on the conversion of the energy of an incoming particle into a magnetic flux change. This is realized by the use of a paramagnetic sensor in a weak magnetic field, which experiences a temperature change upon the absorption of an incoming particle through an absorber that is thermally coupled to the sensor. The temperature increase is accompanied by a change in magnetization of the paramagnet, which creates a flux change that is read out by a direct current superconducting quantum interference device (dc-SQUID). SQUIDs represent highly sensitive state of the art magnetometers with a broad bandwidth. The SQUID can couple to the MMC via a flux transformer setup, where its input coil is connected to the pickup coil of the MMC experiencing the initial flux change $\Delta\Phi$. The induced current change creates a flux change $\Delta\Phi_{\rm s}$ in the input coil that couples into the superconducting SQUID loop, which is intersected by two Josephson junctions. A current-biased dc-SQUID produces a finite voltage across both arms of the loop upon a magnetic flux change. The resulting non-linear current-voltage-characteristics require a broadband FLL feedback readout setup to linearize and amplify the output voltage. To reduce the noise contribution of the feedback electronics, a secondary SQUID is implemented to act as a low temperature preamplifier. The apparent noise in the current-sensor SQUID, however, couples into the MMC, which adds to the intrinsic noise of the detector. For the detectors developed in this working group, this added noise dominates the flux noise spectrum above a few kHz, whereas $1/f$ noise of doped Erbium atoms and thermodynamic energy fluctuations are mostly responsible for low frequency noise \cite{Kempf2018}. The energy resolution $\Delta E_{\rm FWHM}$, which can be described by the signal to noise ratio, is therefore negatively affected by the SQUID readout chain. To quantify this effect, we use the extrinsic energy sensitivity $\epsilon_{\rm p}$ given by the intrinsic noise of the SQUID $S_{\Phi_{\rm s}}$ and the inverse of the flux-to-flux coupling $\Delta\Phi_{\rm s}/\Delta\Phi$. The objective of this thesis was to optimize the coupling between the SQUID and the X-ray detector maXs100 developed in this working group. This was realized by adding a second turn to the input coil of the detector SQUID to match its inductance with the one of the pickup coil, as this maximizes the coupling  $\Delta\Phi_{\rm s}/\Delta\Phi$. We further investigated the resonance and noise behavior to assess that the new design leads to an overall reduction of $\epsilon_{\rm p}$. \\

In chapter 2 we first introduce the theoretical framework of Josephson junctions that is needed to describe the working principle of a dc-SQUID. As such, we cover macroscopic quantum effects such as the Josephson effect and flux quantization. These explain the characteristic properties of Josephson junctions, which are realized as SIS tunnel contacts in this working group. To motivate the use of SQUIDs, it is essential to investigate the behavior of the junction in an external magnetic field. Following this, we cover the general aspects of dc-SQUIDs including their characteristics and noise behavior. Lastly, we discuss parasitic $LC$ resonances arising from geometric properties of the SQUID circuit.

Chapter 3 covers more in depth the realization of practical SQUIDs and how their parameters are chosen to optimize the performance. The importance of using a second stage SQUID representing a low temperature preamplifier is demonstrated, as it significantly improves the signal to noise ratio. The last part of the chapter gives a short introduction to MMCs, summarizing their core features followed by a brief overview of the extrinsic energy sensitivity regarding the SQUID-based readout, which motivates the needed adjustment of the input coil to the pickup coil of the detector, thereby greatly improving the coupling between SQUID and detector.   

In chapter 4 we present the new SQUID design based on the SQUID developed in \cite{Bauer2022}, where a second turn has been added to the input coil with the premise to better match the pickup coil inductance $L_{\rm p}=\qty{6.65}{\nH}$ of the maXs100 detector. Further optimal design parameters are discussed which ensure minimal flux noise as well as smooth current-voltage-characteristics (IVCs). To mitigate the influence of resonances, various damping techniques are applied, such as shunting the $LC$ circuits with resistors. We discuss new damping approaches involving insulated gold layers distributed over the feed lines as well as placing an added gold layer beneath the input coil, which represents a lossy microstrip line.  

The measured characteristic properties of the new SQUID design are presented in chapter 5. Following this, we discuss the measurement of the input coil inductance and compare it to the expected value. To gain insight into the resonance behavior, we measured the IVC's of various new SQUID variants containing different combinations of the damping schemes described in chapter 3. Additionally, a SQUID without input coil has been tested for comparison to help identify input coil related resonances. The chapter ends with the discussion of the noise measurements, which was conducted with variants containing the lossy microstrip input coil, both with and without feed lines with overlaying gold layers. Lastly, the measured noise values are used to obtain intrinsic and extrinsic energy sensitivities, which are compared among the measured variants. We conclude whether the new design is better suited for the readout of the maXs100 detector and propose possible improvements for future works.    