\chapter{Theoretische Grundlagen}


\section{Josephson Kontakte}
Die nach \textit{Brain D. Josephson} benannten \textit{Josephson Kontakte} (engl. \textit{Josephson junctions}) bestehen aus zwei identischen Supraleitern, die schwach miteinander gekoppelt sind. Im Falle der in dieser Arbeitsgruppe hergestellten Kontakte wird eine solche Kopplung durch eine wenige \r{A}ngström dünne Isolationsschicht zwischen den supraleitenden Elektroden realisiert. Aufgrund dessen werden diese auch SIS (Supraleiter-isolator-Supraleiter) Kontakte genannt. Die so entstehende Dreischichtstruktur besteht typischerweise aus Nb/Al-Al$O_x$/Nb, wobei das Niob für die Supraleiter verwendet wird und die Isoationsschicht durch das Aluminiumoxid gegeben ist.  
\subsection{Josephson Effekt}

\subsection{Josephson Kontakte im Magnetfeld}


\section{dc-SQUIDs}

\subsection{Spannungszustand}

\subsection{Rauschen}

\subsection{Inbetriebnahme eines dc-SQUIDs}


\section{Resonanzen eines dc-SQUIDs}

\subsection{Parasitäre Resonanzen}

\subsection{Dämpfungsmethoden}


