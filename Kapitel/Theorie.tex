\chapter{Theoretische Grundlagen}


\section{Josephson Kontakte}
Die nach \textit{Brain D. Josephson} benannten \textit{Josephson Kontakte} (engl. \textit{Josephson junctions}) bestehen aus zwei identischen Supraleitern, die schwach miteinander gekoppelt sind. Im Falle der in dieser Arbeitsgruppe hergestellten Kontakte wird eine solche Kopplung durch eine wenige \r{A}ngström dünne Isolationsschicht zwischen den supraleitenden Elektroden realisiert. Aufgrund dessen werden diese auch SIS (Supraleiter-isolator-Supraleiter) Kontakte genannt. Die so entstehende Dreischichtstruktur besteht typischerweise aus Nb/Al-Al$O_x$/Nb, wobei das Niob für die Supraleiter verwendet wird und die Isoationsschicht durch das Aluminiumoxid gegeben ist. Ein schematischer Aufbau ist in Abb. ? dargestellt. 
Wird der Kontakt nun bei sehr kalten Temperaturen ($\leq$4K) gehalten und an eine Stromquelle angeschlossen, ist entgegen der Erwartungen ein Suprastrom messbar.
        
\subsection{Josephson Effekt}

Der Stromfluss impliziert das Tunneln von Cooperpaaren, da bei diesen Temperaturen Niob überwiegend supraleitend ist  \texttt{${T_c}$} = 9.3). Da die Tunnelwahrscheinlichkeit eines einzelnen Elektrons näherungsweise p = $10^{-4}$ beträgt, ist bei einem Cooperpaar bestehend aus zwei Elektronen von einer wesentlich geringeren Wahrscheinlichkeit auszugehen. Josephson sagte jedoch voraus, dass das Tunnelverhalten von Cooperpaaren und einzelnen Leitungselektronen das gleiche sein muss. Begründet wird dies über das sogenannte \textit{Makroskopische Quantenmodell} 

\subsection{Josephson Kontakte im Magnetfeld}

\Blindtext

\section{dc-SQUIDs}

\blindtext[3]

\subsection{Spannungszustand}

\subsection{Rauschen}

\subsection{Inbetriebnahme eines dc-SQUIDs}


\section{Resonanzen eines dc-SQUIDs}

\subsection{Parasitäre Resonanzen}

\subsection{Dämpfungsmethoden}


