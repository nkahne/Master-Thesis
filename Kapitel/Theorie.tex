\chapter{Theoretische Grundlagen}


\section{Josephson Kontakte}
Die nach \textit{Brain D. Josephson} benannten \textit{Josephson Kontakte} (engl. \textit{Josephson junctions}) bestehen aus zwei identischen Supraleitern, die schwach miteinander gekoppelt sind. Im Falle der in dieser Arbeitsgruppe hergestellten Kontakte wird eine solche Kopplung durch eine wenige nm dünne Isolationsschicht zwischen den supraleitenden Elektroden realisiert.  
\subsection{Josephson Effekt}

\subsection{Josephson Kontakte im Magnetfeld}


\section{dc-SQUIDs}

\subsection{Spannungszustand}

\subsection{Rauschen}

\subsection{Inbetriebnahme eines dc-SQUIDs}


\section{Resonanzen eines dc-SQUIDs}

\subsection{Parasitäre Resonanzen}

\subsection{Dämpfungsmethoden}


