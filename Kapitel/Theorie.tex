\chapter{Theoretische Grundlagen}


\section{Josephson Kontakte}
Die nach \textit{Brain D. Josephson} benannten \textit{Josephson Kontakte} (engl. \textit{Josephson junctions}) bestehen aus zwei identischen Supraleitern, die schwach miteinander gekoppelt sind. Im Falle der in dieser Arbeitsgruppe hergestellten Kontakte wird eine solche Kopplung durch eine wenige \r{A}ngström dünne Isolationsschicht zwischen den supraleitenden Elektroden realisiert. Aufgrund dessen werden diese auch SIS (Supraleiter-Isolator-Supraleiter) Kontakte genannt. Die so entstehende Dreischichtstruktur besteht typischerweise aus Nb/Al-Al$O_x$/Nb, wobei das Niob für die Supraleiter verwendet wird und die Isoationsschicht durch das Aluminiumoxid gegeben ist. Ein schematischer Aufbau ist in Abb. ? dargestellt. 
Wird der Kontakt nun bei sehr kalten Temperaturen ($\leq$4K) gehalten und an eine Stromquelle angeschlossen, ist entgegen der Erwartungen ein Suprastrom messbar.
        
\subsection{Josephson Effekt}

Der Stromfluss impliziert das Tunneln von Cooperpaaren, da bei diesen Temperaturen Niob überwiegend supraleitend ist  (\texttt{${T_c}$} = 9.3). Da die Tunnelwahrscheinlichkeit eines einzelnen Elektrons näherungsweise p = $10^{-4}$ beträgt, ist bei einem Cooperpaar bestehend aus zwei Elektronen von einer wesentlich geringeren Wahrscheinlichkeit auszugehen. Josephson sagte jedoch voraus, dass das Tunnelverhalten von Cooperpaaren und einzelnen Leitungselektronen das gleiche sein muss. Begründet wird dies über das sogenannte \textit{Makroskopische Quantenmodell}. \\
Hierbei liegt das Hauptaugenmerk auf der quantenmechanischen Phase $\theta$. Zum einen sind die Abstände zwischen beiden Elektronen eines Cooperpaares einige nm und damit erheblich größer als der Abstand der Cooperpaare untereinander, wodurch die Wellenfunktionen stark überlappen. Zum anderen unterliegen Cooperpaare aufgrund ihres Gesamtspins von 0 der Bose-Einstein Statistik. Somit teilen sich alle Cooperpaare den gleichen Grundzustand und als Konsequenz sind auch die Energien bzw. Zeitentwicklungen der Phasen gleich. Diese beiden Effekte führen zu dem sogenannten \textit{phase-lock}. Die Phasen benachbarter Paare gleichen sich derart an, dass diese quantenmechanische Eigenschaft nun auf makroskopischer Skala gilt. Dies hat eine makroskopische Wellenfunktion 

\begin{equation}
\Psi(\textbf{r},t) = \Psi_0(\textbf{r},t)e^{i\theta(\textbf{r},t)}
\end{equation}

zur Folge, welche alle Ladungsträger des Supraleiters beschreibt. Beide Elektronen eines Cooperpaares besitzen folglich aufgrund der geteilten Phase dieselbe Tunnelwahrscheinlichkeit wie ein einzelnes Elektron und der Suprastrom wird ermöglicht. Zahlreiche Effekte resultieren aus diesen im Jahre 1953 von Fritz London beschriebenen Modell, darunter die Quantisierung des magnetischen Flusses und der Josepshon Effekt. Diese beiden Phänomene stellen die Grundlage für Josephson Kontakte und deren Anwendungen dar. 

Die Flussquantisierung wird über das Einfangen eines externen magnetischen Flusses in einem supraleitendem Zylinder hergeleitet. Die Wellenfunktion muss nach Umrunden des Zylinders unverändert bleiben, da $\
\subsection{Josephson Kontakte im Magnetfeld}

\Blindtext

\subsection{RCSJ Modell}

\section{dc-SQUIDs}

\blindtext[3]

\subsection{Spannungszustand}

\subsection{Rauschen}

\subsection{Inbetriebnahme eines dc-SQUIDs}


\section{Resonanzen eines dc-SQUIDs}

\subsection{Parasitäre Resonanzen}

\subsection{Dämpfungsmethoden}


