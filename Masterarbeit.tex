\documentclass[12pt,a4paper]{book}

\pdfminorversion=6

% allgemeine Pakete
%\usepackage{ngerman}
\usepackage[english, ngerman]{babel} %whatever comes last defines the language of the document
\usepackage{graphicx}
\usepackage{wrapfig}
\usepackage{blindtext}
\usepackage[utf8]{inputenc}
\usepackage[rightcaption]{sidecap}
\usepackage{ifthen}
%\usepackage[small,normal,bf]{caption2}
\usepackage{dipl_new}
\usepackage{footmisc}	% für mehrere identische Fußnoten, muss vor hyperref kommen, sonst passen die Fußnoten-Links nicht
\usepackage[linkcolor=black,
	citecolor=black,
	filecolor=black,
	menucolor=black,
	urlcolor=black,
	colorlinks=true,
	pdftitle={Titel Diss D.H.},
	pdfsubject={Dissertation},
	pdfkeywords={},
	pdfauthor={Daniel Hengstler},
	]
{hyperref}
\usepackage[intlimits]{amsmath}
\usepackage{cleveref}
%\usepackage{units}
\usepackage[separate-uncertainty = true]{siunitx} % ,multi-part-units=single
\usepackage{nicefrac}
\numberwithin{equation}{chapter}
\usepackage{amssymb}
\usepackage[T1]{fontenc}
\usepackage{multirow, bigdelim, bigstrut}
% \includeonly{theorie,}
\parindent 0em
\usepackage{mathtools}
\usepackage[toc]{appendix}
\usepackage{placeins}
%\usepackage{isotope} % does not work in headings
\usepackage{rotating}
%\usepackage{gensymb} % not needed when using SIunitx
\usepackage{pdflscape} % einzelne Seiten im Querformat
\usepackage{array} % linksbündig in p-Spalten in Tabellen
%\usepackage{bbm} % for unit-matrix
%\usepackage{chemformula} % z.B. für Radikalpunkte

%\usepackage{caption}
%\captionsetup{width=0.9\linewidth}

% für SIUNITX
%\DeclareSIUnit\ph0{\text{\Phi$_{0}$}}
\DeclareSIUnit\percent{\text{\%}}
\DeclareSIUnit{\sqrthz}{\ensuremath{\sqrt{\unit{\hertz}}}}
\DeclareSIUnit{\inch}{\text{in}}
\DeclareSIUnit{\fq}{\Phi_0}	% Flussquant
\sisetup{per-mode=fraction}
% für die Bilder in der Tabelle
%\newcommand{\minifig}[2]{\raggedright #1 \newline \raisebox{-0.9\totalheight}{\includegraphics[width=1.5cm]{#2}}}	 %für Hochformat
\newcommand{\minifig}[2]{ #1 & \raisebox{-0.7\totalheight}{\includegraphics[width=1.5cm]{#2}}}	 %für Querformat

%\hyphenation{ex-peri-men-tal}
%\hyphenation{pa-ra-mag-ne-tic}
%\hyphenation{reso-lu-tion}

%\def\titeldeutsch{Entwicklung und Charakterisierung von zweidimensionalen Arrays aus metallischen magnetischen Kalorimetern für die hochauflösende Röntgenspektroskopie}
%\def\titelenglisch{Development and characterization of two-dimensional metallic magnetic calorimeter arrays for the high-resolution X-ray spectroscopy}

% Eigentliche Arbeit
%\begin{document}
	
	% Keine Kopf- oder Fu?zeilen
%	\pagestyle{empty}
%	\hypersetup{pageanchor=false}
	% include erzeugt ein newpage und fügt danach die angegebene Datei in den Quelltext ein. Die eingefügte Datei wird als normaler Quelltext mitverarbeitet und sollte daher nur aus einer Folge von LaTeX-Befehlen bestehen (ohne Vorspann etc.)
	%\include{kipcover}
	%\include{titel}
	%\pagestyle{empty}
	%\thispagestyle{empty}

	\begin{titlepage}
		\setcounter{page}{-2}
		\setlength{\textheight}{28cm}
		\setlength{\topmargin}{-20mm}
		
		% -------------- ENGLISH ABSTRACT ----------- %
		\begin{center}
			\fbox{\rule{0.5cm}{0cm}\parbox{14.9cm}{\bigskip
					\small
					\noindent
This thesis describes the design and development of a current-sensor dc-SQUID, optimized for the readout of metallic magnetic calorimeter based particle detectors. Maximizing the energy resolution of the detector is achieved by minimizing the intrinsic noise of the SQUID and by matching the input inductance of the SQUID with the inductance of the detector. This was realized by adding a second turn to the input coil of a previously produced SQUID to match the pickup coil inductance $L_{\rm p}=\qty{6.65}{\pH}$ of the maXs100 detector, which is currently being developed in this working group. To mitigate resonant structures in the SQUID dynamics, two new damping techniques were employed. These consisted of gold layers sputtered both atop the feed lines and between SQUID loop and input coil, with the latter representing a lossy microstrip line. Successful damping was achieved by the former as it resulted in substantially smoother current-voltage characteristics. The noise measurements of the new SQUID with various combinations of the damping scheme were performed at a temperature of $\qty{10}{\milli\kelvin}$. These yielded up to $\sqrt{S_{\Phi_{\rm s}, \rm w}}=\qty{0.22}{\micro\fq\per\sqrthz}$ for the frequency independent contribution, whereas the $1/f$ component was as low as $\sqrt{S_{\Phi_{\rm s}, 1/f}}=\qty{2.0}{\micro\fq\per\sqrthz}$, thus being comparable to previous low-noise SQUIDs from this working group. %The intrinsic and extrinsic energy sensitivities were accordingly low, where values of $\epsilon_{\rm s, w}=1.44\, h$ and $\epsilon_{{\rm s}, 1/f}=119.2\, h$ had been achieved for the intrinsic energy sensitivity, while the extrinsic energy sensitivity regarding the maXs100 detector yielded up to $\epsilon_{\rm p, w}=23.11\, h$ and $\epsilon_{{\rm p}, 1/f}=1910\, h$.



%This thesis discusses the optimisation of the $\mathrm{Nb}/\mathrm{Al}$-$\mathrm{AlO}_\mathrm{x}/\mathrm{Nb}$ tri-layer deposition for the fabrication of cross-type based Josephson tunnel junctions. Josephson tunnel junctions (JJs) are the core elements of various superconducting devices such as qubits or superconducting quantum interference devices (SQUIDs). The cross-type JJ geometry is motivated by the reduction of the junction area as well as parasitic capacities compared to the commonly used window-type geometry. The cross-like design removes parasitic effects with the additional benefit of simple and time efficient fabrication steps and relaxed alignment requirements during micro-fabrication. To ensure a reliable wafer-scale fabrication that yields JJs with a reproducible and uniform high quality, the in-house sputter-deposited niobium layers in a new sputtering system had to be investigated regarding their physical properties including the measurement of the critical temperature $T_{\mathrm{c}}$, the stress of the niobium film and junction specific quality features. The parameters for the magnetron sputtering, like the Ar pressure and power of the sputtering source, were optimised accordingly. This resulted in the successful fabrication of high quality cross-type JJs with reduced area sizes by at least a factor 4 and homogeneously distributed quality parameters on wafer-scale, which provides a basis for further developments of cross-type based dc-SQUIDs.


%Very often window-type JJs are used, in which the JJ area is defined by contacts through windows in an insulating layer. The size of the JJ is thus limited by alignment accuracy of the lithographic fabrication step, therfore causing the need for low critical current densities $j_{\mathrm{c}}$ and in turn long oxidation times. 

% 
%We present the optimization of our cross-type $\mathrm{Nb/Al}$-$\mathrm{AlO_x/Nb}$ junctions, for which quality checks of our in-house sputter-deposited niobium films and oxidized aluminium layers were carried out, including the measurement of the critical temperature $T_{\mathrm{c}}$, stress of the niobium film and junction specific quality features. The parameters for the magnetron sputtering, like the Ar pressure and power of the sputtering source, were optimised accordingly. This resulted in the fabrication of high quality cross-type JJs with homogeneously distributed quality parameters on wafer-scale, which provides a basis for further developments of cross-type based dc-SQUIDs.					
					\smallskip}\rule{0.5cm}{0cm}}
		\end{center}
		% ------------ END ENGLISH ABSTRACT ----------- %
		\vfill
		% -------------- GERMAN ABSTRACT ----------- %
		\begin{center}
			\fbox{\rule{0.5cm}{0cm}\parbox{14.9cm}{\smallskip
					\begin{center}
						Design, Herstellung und Charakterisierung von Stromsensor-dc-SQUIDs mit Induktivitätsanpassung zur Auslese von Metallischen Magnetischen Kalorimetern
						%Optimierung der Induktivitätsanpassung von Stromsensor-dc-SQUIDs zur Auslese metallischer magnetischer Kalorimeter   
					\end{center}					
					\selectlanguage{ngerman}
					\small
					\noindent
In der vorliegenden Arbeit werden die durchgeführten Methoden zur Optimierung der $\mathrm{Nb}/\mathrm{Al}$-$\mathrm{AlO}_\mathrm{x}/\mathrm{Nb}$ Dreischicht-Deponierung für die Herstellung von Josephson-Tunnelkontakten auf Basis einer kreuzförmigen Geometrie vorgestellt. Josephson-Tunnelkontakte (JJs) sind die Kernelemente verschiedener supraleitender elektronischer Bauelemente. Die Verwendung der kreuzförmigen JJ-Geometrie ist durch die Verringerung der Kontaktfläche sowie der parasitären Kapazitäten im Vergleich zu der üblicherweise verwendeten Fenstertyp-Geometrie motiviert. Das kreuzförmige Design beseitigt parasitäre Effekte und bietet zudem den Vorteil einfacher und zeitsparender Herstellungsschritte. Um eine zuverlässige Fabrikation auf Wafer-Skala zu gewährleisten, die JJs mit reproduzierbarer und homogen hoher Qualität liefert, mussten die in dem neuen institutsinternen Sputtersystem deponierten Niobschichten auf ihre physikalischen Eigenschaften hin untersucht werden, einschließlich der Messung der kritischen Temperatur $T_{\mathrm{c}}$, der Verspannung des Niobfilms und der JJ-spezifischen Qualitätsmerkmale. Die Parameter für das Magnetron-Sputtern, wie der Ar-Druck und die Leistung der Sputterquelle, wurden entsprechend optimiert. Dies führte zur erfolgreichen Herstellung von qualitativ hochwertigen Kreuztyp-Kontakten mit einer um mindestens den Faktor 4 reduzierten Flächengröße und homogen verteilten Qualitätsparametern auf Wafer-Skala, was eine Grundlage für weitere Entwicklungen von Kreuztyp-Kontakt basierten dc-SQUIDs liefert.

%Josephson-Tunnelkontakte sind die Grundelemente vieler supraleitender elektronischer Bauelemente wie Qubits oder ''Superconducting Quantum Interference Devices'' (SQUIDs). Da für viele Anwendungen eine große Anzahl von Josephson-Kontakten benötigt wird, ist ein zuverlässiger Herstellungsprozess auf Wafer-Skala erforderlich, der Josephson-Kontakte mit reproduzierbarer und einheitlich hoher Qualität liefert. Sehr häufig werden Fenstertyp-Kontakte verwendet, bei denen die Josephson-Kontaktfläche durch Fenster in einer Isolierschicht definiert ist. Die Größe der Kontakte ist somit durch die Ausrichtungsgenauigkeit des lithographischen Fabrikationsschritt begrenzt, was zu niedrigen kritischen Stromdichten $j_{\mathrm{c}}$ und damit zu langen Oxidationszeiten führt. Der Übergang zu kreuzförmigen Josephson-Kontakten ist durch die Reduzierung der Kontaktfläche und der Reduzierung der parasitären Kapazitäten motiviert. Das kreuzförmige Design unterdrückt oder beseitigt parasitäre Effekte mit dem zusätzlichen Vorteil einfacher und zeitsparender Herstellungsschritte. 
%Wir stellen die Optimierung unserer kreuzförmigen $\mathrm{Nb/Al}$-$\mathrm{AlO_x/Nb}$-Kontakte vor, für die Qualitätsprüfungen unserer Instituts-internen sputterabgeschiedenen Niobschichten und oxidierten Aluminiumschichten durchgeführt wurden, einschließlich der Messung der kritischen Temperatur $T_{\mathrm{c}}$, der Verspannung der Niobschicht und der spezifischen Qualitätsmerkmale der Josephson-Kontakte. %Die Parameter für das Magnetronsputtern, wie der Ar-Druck und die Leistung der Sputterquelle, wurden entsprechend optimiert. % Das Ergebnis war die Herstellung von qualitativ hochwertigen Kreuztyp-Kontakten mit homogen verteilten Qualitätsparametern auf Wafer-Skala, was eine Grundlage für weitere Entwicklungen von Cross-Type-basierten dc-SQUIDs liefert.

					\smallskip}\rule{0.5cm}{0cm}}
		\end{center}
		% ------------ END GERMAN ABSTRACT ----------- %
		%
		%
		\selectlanguage{german}
		%\vfill
	\end{titlepage}
	\setcounter{page}{-1}
	\rule{0cm}{17cm}
	\setlength{\textheight}{22.5cm}
	\setlength{\topmargin}{-3mm}
	\normalsize
	
%	\hypersetup{pageanchor=true}

\begin{document}

{
	\pagestyle{headings}	
	\renewcommand{\baselinestretch}{1.4}	
	\pagenumbering{roman}
	\normalsize
	\tableofcontents
	\vfill\eject 
	\ifthenelse{\isodd{\value{page}}}{}{\rule{0cm}{15cm}\vfill\eject}
}

\pagenumbering{arabic}
\pagestyle{headings}   %damit genau wie beim toc die Kopfzeile das schöne Design aus dipl_new übernimmt

\input{Kapitel/theorie}
\chapter{Metallisch Magnetische Kalorimeter} 
















\end{document}