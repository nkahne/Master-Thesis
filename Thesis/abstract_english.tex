\noindent
This thesis describes the design and development of a current-sensor dc-SQUID, optimized for the readout of metallic magnetic calorimeter based particle detectors. Maximizing the energy resolution of the detector is achieved by minimizing the intrinsic noise of the SQUID and by matching the input inductance of the SQUID with the inductance of the detector. Our new design features a two-turn input coil to better match the pickup coil inductance $L_{\rm p}=\qty{6.65}{\pH}$ of the maXs100 detector. To mitigate resonant structures in the SQUID dynamics, two new damping techniques are employed. 
%These consisted of gold layers sputtered both atop the feed lines and between SQUID loop and input coil, with the latter representing a lossy microstrip line. 
On the one hand, we inductively damp the feed lines with a galvanically isolated gold layer. On the other, we fabricated the input coil as a Au/Nb bilayer, forming a lossy microstrip line.  
Successful damping was achieved by the former as it resulted in substantially smoother current-voltage characteristics. The noise measurements of the new SQUID with various combinations of the damping scheme were performed at a temperature of $\qty{10}{\milli\kelvin}$. 
%These yielded up to $\sqrt{S_{\Phi_{\rm s}, \rm w}}=\qty{0.22}{\micro\fq\per\sqrthz}$ for the frequency independent contribution, whereas the $1/f$ component was as low as $\sqrt{S_{\Phi_{\rm s}, 1/f}}=\qty{2.0}{\micro\fq\per\sqrthz}$, thus being comparable to previous low-noise SQUIDs from this working group. %The intrinsic and extrinsic energy sensitivities were accordingly low, where values of $\epsilon_{\rm s, w}=1.44\, h$ and $\epsilon_{{\rm s}, 1/f}=119.2\, h$ had been achieved for the intrinsic energy sensitivity, while the extrinsic energy sensitivity regarding the maXs100 detector yielded up to $\epsilon_{\rm p, w}=23.11\, h$ and $\epsilon_{{\rm p}, 1/f}=1910\, h$.
These yield results comparable to previous low-noise SQUIDs from this working group with noise contributions as low as  $\sqrt{S_{\Phi_{\rm s}, \rm w}}=\qty{0.22}{\micro\fq\per\sqrthz}$ for the white noise and $\sqrt{S_{\Phi_{\rm s}, 1/f}}=\qty{2.0}{\micro\fq\per\sqrthz}$ for the $1/f$ noise contribution. The adjustment of the input coil to $L_{\rm i}=\qty{6.4}{\nH}$ leads to a larger coupling of $\Delta\Phi_{\rm s}/\Delta\Phi=\qty{2.25}{\percent}$. The new design thus represents an improved current-sensor SQUID for the maXs100 detector readout.


%This thesis discusses the optimisation of the $\mathrm{Nb}/\mathrm{Al}$-$\mathrm{AlO}_\mathrm{x}/\mathrm{Nb}$ tri-layer deposition for the fabrication of cross-type based Josephson tunnel junctions. Josephson tunnel junctions (JJs) are the core elements of various superconducting devices such as qubits or superconducting quantum interference devices (SQUIDs). The cross-type JJ geometry is motivated by the reduction of the junction area as well as parasitic capacities compared to the commonly used window-type geometry. The cross-like design removes parasitic effects with the additional benefit of simple and time efficient fabrication steps and relaxed alignment requirements during micro-fabrication. To ensure a reliable wafer-scale fabrication that yields JJs with a reproducible and uniform high quality, the in-house sputter-deposited niobium layers in a new sputtering system had to be investigated regarding their physical properties including the measurement of the critical temperature $T_{\mathrm{c}}$, the stress of the niobium film and junction specific quality features. The parameters for the magnetron sputtering, like the Ar pressure and power of the sputtering source, were optimised accordingly. This resulted in the successful fabrication of high quality cross-type JJs with reduced area sizes by at least a factor 4 and homogeneously distributed quality parameters on wafer-scale, which provides a basis for further developments of cross-type based dc-SQUIDs.


%Very often window-type JJs are used, in which the JJ area is defined by contacts through windows in an insulating layer. The size of the JJ is thus limited by alignment accuracy of the lithographic fabrication step, therfore causing the need for low critical current densities $j_{\mathrm{c}}$ and in turn long oxidation times. 

% 
%We present the optimization of our cross-type $\mathrm{Nb/Al}$-$\mathrm{AlO_x/Nb}$ junctions, for which quality checks of our in-house sputter-deposited niobium films and oxidized aluminium layers were carried out, including the measurement of the critical temperature $T_{\mathrm{c}}$, stress of the niobium film and junction specific quality features. The parameters for the magnetron sputtering, like the Ar pressure and power of the sputtering source, were optimised accordingly. This resulted in the fabrication of high quality cross-type JJs with homogeneously distributed quality parameters on wafer-scale, which provides a basis for further developments of cross-type based dc-SQUIDs.