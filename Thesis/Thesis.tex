\documentclass[12pt,a4paper]{book}

\pdfminorversion=6

% allgemeine Pakete
%\usepackage{ngerman}
\usepackage[ngerman,english]{babel} %whatever comes last defines the language of the document
\usepackage{graphicx}
\usepackage{wrapfig}
\usepackage{blindtext}
\usepackage[utf8]{inputenc}
\usepackage[rightcaption]{sidecap}
\usepackage{ifthen}
%\usepackage[small,normal,bf]{caption2}
\usepackage{dipl2}
\usepackage{footmisc}	% für mehrere identische Fußnoten, muss vor hyperref kommen, sonst passen die Fußnoten-Links nicht
\usepackage[linkcolor=black,
	citecolor=black,
	filecolor=black,
	menucolor=black,
	urlcolor=black,
	colorlinks=true,
	pdftitle={Titel MA N.K.},
	pdfsubject={Master Thesis},
	pdfkeywords={},
	pdfauthor={Nicolas Kahne},
	]
{hyperref}
\usepackage[intlimits]{amsmath}
\usepackage{cleveref}
%\usepackage{units}
\usepackage[separate-uncertainty = true]{siunitx} % ,multi-part-units=single, exponent-product = \cdot
\usepackage{nicefrac}
\numberwithin{equation}{chapter}
\usepackage{amssymb}
\usepackage[T1]{fontenc}
\usepackage{multirow, bigdelim, bigstrut}
% \includeonly{theorie,}
\parindent 0em
\usepackage{mathtools}
\usepackage[toc]{appendix}
\usepackage{placeins}
%\usepackage{isotope} % does not work in headings
\usepackage{rotating}
%\usepackage{gensymb} % not needed when using SIunitx
\usepackage{pdflscape} % einzelne Seiten im Querformat
\usepackage{booktabs}
\usepackage{array} % linksbündig in p-Spalten in Tabellen
%\usepackage{bbm} % for unit-matrix
%\usepackage{chemformula} % z.B. für Radikalpunkte
\usepackage{bm}
%\usepackage{caption}
%\captionsetup{width=0.9\linewidth}

%\usepackage{caption}
%\usepackage{fontspec}

%\newfontfamily\arial{Arial}

% für SIUNITX
%\DeclareSIUnit\ph0{\text{\Phi$_{0}$}}
\DeclareSIUnit\percent{\text{\%}}
\DeclareSIUnit{\sqrthz}{\ensuremath{\sqrt{\unit{\hertz}}}}
\DeclareSIUnit{\inch}{\text{in}}
\DeclareSIUnit{\fq}{\Phi_0}	% Flussquant
\sisetup{per-mode=symbol}
% für die Bilder in der Tabelle
%\newcommand{\minifig}[2]{\raggedright #1 \newline \raisebox{-0.9\totalheight}{\includegraphics[width=1.5cm]{#2}}}	 %für Hochformat
\newcommand{\minifig}[2]{ #1 & \raisebox{-0.7\totalheight}{\includegraphics[width=1.5cm]{#2}}}	 %für Querformat

%\hyphenation{ex-peri-men-tal}
%\hyphenation{pa-ra-mag-ne-tic}
%\hyphenation{reso-lu-tion}

%\def\titeldeutsch{Entwicklung und Charakterisierung von zweidimensionalen Arrays aus metallischen magnetischen Kalorimetern für die hochauflösende Röntgenspektroskopie}
%\def\titelenglisch{Development and characterization of two-dimensional metallic magnetic calorimeter arrays for the high-resolution X-ray spectroscopy}

% Eigentliche Arbeit
\begin{document}
	
	% Keine Kopf- oder Fu?zeilen
	\pagestyle{empty}
	\hypersetup{pageanchor=false}
	% include erzeugt ein newpage und fügt danach die angegebene Datei in den Quelltext ein. Die eingefügte Datei wird als normaler Quelltext mitverarbeitet und sollte daher nur aus einer Folge von LaTeX-Befehlen bestehen (ohne Vorspann etc.)
	\include{kipcover}
	\begin{titlepage}
	
	\thispagestyle{empty}
	\linespread{1.6}     % 1.3= 1.5-zeilig, 1.6=2-zeilig
	\large
	\begin{center}
		{\Huge Faculty of Physics and Astronomy}\\
		{\LARGE Ruprecht-Karls-University Heidelberg}\\
		\vspace{2cm}
		\vfill
		\linespread{1.3}
		\textsc{Master's Thesis}\\ 
		\vspace{-0.2cm}
		in Physics \\
		\bigskip
		submitted by\\
		\vspace{-0.2cm}
		\textbf{Nicolas Robert Kahne}\\
		\vspace{-0.2cm}
		born in Kaiserslautern\\
		\bigskip
		February 2024
		%
	\end{center}
\end{titlepage}


\begin{titlepage}
	\thispagestyle{empty}
	\linespread{1.3}     % 1.3= 1.5-zeilig, 1.6=2-zeilig
	%\large
	\begin{center}
		%\vspace*{1cm}
		%\begin{center}
		\bf
		\Large
		Optimization of inductance matching for current-sensor dc-SQUIDs to readout metallic magnetic calorimeters
		%Placeholder\\
		%Placeholder\\
		%Placeholder
		%\end{center}
		\rm	
		\large
		\vfill
		This Master thesis was carried out by Nicolas Robert Kahne \\
		at the Kirchhoff-Institute for Physics\\
		under the supervision of\\
		\textbf{Prof.\ Dr.\ C. Enss}
	\end{center}
\end{titlepage}


% Leerseite

%\begin{titlepage}
%	\setcounter{page}{-3}
%	\rule{0cm}{15cm}
%\end{titlepage}
	\pagestyle{empty}
	\thispagestyle{empty}

	\begin{titlepage}
		\setcounter{page}{-2}
		\setlength{\textheight}{28cm}
		\setlength{\topmargin}{-20mm}
		
		% -------------- ENGLISH ABSTRACT ----------- %
		\begin{center}
			\fbox{\rule{0.5cm}{0cm}\parbox{14.9cm}{\bigskip
					\small
					\noindent
This thesis describes the design and development of a current-sensor dc-SQUID, optimized for the readout of metallic magnetic calorimeter based particle detectors. Maximizing the energy resolution of the detector is achieved by minimizing the intrinsic noise of the SQUID and by matching the input inductance of the SQUID with the inductance of the detector. This was realized by adding a second turn to the input coil of a previously produced SQUID to match the pickup coil inductance $L_{\rm p}=\qty{6.65}{\pH}$ of the maXs100 detector, which is currently being developed in this working group. To mitigate resonant structures in the SQUID dynamics, two new damping techniques were employed. These consisted of gold layers sputtered both atop the feed lines and between SQUID loop and input coil, with the latter representing a lossy microstrip line. Successful damping was achieved by the former as it resulted in substantially smoother current-voltage characteristics. The noise measurements of the new SQUID with various combinations of the damping scheme were performed at a temperature of $\qty{10}{\milli\kelvin}$. These yielded up to $\sqrt{S_{\Phi_{\rm s}, \rm w}}=\qty{0.22}{\micro\fq\per\sqrthz}$ for the frequency independent contribution, whereas the $1/f$ component was as low as $\sqrt{S_{\Phi_{\rm s}, 1/f}}=\qty{2.0}{\micro\fq\per\sqrthz}$, thus being comparable to previous low-noise SQUIDs from this working group. %The intrinsic and extrinsic energy sensitivities were accordingly low, where values of $\epsilon_{\rm s, w}=1.44\, h$ and $\epsilon_{{\rm s}, 1/f}=119.2\, h$ had been achieved for the intrinsic energy sensitivity, while the extrinsic energy sensitivity regarding the maXs100 detector yielded up to $\epsilon_{\rm p, w}=23.11\, h$ and $\epsilon_{{\rm p}, 1/f}=1910\, h$.



%This thesis discusses the optimisation of the $\mathrm{Nb}/\mathrm{Al}$-$\mathrm{AlO}_\mathrm{x}/\mathrm{Nb}$ tri-layer deposition for the fabrication of cross-type based Josephson tunnel junctions. Josephson tunnel junctions (JJs) are the core elements of various superconducting devices such as qubits or superconducting quantum interference devices (SQUIDs). The cross-type JJ geometry is motivated by the reduction of the junction area as well as parasitic capacities compared to the commonly used window-type geometry. The cross-like design removes parasitic effects with the additional benefit of simple and time efficient fabrication steps and relaxed alignment requirements during micro-fabrication. To ensure a reliable wafer-scale fabrication that yields JJs with a reproducible and uniform high quality, the in-house sputter-deposited niobium layers in a new sputtering system had to be investigated regarding their physical properties including the measurement of the critical temperature $T_{\mathrm{c}}$, the stress of the niobium film and junction specific quality features. The parameters for the magnetron sputtering, like the Ar pressure and power of the sputtering source, were optimised accordingly. This resulted in the successful fabrication of high quality cross-type JJs with reduced area sizes by at least a factor 4 and homogeneously distributed quality parameters on wafer-scale, which provides a basis for further developments of cross-type based dc-SQUIDs.


%Very often window-type JJs are used, in which the JJ area is defined by contacts through windows in an insulating layer. The size of the JJ is thus limited by alignment accuracy of the lithographic fabrication step, therfore causing the need for low critical current densities $j_{\mathrm{c}}$ and in turn long oxidation times. 

% 
%We present the optimization of our cross-type $\mathrm{Nb/Al}$-$\mathrm{AlO_x/Nb}$ junctions, for which quality checks of our in-house sputter-deposited niobium films and oxidized aluminium layers were carried out, including the measurement of the critical temperature $T_{\mathrm{c}}$, stress of the niobium film and junction specific quality features. The parameters for the magnetron sputtering, like the Ar pressure and power of the sputtering source, were optimised accordingly. This resulted in the fabrication of high quality cross-type JJs with homogeneously distributed quality parameters on wafer-scale, which provides a basis for further developments of cross-type based dc-SQUIDs.					
					\smallskip}\rule{0.5cm}{0cm}}
		\end{center}
		% ------------ END ENGLISH ABSTRACT ----------- %
		\vfill
		% -------------- GERMAN ABSTRACT ----------- %
		\begin{center}
			\fbox{\rule{0.5cm}{0cm}\parbox{14.9cm}{\smallskip
					\begin{center}
						Design, Herstellung und Charakterisierung von Stromsensor-dc-SQUIDs mit Induktivitätsanpassung zur Auslese von Metallischen Magnetischen Kalorimetern
						%Optimierung der Induktivitätsanpassung von Stromsensor-dc-SQUIDs zur Auslese metallischer magnetischer Kalorimeter   
					\end{center}					
					\selectlanguage{ngerman}
					\small
					\noindent
In der vorliegenden Arbeit werden die durchgeführten Methoden zur Optimierung der $\mathrm{Nb}/\mathrm{Al}$-$\mathrm{AlO}_\mathrm{x}/\mathrm{Nb}$ Dreischicht-Deponierung für die Herstellung von Josephson-Tunnelkontakten auf Basis einer kreuzförmigen Geometrie vorgestellt. Josephson-Tunnelkontakte (JJs) sind die Kernelemente verschiedener supraleitender elektronischer Bauelemente. Die Verwendung der kreuzförmigen JJ-Geometrie ist durch die Verringerung der Kontaktfläche sowie der parasitären Kapazitäten im Vergleich zu der üblicherweise verwendeten Fenstertyp-Geometrie motiviert. Das kreuzförmige Design beseitigt parasitäre Effekte und bietet zudem den Vorteil einfacher und zeitsparender Herstellungsschritte. Um eine zuverlässige Fabrikation auf Wafer-Skala zu gewährleisten, die JJs mit reproduzierbarer und homogen hoher Qualität liefert, mussten die in dem neuen institutsinternen Sputtersystem deponierten Niobschichten auf ihre physikalischen Eigenschaften hin untersucht werden, einschließlich der Messung der kritischen Temperatur $T_{\mathrm{c}}$, der Verspannung des Niobfilms und der JJ-spezifischen Qualitätsmerkmale. Die Parameter für das Magnetron-Sputtern, wie der Ar-Druck und die Leistung der Sputterquelle, wurden entsprechend optimiert. Dies führte zur erfolgreichen Herstellung von qualitativ hochwertigen Kreuztyp-Kontakten mit einer um mindestens den Faktor 4 reduzierten Flächengröße und homogen verteilten Qualitätsparametern auf Wafer-Skala, was eine Grundlage für weitere Entwicklungen von Kreuztyp-Kontakt basierten dc-SQUIDs liefert.

%Josephson-Tunnelkontakte sind die Grundelemente vieler supraleitender elektronischer Bauelemente wie Qubits oder ''Superconducting Quantum Interference Devices'' (SQUIDs). Da für viele Anwendungen eine große Anzahl von Josephson-Kontakten benötigt wird, ist ein zuverlässiger Herstellungsprozess auf Wafer-Skala erforderlich, der Josephson-Kontakte mit reproduzierbarer und einheitlich hoher Qualität liefert. Sehr häufig werden Fenstertyp-Kontakte verwendet, bei denen die Josephson-Kontaktfläche durch Fenster in einer Isolierschicht definiert ist. Die Größe der Kontakte ist somit durch die Ausrichtungsgenauigkeit des lithographischen Fabrikationsschritt begrenzt, was zu niedrigen kritischen Stromdichten $j_{\mathrm{c}}$ und damit zu langen Oxidationszeiten führt. Der Übergang zu kreuzförmigen Josephson-Kontakten ist durch die Reduzierung der Kontaktfläche und der Reduzierung der parasitären Kapazitäten motiviert. Das kreuzförmige Design unterdrückt oder beseitigt parasitäre Effekte mit dem zusätzlichen Vorteil einfacher und zeitsparender Herstellungsschritte. 
%Wir stellen die Optimierung unserer kreuzförmigen $\mathrm{Nb/Al}$-$\mathrm{AlO_x/Nb}$-Kontakte vor, für die Qualitätsprüfungen unserer Instituts-internen sputterabgeschiedenen Niobschichten und oxidierten Aluminiumschichten durchgeführt wurden, einschließlich der Messung der kritischen Temperatur $T_{\mathrm{c}}$, der Verspannung der Niobschicht und der spezifischen Qualitätsmerkmale der Josephson-Kontakte. %Die Parameter für das Magnetronsputtern, wie der Ar-Druck und die Leistung der Sputterquelle, wurden entsprechend optimiert. % Das Ergebnis war die Herstellung von qualitativ hochwertigen Kreuztyp-Kontakten mit homogen verteilten Qualitätsparametern auf Wafer-Skala, was eine Grundlage für weitere Entwicklungen von Cross-Type-basierten dc-SQUIDs liefert.

					\smallskip}\rule{0.5cm}{0cm}}
		\end{center}
		% ------------ END GERMAN ABSTRACT ----------- %
		%
		%
		\selectlanguage{german}
		%\vfill
	\end{titlepage}
	\setcounter{page}{-1}
	\rule{0cm}{17cm}
	\setlength{\textheight}{22.5cm}
	\setlength{\topmargin}{-3mm}
	\normalsize
	
	\hypersetup{pageanchor=true}

%\begin{document}

{
	\pagestyle{headings}	
	\renewcommand{\baselinestretch}{1.4}	
	\pagenumbering{roman}
	\normalsize
	\tableofcontents
	\vfill\eject 
	\ifthenelse{\isodd{\value{page}}}{}{\rule{0cm}{15cm}\vfill\eject}
}

\pagenumbering{arabic}
\pagestyle{headings}   %damit genau wie beim toc die Kopfzeile das schöne Design aus dipl_new übernimmt

\chapter{Introduction}

Particle detectors requiring a high energy-independent resolution combined with a large bandwidth have been successfully realized in the form of cryogenic metallic magnetic calorimeters (MMC) \cite{Enss2005a}. These high-precision low-temperature detectors are used in a broad range of applications, such as for mass spectrometry of heavy particles \cite{Hengstler2017}, spectroscopy of highly charged heavy ions \cite{Gamer2019} or the investigation of the mass of the electron-neutrino \cite{Gastaldo2017}. \\
The working principle is based on the conversion of the energy of an incoming particle into a magnetic flux change. This is realized by the use of a paramagnetic sensor in a weak magnetic field, which experiences a temperature change upon the absorption of an incoming particle through an absorber that is thermally coupled to the sensor. The temperature increase is accompanied by a change in magnetization of the paramagnet, which creates a flux change that is read out by a direct current superconducting quantum interference device (dc-SQUID). SQUIDs represent highly sensitive state of the art magnetometers with a broad bandwidth. The SQUID can couple to the MMC via a flux transformer setup, where its input coil is connected to the pickup coil of the MMC experiencing the initial flux change $\Delta\Phi$. The induced current change creates a flux change $\Delta\Phi_{\rm s}$ in the input coil that couples into the superconducting SQUID loop, which is intersected by two Josephson junctions. A current-biased dc-SQUID produces a finite voltage across both arms of the loop upon a magnetic flux change. The resulting non-linear current-voltage-characteristics require a broadband FLL feedback readout setup to linearize and amplify the output voltage. To reduce the noise contribution of the feedback electronics, a secondary SQUID is implemented to act as a low temperature preamplifier. The apparent noise in the current-sensor SQUID, however, couples into the MMC, which adds to the intrinsic noise of the detector. \\
For the detectors developed in this working group, this added noise dominates the flux noise spectrum above a few kHz, whereas $1/f$ noise of doped Erbium atoms and thermodynamic energy fluctuations are mostly responsible for low frequency noise \cite{Kempf2018}. The energy resolution $\Delta E_{\rm FWHM}$, which can be described by the signal to noise ratio, is therefore negatively affected by the SQUID readout chain. To quantify this effect, we use the extrinsic energy sensitivity $\epsilon_{\rm p}$ given by the intrinsic noise of the SQUID $S_{\Phi_{\rm s}}$ and the inverse of the flux-to-flux coupling $\Delta\Phi_{\rm s}/\Delta\Phi$. \\
The objective of this thesis was to optimize the coupling between the SQUID and the X-ray detector maXs100 developed in this working group. This was realized by adding a second turn to the input coil of the detector SQUID to match its inductance with the one of the pickup coil, as this maximizes the coupling  $\Delta\Phi_{\rm s}/\Delta\Phi$. Additionally, two new damping strategies using gold layers were employed to mitigate possible resonances in the IVCs. We further investigated the resonance and noise behavior to assess that the new design leads to an overall reduction of $\epsilon_{\rm p}$. \\

In chapter \ref{ch_theo} we first introduce the theoretical framework of Josephson junctions that is needed to describe the working principle of a dc-SQUID. As such, we cover macroscopic quantum effects such as the Josephson effect and flux quantization. These explain the characteristic properties of Josephson junctions, which we realized as SIS tunnel contacts. To motivate the use of SQUIDs, we discuss the behavior of the junction in an external magnetic field. Following this, we cover the general aspects of dc-SQUIDs including their characteristics and noise behavior. Lastly, we discuss parasitic $LC$ resonances arising from geometric properties of the SQUID circuit.

Chapter \ref{ch_methods} covers the realization of practical SQUIDs more in depth and how their parameters are chosen to optimize the performance. We demonstrate the importance of using a second stage SQUID representing a low temperature preamplifier, as it significantly improves the signal to noise ratio. The last part of the chapter gives a short introduction to MMCs, summarizing their core features. We follow with a brief overview of the extrinsic energy sensitivity regarding the SQUID-based readout, which motivates the adaption of the input coil to the pickup coil of the detector, thereby greatly improving the coupling between SQUID and detector.   

In chapter \ref{ch_SQUIDdesign} we present the new SQUID design based on the SQUID developed in \cite{Bauer2022}, where a second turn has been added to the input coil with the premise to better match the pickup coil inductance $L_{\rm p}=\qty{6.65}{\nH}$ of the maXs100 detector. Further optimal design parameters are discussed which ensure minimal flux noise as well as smooth current-voltage-characteristics (IVCs). To mitigate the influence of resonances, various damping techniques are applied, such as shunting the $LC$ circuits with resistors. We discuss new damping approaches involving inductively coupled gold layers distributed over the feed lines as well as placing a gold layer beneath the input coil, which represents a lossy microstrip line.  

The measured characteristic properties of the new SQUID design are presented in chapter \ref{ch_results}. Following this, we discuss the measurement of the input coil inductance and compare it to the expected value. To gain insight into the resonance behavior, we measured the IVC's of various new SQUID variants containing different combinations of the damping schemes described in chapter 3. Additionally, a SQUID without input coil has been tested for comparison to help identify input coil related resonances. The chapter ends with the discussion of the noise measurements, which was conducted with variants containing the lossy microstrip input coil, both with and without damped feed lines. Lastly, the measured noise values are used to obtain intrinsic and extrinsic energy sensitivities, which are compared among the measured variants. We conclude that the new design is better suited for the readout of the maXs100 detector and propose possible improvements for future works.    
\chapter{Theoretical Background}

This chapter provides a short introduction into Josephson junctions and their role in dc-SQUIDs\footnote{\textbf{d}irect \textbf{c}urrent \textbf{S}uperconducting \textbf{QU}antum \textbf{I}nterference \textbf{D}evice}, which will be the main focus of this thesis. We start with a brief overview on macroscopic quantum phenomena such as the Josephson effect and explain the general working principle of superconductor-isolator-superconductor (SIS) tunnel contacts, followed by a summary of their basic properties. They form the theoretical framework to describe SQUIDs, which are developed in this group and optimized within the scope of this thesis. Lastly, we will take a closer look into their resonance behavior and investigate different solution approaches. 

\section{Josephson junctions}


The \textit{Josephson junctions} named after Brian D. Josephson consist of two identical superconductors weakly coupled to each other. In the case of the junctions produced in this working group, such coupling is realized through a few nm thin insulating layer between the superconducting electrodes. Consequently, they are referred to as SIS (Superconductor-Insulator-Superconductor) junctions. The resulting trilayer structure typically consists of Nb/Al-Al$\rm O_x$/Nb, with niobium being used for the superconductors and the insulating layer being provided by the aluminum oxide. A schematic structure is shown in figure \ref{abb:fig:JJschem}. %When the junction is maintained at cold temperatures ($T\leq \qty{4}{\kelvin}$) and connected to a current source a supercurrent is measurable.
By connecting the tunnel junction to a current source they exhibit a non-trivial current-voltage behavior, which will be covered in the following. 


\figurecenter {b!}
{width=\textwidth}
{../Figures/jj_schematic}
%{7.8cm}
{0cm}
{Schematic of a Josephson (SIS) junction. Both superconducting electrodes $\textbf{\textit{S}}_\textbf{1}$ and $\textbf{\textit{S}}_\textbf{2}$ are weakly coupled with each other through a thin tunnel barrier \textbf{\textit{I}}. \bm{$\theta_{\rm 1}$} and \bm{$\theta_{\rm 2}$} represent the macroscopic phases of each superconductor.} 
{fig:JJschem}

%$\theta$ %$\textbf{\textit{\theta}_\textbf{1}}$ and  $\textbf{\textit{\theta}_\textbf{2}}$ represent the macroscopic phases of each superconductor.}
        
\subsection{Josephson effect}\label{subchap_Jeffect}

According to the BCS theory developed by Bardeen, Cooper and Schrieffer in 1957 \cite{Bardeen1957}, electrons in a superconductor form pairs below a material dependent critical temperature $T_{\rm c}$. These composite particles are also referred to as \textit{Cooper pairs} and they represent the superconducting charge carriers with twice the mass and charge of a single electron. Their dissipationless flow causes the current to have zero resistance, which is alongside the Meissner-Ochsenfeld effect \cite{Meissner1933} the most characteristic feature of a superconductor. The latter describes magnetic field expulsion below $T_{\rm c}$, provided the external magnetic field is smaller than a critical field $B_{\rm c}$. Further details on the microscopic theory of superconductivity can be found in \cite{Bardeen1957} and \cite{Ginzburg1950}.

If at $T < \qty{4}{\kelvin}$ an external current source is connected to a Josephson junction, a supercurrent will flow despite the tunnel barrier, implying the tunneling of Cooper pairs as niobium is predominantly superconducting at these temperatures ($T_\mathrm{c} = \qty{9.3}{\kelvin}$). Since the tunneling probability of an individual electron is approximately $p = \num{e-4}$ \cite{Gross2016}, a much lower probability is to be expected for a Cooper pair consisting of two electrons. However, Josephson predicted that the tunneling behavior of Cooper pairs and individual conduction electrons must be the same. This is justified by the so-called \textit{Macroscopic Quantum Model}, formulated by Fritz London in 1953.

The main focus here lies on the quantum mechanical phase $\theta$. On one hand, the distance between both electrons in a Cooper pair is approximately 10 to \qty{1000}{\nm} which is significantly larger than the spacing between Cooper pairs, resulting in strongly overlapping wave functions. On the other hand, Cooper pairs have to obey Bose-Einstein statistics due to their total spin of 0. Thus, all Cooper pairs share the same ground state, and as a consequence, the energies and temporal evolutions of the phases are equal. These two effects lead to what is known as \textit{phase-lock}. The phases of neighboring pairs synchronize such that this quantum mechanical property now holds on a macroscopic scale. This gives rise to a macroscopic wave function

\begin{equation}
\Psi(\textbf{r},t) = \Psi_0(\textbf{r},t)e^{i\theta(\textbf{r},t)} \ \ ,
\end{equation}

which describes all charge carriers of the superconductor. Here, the charge carrier density is given by $\left|\Psi_0(\textbf{r},t)\right|^2 = n_{\rm s}$. \textit{t} denotes the time and \textbf{r} represents the position of the Cooper pair ensemble. As a result of sharing the same phase, both electrons of a Cooper pair consequently possess the same tunneling probability as an individual electron, enabling the supercurrent. This coherence phenomenon is referred to as the \textit{Josephson effect} \cite{Josephson1962}. Another significant consequence of the macroscopic quantum model is flux quantization. Together with the Josephson effect, this forms the basis for Josephson junctions and their applications. 

\figureleft {t!}
{width=\textwidth}
{../Figures/quantized_flux}
{7cm}
{0cm}
{Superconducting ring-shaped cylinder threaded by an external magnetic field. By applying the field at low temperatures, shielding currents arise to expel the field from the superconductor. Upon turning off the external field the shielding currents will remain due to the lack of resistance, causing magnetic flux to be trapped. The dotted blue path \textit{C} is situated at the center of the cylinder wall, which we assume to be current-free due to the London penetration depth $\lambda_{\rm L}$ being much smaller than the thickness of the cylinder wall.} 
{fig:quantflux}

Flux quantization is derived through the capture of an external magnetic flux within a superconducting cylinder (see figure \ref{abb:fig:quantflux}). The wave function must remain unchanged after circumnavigating the cylinder due to $e^{i\theta} = e^{i\theta + 2\pi n}$. As a result, upon integrating along the current-free center of the cylinder wall (path $C$), the following equation holds for the captured flux

\begin{equation}
\Phi = \frac{h}{q_\mathrm{s}}n = \frac{h}{2e}n \equiv \Phi_0n \ \ .
\end{equation}

Here, $n\in\mathbb{Z}$ and \unit{\fq} = \qty{2.07e-15}{\tesla\metre\squared} \cite{CODATA2018} represents the so-called magnetic flux quantum. The captured flux is thus quantized, a consequence solely arising from the macroscopic nature of the phase. This quantity plays a crucial role in the theoretical description of Josephson junctions.


The current and voltage behavior in a SIS junction is described by the \textit{Josephson equations}. Crucial to this description is a critical current \textit{$I_\mathrm{c}$} that is linearly proportional to the applied current \textit{I}, which marks the boundary between two operational modes. Additionally, due to the macroscopic nature of the phase, \textit{I} oscillates with the gauge-invariant phase difference $\varphi$, leading to the \textbf{first Josephson equation} \cite{Josephson1965}

\begin{equation}
\label{1.JE}
I_\mathrm{s} = I_\mathrm{c}\sin(\varphi) \ \ .
\end{equation}

$I_\mathrm{c}$ is proportional to the coupling strength $\kappa$, which describes the overlap of the wave functions $\Psi_1$ and $\Psi_2$ within the insulating layer. The relationship is given by

\begin{equation}
I_\mathrm{c} = \frac{4e\kappa V n_\mathrm{s}}{\hbar} \ \ ,
\end{equation}

where \textit{V} represents the volume of the superconducting electrode and \textit{e} denotes the elementary charge of an electron. We assume that the Cooper pair density $n_\mathrm{s}$ of the two superconductors $S_1$ and $S_2$ is identical, meaning $n_{\mathrm{s}1} = n_{\mathrm{s}2} = n_\mathrm{s}$.

The gauge-invariant phase difference refers to the phases $\theta_1$ and $\theta_2$ of the respective electrodes at the boundary of the insulating layer (positions 1 and 2, see figure \ref{abb:fig:JJschem}). Taking into account possible external electromagnetic fields within the barrier, the general form using the vector potential \textbf{A} is given by \cite{Gross2016}

\begin{equation}
\label{EichInv_Phase}
\varphi(\textbf{r},t) = \theta_2(\textbf{r},t) - \theta_1(\textbf{r},t) - \frac{2\pi}{\Phi_0}\int_{1}^{2}\textbf{A}(\textbf{r},t)\cdot \mathrm{d}\textbf{l} \ \ .
\end{equation}

Assuming a constant supercurrent density $J_\mathrm{s}$ across the junction, taking the time derivative of equation \eqref{EichInv_Phase} yields the \textbf{second Josephson equation} \cite{Josephson1965}

\begin{equation}
\label{2.JE}
\frac{\partial\varphi}{\partial t} = \frac{2\pi}{\Phi_0}V \ \ .
\end{equation}

The first operating mode describes the case for $I<I_\mathrm{c}$. Here, the entire injected current is carried by Cooper pairs, so $I=I_\mathrm{s}=\mathrm{const}$. As a result, $\varphi$ is temporally constant, which, according to equation \eqref{2.JE}, leads to $V=0$. This voltage-free state is known as the \textit{dc Josephson effect}.

For $I>I_\mathrm{c}$ however, Cooper pairs begin to break up such that a portion of the current needs to be carried by quasiparticles, which will then lead to a voltage drop \textit{V}. According to the second Josephson equation, the phase $\varphi$ becomes time dependent, and after integration one obtains

\begin{equation}
\label{phi(t)}
\varphi = \frac{2\pi}{\unit{\fq}}Vt + \varphi_0 = w_\mathrm{J}t + \varphi_0 \ \ \ \mathrm{with} \ \ \ w_\mathrm{J} = \frac{2\pi}{\unit{\fq}}V \ \ .
\end{equation}

Thus, if we insert equation \eqref{phi(t)} into equation \eqref{1.JE}, we observe that  the current $I_\mathrm{s}$ oscillates with the \textit{Josephson frequency} $\frac{f_\mathrm{J}}{V} = \frac{w_\mathrm{J}}{2\pi V} = \frac{1}{\unit{\fq}} \approx \SI{483.6}{\MHz\per\uV}$. Accordingly, this phenomenon is referred to as the \textit{ac Josephson effect}.



\subsection{Josephson Junctions in a Magnetic Field}\label{subsec_jjmag}

\figurecenter {b!}
{width=0.8\textwidth}
{../Figures/shortjj_mag}
%{7.8cm}
{0cm}
{Short Josephson junction connected to a current source in the presence of an external B-field in y-direction, parallel to the junction area. Inside the electrodes the magnetic field decays exponentially according to the London penetration depths $\lambda_{\rm L,1}$ and $\lambda_{\rm L,2}$, visually shown by the purple color gradient. The closed contour \textit{C} is used to derive expressions for the spatially dependent phase difference $\varphi$ and current density $J_{\rm s}$.} 
{fig:JJMag}

To motivate the structure of a dc-SQUID, it is essential to first investigate the current behavior of an extended Josephson junction in the presence of an external magnetic field. So far, all previous formulae apply for point-like junctions, assuming a spatially constant phase difference $\varphi$ and Josephson current density $J_{\rm s}$ across the junction area. This is not the case for three-dimensional (extended) junctions with a length \textit{L} and width \textit{W}. The \textit{Josephson penetration depth} $\lambda_{\rm J}$ is a quantity used to classify an extended junction as short ($\rm W,L \leq\lambda_{\rm J}$) or long ($\rm W,L \geq\lambda_{\rm J}$) and is defined as 

\gl{
\lambda_{\rm J} = \sqrt{\frac{\unit{\fq}}{2\pi\unit{\micro_0}J_{\rm c}t_{\rm B}}
} \ \ .}{lamdaJ}

Here, the magnetic thickness is defined as $t_{\rm B} = d + \lambda_{\rm L,1} + \lambda_{\rm L,2}$. It describes how far an external magnetic field penetrates both superconducting electrodes if applied parallel to the junction area, as depicted in figure \ref{abb:fig:JJMag}. $\lambda_{\rm L,1}$ and $\lambda_{\rm L,2}$ are the respective London penetration depths and $J_{\rm c} = \frac{I_{\rm c}}{WL}$ the critical current density.
This distinction is needed to determine whether the magnetic self-field generated by the supercurrent is negligible in comparison to the external field (short junctions) or not (long junctions). Within the scope of this thesis, only short junctions are used.  

To analyze the current and phase distribution of such a junction we consider the setup shown in figure \ref{abb:fig:JJMag}. A short junction is connected to a current source and is penetrated by an external B-field in y-direction, parallel to the junction area. Now, obtaining an expression for the phase requires a similar approach as the calculation for the quantized flux, where we assumed that the phase changes by $2\pi n$ around a closed loop. Here, we again integrate over a closed contour \textit{C}, with the points $P_{\rm 1}-P_{\rm 4}$ marking the transitions between superconductor and isolator. Using equation \ref{EichInv_Phase}, we find 

\gl{
\frac{\del\varphi}{\del z} = \frac{2\pi}{\unit{\fq}}B_{\rm y}t_{\rm B} \ \ \ \mathrm{and} \ \ \ \frac{\del\varphi}{\del y} = -\frac{2\pi}{\unit{\fq}}B_{\rm z}t_{\rm B} \ \ .
}{phi(z,y)}

In this experiment, however, the magnetic field points in y-direction only, meaning $\varphi$ will only vary along the z-axis. Integrating the first expression in equation \ref{phi(z,y)} then leads to

\gl{
\varphi(z) = \frac{2\pi}{\unit{\fq}}B_{\rm y}t_{\rm B}z + \varphi_0 \ \ .
}{phi(z)} 

Here, the integration constant $\varphi_0$ represents the phase difference for the case $z=0$. Inserting equation \ref{phi(z)} into the first Josephson equation and using $J_{\rm s} = \frac{I_{\rm s}}{WL}$ gives 

\gl{
J_{\rm s}(y,z,t) = J_{\rm c}(y,z)\sin(kz + \varphi_0) \ \ \ \mathrm{with} \ \ \ k = \frac{2\pi}{\unit{\fq}}B_{\rm y}t_{\rm B} \ \ .
}{Js(y,z,t)}

If we now assume the critical current density $J_{\rm c}$ to be constant across the junction area, we can integrate equation \ref{Js(y,z,t)} to get a flux-dependent maximum Josephson current

\gl{
I_{\rm s}^{\rm m}(\Phi)	= I_{\rm c}\left|\frac{\sin(\frac{kL}{2})}{\frac{kL}{2}}\right| = I_{\rm c}\left|\frac{\sin(\frac{\pi\Phi}{\unit{\fq}})}{\frac{\pi\Phi}{\unit{\fq}}}\right| \ \ .
}{Ismax}

This expression describes the so-called Fraunhofer diffraction pattern, shown in figure \ref{abb:fig:fraunhofer}. The result resembles the single slit experiment, where the same pattern is found for the light intensity behind the slit. Here, the analogy works by considering the integral of the critical current density $J_{\rm c}$ as a transmission function which is constant inside the junction and zero outside. 

\figureleft {t!}
{width=\textwidth}
{../Figures/fraunhofer}
{9cm}
{0cm}
{Normalized flux-dependent maximum Josephson current $I_{\rm s}^{\rm m}(\Phi)$ showing a Fraunhofer pattern. It modulates with the flux quantum $\unit{\fq}$, peaking at $\Phi=0$ with subsequent maxima at $\Phi=\pm(\frac{3}{2}+n)\unit{\fq}$ with $n\in\mathbb{N}_0$. For $\Phi=\pm(n+1)\unit{\fq}$ the total net current is zero.} 
{fig:fraunhofer}


\subsection{RCSJ Model}

Until now, we investigated the current-voltage behavior under the assumption of $I<I_{\rm c}$, staying in the so-called zero-voltage state. In this regime, only the dc Josephson effect applies as discussed in subchapter \ref{subchap_Jeffect}. Switching to the voltage stage, i.e. $I>I_{\rm c}$, Cooper pairs start breaking up into quasiparticles if the electric energy $eV$ and/or thermal energy $k_{\rm B}T$ exceeds the sum of both electrodes' gap energies $\Delta_1 + \Delta_2$. Consequently, at the \textit{gap-voltage} 

\gl{
V_{\rm g} = \frac{\Delta_1(T)+\Delta_2(T)}{e}
}{Vgap}

quasiparticles start to cross the tunnel barrier resulting in a steep rise of a resistive normal current $I_{\rm n}$. Under a current source, the condition $I=I_{\rm s}+I_{\rm n}$ must be constantly fulfilled. This results in an oscillating normal current, since $I_{\rm s}$ oscillates with $f_{\rm J}$ according to the ac Josephson effect. $\frac{\mathrm{d}\varphi}{\mathrm{d}t}$ will therefore vary sinusoidally, causing both $I_{\rm s}$ and $I_{\rm n}$ and the resulting voltage to oscillate in a complex manner. As a voltage with such a high frequency cannot be measured, only the time-averaged voltage will be considered in the following discussion. \\
Now, further increasing the energy of the quasiparticles ($T>T_{\rm c}$ and/or $V>V_{\rm g}$) leads to a transition into normal-conducting electrons, which exhibit an ohmic dependence. This behavior can be seen in the typical current-voltage-characteristic (IVC) depicted in figure \ref{abb:IVCs}. \\ 

\twofigurescenter {t!}
{width=0.48\textwidth}
{../Figures/CrossJJ4w13v3_2C12_4_2umJJ_edited_Alex}
{0.02\textwidth} %hspace b/w figures
{width=0.48\textwidth}
{../Figures/JJ_IVC_BetaC_smaller_1_Alex}
{0.5cm} %vspace
{Platzhalter Plots von Alex -> eigene Messung vornehmen?... Measured IVCs from cross-type junctions manufactured in this working group. Left: Underdamped junction showing the typical hysteresis. Right: Overdamped junction with a current-voltage shape that is independent of the current sweep direction. }
{IVCs}

Real junctions, however, are comprised of two electrodes separated by a thin insulating layer, which represent a parallel plate capacitor with the Al-Al$\rm O_x$ layer being the dielectric material. Therefore, a junction capacitance \textit{C} needs to be taken into account. A displacement current $I_{\rm d}$ will flow as a consequence, given we are in the voltage state. Lastly, thermal and 1/f noise cause a small fluctuating current $I_{\rm f}$. All these current channels were defined in the so-called Resistively and Capacitively Shunted Junction (RCSJ) model \cite{Cumber1968}, \cite{Stewart1968}, which models the total current of a lumped (0-dimensional) junction to a sufficiently high accuracy. A schematic of an effective circuit diagram is shown in figure \ref{abb:fig:rcsj} (left). Combining every current channel leads to the \textit{Basic Junction Equation}, which is defined as \cite{Gross2016}

\gl{
I=I_{\rm s}+I_{\rm n}+I_{\rm d}+I_{\rm f}=I_{\rm c}\sin(\varphi)+\frac{1}{R(V)}\frac{\unit{\fq}}{2\pi}\frac{\mathrm{d}\varphi}{\mathrm{d}t}+C\frac{\unit{\fq}}{2\pi}\frac{\mathrm{d}^2\varphi}{\mathrm{d}t^2}+I_{\rm f} \ \ .
}{BJE}

By defining the Josephson coupling energy $U_{\rm J0}=\frac{\hbar I_{\rm c}}{2e}$ and the normalized currents $i=\frac{I}{I_{\rm c}}$ and $i_{\rm f}(t)=\frac{I_{\rm f}(t)}{I_{\rm c}}$, equation \ref{BJE} can be rewritten to 

\gl{
\left(\frac{\hbar}{2 e}\right)^2 C \frac{\mathrm{d}^2 \varphi}{\mathrm{d} t^2}+\left(\frac{\hbar}{2 e}\right)^2 \frac{1}{R(V)} \frac{\mathrm{d} \varphi}{\mathrm{d} t}+\frac{\mathrm{d}}{\mathrm{d} \varphi}\left\{U_{\rm J 0}\left[1-\cos \varphi-i \varphi+i_{\rm f}(t) \varphi\right]\right\}=0 
}{BJE2} \ \ . 

\twofigurescenter {t!}
{width=0.48\textwidth}
{../Figures/RCSJ-Modell}
{0.02\textwidth} %hspace b/w figures
{width=0.48\textwidth}
{../Figures/washboard}
{0.5cm} %vspace
{Left: Schematic circuit of a lumped Josephson junction with all four current channels connected in parallel. The junction is represented by the cross symbol on the left, marking the supercurrent $I_{\rm s}$. The normal current $I_{\rm n}$ is realized with a resistance $R$, while the displacement current $I_{\rm d}$ and the noise $I_{\rm f}$ need a capacitor $C$ and a current source, respectively. Right: Tilted washboard potential for different currents, ranging from 0 to $1.5I_{\rm c}$. The tilt increases with the injected current $I$.}
{fig:rcsj}

The expression in the curly brackets represents the potential energy in the system $U_{\rm J}$, allowing equation \ref{BJE} to be compared to 

\gl{
M\frac{\mathrm{d}^2 x}{\mathrm{d}t^2} + \eta\frac{\mathrm{d} x}{\mathrm{d}t} + \nabla U = 0
}{} \ \ .

This equation describes a particle with mass \textit{M} and damping $\eta$ moving inside the potential \textit{U}. The mechanical analogue therefore allows us to interpret a \textit{phase particle}, where it's motion corresponds to a change of the gauge-invariant phase difference $\varphi$ within a potential $U_{\rm J}$ \cite{Clarke2004}. Consequently, it is attributed with a mass $M=\left(\frac{\hbar}{2 e}\right)^2C$ and damping $\eta=\left(\frac{\hbar}{2 e}\right)^2\frac{1}{R(V)}$. Figure \ref{abb:fig:rcsj} (right) visualizes how this phase particle behaves for different currents \textit{I}. Given the shape of $U_{\rm J}(\varphi)$, the potential is referred to as the \textit{tilted washboard potential}. \\
For $I=0$, the phase particle will remain within one of the potential minima. As the current increases, however, the potential starts to tilt such that the depth of the minima reduces until it vanishes for $I=I_{\rm c}$, thus becoming a saddle point. Up until this point, the phase particle can't overcome the potential barrier to move downward, which confirms the second Josephson equation as the phase difference $\varphi$ should remain constant for $I<I_{\rm c}$. Further increasing the current and therefore the tilt of the potential causes the phase particle to fall along the potential, resulting in a voltage drop across the junction ($\frac{\del\varphi}{\del t}>0$). \\

Reversing the current sweep showcases the importance of the particle's mass $M$ and damping $\eta$, as they determine if the return path equals the above described current shape or not. For the case of a small mass (small $C$) and large damping (small $R$), the phase particle will, due to a lack of momentum, come to a halt as soon as minima reappear in the washboard potential by reducing the current below $I_{\rm c}$. The current path will therefore remain unchanged as $I$ is reduced back to 0, as shown in figure \ref{abb:IVCs} (right). Such a junction is consequently called an \textit{overdamped} junction. \\
The other case describes an \textit{underdamped} junction (figure \ref{abb:IVCs} (left)) and involves a large mass (large $C$) and small damping (large $R$). This allows the phase particle to continue to move downward as it now carries enough momentum to overcome the arising maxima and minima. The finite voltage drop despite the current being below $I_{\rm c}$ is displayed as the steep quasiparticle current curve, which ends with a return current $I_{\rm R}$ that arises with the recapture of the particle in a minimum. This leads to a hysteretic IVC, as depicted in figure \ref{abb:IVCs} (left). $I_{\rm R}$ can be calculated via \cite{Likharev1986}

\gl{
I_{\rm R} = \frac{4}{\pi\sqrt{\beta_{\rm C}}}I_{\rm c} \ \ ,
}{IR}

with $\beta_{\rm C}$ being the dimensionless Stewart-McCumber parameter, that is used to quantitatively distinguish between both junction types. It is given by 

\gl{
\beta_{\rm C} = \frac{2\pi}{\unit{\fq}}I_{\rm c}R^2C
}{betaC} 

with $\beta_{\rm C}\gg 1$ corresponding to a strongly underdamped junction, whereas $\beta_{\rm C}\ll 1$ represents a strongly overdamped junction. The junctions developed and produced within the scope of this thesis aim to be overdamped, which is why we take a closer look on the time-averaged voltage for $I>I_{\rm c}$ in the case of $ \beta_{\rm C}\ll 1$. Neglecting the noise in equation \ref{BJE2}, as well as assuming the resistance to be linear below and above the gap voltage $V_{\rm g}$, i.e. $R(V)=R$, the time-averaged voltage can be derived to 

\gl{
\langle V(t)\rangle=I_{\mathrm{c}} R \sqrt{\left(\frac{I}{I_{\mathrm{c}}}\right)^2-1} \ \ \ \mathrm{for} \ \ \ \frac{I}{I_{\mathrm{c}}}>1 \ \ .
}{}

\section{dc-SQUIDs}

We have now covered the theoretical framework necessary to understand the working principle of a dc-SQUID, which consists of a superconducting ring intersected by two identical Josephson junctions with critical Josephson currents $I_{\rm c}$, as depicted in figure \ref{abb:dcSQUID}. Both junctions are shunted with shunt resistors $R_{\rm s}$ to avoid hysteretic behavior in the respective IVCs. If the loop is then biased with a bias current $I_{\rm b}$ while being threaded by an external magnetic flux $\Phi_{\rm e}$, it is possible to convert small flux variations into a measurable voltage change. dc-SQUIDs are therefore used as highly sensitive flux-to-voltage transducers.

\subsection{Zero Voltage State}

\figureleft {b!}
{width=\textwidth} %sets how much of the fig space is used
{../Figures/dc-SQUID}
{9cm} %sets width of the fig space
{0cm}
{Schematic circuit diagram of a shunted dc-SQUID. A superconducting loop with inductance $L_{\rm s}$ is interrupted by two lumped Josephson junctions such that they form a parallel connection. Operation requires a bias current $I_{\rm b}$ and an external magnetic flux $\Phi$. To avoid hysteresis effects, a shunt resistance $R_{\rm s}$ is connected to each junction.} 
{dcSQUID}

In order to fully understand the working principle of a dc-SQUID it is again necessary to first cover the zero voltage stage as we did for a single junction. 
The parallel connection of the two junctions allows the bias current to split into two supercurrents $I_{\rm s1}$, $I_{\rm s2}$ with identical critical currents, i.e. $I_{\rm c,1}=I_{\rm c,2}=I_{\rm c}$. Here we assume $I_{\rm b}<2I_{\rm c}$ to ensure that no voltage drop across both junctions occurs ($V_{\rm s}=0$). Applying Kirchhoff's law we then obtain the following expression 

\gl{
I_{\rm b} = I_{\mathrm{s}}=I_{\mathrm{c}} \sin \varphi_1+I_{\mathrm{c}} \sin \varphi_2=2 I_{\mathrm{c}} \cos \left(\frac{\varphi_1-\varphi_2}{2}\right) \sin \left(\frac{\varphi_1+\varphi_2}{2}\right) \ \ .
}{squid_Is_tot}

In chapter \ref{subsec_jjmag} we concluded that a magnetic flux $\Phi$ causes the supercurrent to modulate with $\unit{\fq}$. A dc-SQUID can be considered as a single junction with a much larger effective area $A_{\rm eff}$ (loop area), that an external magnetic fux can penetrate. It is therefore reasonabe to expect a similar behavior for a dc-SQUID. The same approach as with a single junction is used to determine the flux dependency of the total supercurrent, where a closed loop integral is performed around the SQUID loop. The calculation leads to the relation ref???  

\gl{
\varphi_2-\varphi_1=\frac{2 \pi \Phi}{\Phi_0}
 \ \ ,}{}

which can be directly inserted into equation \ref{squid_Is_tot} to obtain

\gl{
I_{\mathrm{s}}=2 I_{\mathrm{c}} \cos \left(\pi \frac{\Phi}{\Phi_0}\right) \sin \left(\varphi_1+\pi \frac{\Phi}{\Phi_0}\right) \ \ .
}{}

\gl{
I_{\mathrm{s}}^{\mathrm{m}}(\Phi)=2 I_{\mathrm{c}}\left|\cos \left(\pi \frac{\Phi}{\Phi_0}\right)\right|
}{}

\gl{
\Phi=\Phi_{\rm e} + \Phi_{\rm L} =
 \Phi_{\rm e} - L I_{\rm  c} \sin \left(\pi \frac{\Phi}{\Phi_0}\right) \cos \left(\varphi_1+\pi \frac{\Phi}{\Phi_0}\right)
}{}

\gl{
\Phi=\Phi_{\mathrm{e}}-\frac{1}{2} \beta_L \Phi_0 \sin \left(\pi \frac{\Phi}{\Phi_0}\right) \cos \left(\varphi_1+\pi \frac{\Phi}{\Phi_0}\right)
}{}

\gl{
I_{\mathrm{s}}^{\mathrm{m}}(\Phi_{\rm e})=2 I_{\mathrm{c}}\left|\cos \left(\pi \frac{\Phi_{\rm e}}{\Phi_0}\right)\right|
}{}

\gl{
\frac{I_{\mathrm{s}}^{\mathrm{m}}\left(\Phi_{\mathrm{e}}\right)}{2 I_{\mathrm{c}}} \approx 1-\frac{2 \Phi_{\mathrm{e}}}{\Phi_0 \beta_{\mathrm{L}}}
}{}

\figureleft {b!}
{width=\textwidth}
{../Figures/Phi_betaL}
{9cm}
{0cm}
{Caption} 
{Phi_betaL}

\subsection{Voltage State}

\subsection{Noise}

\subsection{Parasitic Resonances}




\chapter{Experimental Setup}

So far we discussed general aspects of dc-SQUIDs and how their working principle allows for highly sensitive magnetic flux measurements. As already briefly seen in subsection \ref{subsec_para_res}, when it comes to practical SQUIDs many theoretical considerations regarding parameter optimization need to be reevaluated to account for usability in practical experiments. We begin this chapter with general concepts of a practical SQUID design and introduce a typical low-noise setup with a room temperature readout electronic. In this working group, SQUIDs are mainly developed for the readout of \textit{Metallic Magnetic Microcalorimeters (MMCs)} (see section \ref{sec_MMC}). We will see in the following how those SQUIDs need to be designed to optimize their coupling to these detectors. Furthermore, this chapter will cover various methods to reduce quality factors of parasitic resonances, such as adding shunt resistors or coupling to normal conducting gold layers.

\section{Practical dc-SQUIDs}\label{sec_practical_SQUID}

The SQUIDs developed in this working group are used as current sensors for the MMC readout by sending the detected signals from the \textit{pickup coil} of the MMC to the input coil of the SQUID. The requirement of the latter entails a parasitic capacitance resulting in numerous resonances, as discussed in subsection \ref{subsec_para_res}. To achieve high inductive coupling between input coil and SQUID loop, it is necessary to fabricate them closely on top of each other, only separated by a thin insulating layer. The coupling strength is given by the dimensionless parameter 

\gl{
k_{\rm is} = \frac{M_{\rm is}}{\sqrt{L_{\rm i}L_{\rm s}}} \ \ , 
}{} 

where $M_{\rm is}=\Delta\Phi_{\rm s}/\Delta I_{\rm i}$ is the mutual inductance, describing how much flux $\Delta\Phi_{\rm s}$ is generated in the SQUID loop for a current change $\Delta I_{\rm i}$ in the input coil. This allows us to define the so-called coupled energy sensitivity $\epsilon_{\rm c}(f)$ with respect to the input coil, which by using equation \ref{energy_sens} is given as

\gl{
\epsilon_{\rm c}(f) = \frac{\epsilon(f)}{k_{\rm is}^2} = \frac{L_{\rm i}S_{I\rm ,i}}{2} \ \ .
}{}

This expression refers to the apparent current noise $S_{I\rm ,i}=S_{\rm \Phi_s}/M_{\rm is}^2$, which is generated by the flux noise from the SQUID loop through the coupling $M_{\rm is}$. A strong coupling can be achieved by the commonly used square \textit{washer}-geometry with a planar input coil \cite{Jaycox1981}, as shown in figure \ref{abb:washer}. 

\figurecenter {t!}
{width=\textwidth}
{../Figures/Washergeometrie}
{0cm}   %vspace
{Schematic drawing of a typical planar thin-film dc-SQUID. The SQUID loop is realized as a square washer-geometry interrupted by a narrow slit, only connected at the junction area. A thin insulating layer separates the washer from the planar multi-turn input coil above. Left: View from the top. Right: Cross section marked by the dashed line.}
{washer}

Here, the SQUID loop is represented by the washer, whereas each turn of the input coil is symmetrically located on top of it to maximize the coupling between each system. A cross section of this setup is depicted in figure \ref{abb:washer} (right), showing the insulating dielectric layer separating each coil. The washer is intersected by a slit, which starts at the square hole in the middle and ends at the remotely situated junction area that connects each side of the loop. The total inductance of the SQUID loop can be calculated by adding the dominating washer hole inductance $L_{\rm h}$, the slit inductance $L_{\rm t}\approx \qty{0.3}{\pH\per\um}$ and the much smaller parasitic inductance $L_{\rm j}$ associated with the junction area, giving \cite{Ketchen1991}

\gl{
L_{\rm s} = L_{\rm h} + L_{\rm t} + L_{\rm j} \ \ .
}{}

The latter is referred to as parasitic due to it's position outside of the input coil, thus not contributing to the coupling. By neglecting $L_{\rm t}$ and $L_{\rm j}$, we can approximate the washer inductance in the limit of $d\ll w$ to $L_{\rm s}\approx L_{\rm h}\approx 1.25{\rm\mu_0}d$, where $d$ and $w$ are the inner and outer side lengths, respectively \cite{Jaycox1981}. This is a reasonable result considering that the supercurrent will only flow along the inner edge of the washer \cite{Ketchen1982}, thereby being independent of the outer side length $w$. The effective area $A_{\rm eff}$ of the SQUID loop has been calculated to $A_{\rm eff}\approx dw$ \cite{Ketchen1985}, showing that this geometry allows for high sensitivity while keeping the SQUID inductance small. The input coil inductance on the other hand can be approximated by $L_{\rm i}=L_{\rm str}+n^2L_{\rm s}$, where $L_{\rm str}$ is the stripline inductance (see section \ref{sec_damping}) and $n$ is the number of input coil turns \cite{Jaycox1981}. The dc-SQUID designs used in this working group, however, are too complex to provide such analytical expressions and therefore need to be calculated numerically using simulation softwares such as \textit{InductEX}. 

\subsection{Gradiometer}

The high flux sensitivity of a SQUID makes it prone to detect unwanted magnetic bias fields and/or gradients that may be present during its operation. Typical SQUIDs are therefore built in a gradiometric design to counteract this effect \cite{Ketchen1978}. A first order gradiometer consists of two identical conducting loops connected in series or parallel, with opposing orientation as shown in figure \ref{abb:gradiometer} (left, middle). Under the presence of a homogeneous bias field $\textbf{B}$ in x-direction (perpendicular to the gradiometer plane), this configuration produces a zero net current after a field change $\Delta B_{\rm x}$, due to the opposing currents induced in each turn. To also achieve the same effect for a field gradient $\frac{\del \textbf{B}}{\del z}$ or $\frac{\del \textbf{B}}{\del y}$, a second order gradiometer composed of four loops in series or parallel is required, see figure \ref{abb:gradiometer} (right), where only the currents induced in the upper loops are drawn for the sake of overview. In order to incorporate this into a practical SQUID, the input coil and the SQUID loop will consist of four serial and parallel turns, respectively. This configuration enables to combine a small SQUID inductance with a large input coil inductance while maintaining a strong coupling between the two, as each turn of both coils can be produced with similar dimensions. The low SQUID inductance results from the reciprocal summation over each loop inductance $L_{\rm w}$ due to the parallel connection, giving

\gl{
L_{\rm s} = \frac{L_{\rm w}}{4} \ \ .
}{}

Whereas a serial gradiometer gives 

\gl{
L_{\rm i} = 4L_{\rm w}
}{}

for the input coil. This gradiometric setup allows for adapting the input coil to the pickup coil of an MMC by choosing a large enough inductance $L_{\rm i}$, which will be discussed in section \ref{sec_SQUIDdesign}. 

\figurecenter {t!}
{width=\textwidth}
{../Figures/Gradiometer}
{0cm}   %vspace
{Schematic examples of a gradiometric dc-SQUID configuration threaded by a homogeneous magnetic field $\textbf{B}$. A first order gradiometer can be realized by either connecting two loops in series (left) or in parallel (middle). A magnetic field change $\Delta B_{\rm x}$ induces two opposing currents that cancel each other out. Right: Second order gradiometer consisting of four loops connected in series. This geometry results in a net zero current also for an applied field gradient $\frac{\del \textbf{B}}{\del z}$.}
{gradiometer}
   
\section{Operation of a dc-SQUID}

We have seen in section \ref{sec_voltagestate} that the periodic $V\Phi$-characteristic provides an approximately linear dependency at $\Phi = (2n+1)\frac{\Phi_0}{4}$, which only holds for $\Delta\Phi\approx\Phi_0/4$. This restricts the dynamic range greatly, as the linearity vanishes for larger flux changes and for $\Delta\Phi>\Phi_0/2$ the voltage even becomes ambiguous. Such behavior is unsuitable for MMC readout, as they require the highest possible signal to noise ratio and therefore a linearized output voltage.  

\subsection{Flux-Locked Loop}

The standard readout method involves a flux feedback circuit to maintain the operation at the working point independently of the flux change amplitude \cite{Drung2002}. This so-called flux-locked loop (FLL) readout technique first amplifies the output signal of the SQUID $V_s$ with a differential amplifier operated at room temperature, where the voltage $V_{\rm b}$ corresponding to the working point is provided by a voltage source on the second amplifier input. This voltage compensation at the working point ensures that only variations $\Delta V = V_{\rm s}-V_{\rm b}$ that correspond to the flux change $\Delta\Phi$ are amplified. The signal is then fed into an integrator, which integrates it over time and thus creates a rising output voltage $V_{\rm out}$. By now connecting a feedback resistance $R_{\rm fb}$ to the output circuit, a rising feedback current $I_{\rm fb}$ emerges that flows to a feedback coil with inductance $L_{\rm fb}$. This coil is coupled to the SQUID analogous to the input coil (see section \ref{sec_SQUIDdesign}), but with the opposite orientation. A compensation flux $-\Delta\Phi$ is generated up until the initial flux change is fully canceled out, i.e. $V_{\rm s}\to 0$. The integrator will therefore approach a constant value due to the vanishing voltage at the input circuit. This voltage signal is proportional to the current that was needed to completely compensate for the input signal that induced $\Delta\Phi$, leading to the relation 

\gl{
V_{\rm out} = \frac{R_{\rm fb}}{M_{\rm fb}}\Delta\Phi \ \ .	
}{} 

\figurecenter {t!}
{width=\textwidth}
{../Figures/FLL}
{0cm}
{Schematic circuit diagram of a flux-locked loop dc-SQUID readout. The amplified and integrated SQUID voltage signal $V_{\rm s}$, which is caused by a detected flux change $\Delta\Phi$, is fed back to a feedback coil with inductance $L_{\rm fb}$, creating a compensating flux $-\Delta\Phi$. This enables the operation at the working point, while flux changes far greater than $\Phi_0/4$ can be linearized.} 
{FLL}

A schematic for this readout process is shown in figure \ref{abb:FLL}.  With this setup the SQUID is used as a null-detector that allows for the linearization of the quantity of interest, while also providing a large dynamic range. A state-of-the-art, low-noise SQUID readout electronic by the company Magnicon\footnote{Magnicon GmbH, Barkahusenweg 11, 22339 Hamburg} of the type XXF-1, which is used in this working group, provides the necessary current and voltage sources, as well as the room temperature amplifiers within the FLL circuit described above. This SQUID electronic exhibits an intrinsic voltage noise of $\sqrt{S_{V\rm ,el}}\approx\qty{0.33}{\nV\per\sqrthz}$ and intrinsic current noise of $\sqrt{S_{I\rm ,el}}\approx\qty{2.6}{\pA\per\sqrthz}$ \cite{Drung2006}. Furthermore, a large amplifier bandwidth of $\qty{6}{\MHz}$ is provided to ensure high sensitivity for short signal rise times. The intrinsic noise of the SQUIDs produced in this working group, however, typically reaches values of $\sqrt{S_{\Phi_{\rm s}}}\leq\qty{1}{\micro\fq\per\sqrthz}$. Adding these terms together leads to the total apparent flux noise in the SQUID, which is expressed as the spectral power density

\gl{
S_{\rm \Phi_s,SQ} = S_{\rm \Phi_s} + \frac{S_{V\rm ,el}}{V_{\rm \Phi_s}^2} + \frac{S_{I\rm ,el}}{I_{\rm \Phi_s}^2} \ \ .
}{singlestage_noise}  

Typical values for the transfer coefficients of SQUIDs produced within the scope of this thesis, are $V_{\rm \Phi_s}=\qty{80}{\uV\per\fq}$ and $I_{\rm \Phi_s}=\qty{20}{\uA\per\fq}$, leading to the SQUID electronic having a total noise contribution of $\qty{4.13}{\micro\fq\per\sqrthz}$. The amplifier noise therefore dominates the noise level, thereby deteriorating the signal to noise ratio. To avoid this effect, a second SQUID is typically added to act as a low temperature amplifier \cite{Welty1993}. This method significantly reduces the apparent flux noise in the detector SQUID, which is crucial for MMC readout, as the intrinsic noise of a MMC detector should not be lower than that of the readout electronics.


\subsection{Two-Stage Configuration}

Implementing a low temperature amplifier is usually realized through a second stage SQUID, situated between the first stage (detector) SQUID and the room temperature amplifier, as depicted in figure \ref{}. The first stage SQUID, also referred to as a Front-End SQUID, needs to be operated in a voltage bias for this two-stage setup. This can be achieved by connecting a gain resistor $R_{\rm g}$ in parallel with both the Front-End and the input coil of the amplifier SQUID. If a bias current $I_{\rm b}$ is injected into the circuit, all the current will flow through the Front-End, as long as it stays superconducting. Once it becomes normal conducting by further increasing $I_{\rm b}$, the current will start shifting to $R_{\rm g}$, whose resistance is chosen to be much smaller than $R_{\rm dyn}$, until most of the current flows through $R_{\rm g}$. At this point, the resulting voltage across both components becomes approximately constant. This behavior can be visualized through a loadline created by the parallel resistances, which intersects the IVC of the Front-End. The loadline voltage $V_{\rm s}$ between both extremal IV curves will then remain nearly constant, as the slope is given by the small gain resistance $R_{\rm g}$. If a detector signal is now coupled into the Front-End through the input coil with $M_{\rm is}$ , the current in the SQUID will move along the loadline in the $I\Phi$-plane, corresponding to the externally induced flux $\Phi_{\rm s}$. The attached input coil of the second stage SQUID would experience these current changes, hence creating a flux change $\Delta\Phi_{\rm x}$ in the amplifier SQUID, which is being operated in a current bias. To maximize the amplification, the second stage SQUID is typically realized as a $N$-SQUID series array consisting of $N$ serially connected identical SQUID cells. This results in a large voltage drop across the array, given by $V_{\rm array}=NV_{\rm cell}$. Analogous to the single stage readout, the signal will then be amplified at room temperature and fed back to a feedback coil with mutual inductance $M_{\rm f}$ to compensate for the initial flux change $\Delta\Phi_{\rm s}$. An additional feedback coil with mutual inductance $M_{\rm fx}$, spanning symmetrically across every SQUID cell couples a constant flux offset through a bias current $I_{\rm \Phi_x}$ into the array in order to maintain it at its working point. The resulting two-stage $V_{\rm x}\Phi_{\rm s}$-characteristic will strongly depend on the flux gain defined as

\gl{
G_{\rm \Phi} = \frac{\del\Phi_{\rm x}}{\del\Phi_{\rm s}} = \frac{M_{\rm ix}}{R_{\rm g} + R_{\rm dyn}}V_{\rm \Phi_s}	\approx \frac{M_{\rm ix}}{R_{\rm dyn}}V_{\rm \Phi_s} \ \ ,
}{} 

which relates the flux change induced in the second stage SQUID for a given flux change in the detector SQUID. For $\Delta\Phi_{\rm x}=\Delta I_{\rm s}M_{\rm ia}>\Phi_0/2$, additional minima and maxima emerge. These start to overlap for $\Delta\Phi_{\rm x}>\Phi_0$, thereby creating multiple working points that prevent a practical FLL operation. This sets an upper limit for the flux gain, however, it should be chosen as large as possible to reduce the apparent flux noise of the Front-End SQUID. An optimal flux gain has been calculated to $G_{\rm \Phi}\approx\pi$, corresponding to $\Delta\Phi_{\rm x}\approx\Phi_0/2$ \cite{Drung1996a}. \\

\figurecenter {t!}
{width=\textwidth}
{../Figures/2stage}
{0cm}
{caption} 
{label}

The two-stage setup contributes additional noise sources to equation \ref{singlestage_noise}, namely the gain resistor and the amplifier SQUID. However, the resulting conversion to the flux $\Phi_{\rm s}$ of the detector SQUID significantly reduces the influence of the room temperature amplifier, which in turn strongly improves the overall signal to noise ratio. The total apparent flux noise of the Front-End then reads \cite{Drung1996a}\footnote{Auch Stromquellen, siehe F. Kaap?}

\gl{
S_{\rm \Phi_s,SQ} = S_{\rm \Phi_s} + \frac{4k_{\rm B}TR_{\rm g}}{G_{\rm \Phi}^{2}(R_{\rm g}+R_{\rm dyn})^{2}}M_{\rm ia}^{2} + \frac{S_{\rm \Phi_x}}{G_{\rm \Phi}^{2}} + \frac{S_{\rm V,el}}{G_{\rm \Phi}^{2}V_{\rm \Phi_x}^{2}} + \frac{S_{\rm I,el}}{G_{\rm \Phi}^{2}I_{\rm \Phi_x}^{2}} \ \ .
}{total_apparent_fluxnoise_2stage} 

The second term describes the Nyquist current noise caused by the gain resistor, which becomes negligible with a voltage biased detector SQUID where $R_{\rm g}\ll R_{\rm dyn}$. The low and room temperature amplifier terms are reduced by the flux gain parameter, which can't be chosen arbitrarily large as mentioned above. However, using a SQUID array for the second stage increases the voltage swing and thus the transfer coefficients by an $N$-fold, i.e. $V_{\rm \Phi_x} = NV_{\rm \Phi_{cell}}$, where the subscript 'cell' refers to a single array SQUID cell. Consequently, the total noise level can be further reduced by choosing a high number $N$ of SQUID cells. Here it is noteworthy, however, that these considerations only account for the magnetic flux in a single cell, as otherwise the transfer coefficients would remain constant \cite{Stawiasz1993},\cite{Foglietti1993}.  Now, using equation \ref{fluxnoise_psd} we obtain for the SQUID array flux noise

\gl{
\sqrt{S_{\rm \Phi_x}} = \frac{\sqrt{S_{\rm V_x}}}{V_{\rm \Phi_x}} = \frac{\sqrt{NS_{\rm V_{cell}}}}{NV_{\rm \Phi_{cell}}} = \frac{1}{\sqrt{N}}\sqrt{S_{\rm \Phi_{cell}}} \ \ ,
}{}   

hence the intrinsic noise of the second stage gets reduced by a factor of $\frac{1}{\sqrt{N}}$ \cite{Stawiasz1993}. This also has a consequence for the coupled energy sensitivity of the SQUID array, which is calculated by summing the array flux noise over all $N$ cells, giving

%since the total array inductance is the sum of all SQUID cell inductances leading to $L_{\rm x}=NL_{\rm cell}$. Therefore, the coupled energy sensitivity is given by 

%\gl{
%\epsilon_{\rm c,x} = \frac{\epsilon_{\rm x}(f)}{k_{\rm ix}^{2}}	= \frac{S_{\rm \Phi_x}}{2L_{\rm x}k_{\rm ix}^{2}}
%}{}

\gl{
\epsilon_{\rm c,x} = N\frac{S_{\rm \Phi_x}}{2L_{\rm cell}k_{\rm i,cell}} = \frac{S_{\rm \Phi_{cell}}}{2L_{\rm cell}k_{\rm i,cell}} \ \ .	
}{}

with the parameter $k_{\rm i,cell}$ denoting the coupling of a cell with inductance $L_{\rm cell}$ to its respective input coil segment. Connecting \textit{N} SQUIDs in series did therefore not affect the energy sensitivity, provided that $k_{\rm i,cell}$ remains constant across the array. The arrays produced in this working group either contain 16 \cite{Kempf2015} or 18 \cite{Krä2023} cells. Applying this to equation \ref{total_apparent_fluxnoise_2stage}, we would reduce the above-mentioned contribution of the room temperature amplifier to\\





\section{Metallic Magnetic Microcalorimeters} \label{sec_MMC}

Figure: Schematic of MMC (?)

Short explanation of working principle

\subsection{Readout with Coupled dc-SQUIDs}

Extrinsic energy sensitivity between FE and pickup coil:
\gl{
\epsilon_{\rm p} = \frac{S_{\rm \Phi_s,p}}{2L_{\rm p}}
}{}


\gl{
I_{\rm in} = \frac{\Delta\Phi_{\rm p}}{L_{\rm i} + L_{\rm par} + L_{\rm p}}
}{}

\gl{
\frac{\Delta\Phi_{\rm s}}{\Delta\Phi_{\rm p}} = \frac{M_{\rm is}}{L_{\rm i} + L_{\rm par} + L_{\rm p}}
}{}

\gl{
L_{\rm s}' = L_{\rm s}()1-k_{\rm is}^2s_{\rm in})
}{}

\section{dc-SQUID Design} \label{sec_SQUIDdesign}

\subsection{dc-SQUID with a Two-Turn Input Coil}

%\subsection{Integrated Two-Stage Chip}

\section{Damping Methods} \label{sec_damping}

\subsection{Lossy Input Coil}

\subsection{Inductive Damping}
\chapter{dc-SQUID Design} \label{ch_SQUIDdesign}

The main objective for this thesis was the optimization of an existing Front-End SQUID design for an improved coupling to one of the detectors developed in this working group. As we have seen in the previous section \ref{sec_MMC}, adjusting $L_{\rm i}$ to the detector coil ensures the maximization of the flux-to-flux coupling and therefore minimizes the extrinsic energy sensitivity. The parasitic inductance $L_{\rm par}$, that arises from the aluminum bonds between the SQUID and the detector substrate to form the flux transformer, has been estimated to \qty{0.5}{\nano\henry} \cite{Hengstler2017}. The previous SQUID design exhibits a design value of $L_{\rm i}=1.64$ for the input coil inductance \cite{Bauer2022}, which therefore fulfills the condition $L_{\rm i}=L_{\rm p}+L_{\rm par}$ for the pickup coil inductance of the ECHo-100k detector of $L_{\rm p}=\qty{1.14}{\nano\henry}$ \cite{Mantegazzini2021}. Other MMC detectors from this working group such as the 4k-pixel molecule camera MOCCA and the X-ray detector maXs100 require higher input inductances, as their pickup coil inductances are $L_{\rm p}=\qty{8.8}{\nano\henry}$ and $L_{\rm p}=\qty{6.65}{\nano\henry}$, respectively. In \cite{Bauer2022} SQUIDs with matching input inductances for the MOCCA and maXs100 detector were developed for the first time using an intermediary coupling transformer. These improved the energy resolution $\Delta E_{\rm FWHM}$ of the detectors, although the effect was minimal for the latter. Specifically for the maXs100 detector, a different approach was therefore followed in the framework of this thesis to achieve a better coupling while avoiding a significant increase of the detector noise.  

\section{dc-SQUID with a Two-Turn Input Coil} \label{sec_FEdesign}

Increasing the input coil inductance can be realized either by changing the geometry of the coil itself, or by implementing a double flux transformer structure, with the benefit of easily adapting the inductance independently of the SQUID design. The latter, however, was accompanied with a reduction of the effective coupling constant $k_{\rm is}'$ in the work of \cite{Bauer2022} regarding the maXs100 detector readout, which led to a lower flux-to-flux coupling despite the higher inductance of $L_{\rm i}'=\qty{5.47}{\nano\henry}$. Only the white noise reduction of the SQUID, which was caused by the shielding effects of the added flux transformer, resulted in a small overall improvement of $\epsilon_{\rm p}$ and $\Delta E_{\rm FWHM}$. \\
In this work we designed a new detector SQUID with \textit{window-type} Josephson junctions, which is based on the design developed in \cite{Bauer2022}. These type of junctions are realized by structuring window-shaped vias on top of the square junction area, which connects the junction with the overlying niobium layer. A schematic of this new SQUID is shown in figure \ref{abb:NL_FE}. Four large oval loops form the second order gradiometer described in subsection \ref{subsec_gradio}, where the lower fabricated niobium layer (Nb1) contains the SQUID loop as a parallel gradiometer. The feed line coming from the top supplies the serially connected input coil, which was fabricated as a second niobium layer (Nb2) on top of the SQUID loop, only separated by an insulating SiO$_{\rm 2}$ layer. This geometry allows to combine a small SQUID loop inductance $L_{\rm s}=\frac{L_{\rm l}}{4}$ with a large input inductance $L_{\rm i}=4L_{\rm l}$. Also in the Nb2 layer, another serial gradiometer of second order representing the feedback coil is located below the input coil as shown in figure \ref{abb:NL_FE}, with feed lines on the bottom left. This coil has a design inductance of $L_{\rm f}=\qty{336}{\pico\henry}$ with a line width of \qty{3}{\micro\meter}. Both coils exhibit the same geometry as the corresponding underlying washer strip to maximize the overlap and therefore the coupling. At the same time, the proximity to the input coil is kept at a minimum to minimize cross talk between the two coils. The feed lines of the SQUID loop at the bottom center lead to the junction area, which is shown in the zoomed in section. Both square-shaped Nb/Al-Al$\rm O_x$/Nb junctions are realized with the dimensions $\qtyproduct{4.5 x 4.5}{\micro\meter}$ and a targeted critical current of $I_{\rm c}=\qty{6}{\micro\ampere}$, leading to a critical current density of $j_{\rm c}=\qty{30}{\ampere\per\centi\meter\squared}$. Two AuPd junction shunt resistors $R_{\rm s}$ are located on the left and right side of the junction area, respectively. Both are attached to a large heat sink made of two gold layers, a thick galvanized layer on top of a sputtered, thin one. These so-called \textit{cooling fins} provide a better electron-phonon coupling thanks to their large volume, which reduces the electron temperature of the normal-conducting shunts and thus mitigates the corresponding thermal noise \cite{Mazibrada2024}. A third AuPd resistor $R_{\rm d}$ with the same dimensions as the shunt resistors is placed above the junction area and connected in parallel with the washer loop. This so-called \textit{washer shunt} provides damping properties to reduce quality factors of parasitic resonances, as will be discussed in section \ref{sec_damping}. \\

\figurecenter {t!}
{width=\textwidth}
{../Figures/FE_NL_design}
{0cm}
{(Hier kommt noch ein Mikroskop Bild dazu wie bei Fabian, Beschriftungen werden im Anschluss gemacht ...) Schematic of the new dc-SQUID design with a two-turn input coil, including a zoom into the junction area (bottom). Four large oval-shaped loops follow the microstrip transmission line structure given by the washer SQUID loop in the lower niobium layer Nb1 and the input coil in the upper niobium layer Nb2. Fabricated in the same manner is a small feedback coil between the junction area and the input coil. Both coils are free of $\rm{SiO_2}$, visualized through the inverse drawing of the $\rm{SiO_2}$ layer.} 
{NL_FE}

As opposed to the previous design, this input coil is realized with two turns instead of one. Neglecting the stripline inductance $L_{\rm str}$ (see section \ref{sec_resonance_results}), the input inductance becomes approximately proportional to the number of turns squared $n^2$ \cite{Ketchen1981,Jaycox1981}, giving a theoretical value of $L_{\rm i}^{\rm theo}\approx2^{2}\cdot \qty{1.64}{\nano\henry}\approx\qty{6.56}{\nano\henry}$. To implement the second turn a wider washer loop line width of $w_{\rm s}=\qty{10}{\micro\meter}$ was needed. The line width of the input coil $w_{\rm i}$ with two turns remained at $\qty{3}{\micro\meter}$, which would have not been possible to fabricate on top of the previous washer width of $w_{\rm s}=\qty{5}{\micro\meter}$ without sacrificing coupling strength. A general increase in $w_{\rm i}$, however, entails the risk of capturing noise inducing flux vortices. These form if the B-field pointing perpendicular to the SQUID plane exceeds a critical field given by \cite{Kuit2008} 

\gl{
B_{\rm v,crit} = 1.65\frac{\unit{\fq}}{w_{\rm s}^2}	\ \ .
}{}

At the earths surface, its magnetic field reaches a maximum amplitude of \qty{65}{\micro\tesla}, which would give a threshold width of $w_{\rm s}=\qty{7.2}{\micro\meter}$. The dilution refrigerators used to cool down both the SQUIDs and the detectors typically provide magnetic shielding, such that we consider a width of $w_{\rm s}=\qty{10}{\micro\meter}$ to have a negligible impact on the flux noise attributed to flux vortices. As the input inductance was the only parameter necessary to adjust, we attempted to keep the other design parameters unaltered. The widened washer width would therefore need to be compensated with an approximately \qty{15}{\%} larger washer hole circumference in order to maintain the same SQUID loop inductance. For the sake of safety, the circumference was increased only by \qty{10}{\%} to prevent possible hysteretic behavior, that can occur if the SQUID loop inductance $L_{\rm s}$ and therefore the screening parameter $\beta_{\rm L}$ grows too large. The increase was estimated by modeling the rather complicated oval washer loop geometry as a ring-shaped structure whose inductance can be calculated with the relation $L={\rm \mu_0}R\left(\ln\left(\frac{8R}{a}\right)-2\right)$, where \textit{R} denotes the loop radius and \textit{a} the radius of the wire \cite{Dengler2016}. In addition to these geometric adjustments, a comprehensive structuring of the SiO2 layer in the area of the washer loops was also omitted, allowing the empty interior to remain free from the insulating layer. This intends to prevent potential flux noise induced by unavoidable magnetic impurities in the $\rm SiO_2$. The same method was applied to the smaller loop areas formed by the feedback coil. The absence of insulation within the loops is visualized by the green-colored areas in figure \ref{abb:NL_FE}. \\

The design values for $R_{\rm s}$ and $L_{\rm s}$ are chosen such that the extrinsic energy sensitivity given in equation \ref{extr_energy_sens} is minimized, which requires the maximization of the flux-to-flux coupling ($L_{\rm i}=L_{\rm p}+L_{\rm par}$) and the minimization of the intrinsic white noise of the SQUID. In \cite{Bauer2022} this numerical calculation was done with the constraints $\beta_{\rm C}\leq 0.7$ and $\beta_{\rm L}\leq 1$ to avoid hysteretic behavior, which led to the optimal parameters $\beta_{\rm C}=0.7$ and $\beta_{\rm L}=0.86$. Consequently, additional noise through voltage jumps caused by hysteretic IVCs as well as Nyquist noise from higher harmonics of the Josephson frequencies \cite{Clarke1996} have been neglected for this minimization, which was not the case for the derivation of equation \ref{voltagenoise_psd}. The intrinsic white noise of the SQUID used for the numerical calculation is therefore given by the adjused expression \cite{Knuutila1988}

\gl{
S_{\rm \Phi_s} = 2k_{\rm B}T\frac{L_{\rm s}^2}{R_{\rm s}}\left[(1-k_{\rm is}^2s_{\rm in})^2+\frac{\sqrt{2}(1+\beta_{\rm L})^2}{\beta_{\rm L}^2}\right] \ \ .	
}{intr_FEnoise_minimize}

To achieve the targeted critical current of $I_{\rm c}=\qty{6}{\micro\ampere}$ with the given junction dimension of $\qtyproduct{4.5 x 4.5}{\micro\meter}$, we chose a critical current density of $j_{\rm c}=\qty{29.63}{\ampere\per\cm\squared}$. The current density can be used to calculate the junction capacitance \textit{C} by using the empirical relation $\frac{1}{C'}=p_1 + p_2\log_{10}j_{\rm c}$, where $p_{\rm 1}$ and $p_{\rm 2}$ are constant fit parameters \cite{Maezawa1995}. Here, the intrinsic capacitance $C'$ excludes any parasitic capacitances arising from the window-type fabrication technique. For simplicity reasons, we assume $C\approx C'$ and thus obtain $C=\qty{0.95}{\pico\farad}$. The optimal Stewart McCumber and screening parameter then provide the values $R_{\rm s}=\qty{6.3}{\ohm}$ and $L_{\rm s}=\qty{147}{\pico\henry}$, respectively. The designed shunt resistor in both the previous and the new design was rounded to \qty{6}{\ohm}, which results with Ohm's circuit law in a normal resistance of $R_{\rm n}=\qty{3}{\ohm}$ for the whole SQUID. This consequently corresponds to a slightly lower damping parameter of $\beta_{\rm C}=0.62$. The coupling constant was set to an upper limit of $k_{\rm is}=0.75$, which is typically the highest achievable value for the SQUIDs produced in this working group. Under the assumption of $k_{\rm is}$ being maximal and $L_{\rm i}=L_{\rm p}+L_{\rm par}=\qty{7.15}{\nano\henry}$ for the maXs100 detector read-out, the theoretically obtainable flux-to-flux coupling regarding a single pickup coil is $\frac{\Delta\Phi_{\rm s}}{\Delta\Phi_{\rm p}}=5.38\%$. Lastly, for the extrinsic energy sensitivity we would obtain with equation \ref{intr_FEnoise_minimize} $\epsilon_{\rm p}=\qty{0.53}{h}$, given the typical detector operation temperature of $T=\qty{20}{\milli\kelvin}$.   

\section{Damping Methods} \label{sec_damping}

As discussed in subsection \ref{subsec_para_res}, several SQUID parameters can be optimized to mitigate the influence of various resonances in the circuit. However, we would like to choose these parameters accordingly to the minimization of the extrinsic energy sensitivity as discussed in the previous section \ref{sec_FEdesign}. This choice imposes substantial limitations on the extent to which additional parameter modifications can be implemented. For instance, increasing the length of the input coil $l_{\rm i}$ would shift the corresponding strip line resonance given by equation \ref{stripline_res_general} away from the operation frequency, which in turn would lead to a larger input inductance, thus impeding the maximization of the flux-to-flux coupling. Furthermore, even resonances far away from the operation frequency can result problematic as thermally activated transitions between different states increase the noise level \cite{Sepp1987}. This motivates to follow a more practical approach to suppress \textit{LC} resonances, which can be realized through damping with attenuators, such as the damping resistor $R_{\rm d}$ shown in the top of junction area in figure \ref{abb:NL_FE}. These are typically connected in parallel to the resonant circuit, as this reduces the quality factor \textit{Q} of the corresponding \textit{LC} resonance given an appropriate dimensioning of the resistor. Consequently, the $L_{\rm s}C_{\rm p}$ and $L_{\rm s}C$ resonances can be damped by implementing a \textit{washer shunt} $R_{\rm d}$ \cite{Ono1997, Ryh1992}. Although the IVC intersection caused by the latter could not be fully eliminated in previous works of this group, they showed a significant smoothing of the curves as the accompanying step structures decreased \cite{Bauer2018}. The current noise introduced through this resistor, on the other hand, deteriorates the energy sensitivity and thus limits the damping benefit. However, this effect is minimal for the condition $R_{\rm d}\approx R_{\rm s}$ with $\beta_{\rm L}=1$ \cite{Enpuku1986, Ryh1992}, which is why we choose $R_{\rm d}=\qty{6}{\ohm}$. The input circuit is shunted with an $R_{\rm x}C_{\rm x}$ attenuator to damp the $L_{\rm i}C_{\rm p}$ resonance, where the added capacitance $C_{\rm x}$ blocks low frequency current noise \cite{Sepp1987}. Both damping techniques to suppress $C_{\rm p}$-related resonances were proven to be effective for various SQUID designs in many works \cite{Knuutila1987, Enpuku1986,Can1991,Bauer2018}. As for the $\lambda/2$ resonances, both the $R_{\rm d}$ and the $R_{\rm x}C_{\rm x}$ attenuator provide good damping as well \cite{Can1991}, since they terminate the microstrip lines and thus avoid impedance mismatches given a suitable dimensioning of the resistors (see section \ref{sec_resonance_results}). A schematic of the resulting circuit diagram of the coupled dc-SQUID with all damping components is depicted in figure \ref{abb:RxCx_circuit} (left). Shown on the right is the design of the $R_{\rm x}C_{\rm x}$ shunt, which has been adapted from \cite{Bauer2022}. The capacitance $C_{\rm x}=\qty{10}{\pico\farad}$ is divided into two parallelly connected square-shaped capacitors with $\frac{C_{\rm x}}{2}$ to reduce the space needed on the chip. The top and bottom plate of each capacitor are fabricated in the Nb2 and Nb1 layer, respectively. The shunt $R_{\rm x}$ is split as well into two parallel AuPd resistors with $2R_{\rm x}$ each. Optimal values for these components will be discussed in chapter \ref{ch_results}.

\twofigurescenter{t!}
{width=0.48\textwidth}
{../Figures/gekoppeltes_SQUID}
{0.02\textwidth} %hspace
{width=0.48\textwidth}
{../Figures/FE_RxCx}
{0.5cm} %vspace
{Left: Schematic circuit of a coupled dc-SQUID with damping components $R_{\rm d}$ and $R_{\rm x}C_{\rm x}$. The input coil forms a flux transformer with a pickup coil of inductance $L_{\rm p}$. The parasitic capacitance between the input circuit and the SQUID loop is connected in parallel to both $L_{\rm i}$ and $L_{\rm s}$. Right: Structure design of the $R_{\rm x}C_{\rm x}$ attenuator located above the dc-SQUID. Both devices $R_{\rm x}$ and $C_{\rm x}$ are each realized as two parallel components. (Beschriftungen werden noch hizugefügt)}
{RxCx_circuit}

\subsection{Lossy Input Coil}\label{subsec_L_FE}

Although the above-mentioned damping methods provide significant improvement regarding the resonance behavior, further suppressing individual resonances is, depending on the SQUID design, typically favourable or even necessary. For this reason, we apply two additional damping techniques, which require resistive gold layers electrically or inductively coupled to the circuit. The former will be covered in the following, whereas the latter is explained in subsection \ref{subsec_ind_damp}. \\
Several approaches to reduce \textit{Q} values of $\lambda/2$ resonances associated with microstrip lines have been investigated in \cite{Boyd2022}. The experimental setup consists of two parallel meanders with each having a length of \qty{6.5}{\milli\metre} and and an inductance of \qty{2}{\nH}. They are fabricated on top of each other with insulating $\rm SiO_2$ between them and in a direct-coupled MMC setup, where as opposed to the flux transformer setup in section \ref{sec_MMC}, the SQUID is directly coupled to the sensor. This setup produced high \textit{Q} resonances at integer and half-integer wavelengths. Whereas placing an individual resistor in parallel to one meander only damped the half-integer modes, a more distributed damping scheme in the form of an insulated gold layer between both meanders provided strong damping of all microstrip resonances while maintaining a low detector noise level. The \textit{Q} values have been further reduced by structuring the Au layer with the same geometry as the microstrip lines and electrically connecting it to one of the meanders, thereby preventing large noise inducing, normal-conducting loops. The attenuation constant $\alpha$ of this lossy microstrip line increases with frequency and thickness of the Au layer, while the noise stays insensitive upon thickness changes. \\
Within the scope of this thesis, such damping techniques were implemented and tested on the dc-SQUID described in section \ref{sec_FEdesign}. For this, we sputtered a gold layer between $\rm SiO_2$ and Nb2 with the same geometry as the input coil running above the SQUID loop. This gold layer was omitted around the vias between the washer loops to avoid additional normal resistances. Gold and niobium were structured together as a bilayer in the same microfabrication step by sputtering the upper niobium layer directly after the gold. To connect the four input coil segments over vias, the original second niobium layer Nb2 was used. A schematic of our Front-End design with this lossy input coil, including an image obtained with an optical microscope, is shown in figure \ref{abb:FE_L}. Our Front-End SQUIDs with this gold layer adjacent to the input coil will be referred to as 'lossy' for the upcoming discussions. 

\figurecenter {t!}
{width=\textwidth}
{../Figures/FE_L_design}
{0cm}
{(siehe figure 3.1) dc-SQUID design with a two-turn input coil realized as a lossy microstrip line. A gold layer is structured between the insulating $\rm SiO_2$ and an upper niobium layer and has been fabricated in a single step as a bilayer. The overlying niobium in the Nb2 layer connects all four input coil segments over their respective washer loop to prevent the vias to acquire a normal resistance.} 
{FE_L}

\subsection{Inductive Damping} \label{subsec_ind_damp}

Together with the attempt of suppressing resonances through a lossy microstrip line, we introduce a second new damping technique denoted as inductive damping. This method is based on the principle of magnetic damping, where a change in magnetic flux creates eddy currents in a nearby conductor, which following Lenz's law induces a flux trying to compensate the initial one. The flux change is therefore damped by effectively transferring part of the magnetic energy to the kinetic energy of the induced current, which in the case of a normal conductor emits heat. This phenomenon also causes the reduction of the geometric inductance of the SQUID loop due to the shielding effect of the flux transformer (see subsection \ref{subsec_extr_sens_theo}). A strong indicator that this mechanism can be applied to SQUIDs shows the first mentioned experiment in \cite{Boyd2022}, where a square gold layer representing the MMC sensor was placed at a hight of \qty{300}{\nano\meter} above an isolated meander. This lead to a significant reduction of the high \textit{Q} values of the meander modes. The concept is now applied to the feed lines on our SQUID chip, where large gold pads have been placed across the SQUID washer and feedback coil feed lines, as shown in figure \ref{abb:damped_chip}. The sharp voltage spikes associated with high \textit{Q} resonances, that can be present within those lines would therefore be damped by partly converting their energy into heat in the normal conducting gold layer. The gold pads are fabricated in the same layers as the heat sinks for the shunt resistors and are consequently sputtered first before being electroplated. The latter step significantly increases their volume, which allows for larger and more effective eddy currents. Additionally, the generated heat is expected to better dissipate into the chip substrate as the electron-phonon coupling increases with volume. All feed lines are typically realized as microstrip lines, but for this case we adjusted them into a coplanar structure as there is no insulation layer after Nb2. Figure \ref{abb:damped_chip} shows the chip design consisting of four distinct Front-End variants, each provided with the inductive damping scheme. The first channel at the top is realized without an input coil in order to better allocate possible resonance structures visible in the SQUID's IVCs. The design introduced in \ref{sec_FEdesign}, also referred to as 'non-lossy', is represented in channel 2, followed by the lossy variant in channel 3. The last channel contains a lossy Front-End as well, however, the washer loop interiors were not kept $\rm SiO_2$-free. This allows to investigate the influence of possible magnetic impurities in the insulation material and assess whether it can be regarded as negligible or not. The same chip design has been produced with and without gold pads on the feed lines, resulting in 8 different Front-End SQUIDs that were developed and tested within the scope of this thesis. We present our results in the following chapter.     

\figureleft {t!}
{width=\textwidth} %sets how much of the fig space is used
{../Figures/damped_chip}
{9cm} %sets width of the fig space
{0cm}
{(siehe figure 3.1 ...) Inductive damping scheme on the Front-End SQUID chip of the type 4x100i6 v1.4. All feedback coil and SQUID loop feed lines are covered with large rectangular, insulated gold pads. Each SQUID channel is occupied with a different SQUID variant. Channel 1: Front-End without an input coil. Channel 2: Non-lossy SQUID design presented in section \ref{sec_FEdesign}. Channel 3: Lossy SQUID design from section \ref{subsec_L_FE}. Channel 4: Lossy design with $\rm SiO_2$ inside the washer loop interiors.}
{damped_chip}


\chapter{Experimental Results} \label{ch_results}

This chapter will provide an overview over the general performance of the first stage dc-SQUIDs developed within the framework of this thesis. Particularly, the overall resonance behavior and the noise spectra were investigated. We will begin with a summary of characteristic parameters obtained by our measurements and compare them with the target values. The SQUIDs were produced in the institute's cleanroom and then tested both in a single- and a two-stage setup as described in section \ref{sec_operation}. The single-stage measurement was carried out at $T=\qty{4.2}{\kelvin}$ in a liquid helium transport vessel as well as in a dilution cryostat with $T=\qty{10}{\milli\kelvin}$. The former submerges the SQUIDs in liquid helium via a dipstick, which provides a sample holder for PCBs. The SQUIDs are glued onto those PCBs and can be electrically connected to them through aluminum bond wires, by utilizing the bond pads shown around the border of the chip in figure \ref{abb:damped_chip}. The sample holder is equipped with both a superconducting (niobium) and a soft-magnetic shield to suppress external magnetic fields. The read-out is enabled by connecting the broadband SQUID electronic of the type XXF-1 (see subsection \ref{subsec_fll}) to the dipstick and using it to both supply the necessary bias and ramp current signals to the SQUID as well as provide the FLL feedback circuit. The SQUID electronic is then controlled via software and the voltage output observed on a keysight InfiniiVision\footnote{Keysight Technologies Deutschland GmbH, Herrenberger Straße 130, 71034 Böblingen} oscilloscope. The noise measurement has been conducted with the two-stage setup at $T=\qty{10}{\milli\kelvin}$ in the cryostat, whereas the single-stage measurements to obtain characteristic Front-End properties have been done both in the helium vessel and the cryostat. 

\section{Characteristic dc-SQUID Parameters}

Despite all above-mentioned distinctions between the Fron-End variants, most characteristic parameters such as the SQUID loop inductance $L_{\rm s}$ or shunt resistance $R_{\rm s}$ are unaffected by these variations. We therefore consider the following measurements to be representative for all variants and assume the errors to stem from fabrication-related deviations across the wafer. \\
To qualitatively explain the following considerations, we present the current-voltage as well as the voltage-flux characteristics of one of the measured SQUIDs, shown in figure \ref{abb:iv_vphi_result}. This SQUID with the label HDSQ17-W1-3C16 is of the type 'non-lossy' with inductive damping and has been measured at $T=\qty{10}{\milli\kelvin}$. We start the characterization by measuring the slope of the IV curves across the ohmic regime, which corresponds to the normal resistance $R_{\rm n}$ of the SQUID. Including the SQUID chosen for figure \ref{abb:damped_chip}, we measured 10 SQUIDs from the wafer HDSQ17-W1 covering all above-mentioned variants. The median normal resistance taken from these SQUIDs yielded the value $R_{\rm n}=\qty{3}{\ohm}$. This corresponds to a shunt resistance $R_{\rm s}=2R_{\rm n}$ of $R_{\rm s}=\qty{6}{\ohm}$, which exactly corresponds to the targeted value of $R_{\rm s}=\qty{6}{\ohm}$. The maximum voltage swing as seen in figure \ref{abb:iv_vphi_result} (right) is obtained by applying a bias current $I_{\rm b}^{\rm max}$ supplied by the SQUID electronic, while driving a ramp signal through one of the coupled coils to provide a varying external flux. This yielded median values of $\Delta V_{\rm max}=\qty{29.95}{\micro\volt}$ and $I_{\rm b}^{\rm max}=\qty{11.62}{\micro\ampere}$. For $T=\qty{0}{\kelvin}$, $I_{\rm b}^{\rm max}$ should correspond to twice the critical current $I_{\rm c}$ (compare figure \ref{abb:IV_VPhi_theo}). Due to the finite temperature we obtain noticeable thermal smoothening visible in the IVC of figure \ref{abb:iv_vphi_result} (left), such that the critical current is better approximated by \cite{Drung1996}

\twofigurescenter {t!}
{width=0.48\textwidth}
{../Figures/IV_3c16_ch2}
{0.02\textwidth} %hspace b/w figures
{width=0.48\textwidth}
{../Figures/VPhi_3c16_ch2}
{0.5cm} %vspace
{Current-voltage (left) and voltage-flux characteristics (right) of a non-lossy SQUID with inductive damping, labelled HDSQ17-W1-3C16. The family of IV curves are measured by varying the bias current for either the input or the feedback coil, while a current ramp signal is driven through the SQUID. To obtain the corresponding $V\Phi$ characteristic, the roles for the bias and ramp signal are switched. The curve was measured at $I_{\rm b}^{\rm max}=\qty{11.33}{\uA}$, resulting in a maximal voltage swing $\Delta V_{\rm max}=\qty{30.7}{\uV}$. Note that both plots have different scaling on the y-axis.}
{iv_vphi_result}


\gl{
I_{\rm c} \approx \frac{I_{\rm b}^{\rm max}}{2} + \frac{k_{\rm B}T}{\unit{\fq}}\left(1+\sqrt{1+\frac{I_{\rm b}^{\rm max}\unit{\fq}}{k_{\rm B}T}}\right) \ \ .
}{}

We therefore calculate the critical current to\footnote{Calculated with the median of all 10 Ic's. If I calculate Ic once with the median of Ib,max, then the result is Ib,max/2 $\approx$ 5.81. Which is better? For the above argument, 5.84 fits better.} $I_{\rm c}=\qty{5.84}{\micro\ampere}$. This corresponds to a critical current density of $j_{\rm c}=\qty{28.8}{\ampere\per\centi\meter\squared}$, such that both values only deviate \qty{2.7}{\percent} from the design values. The junction capacitance thus becomes slightly smaller than the design value according to the empirical relation introduced in section \ref{sec_FEdesign}. Using the fitted function $\frac{1}{C'}=0.132-0.053\log_{10}j_{\rm c}$ obtained in \cite{Bauer2022} and by again assuming $C'=C$, the capacitance yields $C=\qty{0.948}{\pico\farad}$, which is in excellent agreement with the target value of \qty{0.95}{\pico\farad}. The McCumber parameter can now be calculated to $\beta_{\rm C}=0.61$, also fitting well with the design value of 0.62. The screening parameter is typically derived by using the $\beta_{\rm L}$-dependent normalized current swing $\frac{\Delta I_{\rm max}}{2I_{\rm c}}$ at $V=0$, which has been numerically simulated in \cite{Tesche1977}. We estimated the current swing by extrapolating both extremal IV curves for the theoretical case of $T=0$, thus neglecting the thermal rounding and obtaining a minimal and maximal critical current $I_{\rm c,1}$ and $I_{\rm c,2}$. For our SQUIDs, the extremal IV curve with the lower critical current $I_{\rm c,1}$ does not correspond exactly to $\Phi = (n+\frac{1}{2}\Phi_0)$ (compare figure \ref{abb:iv_vphi_result}), 
which is a consequence of an asymmetric current injection (see below). The median of the maximal current swing $\Delta I_{\rm max}$ therefore yields \qty{7.18}{\micro\ampere}, ranging from \qty{6.09}{\micro\ampere} to \qty{7.73}{\micro\ampere} for the lowest and highest measured value. The value for $\Delta I_{\rm max}$, which approximately corresponds to the current swing of the voltage-biased SQUID in a two-stage configuration, indicates that the SQUID is well adapted to the arrays produced in this working group \cite{Kempf2015}\footnote{Die arrays in diesem paper sind andere als die die wir nutzen und haben $M^{-1}=12.9$ µA/phi0. Wo finde ich veröffentlichungen zu HDSQ15w3 arrays?}. These arrays have a typical reciprocal mutual inductance of $\frac{1}{M_{\rm ix}}=\qty{11.7}{\micro\ampere\per\fq}$, such that the flux $\Delta\Phi_{\rm x}$ coupled to the array is $\Delta\Phi_{\rm x}=M_{\rm ix}\Delta I_{\rm max}=\qty{0.61}{\fq}$, which is close to the optimal value of $\frac{\unit{\fq}}{2}$ to achieve a flux gain of $G_{\rm \Phi}\approx 3$ (see section \ref{subsec_2stage_theo}). The current swing allows us to calculate the screening parameter to $\beta_{\rm L}=0.60$, which is rather low compared to the design value of $\beta_{\rm L}=0.86$. Hysteretic behavior related to the obtained $\beta_{\rm L}$ and $\beta_{\rm C}$ should nevertheless not occur and thus be absent in the IVCs. The intersections and current steps seen in figure \ref{abb:iv_vphi_result} likely stem from resonances, which will be discussed in section \ref{sec_resonance_results}. \\ 

With $I_{\rm c}=5.84$ and $\beta_{\rm L}=0.60$ we obtain a SQUID loop inductance of $L_{\rm s}=\qty{103}{\pico\henry}$, which fits very well with a simulated value of $L_{\rm s}=\qty{106}{\pico\henry}$. This significantly deviates from the design value of $L_{\rm s}=\qty{147}{\pico\henry}$, which can be partly explained by the conservative estimation for the geometric adjustment of the washer loop size, tendentially resulting in a smaller inductance. Another reason is a that the SQUID loop inductance of the previous design resulted in a lower value than intended in \cite{Bauer2022}. On the wafer HDSQ17-W1 we included for comparison the original design with a single turn input coil, where we measured two channels of the chip HDSQ17-W1-3A09 alongside the other 10 new SQUIDs discussed in this section. These provided the respective values $L_{\rm s}=\qty{127}{\pico\henry}$ ($\beta_{\rm L}=0.80$) and $L_{\rm s}=\qty{138}{\pico\henry}$ ($\beta_{\rm L}=0.87$), which further supports the assumption of the original SQUID loop design being too small. A better adjustment of $L_{\rm s}$ can be realized in future works. \\ 
 
The mutual inductances of the input and feedback coil were obtained by both sending a bias current to the SQUID loop and a current ramp signal to the coils. The input coil was only connected to the SQUID electronic during the measurements in the helium vessel at \qty{4.2}{\kelvin}, where we calculated $M_{\rm is}$ of 22 Front-End channels from 9 chips, covering all non-lossy and lossy variants. Counting the obtained voltage oscillations per given current ramp amplitude leads to the measured median values $M_{\rm is}=\qty{611}{\pico\farad}$ and $M_{\rm fs}=\qty{51}{\pico\farad}$. The latter coincides well with the measured value of \qty{50}{\pico\farad} from \cite{Bauer2022}. The input coil current sensitivity is almost twice as large as the value determined with the previous design with a single-turn input coil, which yielded $M_{\rm is}=\qty{328}{\pico\farad}$ \cite{Bauer2022}. This result can be understood by the general linear dependence of the mutual inductance from the number of turns \textit{n} of the input coil and the SQUID loop inductance, i.e. $M_{\rm is}\approx nL_{\rm s}$ \cite{Ketchen1981}. This expression would need to be multiplied by a factor of 4 due to the parallel connection of the four washer loops. The result doesn't match the lower measured value of the previous design for $n=1$, however, one needs to account for the parasitic inductance at the junction area that doesn't contribute to M, which in turn reduces the mutual inductance. The fact that $M_{\rm is}\approx 2\cdot \qty{328}{\pico\farad}>\qty{611}{\pico\farad}$ can be explained by two reasons. On one hand, the inductance $L_{\rm s}$ has been reduced by approximately \qty{20}{\percent} as compared to the previous design, which according to equation \ref{kis_theo} leads to a decreased current sensitivity $M_{\rm is}$. On the other hand, the geometric two-turn input coil structure might deteriorate the coupling constant $k_{\rm is}$, as compared to a single turn design. This parameter will be determined together with the input coil inductance in the following subsection \ref{subsec_Li}.

As for the transfer coefficients, we obtained different values for the positive and negative slopes of the $I\Phi$ curve. This is attributed to the intended asymmetric current injection, which is realized by differing SQUID loop inductances between each arm of the loop, ultimately leading to a unilaterally larger transfer coefficient $I_{\rm \Phi}$ \cite{Ferring2015}. The values are $I_{\rm \Phi,+}=\qty{10.2}{\micro\ampere\per\fq}$ and $I_{\rm \Phi,-}=\qty{23.1}{\micro\ampere\per\fq}$ for the positive and negative slope, respectively. The voltage transfer coefficient is approximately symmetric and yielded $V_{\rm \Phi,+}=\qty{94.0}{\micro\volt\per\fq}$ and $V_{\rm \Phi,-}=\qty{93.0}{\micro\volt\per\fq}$. Except for $I_{\rm \Phi,+}$, the variance of these transfer parameters was rather large, with deviations from the median of up to \qty{39}{\percent}. These results are nevertheless comparable to previous SQUIDs produced in this working group.      


%Discuss various measured/simulated parameters such as Rs (mention sputtered thickness and sheet resistance),Rn,$\Delta V$,Ic,beta's,Ls,Li,M's and calculate $\frac{\Delta\Phi_{\rm s}}{\Delta\Phi_{\rm p}}$, $\epsilon_p$. Summarize the values in a table. Compare with previous FEs and literature. 


\subsection{Input Coil Inductance} \label{subsec_Li}

To determine whether the new SQUID design with increased input inductance exhibits better coupling to the maXs100 detector, it is essential to measure $L_{\rm i}$. For this measurement, the input coil is electrically shorted via its bond pads with aluminum bond wires while the Front-End is in a two-stage setup at \qty{4.2}{\kelvin}. In this case, the detector SQUID is in an integrated chip that contains both the new non-lossy Front-End design as well as an array for the second stage, which has been developed and tested by \cite{Kraemer2023}. These integrated chips were produced on the wafer labeled as HDSQ16-W1 and were designed with the same material thicknesses, such that we expect the following considerations to be representative for the non-lossy SQUIDs from HDSQ17-W1. The aluminum bond wires are normal conducting at these temperatures and thus provide a resistance $R_{\rm bond}$. This closed loop couples via $M_{\rm is}$ thermal noise from the resistive wires into the SQUID loop, which is added to the apparent noise of the first stage SQUID. The resistance $R_{\rm bond}$ forms together with the total inductance $L_{\rm tot}$ consisting of the input coil, its feed line and the wire inductance an \textit{RL} lowpass filter. The attributed cut-off frequency damps higher noise frequencies up to the point where this noise contribution becomes negligible, as can be seen in the measured noise spectrum of figure \ref{abb:Li_meas}. The second drop at even higher frequencies represents the lowpass characteristic of the SQUID electronic, as it provides a limited bandwidth of up to \qty{7}{\mega\hertz}\footnote{6 oder 7?}. The total apparent noise of the Front-End SQUID is then given by     

\figureleft {t!}
{width=\textwidth} %sets how much of the fig space is used
{../Figures/Li_meas}
{9cm} %sets width of the fig space
{0cm}
{Noise spectrum of the input coil inductance $L_{\rm i}$ measurement. The input coil is resistively shorted through aluminum bond wires with resistance $R_{\rm bond}$, which results in thermal noise that couples into the SQUID. The added noise has a cut-off frequency $f_{\rm c}=\frac{R_{\rm bond}}{2\pi L_{\rm tot}}$ that allows to extract the added white noise component by a numerical fit. The input inductance can be derived by substracting the parasitic inductances from $L_{\rm tot}$, which is a fit parameter alongside $R_{\rm bond}$.}
{Li_meas}

\gl{
S_{\rm \Phi_s,SQ} = M_{\rm is}^2\frac{4k_{\rm B}T}{R_{\rm bond}}\left[
\frac{1}{1+(\frac{2\pi fL_{\rm tot}}{R_{\rm bond}})^{2}}\right]	+ S_{\rm \Phi_s,w} \ \ ,
}{}

where the first term describes the added current noise of the shorted circuit, which is schematically shown in the inset of figure \ref{abb:Li_meas}. The second term represents the apparent white noise of the SQUID. This expression is numerically fitted to the measured data to obtain both the white noise component $ M_{\rm is}^2\frac{4k_{\rm B}T}{R_{\rm bond}}$ and the cut-off frequency $f_{\rm c}=\frac{R_{\rm bond}}{2\pi L_{\rm tot}}$. Two measurements from separate chips (2C14 and 2C23) were carried out, where the one from 2C14 is depicted in figure \ref{abb:Li_meas}. The fits resulted in $R_{\rm bond}^{\rm 2C14}=\qty{2.07}{\milli\ohm}$ and $L_{\rm tot}^{\rm 2C14}=\qty{6.86}{\nH}$ for the chip 2C14, whereas $R_{\rm bond}^{\rm 2C23}=\qty{1.86}{\milli\ohm}$ and $L_{\rm tot}^{\rm 2C23}=\qty{6.74}{\nH}$ for the chip 2C23. It has been shown in \cite{Hengstler2017}, that the aluminum wires exhibit an inductance of $L_{\rm bond}=\qty{0.14}{\nano\henry\per\milli\ohm}$, which consequently results in $L_{\rm bond}^{\rm 2C14}=\qty{0.29}{\nH}$ and  $L_{\rm bond}^{\rm 2C23}=\qty{0.26}{\nH}$. The other parasitic inductance that stems from the input coil feed lines can be estimated with the microstrip inductance per unit length \cite{Chang1979}

\gl{
L_{\rm str}=\mu_0\frac{h+2\lambda}{w_{\rm i}+2h+4\lambda} \ \ ,	
}{Lstr} 

with $\lambda$ being the London penetration depth of the superconductor, $w_{\rm i}$ the width of the upper line and $h$ the thickness of the insulating layer. The length of the feed lines is approximately \qty{724}{\um}, the width \qty{3}{\um} and $\rm{SiO_2}$ has been fabricated as a \qty{375}{\nm} thick layer. We also assume $\lambda=\qty{90}{\nm}$, which is a typical value for the fabricated niobium in this working group. This yields a smaller contribution of $\qty{0.12}{\nH}$ for the feed line inductance. Thus, we finally obtain the input inductances $L_{\rm i}^{\rm 2C14}=\qty{6.45}{\nH}$ and $L_{\rm i}^{\rm 2C23}=\qty{6.36}{\nH}$. Even though these are close to the in the beginning of section \ref{sec_FEdesign} approximated $L_{\rm i}^{\rm theo}\approx n^{2}\qty{1.64}{\nH}\approx\qty{6.56}{\nH}$, we should take into account the smaller measured inductance $L_{\rm i}=\qty{1.27}{\nH}$ for the single turn input coil \cite{Bauer2022}. Furthermore, the microstrip inductance should not be neglected anymore as the length $l_{\rm i}$ of the input coil with two turns is now approximately twice as large. $L_{\rm i}$ is only linear proportional to the total microstrip inductance $l_{\rm i}L_{\rm str}$ \cite{Ketchen1991}, which would further reduce the increase of $L_{\rm i}$ upon adding a second turn. However, the enlargement of the washer loop has a stronger influence on $L_{\rm i}$ than on $L_{\rm s}$ due to the series connection of the input coil, such that the inductance increase per washer loop is multiplied by 4. This effect seems to narrowly compensate the two reductions, such that the measured values are only slightly below $L_{\rm i}^{theo}$. With InductEx we obtained a simulated value of $L_{\rm i}=\qty{5.7}{\nH}$?, which deviates \qty{20}{\percent}? and \qty{20}{\percent}? from $L_{\rm i}^{\rm 2C14}$ and $L_{\rm i}^{\rm 2C23}$, respectively. These large deviations are in accordance with previous inductance simulations in several works of this working group \cite{Ferring2015, Bauer2022}. For the following discussions, we will use the average $L_{\rm i}=\qty{6.40}{\nH}$ of the two measurements. \\

We have now all parameters to determine the coupling constant $k_{\rm is}$. Using equation \ref{kis_theo}, we obtain $k_{\rm is}=0.75$, which corresponds to the target value that has been set as a realistic upper limit for the minimization of the extrinsic energy sensitivity $\epsilon_{\rm p}$. This suggests that the reduction of the input coil current sensitivity $M_{\rm is}$ solely stems from the small measured SQUID loop inductance $L_{\rm s}$. It is noteworthy, however, that the variance of both the measured critical currents and the current swings was rather large with deviating up to \qty{11}{\percent} and \qty{15}{\percent}, respectively. This of course provides an uncertainty for $\beta_{\rm L}$ and $L_{\rm s}$ that needs to be taken into account. Such variances have also been observed in \cite{Bauer2022}, where it was shown that SQUIDs with \textit{cross-type} junctions exhibit a smaller variance across the wafer, making them more reliable than window-type junctions regarding $I_{\rm c}$. \\    
An overview of the most relevant measured parameters compared to their respective design values is shown in table \ref{tab:SQUIDparameters}.

\begin{table}[htb]
	\centering
	\begin{tabular}{c|*{9}{c}}
	\toprule
		Parameter & $R_{\rm s}$ & $I_{\rm c}$ & $M_{\rm is}$ & $M_{\rm fs}$ & $L_{\rm s}$ & $L_{\rm i}$ & $\beta_{\rm L}$ & $\beta_{\rm C}$ & $k_{\rm is}$ \\
		 & $[\Omega]$ & $[\unit{\micro\ampere}]$ & $[\unit{\pH}]$ & $[\unit{\pH}]$ & $[\unit{\pH}]$ & $[\unit{\nH}]$ &  &  &  \\
		\midrule
		Measured & 6 & 5.84 & 611 & 51 & 103 & 6.40 & 0.60 & 0.61 & 0.75 \\
		Design & 6 & 6 & - & 50 & 147 & 6.56 & 0.86 & 0.62 & 0.75 \\
	\end{tabular}
	\caption{Summary of measured characteristic parameters of the new dc-SQUID design with a two-turn input coil, which are compared with the corresponding target values.}
	\label{tab:SQUIDparameters}
\end{table}

\section{Resonance Behavior}\label{sec_resonance_results}

Discuss in more detail all types of resonances with formulae and calculate the frequencies (there are still two of them that are unclear/need to be discussed in private or in a meeting)

\textit{Figures}: All measured FE IVCs, like non-lossy, lossy, damping variants, iso, no-iso, no input coil, Fabiennes FE... (should I also show the corresponding VPhi curves?)

Try to identify the resonances in the plots. Discuss what actually helped damping and what not. 

%Possible reason for LsC being too low: betaL theo adapted for symmetric squids, might need correction for asymm. -> could mean that betaL is too small, and therefore Ls. Would make sense as we did not expect Ls to be \textit{that} small, also, InductEX seems to underestimate inductances and it simulated Ls=106pH, favoring a larger "real" Ls.

\section{Noise Performance}

\subsection{Lumped Element Two-Stage Setup}

\textit{Figures}: Show noise measurements at mK (3 setups from SQUID Cryo, 1 setup from Mocca Cryo) 

Discuss all contributions, especially from the FE -> Difference if input coil is shorted or not?

Mention how it would change with a cross JJ SQUID (lower white noise) -> here or in the summary?

Are 2stage VPhi curves interesting or do we restrict ourselves to noise plots?

\subsection{Integrated Two-Stage Setup}

\textit{Figure}: Int. 2stage noise measurement

Summarize Fabians measurements and compare with ours. 
\chapter{Summary}

-Slightly increase Rs to obtain ideal betaC \\
-Same with Ls \\
-Try crosstype for even lower noise

\bibliographystyle{bibstyle_andi_english} 
\bibliography{library}
\pagestyle{empty}
\vfill
\newpage
%\chapter*{Erklärung}


%\nonumber\kap{}{Erklärung:}

\cleardoublepage

\def\makeheadline{}
%\makeheadline

\rule{0cm}{5cm}

\bigskip
\noindent Ich versichere, dass ich diese Arbeit selbstst\"{a}ndig verfasst und keine anderen als die angegebenen Quellen und Hilfsmittel benutzt habe. \vskip 2.5 cm \noindent
\hbox{Heidelberg, den 04. Februar 2024}\hfill{ 
	...........................................}

\hskip 11.0 cm (Nicolas Robert Kahne)

%\dateiende



\end{document}