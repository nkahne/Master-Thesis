\documentclass[12pt,a4paper]{book}

\pdfminorversion=6

% allgemeine Pakete
%\usepackage{ngerman}
\usepackage[ngerman,english]{babel} %whatever comes last defines the language of the document
\usepackage{graphicx}
\usepackage{wrapfig}
\usepackage{blindtext}
\usepackage[utf8]{inputenc}
\usepackage[rightcaption]{sidecap}
\usepackage{ifthen}
%\usepackage[small,normal,bf]{caption2}
\usepackage{dipl2}
\usepackage{footmisc}	% für mehrere identische Fußnoten, muss vor hyperref kommen, sonst passen die Fußnoten-Links nicht
\usepackage[linkcolor=black,
	citecolor=black,
	filecolor=black,
	menucolor=black,
	urlcolor=black,
	colorlinks=true,
	pdftitle={Titel MA N.K.},
	pdfsubject={Master Thesis},
	pdfkeywords={},
	pdfauthor={Nicolas Kahne},
	]
{hyperref}
\usepackage[intlimits]{amsmath}
\usepackage{cleveref}
%\usepackage{units}
\usepackage[separate-uncertainty = true]{siunitx} % ,multi-part-units=single, exponent-product = \cdot
\usepackage{nicefrac}
\numberwithin{equation}{chapter}
\usepackage{amssymb}
\usepackage[T1]{fontenc}
\usepackage{multirow, bigdelim, bigstrut}
% \includeonly{theorie,}
\parindent 0em
\usepackage{mathtools}
\usepackage[toc]{appendix}
\usepackage{placeins}
%\usepackage{isotope} % does not work in headings
\usepackage{rotating}
%\usepackage{gensymb} % not needed when using SIunitx
\usepackage{pdflscape} % einzelne Seiten im Querformat
\usepackage{array} % linksbündig in p-Spalten in Tabellen
%\usepackage{bbm} % for unit-matrix
%\usepackage{chemformula} % z.B. für Radikalpunkte
\usepackage{bm}
%\usepackage{caption}
%\captionsetup{width=0.9\linewidth}

%\usepackage{caption}
%\usepackage{fontspec}

%\newfontfamily\arial{Arial}

% für SIUNITX
%\DeclareSIUnit\ph0{\text{\Phi$_{0}$}}
\DeclareSIUnit\percent{\text{\%}}
\DeclareSIUnit{\sqrthz}{\ensuremath{\sqrt{\unit{\hertz}}}}
\DeclareSIUnit{\inch}{\text{in}}
\DeclareSIUnit{\fq}{\Phi_0}	% Flussquant
\sisetup{per-mode=fraction}
% für die Bilder in der Tabelle
%\newcommand{\minifig}[2]{\raggedright #1 \newline \raisebox{-0.9\totalheight}{\includegraphics[width=1.5cm]{#2}}}	 %für Hochformat
\newcommand{\minifig}[2]{ #1 & \raisebox{-0.7\totalheight}{\includegraphics[width=1.5cm]{#2}}}	 %für Querformat

%\hyphenation{ex-peri-men-tal}
%\hyphenation{pa-ra-mag-ne-tic}
%\hyphenation{reso-lu-tion}

%\def\titeldeutsch{Entwicklung und Charakterisierung von zweidimensionalen Arrays aus metallischen magnetischen Kalorimetern für die hochauflösende Röntgenspektroskopie}
%\def\titelenglisch{Development and characterization of two-dimensional metallic magnetic calorimeter arrays for the high-resolution X-ray spectroscopy}

% Eigentliche Arbeit
%\begin{document}
	
	% Keine Kopf- oder Fu?zeilen
%	\pagestyle{empty}
%	\hypersetup{pageanchor=false}
	% include erzeugt ein newpage und fügt danach die angegebene Datei in den Quelltext ein. Die eingefügte Datei wird als normaler Quelltext mitverarbeitet und sollte daher nur aus einer Folge von LaTeX-Befehlen bestehen (ohne Vorspann etc.)
	%\include{kipcover}
	%\include{titel}
	%\pagestyle{empty}
	%\thispagestyle{empty}

	\begin{titlepage}
		\setcounter{page}{-2}
		\setlength{\textheight}{28cm}
		\setlength{\topmargin}{-20mm}
		
		% -------------- ENGLISH ABSTRACT ----------- %
		\begin{center}
			\fbox{\rule{0.5cm}{0cm}\parbox{14.9cm}{\bigskip
					\small
					\noindent
This thesis describes the design and development of a current-sensor dc-SQUID, optimized for the readout of metallic magnetic calorimeter based particle detectors. Maximizing the energy resolution of the detector is achieved by minimizing the intrinsic noise of the SQUID and by matching the input inductance of the SQUID with the inductance of the detector. This was realized by adding a second turn to the input coil of a previously produced SQUID to match the pickup coil inductance $L_{\rm p}=\qty{6.65}{\pH}$ of the maXs100 detector, which is currently being developed in this working group. To mitigate resonant structures in the SQUID dynamics, two new damping techniques were employed. These consisted of gold layers sputtered both atop the feed lines and between SQUID loop and input coil, with the latter representing a lossy microstrip line. Successful damping was achieved by the former as it resulted in substantially smoother current-voltage characteristics. The noise measurements of the new SQUID with various combinations of the damping scheme were performed at a temperature of $\qty{10}{\milli\kelvin}$. These yielded up to $\sqrt{S_{\Phi_{\rm s}, \rm w}}=\qty{0.22}{\micro\fq\per\sqrthz}$ for the frequency independent contribution, whereas the $1/f$ component was as low as $\sqrt{S_{\Phi_{\rm s}, 1/f}}=\qty{2.0}{\micro\fq\per\sqrthz}$, thus being comparable to previous low-noise SQUIDs from this working group. %The intrinsic and extrinsic energy sensitivities were accordingly low, where values of $\epsilon_{\rm s, w}=1.44\, h$ and $\epsilon_{{\rm s}, 1/f}=119.2\, h$ had been achieved for the intrinsic energy sensitivity, while the extrinsic energy sensitivity regarding the maXs100 detector yielded up to $\epsilon_{\rm p, w}=23.11\, h$ and $\epsilon_{{\rm p}, 1/f}=1910\, h$.



%This thesis discusses the optimisation of the $\mathrm{Nb}/\mathrm{Al}$-$\mathrm{AlO}_\mathrm{x}/\mathrm{Nb}$ tri-layer deposition for the fabrication of cross-type based Josephson tunnel junctions. Josephson tunnel junctions (JJs) are the core elements of various superconducting devices such as qubits or superconducting quantum interference devices (SQUIDs). The cross-type JJ geometry is motivated by the reduction of the junction area as well as parasitic capacities compared to the commonly used window-type geometry. The cross-like design removes parasitic effects with the additional benefit of simple and time efficient fabrication steps and relaxed alignment requirements during micro-fabrication. To ensure a reliable wafer-scale fabrication that yields JJs with a reproducible and uniform high quality, the in-house sputter-deposited niobium layers in a new sputtering system had to be investigated regarding their physical properties including the measurement of the critical temperature $T_{\mathrm{c}}$, the stress of the niobium film and junction specific quality features. The parameters for the magnetron sputtering, like the Ar pressure and power of the sputtering source, were optimised accordingly. This resulted in the successful fabrication of high quality cross-type JJs with reduced area sizes by at least a factor 4 and homogeneously distributed quality parameters on wafer-scale, which provides a basis for further developments of cross-type based dc-SQUIDs.


%Very often window-type JJs are used, in which the JJ area is defined by contacts through windows in an insulating layer. The size of the JJ is thus limited by alignment accuracy of the lithographic fabrication step, therfore causing the need for low critical current densities $j_{\mathrm{c}}$ and in turn long oxidation times. 

% 
%We present the optimization of our cross-type $\mathrm{Nb/Al}$-$\mathrm{AlO_x/Nb}$ junctions, for which quality checks of our in-house sputter-deposited niobium films and oxidized aluminium layers were carried out, including the measurement of the critical temperature $T_{\mathrm{c}}$, stress of the niobium film and junction specific quality features. The parameters for the magnetron sputtering, like the Ar pressure and power of the sputtering source, were optimised accordingly. This resulted in the fabrication of high quality cross-type JJs with homogeneously distributed quality parameters on wafer-scale, which provides a basis for further developments of cross-type based dc-SQUIDs.					
					\smallskip}\rule{0.5cm}{0cm}}
		\end{center}
		% ------------ END ENGLISH ABSTRACT ----------- %
		\vfill
		% -------------- GERMAN ABSTRACT ----------- %
		\begin{center}
			\fbox{\rule{0.5cm}{0cm}\parbox{14.9cm}{\smallskip
					\begin{center}
						Design, Herstellung und Charakterisierung von Stromsensor-dc-SQUIDs mit Induktivitätsanpassung zur Auslese von Metallischen Magnetischen Kalorimetern
						%Optimierung der Induktivitätsanpassung von Stromsensor-dc-SQUIDs zur Auslese metallischer magnetischer Kalorimeter   
					\end{center}					
					\selectlanguage{ngerman}
					\small
					\noindent
In der vorliegenden Arbeit werden die durchgeführten Methoden zur Optimierung der $\mathrm{Nb}/\mathrm{Al}$-$\mathrm{AlO}_\mathrm{x}/\mathrm{Nb}$ Dreischicht-Deponierung für die Herstellung von Josephson-Tunnelkontakten auf Basis einer kreuzförmigen Geometrie vorgestellt. Josephson-Tunnelkontakte (JJs) sind die Kernelemente verschiedener supraleitender elektronischer Bauelemente. Die Verwendung der kreuzförmigen JJ-Geometrie ist durch die Verringerung der Kontaktfläche sowie der parasitären Kapazitäten im Vergleich zu der üblicherweise verwendeten Fenstertyp-Geometrie motiviert. Das kreuzförmige Design beseitigt parasitäre Effekte und bietet zudem den Vorteil einfacher und zeitsparender Herstellungsschritte. Um eine zuverlässige Fabrikation auf Wafer-Skala zu gewährleisten, die JJs mit reproduzierbarer und homogen hoher Qualität liefert, mussten die in dem neuen institutsinternen Sputtersystem deponierten Niobschichten auf ihre physikalischen Eigenschaften hin untersucht werden, einschließlich der Messung der kritischen Temperatur $T_{\mathrm{c}}$, der Verspannung des Niobfilms und der JJ-spezifischen Qualitätsmerkmale. Die Parameter für das Magnetron-Sputtern, wie der Ar-Druck und die Leistung der Sputterquelle, wurden entsprechend optimiert. Dies führte zur erfolgreichen Herstellung von qualitativ hochwertigen Kreuztyp-Kontakten mit einer um mindestens den Faktor 4 reduzierten Flächengröße und homogen verteilten Qualitätsparametern auf Wafer-Skala, was eine Grundlage für weitere Entwicklungen von Kreuztyp-Kontakt basierten dc-SQUIDs liefert.

%Josephson-Tunnelkontakte sind die Grundelemente vieler supraleitender elektronischer Bauelemente wie Qubits oder ''Superconducting Quantum Interference Devices'' (SQUIDs). Da für viele Anwendungen eine große Anzahl von Josephson-Kontakten benötigt wird, ist ein zuverlässiger Herstellungsprozess auf Wafer-Skala erforderlich, der Josephson-Kontakte mit reproduzierbarer und einheitlich hoher Qualität liefert. Sehr häufig werden Fenstertyp-Kontakte verwendet, bei denen die Josephson-Kontaktfläche durch Fenster in einer Isolierschicht definiert ist. Die Größe der Kontakte ist somit durch die Ausrichtungsgenauigkeit des lithographischen Fabrikationsschritt begrenzt, was zu niedrigen kritischen Stromdichten $j_{\mathrm{c}}$ und damit zu langen Oxidationszeiten führt. Der Übergang zu kreuzförmigen Josephson-Kontakten ist durch die Reduzierung der Kontaktfläche und der Reduzierung der parasitären Kapazitäten motiviert. Das kreuzförmige Design unterdrückt oder beseitigt parasitäre Effekte mit dem zusätzlichen Vorteil einfacher und zeitsparender Herstellungsschritte. 
%Wir stellen die Optimierung unserer kreuzförmigen $\mathrm{Nb/Al}$-$\mathrm{AlO_x/Nb}$-Kontakte vor, für die Qualitätsprüfungen unserer Instituts-internen sputterabgeschiedenen Niobschichten und oxidierten Aluminiumschichten durchgeführt wurden, einschließlich der Messung der kritischen Temperatur $T_{\mathrm{c}}$, der Verspannung der Niobschicht und der spezifischen Qualitätsmerkmale der Josephson-Kontakte. %Die Parameter für das Magnetronsputtern, wie der Ar-Druck und die Leistung der Sputterquelle, wurden entsprechend optimiert. % Das Ergebnis war die Herstellung von qualitativ hochwertigen Kreuztyp-Kontakten mit homogen verteilten Qualitätsparametern auf Wafer-Skala, was eine Grundlage für weitere Entwicklungen von Cross-Type-basierten dc-SQUIDs liefert.

					\smallskip}\rule{0.5cm}{0cm}}
		\end{center}
		% ------------ END GERMAN ABSTRACT ----------- %
		%
		%
		\selectlanguage{german}
		%\vfill
	\end{titlepage}
	\setcounter{page}{-1}
	\rule{0cm}{17cm}
	\setlength{\textheight}{22.5cm}
	\setlength{\topmargin}{-3mm}
	\normalsize
	
%	\hypersetup{pageanchor=true}

\begin{document}

{
	\pagestyle{headings}	
	\renewcommand{\baselinestretch}{1.4}	
	\pagenumbering{roman}
	\normalsize
	\tableofcontents
	\vfill\eject 
	\ifthenelse{\isodd{\value{page}}}{}{\rule{0cm}{15cm}\vfill\eject}
}

\pagenumbering{arabic}
\pagestyle{headings}   %damit genau wie beim toc die Kopfzeile das schöne Design aus dipl_new übernimmt

\chapter{Theoretical Background}

This chapter provides a short introduction into Josephson junctions and their role in dc-SQUIDs\footnote{\textbf{d}irect \textbf{c}urrent \textbf{S}uperconducting \textbf{QU}antum \textbf{I}nterference \textbf{D}evice}, which will be the main focus of this thesis. We start with a brief overview on macroscopic quantum phenomena such as the Josephson effect and explain the general working principle of superconductor-isolator-superconductor (SIS) tunnel contacts, followed by a summary of their basic properties. They form the theoretical framework to describe SQUIDs, which are developed in this group and optimized within the scope of this thesis. Lastly, we will take a closer look into their resonance behavior and investigate different solution approaches. 

\section{Josephson junctions}


The \textit{Josephson junctions} named after Brian D. Josephson consist of two identical superconductors weakly coupled to each other. In the case of the junctions produced in this working group, such coupling is realized through a few nm thin insulating layer between the superconducting electrodes. Consequently, they are referred to as SIS (Superconductor-Insulator-Superconductor) junctions. The resulting trilayer structure typically consists of Nb/Al-Al$\rm O_x$/Nb, with niobium being used for the superconductors and the insulating layer being provided by the aluminum oxide. A schematic structure is shown in figure \ref{abb:fig:JJschem}. %When the junction is maintained at cold temperatures ($T\leq \qty{4}{\kelvin}$) and connected to a current source a supercurrent is measurable.
By connecting the tunnel junction to a current source they exhibit a non-trivial current-voltage behavior, which will be covered in the following. 


\figurecenter {b!}
{width=\textwidth}
{../Figures/jj_schematic}
%{7.8cm}
{0cm}
{Schematic of a Josephson (SIS) junction. Both superconducting electrodes $\textbf{\textit{S}}_\textbf{1}$ and $\textbf{\textit{S}}_\textbf{2}$ are weakly coupled with each other through a thin tunnel barrier \textbf{\textit{I}}. \bm{$\theta_{\rm 1}$} and \bm{$\theta_{\rm 2}$} represent the macroscopic phases of each superconductor.} 
{fig:JJschem}

%$\theta$ %$\textbf{\textit{\theta}_\textbf{1}}$ and  $\textbf{\textit{\theta}_\textbf{2}}$ represent the macroscopic phases of each superconductor.}
        
\subsection{Josephson effect}\label{subchap_Jeffect}

According to the BCS theory developed by Bardeen, Cooper and Schrieffer in 1957 \cite{Bardeen1957}, electrons in a superconductor form pairs below a material dependent critical temperature $T_{\rm c}$. These composite particles are also referred to as \textit{Cooper pairs} and they represent the superconducting charge carriers with twice the mass and charge of a single electron. Their dissipationless flow causes the current to have zero resistance, which is alongside the Meissner-Ochsenfeld effect \cite{Meissner1933} the most characteristic feature of a superconductor. The latter describes magnetic field expulsion below $T_{\rm c}$, provided the external magnetic field is smaller than a critical field $B_{\rm c}$. Further details on the microscopic theory of superconductivity can be found in \cite{Bardeen1957} and \cite{Ginzburg1950}.

If at $T < \qty{4}{\kelvin}$ an external current source is connected to a Josephson junction, a supercurrent will flow despite the tunnel barrier, implying the tunneling of Cooper pairs as niobium is predominantly superconducting at these temperatures ($T_\mathrm{c} = \qty{9.3}{\kelvin}$). Since the tunneling probability of an individual electron is approximately $p = \num{e-4}$ \cite{Gross2016}, a much lower probability is to be expected for a Cooper pair consisting of two electrons. However, Josephson predicted that the tunneling behavior of Cooper pairs and individual conduction electrons must be the same. This is justified by the so-called \textit{Macroscopic Quantum Model}, formulated by Fritz London in 1953.

The main focus here lies on the quantum mechanical phase $\theta$. On one hand, the distance between both electrons in a Cooper pair is approximately 10 to \qty{1000}{\nm} which is significantly larger than the spacing between Cooper pairs, resulting in strongly overlapping wave functions. On the other hand, Cooper pairs have to obey Bose-Einstein statistics due to their total spin of 0. Thus, all Cooper pairs share the same ground state, and as a consequence, the energies and temporal evolutions of the phases are equal. These two effects lead to what is known as \textit{phase-lock}. The phases of neighboring pairs synchronize such that this quantum mechanical property now holds on a macroscopic scale. This gives rise to a macroscopic wave function

\begin{equation}
\Psi(\textbf{r},t) = \Psi_0(\textbf{r},t)e^{i\theta(\textbf{r},t)} \ \ ,
\end{equation}

which describes all charge carriers of the superconductor. Here, the charge carrier density is given by $\left|\Psi_0(\textbf{r},t)\right|^2 = n_{\rm s}$. \textit{t} denotes the time and \textbf{r} represents the position of the Cooper pair ensemble. As a result of sharing the same phase, both electrons of a Cooper pair consequently possess the same tunneling probability as an individual electron, enabling the supercurrent. This coherence phenomenon is referred to as the \textit{Josephson effect} \cite{Josephson1962}. Another significant consequence of the macroscopic quantum model is flux quantization. Together with the Josephson effect, this forms the basis for Josephson junctions and their applications. 

\figureleft {t!}
{width=\textwidth}
{../Figures/quantized_flux}
{7cm}
{0cm}
{Superconducting ring-shaped cylinder threaded by an external magnetic field. By applying the field at low temperatures, shielding currents arise to expel the field from the superconductor. Upon turning off the external field the shielding currents will remain due to the lack of resistance, causing magnetic flux to be trapped. The dotted blue path \textit{C} is situated at the center of the cylinder wall, which we assume to be current-free due to the London penetration depth $\lambda_{\rm L}$ being much smaller than the thickness of the cylinder wall.} 
{fig:quantflux}

Flux quantization is derived through the capture of an external magnetic flux within a superconducting cylinder (see figure \ref{abb:fig:quantflux}). The wave function must remain unchanged after circumnavigating the cylinder due to $e^{i\theta} = e^{i\theta + 2\pi n}$. As a result, upon integrating along the current-free center of the cylinder wall (path $C$), the following equation holds for the captured flux

\begin{equation}
\Phi = \frac{h}{q_\mathrm{s}}n = \frac{h}{2e}n \equiv \Phi_0n \ \ .
\end{equation}

Here, $n\in\mathbb{Z}$ and \unit{\fq} = \qty{2.07e-15}{\tesla\metre\squared} \cite{CODATA2018} represents the so-called magnetic flux quantum. The captured flux is thus quantized, a consequence solely arising from the macroscopic nature of the phase. This quantity plays a crucial role in the theoretical description of Josephson junctions.


The current and voltage behavior in a SIS junction is described by the \textit{Josephson equations}. Crucial to this description is a critical current \textit{$I_\mathrm{c}$} that is linearly proportional to the applied current \textit{I}, which marks the boundary between two operational modes. Additionally, due to the macroscopic nature of the phase, \textit{I} oscillates with the gauge-invariant phase difference $\varphi$, leading to the \textbf{first Josephson equation} \cite{Josephson1965}

\begin{equation}
\label{1.JE}
I_\mathrm{s} = I_\mathrm{c}\sin(\varphi) \ \ .
\end{equation}

$I_\mathrm{c}$ is proportional to the coupling strength $\kappa$, which describes the overlap of the wave functions $\Psi_1$ and $\Psi_2$ within the insulating layer. The relationship is given by

\begin{equation}
I_\mathrm{c} = \frac{4e\kappa V n_\mathrm{s}}{\hbar} \ \ ,
\end{equation}

where \textit{V} represents the volume of the superconducting electrode and \textit{e} denotes the elementary charge of an electron. We assume that the Cooper pair density $n_\mathrm{s}$ of the two superconductors $S_1$ and $S_2$ is identical, meaning $n_{\mathrm{s}1} = n_{\mathrm{s}2} = n_\mathrm{s}$.

The gauge-invariant phase difference refers to the phases $\theta_1$ and $\theta_2$ of the respective electrodes at the boundary of the insulating layer (positions 1 and 2, see figure \ref{abb:fig:JJschem}). Taking into account possible external electromagnetic fields within the barrier, the general form using the vector potential \textbf{A} is given by \cite{Gross2016}

\begin{equation}
\label{EichInv_Phase}
\varphi(\textbf{r},t) = \theta_2(\textbf{r},t) - \theta_1(\textbf{r},t) - \frac{2\pi}{\Phi_0}\int_{1}^{2}\textbf{A}(\textbf{r},t)\cdot \mathrm{d}\textbf{l} \ \ .
\end{equation}

Assuming a constant supercurrent density $J_\mathrm{s}$ across the junction, taking the time derivative of equation \eqref{EichInv_Phase} yields the \textbf{second Josephson equation} \cite{Josephson1965}

\begin{equation}
\label{2.JE}
\frac{\partial\varphi}{\partial t} = \frac{2\pi}{\Phi_0}V \ \ .
\end{equation}

The first operating mode describes the case for $I<I_\mathrm{c}$. Here, the entire injected current is carried by Cooper pairs, so $I=I_\mathrm{s}=\mathrm{const}$. As a result, $\varphi$ is temporally constant, which, according to equation \eqref{2.JE}, leads to $V=0$. This voltage-free state is known as the \textit{dc Josephson effect}.

For $I>I_\mathrm{c}$ however, Cooper pairs begin to break up such that a portion of the current needs to be carried by quasiparticles, which will then lead to a voltage drop \textit{V}. According to the second Josephson equation, the phase $\varphi$ becomes time dependent, and after integration one obtains

\begin{equation}
\label{phi(t)}
\varphi = \frac{2\pi}{\unit{\fq}}Vt + \varphi_0 = w_\mathrm{J}t + \varphi_0 \ \ \ \mathrm{with} \ \ \ w_\mathrm{J} = \frac{2\pi}{\unit{\fq}}V \ \ .
\end{equation}

Thus, if we insert equation \eqref{phi(t)} into equation \eqref{1.JE}, we observe that  the current $I_\mathrm{s}$ oscillates with the \textit{Josephson frequency} $\frac{f_\mathrm{J}}{V} = \frac{w_\mathrm{J}}{2\pi V} = \frac{1}{\unit{\fq}} \approx \SI{483.6}{\MHz\per\uV}$. Accordingly, this phenomenon is referred to as the \textit{ac Josephson effect}.



\subsection{Josephson Junctions in a Magnetic Field}\label{subsec_jjmag}

\figurecenter {b!}
{width=0.8\textwidth}
{../Figures/shortjj_mag}
%{7.8cm}
{0cm}
{Short Josephson junction connected to a current source in the presence of an external B-field in y-direction, parallel to the junction area. Inside the electrodes the magnetic field decays exponentially according to the London penetration depths $\lambda_{\rm L,1}$ and $\lambda_{\rm L,2}$, visually shown by the purple color gradient. The closed contour \textit{C} is used to derive expressions for the spatially dependent phase difference $\varphi$ and current density $J_{\rm s}$.} 
{fig:JJMag}

To motivate the structure of a dc-SQUID, it is essential to first investigate the current behavior of an extended Josephson junction in the presence of an external magnetic field. So far, all previous formulae apply for point-like junctions, assuming a spatially constant phase difference $\varphi$ and Josephson current density $J_{\rm s}$ across the junction area. This is not the case for three-dimensional (extended) junctions with a length \textit{L} and width \textit{W}. The \textit{Josephson penetration depth} $\lambda_{\rm J}$ is a quantity used to classify an extended junction as short ($\rm W,L \leq\lambda_{\rm J}$) or long ($\rm W,L \geq\lambda_{\rm J}$) and is defined as 

\gl{
\lambda_{\rm J} = \sqrt{\frac{\unit{\fq}}{2\pi\unit{\micro_0}J_{\rm c}t_{\rm B}}
} \ \ .}{lamdaJ}

Here, the magnetic thickness is defined as $t_{\rm B} = d + \lambda_{\rm L,1} + \lambda_{\rm L,2}$. It describes how far an external magnetic field penetrates both superconducting electrodes if applied parallel to the junction area, as depicted in figure \ref{abb:fig:JJMag}. $\lambda_{\rm L,1}$ and $\lambda_{\rm L,2}$ are the respective London penetration depths and $J_{\rm c} = \frac{I_{\rm c}}{WL}$ the critical current density.
This distinction is needed to determine whether the magnetic self-field generated by the supercurrent is negligible in comparison to the external field (short junctions) or not (long junctions). Within the scope of this thesis, only short junctions are used.  

To analyze the current and phase distribution of such a junction we consider the setup shown in figure \ref{abb:fig:JJMag}. A short junction is connected to a current source and is penetrated by an external B-field in y-direction, parallel to the junction area. Now, obtaining an expression for the phase requires a similar approach as the calculation for the quantized flux, where we assumed that the phase changes by $2\pi n$ around a closed loop. Here, we again integrate over a closed contour \textit{C}, with the points $P_{\rm 1}-P_{\rm 4}$ marking the transitions between superconductor and isolator. Using equation \ref{EichInv_Phase}, we find 

\gl{
\frac{\del\varphi}{\del z} = \frac{2\pi}{\unit{\fq}}B_{\rm y}t_{\rm B} \ \ \ \mathrm{and} \ \ \ \frac{\del\varphi}{\del y} = -\frac{2\pi}{\unit{\fq}}B_{\rm z}t_{\rm B} \ \ .
}{phi(z,y)}

In this experiment, however, the magnetic field points in y-direction only, meaning $\varphi$ will only vary along the z-axis. Integrating the first expression in equation \ref{phi(z,y)} then leads to

\gl{
\varphi(z) = \frac{2\pi}{\unit{\fq}}B_{\rm y}t_{\rm B}z + \varphi_0 \ \ .
}{phi(z)} 

Here, the integration constant $\varphi_0$ represents the phase difference for the case $z=0$. Inserting equation \ref{phi(z)} into the first Josephson equation and using $J_{\rm s} = \frac{I_{\rm s}}{WL}$ gives 

\gl{
J_{\rm s}(y,z,t) = J_{\rm c}(y,z)\sin(kz + \varphi_0) \ \ \ \mathrm{with} \ \ \ k = \frac{2\pi}{\unit{\fq}}B_{\rm y}t_{\rm B} \ \ .
}{Js(y,z,t)}

If we now assume the critical current density $J_{\rm c}$ to be constant across the junction area, we can integrate equation \ref{Js(y,z,t)} to get a flux-dependent maximum Josephson current

\gl{
I_{\rm s}^{\rm m}(\Phi)	= I_{\rm c}\left|\frac{\sin(\frac{kL}{2})}{\frac{kL}{2}}\right| = I_{\rm c}\left|\frac{\sin(\frac{\pi\Phi}{\unit{\fq}})}{\frac{\pi\Phi}{\unit{\fq}}}\right| \ \ .
}{Ismax}

This expression describes the so-called Fraunhofer diffraction pattern, shown in figure \ref{abb:fig:fraunhofer}. The result resembles the single slit experiment, where the same pattern is found for the light intensity behind the slit. Here, the analogy works by considering the integral of the critical current density $J_{\rm c}$ as a transmission function which is constant inside the junction and zero outside. 

\figureleft {t!}
{width=\textwidth}
{../Figures/fraunhofer}
{9cm}
{0cm}
{Normalized flux-dependent maximum Josephson current $I_{\rm s}^{\rm m}(\Phi)$ showing a Fraunhofer pattern. It modulates with the flux quantum $\unit{\fq}$, peaking at $\Phi=0$ with subsequent maxima at $\Phi=\pm(\frac{3}{2}+n)\unit{\fq}$ with $n\in\mathbb{N}_0$. For $\Phi=\pm(n+1)\unit{\fq}$ the total net current is zero.} 
{fig:fraunhofer}


\subsection{RCSJ Model}

Until now, we investigated the current-voltage behavior under the assumption of $I<I_{\rm c}$, staying in the so-called zero-voltage state. In this regime, only the dc Josephson effect applies as discussed in subchapter \ref{subchap_Jeffect}. Switching to the voltage stage, i.e. $I>I_{\rm c}$, Cooper pairs start breaking up into quasiparticles if the electric energy $eV$ and/or thermal energy $k_{\rm B}T$ exceeds the sum of both electrodes' gap energies $\Delta_1 + \Delta_2$. Consequently, at the \textit{gap-voltage} 

\gl{
V_{\rm g} = \frac{\Delta_1(T)+\Delta_2(T)}{e}
}{Vgap}

quasiparticles start to cross the tunnel barrier resulting in a steep rise of a resistive normal current $I_{\rm n}$. Under a current source, the condition $I=I_{\rm s}+I_{\rm n}$ must be constantly fulfilled. This results in an oscillating normal current, since $I_{\rm s}$ oscillates with $f_{\rm J}$ according to the ac Josephson effect. $\frac{\mathrm{d}\varphi}{\mathrm{d}t}$ will therefore vary sinusoidally, causing both $I_{\rm s}$ and $I_{\rm n}$ and the resulting voltage to oscillate in a complex manner. As a voltage with such a high frequency cannot be measured, only the time-averaged voltage will be considered in the following discussion. \\
Now, further increasing the energy of the quasiparticles ($T>T_{\rm c}$ and/or $V>V_{\rm g}$) leads to a transition into normal-conducting electrons, which exhibit an ohmic dependence. This behavior can be seen in the typical current-voltage-characteristic (IVC) depicted in figure \ref{abb:IVCs}. \\ 

\twofigurescenter {t!}
{width=0.48\textwidth}
{../Figures/CrossJJ4w13v3_2C12_4_2umJJ_edited_Alex}
{0.02\textwidth} %hspace b/w figures
{width=0.48\textwidth}
{../Figures/JJ_IVC_BetaC_smaller_1_Alex}
{0.5cm} %vspace
{Platzhalter Plots von Alex -> eigene Messung vornehmen?... Measured IVCs from cross-type junctions manufactured in this working group. Left: Underdamped junction showing the typical hysteresis. Right: Overdamped junction with a current-voltage shape that is independent of the current sweep direction. }
{IVCs}

Real junctions, however, are comprised of two electrodes separated by a thin insulating layer, which represent a parallel plate capacitor with the Al-Al$\rm O_x$ layer being the dielectric material. Therefore, a junction capacitance \textit{C} needs to be taken into account. A displacement current $I_{\rm d}$ will flow as a consequence, given we are in the voltage state. Lastly, thermal and 1/f noise cause a small fluctuating current $I_{\rm f}$. All these current channels were defined in the so-called Resistively and Capacitively Shunted Junction (RCSJ) model \cite{Cumber1968}, \cite{Stewart1968}, which models the total current of a lumped (0-dimensional) junction to a sufficiently high accuracy. A schematic of an effective circuit diagram is shown in figure \ref{abb:fig:rcsj} (left). Combining every current channel leads to the \textit{Basic Junction Equation}, which is defined as \cite{Gross2016}

\gl{
I=I_{\rm s}+I_{\rm n}+I_{\rm d}+I_{\rm f}=I_{\rm c}\sin(\varphi)+\frac{1}{R(V)}\frac{\unit{\fq}}{2\pi}\frac{\mathrm{d}\varphi}{\mathrm{d}t}+C\frac{\unit{\fq}}{2\pi}\frac{\mathrm{d}^2\varphi}{\mathrm{d}t^2}+I_{\rm f} \ \ .
}{BJE}

By defining the Josephson coupling energy $U_{\rm J0}=\frac{\hbar I_{\rm c}}{2e}$ and the normalized currents $i=\frac{I}{I_{\rm c}}$ and $i_{\rm f}(t)=\frac{I_{\rm f}(t)}{I_{\rm c}}$, equation \ref{BJE} can be rewritten to 

\gl{
\left(\frac{\hbar}{2 e}\right)^2 C \frac{\mathrm{d}^2 \varphi}{\mathrm{d} t^2}+\left(\frac{\hbar}{2 e}\right)^2 \frac{1}{R(V)} \frac{\mathrm{d} \varphi}{\mathrm{d} t}+\frac{\mathrm{d}}{\mathrm{d} \varphi}\left\{U_{\rm J 0}\left[1-\cos \varphi-i \varphi+i_{\rm f}(t) \varphi\right]\right\}=0 
}{BJE2} \ \ . 

\twofigurescenter {t!}
{width=0.48\textwidth}
{../Figures/RCSJ-Modell}
{0.02\textwidth} %hspace b/w figures
{width=0.48\textwidth}
{../Figures/washboard}
{0.5cm} %vspace
{Left: Schematic circuit of a lumped Josephson junction with all four current channels connected in parallel. The junction is represented by the cross symbol on the left, marking the supercurrent $I_{\rm s}$. The normal current $I_{\rm n}$ is realized with a resistance $R$, while the displacement current $I_{\rm d}$ and the noise $I_{\rm f}$ need a capacitor $C$ and a current source, respectively. Right: Tilted washboard potential for different currents, ranging from 0 to $1.5I_{\rm c}$. The tilt increases with the injected current $I$.}
{fig:rcsj}

The expression in the curly brackets represents the potential energy in the system $U_{\rm J}$, allowing equation \ref{BJE} to be compared to 

\gl{
M\frac{\mathrm{d}^2 x}{\mathrm{d}t^2} + \eta\frac{\mathrm{d} x}{\mathrm{d}t} + \nabla U = 0
}{} \ \ .

This equation describes a particle with mass \textit{M} and damping $\eta$ moving inside the potential \textit{U}. The mechanical analogue therefore allows us to interpret a \textit{phase particle}, where it's motion corresponds to a change of the gauge-invariant phase difference $\varphi$ within a potential $U_{\rm J}$ \cite{Clarke2004}. Consequently, it is attributed with a mass $M=\left(\frac{\hbar}{2 e}\right)^2C$ and damping $\eta=\left(\frac{\hbar}{2 e}\right)^2\frac{1}{R(V)}$. Figure \ref{abb:fig:rcsj} (right) visualizes how this phase particle behaves for different currents \textit{I}. Given the shape of $U_{\rm J}(\varphi)$, the potential is referred to as the \textit{tilted washboard potential}. \\
For $I=0$, the phase particle will remain within one of the potential minima. As the current increases, however, the potential starts to tilt such that the depth of the minima reduces until it vanishes for $I=I_{\rm c}$, thus becoming a saddle point. Up until this point, the phase particle can't overcome the potential barrier to move downward, which confirms the second Josephson equation as the phase difference $\varphi$ should remain constant for $I<I_{\rm c}$. Further increasing the current and therefore the tilt of the potential causes the phase particle to fall along the potential, resulting in a voltage drop across the junction ($\frac{\del\varphi}{\del t}>0$). \\

Reversing the current sweep showcases the importance of the particle's mass $M$ and damping $\eta$, as they determine if the return path equals the above described current shape or not. For the case of a small mass (small $C$) and large damping (small $R$), the phase particle will, due to a lack of momentum, come to a halt as soon as minima reappear in the washboard potential by reducing the current below $I_{\rm c}$. The current path will therefore remain unchanged as $I$ is reduced back to 0, as shown in figure \ref{abb:IVCs} (right). Such a junction is consequently called an \textit{overdamped} junction. \\
The other case describes an \textit{underdamped} junction (figure \ref{abb:IVCs} (left)) and involves a large mass (large $C$) and small damping (large $R$). This allows the phase particle to continue to move downward as it now carries enough momentum to overcome the arising maxima and minima. The finite voltage drop despite the current being below $I_{\rm c}$ is displayed as the steep quasiparticle current curve, which ends with a return current $I_{\rm R}$ that arises with the recapture of the particle in a minimum. This leads to a hysteretic IVC, as depicted in figure \ref{abb:IVCs} (left). $I_{\rm R}$ can be calculated via \cite{Likharev1986}

\gl{
I_{\rm R} = \frac{4}{\pi\sqrt{\beta_{\rm C}}}I_{\rm c} \ \ ,
}{IR}

with $\beta_{\rm C}$ being the dimensionless Stewart-McCumber parameter, that is used to quantitatively distinguish between both junction types. It is given by 

\gl{
\beta_{\rm C} = \frac{2\pi}{\unit{\fq}}I_{\rm c}R^2C
}{betaC} 

with $\beta_{\rm C}\gg 1$ corresponding to a strongly underdamped junction, whereas $\beta_{\rm C}\ll 1$ represents a strongly overdamped junction. The junctions developed and produced within the scope of this thesis aim to be overdamped, which is why we take a closer look on the time-averaged voltage for $I>I_{\rm c}$ in the case of $ \beta_{\rm C}\ll 1$. Neglecting the noise in equation \ref{BJE2}, as well as assuming the resistance to be linear below and above the gap voltage $V_{\rm g}$, i.e. $R(V)=R$, the time-averaged voltage can be derived to 

\gl{
\langle V(t)\rangle=I_{\mathrm{c}} R \sqrt{\left(\frac{I}{I_{\mathrm{c}}}\right)^2-1} \ \ \ \mathrm{for} \ \ \ \frac{I}{I_{\mathrm{c}}}>1 \ \ .
}{}

\section{dc-SQUIDs}

We have now covered the theoretical framework necessary to understand the working principle of a dc-SQUID, which consists of a superconducting ring intersected by two identical Josephson junctions with critical Josephson currents $I_{\rm c}$, as depicted in figure \ref{abb:dcSQUID}. Both junctions are shunted with shunt resistors $R_{\rm s}$ to avoid hysteretic behavior in the respective IVCs. If the loop is then biased with a bias current $I_{\rm b}$ while being threaded by an external magnetic flux $\Phi_{\rm e}$, it is possible to convert small flux variations into a measurable voltage change. dc-SQUIDs are therefore used as highly sensitive flux-to-voltage transducers.

\subsection{Zero Voltage State}

\figureleft {b!}
{width=\textwidth} %sets how much of the fig space is used
{../Figures/dc-SQUID}
{9cm} %sets width of the fig space
{0cm}
{Schematic circuit diagram of a shunted dc-SQUID. A superconducting loop with inductance $L_{\rm s}$ is interrupted by two lumped Josephson junctions such that they form a parallel connection. Operation requires a bias current $I_{\rm b}$ and an external magnetic flux $\Phi$. To avoid hysteresis effects, a shunt resistance $R_{\rm s}$ is connected to each junction.} 
{dcSQUID}

In order to fully understand the working principle of a dc-SQUID it is again necessary to first cover the zero voltage stage as we did for a single junction. 
The parallel connection of the two junctions allows the bias current to split into two supercurrents $I_{\rm s1}$, $I_{\rm s2}$ with identical critical currents, i.e. $I_{\rm c,1}=I_{\rm c,2}=I_{\rm c}$. Here we assume $I_{\rm b}<2I_{\rm c}$ to ensure that no voltage drop across both junctions occurs ($V_{\rm s}=0$). Applying Kirchhoff's law we then obtain the following expression 

\gl{
I_{\rm b} = I_{\mathrm{s}}=I_{\mathrm{c}} \sin \varphi_1+I_{\mathrm{c}} \sin \varphi_2=2 I_{\mathrm{c}} \cos \left(\frac{\varphi_1-\varphi_2}{2}\right) \sin \left(\frac{\varphi_1+\varphi_2}{2}\right) \ \ .
}{squid_Is_tot}

In chapter \ref{subsec_jjmag} we concluded that a magnetic flux $\Phi$ causes the supercurrent to modulate with $\unit{\fq}$. A dc-SQUID can be considered as a single junction with a much larger effective area $A_{\rm eff}$ (loop area), that an external magnetic fux can penetrate. It is therefore reasonabe to expect a similar behavior for a dc-SQUID. The same approach as with a single junction is used to determine the flux dependency of the total supercurrent, where a closed loop integral is performed around the SQUID loop. The calculation leads to the relation ref???  

\gl{
\varphi_2-\varphi_1=\frac{2 \pi \Phi}{\Phi_0}
 \ \ ,}{}

which can be directly inserted into equation \ref{squid_Is_tot} to obtain

\gl{
I_{\mathrm{s}}=2 I_{\mathrm{c}} \cos \left(\pi \frac{\Phi}{\Phi_0}\right) \sin \left(\varphi_1+\pi \frac{\Phi}{\Phi_0}\right) \ \ .
}{}

\gl{
I_{\mathrm{s}}^{\mathrm{m}}(\Phi)=2 I_{\mathrm{c}}\left|\cos \left(\pi \frac{\Phi}{\Phi_0}\right)\right|
}{}

\gl{
\Phi=\Phi_{\rm e} + \Phi_{\rm L} =
 \Phi_{\rm e} - L I_{\rm  c} \sin \left(\pi \frac{\Phi}{\Phi_0}\right) \cos \left(\varphi_1+\pi \frac{\Phi}{\Phi_0}\right)
}{}

\gl{
\Phi=\Phi_{\mathrm{e}}-\frac{1}{2} \beta_L \Phi_0 \sin \left(\pi \frac{\Phi}{\Phi_0}\right) \cos \left(\varphi_1+\pi \frac{\Phi}{\Phi_0}\right)
}{}

\gl{
I_{\mathrm{s}}^{\mathrm{m}}(\Phi_{\rm e})=2 I_{\mathrm{c}}\left|\cos \left(\pi \frac{\Phi_{\rm e}}{\Phi_0}\right)\right|
}{}

\gl{
\frac{I_{\mathrm{s}}^{\mathrm{m}}\left(\Phi_{\mathrm{e}}\right)}{2 I_{\mathrm{c}}} \approx 1-\frac{2 \Phi_{\mathrm{e}}}{\Phi_0 \beta_{\mathrm{L}}}
}{}

\figureleft {b!}
{width=\textwidth}
{../Figures/Phi_betaL}
{9cm}
{0cm}
{Caption} 
{Phi_betaL}

\subsection{Voltage State}

\subsection{Noise}

\subsection{Parasitic Resonances}




\chapter{Metallic Magnetic Microcalorimeters} 



\bibliographystyle{bibstyle_andi_english} 
\bibliography{library}





\end{document}