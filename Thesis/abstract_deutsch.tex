\noindent
In der vorliegenden Arbeit werden die durchgeführten Methoden zur Optimierung der $\mathrm{Nb}/\mathrm{Al}$-$\mathrm{AlO}_\mathrm{x}/\mathrm{Nb}$ Dreischicht-Deponierung für die Herstellung von Josephson-Tunnelkontakten auf Basis einer kreuzförmigen Geometrie vorgestellt. Josephson-Tunnelkontakte (JJs) sind die Kernelemente verschiedener supraleitender elektronischer Bauelemente. Die Verwendung der kreuzförmigen JJ-Geometrie ist durch die Verringerung der Kontaktfläche sowie der parasitären Kapazitäten im Vergleich zu der üblicherweise verwendeten Fenstertyp-Geometrie motiviert. Das kreuzförmige Design beseitigt parasitäre Effekte und bietet zudem den Vorteil einfacher und zeitsparender Herstellungsschritte. Um eine zuverlässige Fabrikation auf Wafer-Skala zu gewährleisten, die JJs mit reproduzierbarer und homogen hoher Qualität liefert, mussten die in dem neuen institutsinternen Sputtersystem deponierten Niobschichten auf ihre physikalischen Eigenschaften hin untersucht werden, einschließlich der Messung der kritischen Temperatur $T_{\mathrm{c}}$, der Verspannung des Niobfilms und der JJ-spezifischen Qualitätsmerkmale. Die Parameter für das Magnetron-Sputtern, wie der Ar-Druck und die Leistung der Sputterquelle, wurden entsprechend optimiert. Dies führte zur erfolgreichen Herstellung von qualitativ hochwertigen Kreuztyp-Kontakten mit einer um mindestens den Faktor 4 reduzierten Flächengröße und homogen verteilten Qualitätsparametern auf Wafer-Skala, was eine Grundlage für weitere Entwicklungen von Kreuztyp-Kontakt basierten dc-SQUIDs liefert.

%Josephson-Tunnelkontakte sind die Grundelemente vieler supraleitender elektronischer Bauelemente wie Qubits oder ''Superconducting Quantum Interference Devices'' (SQUIDs). Da für viele Anwendungen eine große Anzahl von Josephson-Kontakten benötigt wird, ist ein zuverlässiger Herstellungsprozess auf Wafer-Skala erforderlich, der Josephson-Kontakte mit reproduzierbarer und einheitlich hoher Qualität liefert. Sehr häufig werden Fenstertyp-Kontakte verwendet, bei denen die Josephson-Kontaktfläche durch Fenster in einer Isolierschicht definiert ist. Die Größe der Kontakte ist somit durch die Ausrichtungsgenauigkeit des lithographischen Fabrikationsschritt begrenzt, was zu niedrigen kritischen Stromdichten $j_{\mathrm{c}}$ und damit zu langen Oxidationszeiten führt. Der Übergang zu kreuzförmigen Josephson-Kontakten ist durch die Reduzierung der Kontaktfläche und der Reduzierung der parasitären Kapazitäten motiviert. Das kreuzförmige Design unterdrückt oder beseitigt parasitäre Effekte mit dem zusätzlichen Vorteil einfacher und zeitsparender Herstellungsschritte. 
%Wir stellen die Optimierung unserer kreuzförmigen $\mathrm{Nb/Al}$-$\mathrm{AlO_x/Nb}$-Kontakte vor, für die Qualitätsprüfungen unserer Instituts-internen sputterabgeschiedenen Niobschichten und oxidierten Aluminiumschichten durchgeführt wurden, einschließlich der Messung der kritischen Temperatur $T_{\mathrm{c}}$, der Verspannung der Niobschicht und der spezifischen Qualitätsmerkmale der Josephson-Kontakte. %Die Parameter für das Magnetronsputtern, wie der Ar-Druck und die Leistung der Sputterquelle, wurden entsprechend optimiert. % Das Ergebnis war die Herstellung von qualitativ hochwertigen Kreuztyp-Kontakten mit homogen verteilten Qualitätsparametern auf Wafer-Skala, was eine Grundlage für weitere Entwicklungen von Cross-Type-basierten dc-SQUIDs liefert.
