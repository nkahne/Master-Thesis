\noindent
In der vorliegenden Arbeit wird das Design und die Entwicklung eines Stromsensor-dc-SQUIDs beschrieben, welches hinsichtlich der Auslese von Teilchendetektoren auf Basis metallischer magnetischer Kalorimeter optimiert wurde. Die Maximierung der Energieauflösung des Detektors wird durch Minimierung des intrinsischen SQUID Rauschens und durch die Anpassung der Eingangsinduktivität des SQUIDs an die Induktivität des Detektors erreicht. 
Das neue Design verwendet eine Einkoppelspule mit zwei Windungen, um die Eingangsinduktivität an die Detektorinduktivität $L_{\rm p}=\qty{6.65}{\pH}$ des maXs100-Detektors anzugleichen.
%Dies wurde durch das Erweitern der Einkoppelspule eines zuvor hergestellten SQUIDs mithilfe einer zweiten Windung realisiert, sodass diese der Induktivität der Detektorspule $L_{\rm p}=\qty{6.65}{\pH}$ des maXs100-Detektors angepasst wird, welcher derzeit in dieser Arbeitsgruppe entwickelt wird. 
Um Resonanzstrukturen in den SQUID Kennlinien zu minimieren, werden zwei neue Dämpfungstechniken eingesetzt. Einerseits werden die Zuleitungen durch eine galvanisch getrennte Goldschicht induktiv gedämpft, andererseits wird die Einkoppelspule als Au/Nb Zweischichtstruktur strukturiert, wobei letzteres eine verlustreiche Mikrostreifenleitung darstellt. Mit Ersterem wurde eine erfolgreiche Dämpfung erreicht, was durch wesentlich glattere Strom-Spannungs-Kennlinien demonstriert werden konnte. Die Rauschmessungen wurden für das neue SQUID mit verschiedenen Kombinationen der Dämpfungsmethoden bei einer Temperatur von $\qty{10}{\milli\kelvin}$ durchgeführt. 
Diese ergaben 
Diese ergaben bis zu $\sqrt{S_{\Phi_{\rm s}, \rm w}}=\qty{0.22}{\micro\fq\per\sqrthz}$ für den frequenzunabhängigen Beitrag, während für die $1/f$-Komponente ein Wert von $\sqrt{S_{\Phi_{\rm s}, 1/f}}=\qty{2.0}{\micro\fq\per\sqrthz}$ erreicht werden konnte, was vergleichbar mit früheren rauscharmen SQUIDs aus dieser Arbeitsgruppe ist. Die Anpassung der Eingangsinduktivität zu $L_{\rm i}=\qty{6.4}{\nH}$ führt zu einer verbesserten Kopplung $\Delta\Phi_{\rm s}/\Delta\Phi=\qty{2.25}{\percent}$. Damit stellt das neue Design ein verbessertes Stromsensor SQUID zur Auslese des maXs100 Detektors dar. 
%Die intrinsische und extrinsische Energiesensitivität waren entsprechend niedrig, wobei Werte von $\epsilon_{\rm s, w}=1.44, h$ und $\epsilon_{{\rm s}, 1/f}=119.2, h$ für die intrinsische Energiesensitivität erreicht wurden, während die extrinsische Energiesensitivität für den maXs100-Detektor bis zu $\epsilon_{\rm p, w}=23.11, h$ und $\epsilon_{{\rm p}, 1/f}=1910, h$ ergab.


%In der vorliegenden Arbeit werden die durchgeführten Methoden zur Optimierung der $\mathrm{Nb}/\mathrm{Al}$-$\mathrm{AlO}_\mathrm{x}/\mathrm{Nb}$ Dreischicht-Deponierung für die Herstellung von Josephson-Tunnelkontakten auf Basis einer kreuzförmigen Geometrie vorgestellt. Josephson-Tunnelkontakte (JJs) sind die Kernelemente verschiedener supraleitender elektronischer Bauelemente. Die Verwendung der kreuzförmigen JJ-Geometrie ist durch die Verringerung der Kontaktfläche sowie der parasitären Kapazitäten im Vergleich zu der üblicherweise verwendeten Fenstertyp-Geometrie motiviert. Das kreuzförmige Design beseitigt parasitäre Effekte und bietet zudem den Vorteil einfacher und zeitsparender Herstellungsschritte. Um eine zuverlässige Fabrikation auf Wafer-Skala zu gewährleisten, die JJs mit reproduzierbarer und homogen hoher Qualität liefert, mussten die in dem neuen institutsinternen Sputtersystem deponierten Niobschichten auf ihre physikalischen Eigenschaften hin untersucht werden, einschließlich der Messung der kritischen Temperatur $T_{\mathrm{c}}$, der Verspannung des Niobfilms und der JJ-spezifischen Qualitätsmerkmale. Die Parameter für das Magnetron-Sputtern, wie der Ar-Druck und die Leistung der Sputterquelle, wurden entsprechend optimiert. Dies führte zur erfolgreichen Herstellung von qualitativ hochwertigen Kreuztyp-Kontakten mit einer um mindestens den Faktor 4 reduzierten Flächengröße und homogen verteilten Qualitätsparametern auf Wafer-Skala, was eine Grundlage für weitere Entwicklungen von Kreuztyp-Kontakt basierten dc-SQUIDs liefert.

%Josephson-Tunnelkontakte sind die Grundelemente vieler supraleitender elektronischer Bauelemente wie Qubits oder ''Superconducting Quantum Interference Devices'' (SQUIDs). Da für viele Anwendungen eine große Anzahl von Josephson-Kontakten benötigt wird, ist ein zuverlässiger Herstellungsprozess auf Wafer-Skala erforderlich, der Josephson-Kontakte mit reproduzierbarer und einheitlich hoher Qualität liefert. Sehr häufig werden Fenstertyp-Kontakte verwendet, bei denen die Josephson-Kontaktfläche durch Fenster in einer Isolierschicht definiert ist. Die Größe der Kontakte ist somit durch die Ausrichtungsgenauigkeit des lithographischen Fabrikationsschritt begrenzt, was zu niedrigen kritischen Stromdichten $j_{\mathrm{c}}$ und damit zu langen Oxidationszeiten führt. Der Übergang zu kreuzförmigen Josephson-Kontakten ist durch die Reduzierung der Kontaktfläche und der Reduzierung der parasitären Kapazitäten motiviert. Das kreuzförmige Design unterdrückt oder beseitigt parasitäre Effekte mit dem zusätzlichen Vorteil einfacher und zeitsparender Herstellungsschritte. 
%Wir stellen die Optimierung unserer kreuzförmigen $\mathrm{Nb/Al}$-$\mathrm{AlO_x/Nb}$-Kontakte vor, für die Qualitätsprüfungen unserer Instituts-internen sputterabgeschiedenen Niobschichten und oxidierten Aluminiumschichten durchgeführt wurden, einschließlich der Messung der kritischen Temperatur $T_{\mathrm{c}}$, der Verspannung der Niobschicht und der spezifischen Qualitätsmerkmale der Josephson-Kontakte. %Die Parameter für das Magnetronsputtern, wie der Ar-Druck und die Leistung der Sputterquelle, wurden entsprechend optimiert. % Das Ergebnis war die Herstellung von qualitativ hochwertigen Kreuztyp-Kontakten mit homogen verteilten Qualitätsparametern auf Wafer-Skala, was eine Grundlage für weitere Entwicklungen von Cross-Type-basierten dc-SQUIDs liefert.
